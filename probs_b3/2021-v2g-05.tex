\setAuthor{Jaan Toots}
\setRound{piirkonnavoor}
\setYear{2021}
\setNumber{G 5}
\setDifficulty{5}
\setTopic{TODO}

\prob{Vooluallikas}
Elektriskeem koosneb vooluallikast, mis annab välja konstantset voolu $I_0$ ja sellega rööpselt ühendatud sisetakistusest $R$. Leidke maksimaalne võimsus, mis saab klemmide $A$ ja $B$ vahele ühendatud tarbijal eralduda.
\begin{figure}[H]
  \centering
  \begin{circuitikz} \draw
    (0,0) to[I=$I_0$] (0,2)
    to[short] (2,2)
    to[R,l=$R$] (2,0)
    to[short] (0,0)

    (2,2) to[short, -o] (4,2)node[right]{$A$}
    (2,0) to[short, -o] (4,0)node[right]{$B$}
    % (3.3,2) to[open, v^=$V$] (3.3,0)
    ;
  \end{circuitikz}
\end{figure}


\hint

\solu
Olgu pinge koormise klemmidel (ja ühtlasi ka sisetakisti klemmidel) $V$.
Siis voolutugevus läbi koormise on $I = I_0 - \frac VR$ \pp2. Seega võimsus on
$$P = IV = I_0V - \frac{V^2}R \quad \pp3$$
Järelikult võimsus koormisel avaldub kui ruutfunktsioon pingest. Selle ruutfunktsiooni maksimumi leidmiseks viime võimsuse avaldise kujule
$$P=-\frac1R \left(V - \frac{I_0R}2\right)^2 + \frac{I_0^2R}4 \quad \pp2$$
Paneme tähele, et $R>0$ ja $(V - \frac{I_0R}{2})^2 \ge 0$. \par
Seega on maksimaalne võimsus $P_0 = \frac{I_0^2R}{4}$ \pp1 ja seda pingel $V = \frac{I_0R}{2}$.
\probend