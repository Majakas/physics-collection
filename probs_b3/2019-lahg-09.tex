\setAuthor{Jaan Kalda}
\setRound{lahtine}
\setYear{2019}
\setNumber{G 9}
\setDifficulty{9}
\setTopic{TODO}

\prob{Hantel}
Kaks metallkera raadiusega $R$ ja massiga $m$ on ühendatud koaksiaalselt metallvardaga, mille pikkus $L$ on hulga suurem $R$-st ja mille mass ning diameeter on tühiselt väiksed. Kogu süsteem asub kaalutuse tingimustes teljega ristsihilises homogeenses magnetväljas induktsiooniga $B$. Ühele kuulikestest kantakse hetkeliselt elektrilaeng $Q$; visandage ühel ja samal joonisel mõlema kuulikese trajektoorid edasise liikumise käigus. Eeldada, et varda elektriline takistus on tühiselt väike ja et $\varepsilon_0B^2L^2R\ll m$, kus $\varepsilon_0$ tähistab vaakumi dielektrilist läbitavust.



\hint

\solu
Tagamaks, et hantel on ekvipotentsiaalne, voolab pool laengust teisele kerale; see toimub väga kiiresti, sest varda takistus on tühiselt väike (eeldame, et $RC$-aeg on hulga väiksem tsüklotronperioodist). Laengu voolamise ajal mõjub vardale Ampère'i jõud $F=BL\frac {\mathrm d q}{\mathrm dt}$, kus $q$ tähistab teise kera laengut. Seetõttu on ülekantav jõuimpulss leitav kui $\Delta p=\int F\mathrm dt=\int BL\mathrm dq = BLQ/2$. Järelikult hakkab hantel liikuma kiirusega $v=\Delta p/2m=BLQ/4m$. Kuivõrd mõlema kuuli massid ja laengud on samad, siis hakkavad nad liikuma magnetväljas ühtmoodi, tsüklotronsagedusega $\omega=BQ/2m$ ja tsüklotronraadiusega $r=v/\omega=L/2$. Järelikult on kuulikeste trajektoorid ringjooned raadiusega $L/2$ ja diameetriga $L$, st tegemist on kahe üksteist puudutava ringjoonega.
\probend