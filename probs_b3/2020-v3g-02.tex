\setAuthor{}
\setRound{lõppvoor}
\setYear{2020}
\setNumber{G 2}
\setDifficulty{2}
\setTopic{TODO}

\prob{Kohukesed}
Richard ostis kohukesi, mis on ühtlase kihina poekoti põhjas. Heatujulisena teeb
ta kotiga vertikaalseid ringe, aga siis hakkab kohukeste pärast muretsema. Õnneks
selgub, et need ei kukkunud välja ja lömaks ka ei läinud. Näidake, et kui Richard
kotiga ei vehi (ega seda keeruta), siis võib ta panna koti põhja kaks korda
paksema kihi kohukesi, ilma et ükski kohuke lömastuks. Kohukesi võib käsitleda
vedelikuna. Võib eeldada, et kohuke lömastub siis, kui rõhk tema juures ületab
mingi kriitilise väärtuse. Vertikaalsete ringide tegemisel on Richardi käe
nurkkiirus konstantne. Eeldada, et kohukeste kihi paksus on palju väiksem kui
ringi raadius.


\hint

\solu
Koti mõõtmed on Richardi käe pikkusega võrreldes väikesed, seega võime eeldada, et nurkkiirendus on kõigile kohukestele ühtemoodi $\omega^2 \ell$, kus $\ell$ on Richardi käe pikkus. Kohukesed ei kuku kotist välja, seega ringi ülemisest punktist saame $\omega^2\ell\geq g$. Kohukesed ei lähe alumises punktis lömaks, seega rõhk $\rho (g+\omega^2 \ell) h$ pole piisav, et kohukest lömastada. Eelnevast
\[\rho (g+\omega^2 \ell) h\geq \rho g 2h.\]

Seega saaks kotti panna kõrgusega $2h$ kihi kohukesi, ilma et need lömaks läheks.
\probend