\setAuthor{Kaarel Kivisalu}
\setRound{lahtine}
\setYear{2020}
\setNumber{G 1}
\setDifficulty{1}
\setTopic{TODO}

\prob{Pudel}
Jäigast materjalist pudel (nt klaaspudel) on täidetud osaliselt veega. Rael pani tähele, et väikse pudelikaelaga on pudelist raske kogu vedelikku ära juua. Leidke, mis on maksimaalne vedeliku ruumala, mida on võimalik ära juua ilma pudelisse õhku juurde puhumata. Pudeli ruumala on $V$ ja Rael suudab pudelisse tekitada alarõhu $\Delta P=\num{0.25} p_0$ võrreldes atmosfäärirõhuga, kus $p_0$ on atmosfäärirõhk. Eeldada, et Rael joob vett piisavalt aeglaselt, et pudelis olev õhk on soojuslikus tasakaalus välisõhuga.
	

	
\hint

\solu
Olgu algne vee ruumala pudelis $V_0$. Selleks, et ära joodud vee ruumala maksimeerida, peaks ilmselt lõpus pudelit ainult õhk täitma. Vastasel korral on võimalik algset vee ruumala vähendada ning suurema õhu osakaalu arvelt rohkem vett ära juua.

Pudelist joomine on isotermiline protsess. Seega,
\[
p_0 (V-V_0)=(p_0-\Delta p)V.
\]
Siit avaldades $V_0$ saame, et
\[
V_0=\frac{\Delta P}{p_0} V = \num{0.25} V.
\]
\probend