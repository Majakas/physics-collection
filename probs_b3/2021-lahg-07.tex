\setAuthor{Jaan Kalda}
\setRound{lahtine}
\setYear{2021}
\setNumber{G 7}
\setDifficulty{7}
\setTopic{TODO}

\prob{Hantel}
Kaks pisikest kera, kumbki massiga $m$, on ühendatud üksteise külge kerge jäiga varda abil, mille pikkus on $l$; nimetagem seda süsteemi edaspidi hantliks. Üks kera kannab laengut $q$, kuid teine on ilma laenguta. Alghetkel on hantel horisontaalne ning liikumatu ja asub vertikaalses elektriväljas tugevusega $E$; raskusväli puudub. Milline on hantli telje pöörlemise maksimaalne nurkkiirus $\omega$ edasise liikumise käigus?


\hint

\solu
Hantli massikeskme kiirendus on $a=\frac{Eq}{2m}$. Kasutame süsteemi, mis liigub kiirendusega $\frac{Eq}{2m}$ ja kus hantli massikese on paigal. Selles süsteemis mõjub kummalegi kerale elektriväljale vastassuunaline inertsijõud $F=am=\frac{Eq}{2}$, st laenguga kerale mõjub elektrivälja suunaline resultantjõud $\frac{Eq}{2}$ ning laenguta kerale mõjub samasuur kuid vastassuunaline jõud. Nende jõudude potentsiaalse energia muutus (võrreldes algasendiga) on maksimaalne hetkel, kui hantel on pöördunud täisnurga võrra: $$\Delta E_p=2\cdot \frac{Eq}2\cdot\frac l2=\frac{Eql}2;$$ selles valemis arvestasime, et kummagi kera jõusihiline nihe on $\frac l 2$. Energia jäävusest tulenevalt on siis maksimaalne ka kineetiline energia ja nurkkiirus.  Seega saame energia jäävuse seadusest $$\frac{Eql}2=2\cdot \frac{mv^2}2,$$ kus $v$ on kummagi kuulikese kiirus meie taustsüsteemis antud ajahetkel. Järelikult $v=\sqrt{\frac{Eql}{2m}}$ ning otsitav nurkkiirus $$\omega=\frac{\sqrt{\frac{Eql}{2m}}}{\frac{l}{2}}=\sqrt{\frac{2Eq}{ml}}.$$
\probend