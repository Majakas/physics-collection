\setAuthor{Kaur Aare Saar}
\setRound{piirkonnavoor}
\setYear{2021}
\setNumber{G 7}
\setDifficulty{7}
\setTopic{TODO}

\prob{Rippuv laeng}
Punktlaeng $q_1$ ripub lakke kinnitatud nööri otsas, mille pikkus on $L$. Teine samamärgiline punktlaeng $q_2$ on nöörist sõltumatult fikseeritud täpselt nööri kinnituskoha alla kaugusele $2h$ laest. Leidke milline peab olema esimese punktlaengu mass, et selle kaugus laest oleks $h$.



\hint

\solu
Selleks, et rippuv laeng püsiks paigal, peab talle mõjuvate jõuvektorite summa olema $0$. Rippuvale laengule mõjub kolm jõudu, raskusjõud $mg$, nööri pinge $T$ ning laengute vaheline elektrostaatiline jõud $F$. \pp1

\begin{figure}[H]
	\centering
	\begin{tikzpicture}[scale=1.5]
    \coordinate (A) at (0,0);
    \coordinate (q1) at (1,-2);
    \coordinate (q2) at (0,-4);
    \coordinate (D) at (1,-0);
    \coordinate (E) at (0.5,-1);
    \coordinate (F) at (1.5,-1);
    \coordinate (G) at (1,-4);

    % kinnitus
    \fill [pattern = north east lines] (-0.5,0) rectangle (0.5,0.2);
    \draw[thick] (-0.5,0) -- (0.5,0);

    %nöör
    \draw [thick] (q1)	-- (A) node[yshift=-1.2cm,xshift=0.9cm] {$L$};

    %abijooned
    \draw [dashed] (q2) -- (A) node[midway,left] {$2h$};
    \draw [dashed] (q1) -- (q2) node[midway, right] {$L$};
    \draw [dashed] (q1) -- (D) node[right,yshift=-1.2cm] {$h$};

    %jõud
    \draw[line width=2pt,blue,-stealth](q1)--(E) node[anchor=north east]{$T$};
    \draw[line width=2pt,blue,-stealth](q1)--(F) node[anchor=north west]{$F$};
    \draw[line width=2pt,blue,-stealth](q1)--(G) node[anchor=north west]{$mg$};

    %nurgad
    \draw pic [text="$\theta$", draw, angle radius=1.1cm] {angle=q2--A--q1};
    \draw pic [text="$\theta$", draw, angle radius=1.05cm] {angle=D--q1--A};
    \draw pic [text="$\theta$", draw, angle radius=1.1cm] {angle=F--q1--D};    

    %laengud
    \node[circle,fill=black,inner sep=1mm] at (q1) {};
    \node[circle,fill=black,inner sep=1mm] at (q2) {};
  \end{tikzpicture}
\end{figure}

Paneme tähele, et kinnituskoht, rippuv laeng ning fikseeritud laeng moodustavad võrdhaarse kolmnurga, mistõttu on laengute vaheline kaugus~$L$~\pp1.

Seega laengute vaheline elektrostaatiline jõud on:
$$F=\frac{kq_1q_2}{L^2} \quad \pp1$$

Raskusjõul puudub horisontaalkomponent. Järelikult on elektrostaatilise jõu ja nööri pinge horisontaalkomponendid võrdsed ja vastassuunalised \pp1.

Seega kuna elektrostaatilise jõu ja nööri pinge horisontaalkomponendid on võrdsed, siis sümmeetria tõttu on need jõud absoluutväärtuselt võrdsed $T=F$ \pp1.

Nüüd vertikaalsuunalisest jõudude tasakaalust saame
$$mg=T\cos\theta + F\cos (-\theta) = 2F\cos\theta = 2 \cos\theta \frac{kq_1q_2}{L^2} \quad \pp3,$$
kus $\theta$ on nurk vertikaalsihi ja nööri vahel.
Täisnurksest kolmnurgast saame $\cos\theta = \frac hL$ \pp1.

Nüüd asendades sisse $\cos\theta$ vertikaalsuunalise jõudude tasakaalu võrrandisse ja avaldades sealt $m$, saame
$$m=\frac{2kq_1q_2h}{gL^3} \quad \pp1$$
\probend