\setAuthor{Kaarel Kivisalu}
\setRound{lõppvoor}
\setYear{2023}
\setNumber{G 4}
\setDifficulty{4}
\setTopic{TODO}

\prob{Laev}
Laev sõitis läbi Suessi kanali ning jäi sinna kinni nii, et blokeeris kogu kanali. Laeva pikkus on $l$, laius on $w$, kõrgus on $h$ ja mass on $m$. Veepiirist allpool on $k$ osa laeva ruumalast, kusjuures $klwh\rho < m$. Võib eeldada, et laev on ühtlase massijaotusega risttahukas ja $l \gg w$. Vee tihedus on $\rho$, raskuskiirendus $g$ ja kanali laius $d$. Hõõrdetegur laeva kere ja kanali vahel on $\mu$. Laeva mõlemat otsa tõmbavad puksiirlaevad kanali sihis eri suundades. Kui suure jõuga $F$ peavad puksiirlaevad tõmbama, et kinni jäänud laev hakkaks liikuma?


\hint

\solu
\par
Nurk laeva ja kanali ristsihi vahel on $ \alpha = \arccos (d/l)$. Kanali põhja poolt laevale avaldatav toereaktsioon on $N = (m-\rho lwh k) g$.

Sümmeetri tõttu hakkab laev pöörlema ümber oma keskpunkti. Seega puksiiride efektiivne jõud on risti laeva sihiga. Seega $F \cos \alpha = \mu N/2$. Nendest võrranditest avaldades saame, et
\begin{equation*}
F=\frac{\mu g l}{2d} (m-\rho l w h k).
\end{equation*}
\probend