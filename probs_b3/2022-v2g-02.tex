\setAuthor{Jaan Kalda}
\setRound{piirkonnavoor}
\setYear{2022}
\setNumber{G 2}
\setDifficulty{2}
\setTopic{TODO}

\prob{Juhe}
Peeter tahab vältida kasvuhoones taimede külmumist ja viib sinna elektriradiaatori nimivõimsusega $P=\qty{2}{\kW}$ ja nimipingega $V_0=\qty{230}{\V}$. Ta kasutab selleks pikendusjuhet pikkusega $L=\qty{40}{\m}$, mis  sisaldab kahte kõrvuti paiknevat vasktraati ristlõikepindalaga $S=\qty{1}{\mm\squared}$. Vase eritakistus $\rho=\qty{17}{\mohm\mm\squared\per\m}$. Milline soojuslik võimsus eraldub juhtmetes, kui tegelik võrgupinge pistikus on $V_p=\qty{240}{V}$?


\hint

\solu
$P=V_0^2/R$, millest $R=V_0^2/P=\qty{26.45}{\ohm}$. Leiame vasktraadist juhtme takistuse $r=2L\rho/S=\qty{1.36}{\ohm}$. Vool juhtmes $I=V_p/(R+r)=\qty{8.6}{\A}$ ning juhtmes eralduv võimsus $P_j=rI^2=rV_p^2/(R+r)^2\approx\qty{101}{\W}$.
\probend