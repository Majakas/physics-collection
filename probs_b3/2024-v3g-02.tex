\setAuthor{Taavi Pungas}
\setRound{lõppvoor}
\setYear{2024}
\setNumber{G 2}
\setDifficulty{2}
\setTopic{TODO}

\prob{Ragulka}
Eva leidis pargist ragulka ning soovis sellega lasta kivikese vertikaalselt üles nii, et see täpselt puudutaks tema kohal asuvat puuoksa. Kui Eva venitab ragulka kummi lõdvast olekust \SI{3}{\cm} kaugusele, siis jääb kivil puudu $1/4$ vahemaast oksani. Kui kaugele peaks Eva venitama ragulka kummi, et kivike jõuaks täpselt oksani?


\hint

\solu
Kivike on raske ja väike, seega õhutakistusega pole heas lähenduses vaja arvestada. See võimaldab meil kasutada energia jäävuse seadust. Lähendame ragulka kummi kui lineaarse elastsusega materjali, st kehtib Hooke'i seadus. Kivikese lennu haripunktis on elastsusjõu potentsiaalne energia ragulka kummi venitamisel täpselt teisenenud raskusjõu potentsiaalseks energiaks:
\[
  \frac{kx^2}{2} = mgh.
\]
Jagades läbi selle valemi teise ja esimese katse jaoks, saame
\[
  \frac{x_2^2}{x_1^2} = \frac{h_2}{h_1},
\]
seega
\[
  x_2 = x_1 \cdot \sqrt{\frac{h_2}{h_1}} = \SI{3}{\cm} \cdot \sqrt{\frac{1}{1- \frac{1}{4}}} \approx \SI{3,5}{\cm}.
\]
\probend