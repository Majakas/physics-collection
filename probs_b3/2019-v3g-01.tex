\setAuthor{}
\setRound{lõppvoor}
\setYear{2019}
\setNumber{G 1}
\setDifficulty{1}
\setTopic{TODO}

\prob{Autod}
Kaks autot sõidavad teineteise poole. Esimese auto kiirus $v_1=\SI{80}{km/h}$ ning teise auto kiirus $v_2 = \SI{100}{km/h}$. Autod märkavad teinetest, kui nende vahekaugus $s=\SI{600}{m}$ ning hakkavad samal hetkel pidurdama. Autod jäävad seisma samal hetkel, vahetult enne kokkupõrget. Kui kaua võttis autodel aega peatumine? 

\hint

\solu
Kuna autod jäävad seisma samaaegselt, siis läheme ühe ühe autoga seotud taustsüsteemi.
Autode suhteline kiirus üksteise suhtes on $v = v_1+v_2 = \SI{180}{km/h}=\SI{50}{m/s}$\\
Autode pidurdusteekond on kokku $2=\SI{600}{m}$ ning lõppkiirus on $\SI{0}{m/s}$, siis pidurdamiseks kulunud aeg on
\[ s = \frac{v + v_0}{2}\cdot t \quad\quad\Rightarrow\quad\quad t = \frac{2s}{v+v_0} = \SI{24}{s} \]
\probend