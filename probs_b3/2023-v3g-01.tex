\setAuthor{Jaan Kalda}
\setRound{lõppvoor}
\setYear{2023}
\setNumber{G 1}
\setDifficulty{1}
\setTopic{TODO}

\prob{Sujuv autosõit}
Bussijuht tahab sõita sujuvalt, st et reisijatel, kes bussis püsti seisavad ja kusagilt kinni ei hoia, ei tekiks äkilise kiirendamise või pidurdamise tõttu tasakaalu kaotamise ja kukkumise ohtu. Seepärast suurendab ta pidurdades survet piduripedaalile tasapisi kuni bussi peatumiseni. Kas selline sõit on sujuv? Kui on sujuv, siis põhjendada, miks see nii on. Kui ei ole sujuv, siis selgitada, mis hetkel on seisvatel reisijatel oht tasakaal kaotada, mis suunas on neil oht kukkuda ning kuidas tuleks pidurdada, et pidurdamine oleks sujuv, st et seisvatel reisijatel ei tekiks kordagi ohtu tasakaalu kaotada?


\hint

\solu
\par
Ei ole sujuv: seismajäämise hetkel muutub hõõrdejõud hetkeliselt nulliks, mis tähendab, et inimesed, kes olid pidurdamise ajal kergelt tahapoole kallutanud, et seista neile jalgade juures mõjuva hõõrdejõu ja toereaktsiooniga paralleelselt, kaotavad tasakaalu ja hakkavad tahapoole kukkuma.
\probend