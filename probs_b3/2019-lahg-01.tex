\setAuthor{Markus Rene Pae}
\setRound{lahtine}
\setYear{2019}
\setNumber{G 1}
\setDifficulty{1}
\setTopic{Kinemaatika}

\prob{Ristmik}
Juku, sõites autoga ($v = \SI{90}{\kilo\meter\per\hour}$), läheneb Y-kujulisele ristmikule. Kui tal on ristmikuni jäänud veel $l = \SI{150}{\meter}$ märkab Juku, et kõrvalharu pealt sõidab ristmiku poole ka teine auto, mille ristmikuni jäänud vahemaa ja kiiruse projektsioonid Juku sõidusuunale on võrdsed Juku auto omadega. Kokkupõrke vältimiseks kiirendab Juku kiirendusega $a = \SI{0.5}{\meter\per\second\squared}$. Mis on autode vahemaa, kui teine auto jõuab ristmikuni? Teede vaheline nurk on \ang{15}.


\hint
Kõigepealt tuleb leida kui palju kulub Jukul kiirendamata aega ristmikuni jõudmiseks, sest see ühtib ajaga, millega teine auto ristmikuni jõuab.\solu
Kui Juku ja teise auto projitseeritud kaugus ristmikust ja kiirused on samad, siis nad jõuavad ristmikule täpselt sama ajaga
\[
t = s/v = \frac{\SI{150}{\meter}}{\SI{90}{\kilo\meter\per\hour}} = \frac{\SI{150}{\meter}}{\SI{25}{\meter\per\second}} = \SI{6}{\second}.
\]
Kui Juku saavutaks hetkeliselt kiirenduse $a = \SI{0.5}{\meter\per\second\squared}$, siis selle ajaga jõuaks ta läbida
\[
s = v_0 t + a t^2/2 = \SI{159}{\meter}.
\]
Seega on teise auto ristmikule jõudes autode vahemaa $\SI{9}{\meter}$. Nurk ei oma antud ülesandes tähtsust.\probend