\setAuthor{Markus Rene Pae}
\setRound{lahtine}
\setYear{2019}
\setNumber{G 1}
\setDifficulty{1}
\setTopic{TODO}

\prob{Ristmik}
Juku, sõites autoga ($v = \SI{90}{\kilo\meter\per\hour}$), läheneb Y-kujulisele ristmikule. Kui tal on ristmikuni jäänud veel $l = \SI{150}{\meter}$ märkab Juku, et kõrvalharu pealt sõidab ristmiku poole ka teine auto, mille ristmikuni jäänud vahemaa ja kiiruse projektsioonid Juku sõidusuunale on võrdsed Juku auto omadega. Kokkupõrke vältimiseks kiirendab Juku kiirendusega $a = \SI{0.5}{\meter\per\second\squared}$. Mis on autode vahemaa, kui teine auto jõuab ristmikuni? Teede vaheline nurk on \ang{15}.


\hint

\solu
Kui Juku ja teise auto projitseeritu kaugus ristmikust ja kiirused on samad, siis nad jõuavad ristmikule täpselt sama ajaga $t = s/v = \SI{150}{\meter} / \SI{90}{\kilo\meter\per\hour} = \SI{150}{\meter} / \SI{25}{\meter\per\second} = \SI{6}{\second}$. Kui Juku saavuitaks hetkeliselt kiirenduse $a = \SI{0.5}{\meter\per\second\squared}$, siis selle ajaga jõuaks ta läbida $s = v_0 t + a t^2/2 = \SI{25}{\meter\per\second} \cdot \SI{6}{\second} + \SI{0.5}{\meter\per\second\squared} (\SI{6}{\second})^2 / 2 = \SI{159}{\meter}$. Seega on teise auto ristmikule jõudes autode vahemaa $\SI{9}{\meter}$. Nurk ei oma antud ülesandes tähtsust.
\probend