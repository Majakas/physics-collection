\setAuthor{}
\setRound{piirkonnavoor}
\setYear{2019}
\setNumber{G 7}
\setDifficulty{7}
\setTopic{TODO}

\prob{Lennurada}
Lennuk vajab õhkutõusmiseks vajaliku üleslükkejõu saavutamiseks merepinna kõrgusel vähemalt $L=\SI{2}{km}$ pikkust hoorada. Kui pikka hoorada vajaks sama lennuk maailma kõrgeimal tsiviillennuväljal, mis asub merepinnast $\SI{4}{km}$ kõrgusel? Õhu tihedus merepinnal $\rho_1 = \SI{1,23}{kg/m^3}$ ning $\SI{4}{km}$ kõrgusel $\rho_2 = \SI{0,82 }{kg/m^3}$. Lennuki tiibade poolt tekitatav üleslükkejõud on võrdeline õhu tiheduse ning lennuki kiiruse ruuduga. Eeldada, et ilm on mõlemal juhul tuulevaikne ning lennuki kiirendus on kogu hoovõtu jooksul konstantne. 

\hint

\solu
Teame, et jõud on võrdeline õhu tiheduse ning kiiruse ruuduga. Seega olgu 
$$F=C\rho v^2 \Rightarrow v^2=\frac{F}{C\rho}$$
kus $C$ on võrdetegur. \pp{2} Kuna lennuk alustab hoovõttu paigalseisust ühtlase kiirendusega $a$, avaldub läbitud teepikkus $s=\frac{v^2}{2a}$. \pp{1} Kuna lennuki mass (ning seega ka õhkutõusuks vajaminev üleslükkejõud) ja kiirendus on mõlemal juhul samad \pp{1}, avalduvad teepikkused vastavalt:
$$s_1=\frac{v_1^2}{2a}=\frac{F}{2Ca\rho_1}\quad\pp{1}$$ 
$$s_2=\frac{v_2^2}{2a}=\frac{F}{2Ca\rho_2}\quad\pp{1}$$   

Võrrandid omavahel jagades saame:
$$\frac{s_1}{s_2}=\frac{\rho_2}{\rho_1}\quad\pp{1}$$ 
Siit saame avaldada vajaliku lennuraja pikkuse $s_2 = s_1\frac{\rho_1}{\rho_2} \approx \SI{3}{km}\quad  \pp{1}$.
\probend