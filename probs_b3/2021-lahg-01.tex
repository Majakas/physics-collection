\setAuthor{Jaan Kalda}
\setRound{lahtine}
\setYear{2021}
\setNumber{G 1}
\setDifficulty{1}
\setTopic{TODO}

\prob{Võrkpall}
Rannavõrkpallis peab olema rõhk (st manomeetri näit manomeetriga mõõtmisel) vahemikus $p_-=\SI{17.5}{kPa}$ kuni $p_+=\SI{22.5}{kPa}$. Pall pumbatakse normaaltingimustel (temperatuuril $t_0=\SI{20}{\celsius}$ ja atmosfäärirõhul $p_a=\SI{101.325}{kPa}$) rõhuni $p_0=\SI{17.5}{kPa}$ ning viiakse rannaliivale, kus see kuumeneb temperatuurini $t_1=\SI{80}{\celsius}$. Millise rõhu omandab pall?


\hint

\solu
Ideaalse gaasi olekuvõrrandi saab uue pallirõhu $p_1$ abil kirjutada kujul $p_1+p_a=(p_0+p_a)T_1/T_0$, kus alg- ja lõpptemperatuurid on kelvinites. Siit saame $p_1=(p_0+p_a)T_1/T_0-p_0\approx \SI{42}{kPa}$, st pall on nüüd liiga suure rõhu all.
\probend