\setAuthor{Taavet Kalda}
\setRound{piirkonnavoor}
\setYear{2022}
\setNumber{G 7}
\setDifficulty{7}
\setTopic{TODO}

\prob{Plastiliin}
Laua peal asub vedrudel toetuv tasakaaluasendis olev plaat. Vedrude summaarne jäikustegur on $k$. Plaadi pinnale kõrguselt $h$ kukutatakse ideaalselt plastne plastiliinitükk massiga $m$. Mõne aja pärast kukutatakse samalt kõrguselt teine identse massiga plastiliinitükk nii, et plastiliinitükk langeb plaadi pinnale hetkel, mil plaat on liikumas üles ja esialgse tasakaaluasendiga võrreldes samal kõrgusel. Mis on plaadi edasise trajektoori kõige madalama asendi kõrguse ja esialgse tasakaaluasendi kõrguse vahe? Plaadi mass on plastiliini massiga võrreldes tühine, plastiliin kleepub pärast põrget plaadi külge ning õhutakistust ja vedrude hõõrdetakistust võib ignoreerida. Raskuskiirendus on $g$.



\hint

\solu
Peale esimest kokkupõrget nihkub plaadi tasakaaluasend $mg/k$ võrra allapoole ning plaat hakkab uue tasakaaluasendi ümber teatud amplituudiga võnkuma (amplituudi väärtust pole punktide saamiseks vaja leida) \p3. Kuna plaadi mass on plastiliini massiga võrreldes tühine, omandab plaat kokkupõrke hetkel impulsi jäävuse tõttu sama kiiruse nagu plastiliin, seega plaadi kiirus on energia jäävusest $v_0 = \sqrt{2gh}$ ($v_0$ avaldist pole punktide saamiseks vaja leida). Energia jäävusest järeldame, et plaadi kiirus on samuti $v_0$ siis, kui teine plastiliinitükk plaadiga kontakti loob \p2. See tuleneb asjaolust et plaadi kõrgus lauast, ehk teisisõnu plaadi potentsiaalne energia, on hetk enne teist kokkupõrget sama mis see oli just peale esimest kokkupõrget. Seega on teise plastiliini kokkupõrge plaadiga efektiivselt sama kui kahe identse massi ja kiirusega objektide lauskokkupõrge. Seega jääb plaat peale teist kokkupõrget seisma \p2. Samas nihkub plaadi tasakaaluasend veel $mg/k$ võrra allapoole ning plaat hakkab amplituudiga $2mg/k$ uue tasakaaluasendi ümber võnkuma \p2. Seega on järgneva liikumise käigus plaadi kõige alumine asend $4mg/k$ võrra esialgsest asendist all pool \p1.
\probend