\setAuthor{Jaan Kalda}
\setRound{lõppvoor}
\setYear{2022}
\setNumber{G 9}
\setDifficulty{9}
\setTopic{TODO}

\prob{Satelliit}
Millise temperatuuri omandab väike kuubikujuline absoluutselt must satelliit, mis liigub Maa ümber madalal ringorbiidil (kõrgus maapinnast on hulga väiksem, kui Maa raadius) ja on parajasti öö külje peal. \\
Vihje: Maad võib vaadelda kui lõpmatut tasandit, mis on temperatuuril $T_0=\SI{15}\celsius$ ja omab kiirgustegurit $\epsilon=\num{0.6}$, st tema soojuskiirgustihedus ($\SI{}{\watt / \metre\squared}$) on 60\% samasuguse absoluutselt musta keha soojuskiirgustihedusest $\sigma T^4$, kus $\sigma$ tähistab Stefan‐Boltzmanni konstanti.


\hint

\solu
\
Kui satelliit oleks kahe absoluutselt musta plaadi vahel, mille temperatuur on $T_0$, siis saavutaks ta peatselt soojusliku tasakaalu, st omandaks temperatuuri $T_0$. Sellisel juhul kiirgaks ta soojust koguvõimsusega $P_0=S \sigma T^4_0$, kus $S$ on ta kogupindala. Nüüd saab ta aga soojuskiirgust kaks korda vähem seetõttu, et on vaid üks plaat. Arvestades Maa ja musta plaadi soojuskiirguste tiheduste suhtega $\epsilon$, saame satelliidile langeva koguvõimsuse $P_1=\epsilon S \sigma T^4_0/2$. Soojustasakaalu korral kiirgab satelliit sama palju, kui ta saab soojust, st $\epsilon S \sigma T^4_0/2=S \sigma T^4$. Siit saame avaldada satelliidi temperatuuri:
$$
T=(\epsilon/2)^{1/4}T_0 \approx \SI{213}{\kelvin} \approx \SI{-60}{\celsius}. 
$$
\probend