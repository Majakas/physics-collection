\setAuthor{Oleg Košik}
\setRound{lõppvoor}
\setYear{2021}
\setNumber{G 1}
\setDifficulty{1}
\setTopic{TODO}

\prob{Lumepall}
Hannes istub $H=\SI{7.7}{\m}$ kõrgusel puu otsas ja tal on käes lumepall. Ta märkab otse tema suunas kiirusega $v=\SI{6.0}{\km\per\hour}$ lähenevat Richardit, kes on puust $l=\SI{7.0}{\m}$ kaugusel, ja otsustab teda lumepalliga visata. Millise kiirusega peaks ta lumepalli horisontaalselt viskama, et tabada Richardi pead, kui Richard on $h=\SI{1.8}{\m}$ pikk? Raskuskiirendus $g=\SI{9.8}{\m\per\s\squared}$.


\hint

\solu
Lumepall peab läbima kiirendusega $g$ vertikaalse teepikkuse $H-h$, seega saame võrduse
$$H-h=\frac{gt^2}{2},$$ kus $t$ on lumepalli lennuaeg. Olgu $w$ lumepalli algkiirus, siis tema horisontaalsuunaline kiirus läheneva Richardi suhtes on $v+w$. Peab kehtima võrdus $$(v+w)t=l.$$ Avaldades esimesest võrdusest $t$ ja teisest $w$ leiame
$$w=l\sqrt{\frac{g}{2(H-h)}}-v\approx \SI{4,7}{m/s}.$$
\probend