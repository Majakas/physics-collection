\setAuthor{Jaan Kalda}
\setRound{lõppvoor}
\setYear{2023}
\setNumber{G 2}
\setDifficulty{2}
\setTopic{TODO}

\prob{Elektrikarjus}
Elektrikarjusega karjatamisel ümbritseb karjamaad pikk traat, mis on postide abil maast elektriliselt isoleeritud. Elektrikarjuses olev generaator saadab sellesse traati impulsspinge: pingevabad perioodid vahelduvad lühikeste pingega perioodidega. Pingeimuplsi ajal võib elektrikarjuse pingegeneraatorit vaadelda kui elektromotoorjõudu $\mathcal E$, mis omab teatud sisetakistust $R$. Elektriimpulss on eluohtlik, kui inimest läbib vool, mis on suurem kui $I_0=\SI{30}{mA}$. Teatud marki elektrikarjuse kohta on teada järgmist: kui pingegeneraatori väljundklemmidest üks on maandatud ja teisest lähtuv karjaaia traat on maapinnast ideaalselt isoleeritud, siis traadi ja maapinna vaheline pinge on impulsi ajal $U_m=\SI{15}{kV}$. Inimene, kes kõnnib paljajalu ja on seetõttu heas elektrilises kontaktis maapinnaga, puudutab kuiva käega karjuse traati ning saab elektrilöögi. Eeldage, et inimese keha takistus on hulga väiksem, kui kuiva käenaha takistus $r=\SI 5\kohm$.\\
\osa Joonistage elektriskeem, mis kirjeldab olukorda, kui inimene saab parajasti karjuselt elektrilööki.\\
\osa Millised sisetakistuse $R$ väärtused on lubatavad?



\hint

\solu
\par
Kui elektrikarjuse traat on maapinnast isoleeritud, siis seal voolu pole, pingelangu pole ja järelikult on pinge maa suhtes võrdne elektromotoorjõuga, seega $\mathcal E=\SI {15}{\kV}$. Kui inimene puudutab traati, siis ta sisuliselt lühistab selle, st elektromotoorjõule langeb selle sisetakistus. Elektromotoorjõud $\mathcal E$ sisetakistusega $R$ on ühendatud takistile $r$:
\begin{center}
  \begin{circuitikz}
    \draw (0,0) to[battery1, l=$\mathcal{E}$, invert] (2,0) to[R=$R$] (4,0) -- (4,2) to[R=$r$] (0,2) -- (0,0);
  \end{circuitikz}
\end{center}
Järelikult $R_{\min{}}+r=\mathcal E /I_{\max{}}=\SI{500}{\kohm}$. Näeme, et $r\ll R_{\min{}}$, seega  $R_{\min{}}\approx \mathcal E /I_{\max{}}=\SI{500}{\kohm}$.
\probend