\setAuthor{}
\setRound{piirkonnavoor}
\setYear{2020}
\setNumber{G 6}
\setDifficulty{6}
\setTopic{TODO}

\prob{Kell}
Seinakellal on minuti ja tunniosuti, mis kaaluvad vastavalt
$m_1=\SI{5}{\gram}$ ja $m_2=\SI{25}{\gram}$, mille pikkused on vastavalt
$L_1=\SI{15}{\centi\meter}$ ja $L_2=\SI{10}{\centi\meter}$ ning mille otspunktid
on kinnitatud kella keskpunkti. Kella vooluallikas on tühjenemas ning täpselt kell kuus annab see maksimaalselt voolu $I=\SI{1}{\milli\ampere}$ pingel
$V=\SI{5}{\volt}$. Elektrienergiast jõuab osutiteni mehaaniline energia
kasuteguriga $\eta=0.3$. Leia minuti täpsusega aeg, mil kell jääb seisma. 

\hint

\solu
Leiame kummagi osuti jaoks maksimaalse võimsuse, mis nende liigutamiseks saab
kuluda. See juhtub siis, kui vastav seier on $9$ peal, sest tööd tuleb teha
gravitatsioonijõu vastu ja konstanse kiiruse korral on vajaminev jõud suurim
siis kui jõud mille vastu tööd tehakse on antiparalleelne masskeskme
kiirusvektoriga.
$$P_{max} = F v = mg \frac{2 \pi \frac{L}{2}}{T} = \frac{\pi mgL}{T},$$
kus $T$ on vastava seieri täispöörde tegemiseks kuluv aeg.
Minutiosuti jaoks on vastav periood $T_1={60\cdot60}$ s ning võimsus $P_1 =\SI{6.41}{\micro\watt}$.
Tunniosuti jaoks on vastav periood $T_2={12\cdot 60\cdot60}$ s ning võimsus $P_2=\SI{1.78}{\micro\watt}$. Seega summaarselt saaks
seierite liigutamine võtta võimsust $P_1+P_2 = \SI {8.20}{\micro\watt}$.
Vooluallikas suudab seiereid liigutada kasuliku võimsusega
$P=\eta VI=\SI{1.5}{\milli\watt}$. See on aga rohkem kui osutite liigutamiseks
vaja läheb, seega kell ei jää seisma.

\textbf{Vastus:}
Kell ei jää seisma, kuniks kella vooluallikas annab voolu vähemalt
$\SI {8.20}{\micro\watt}$.
\probend