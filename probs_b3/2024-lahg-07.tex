\setAuthor{Jarl Patrick Paide}
\setRound{lahtine}
\setYear{2024}
\setNumber{G 7}
\setDifficulty{7}
\setTopic{TODO}

\prob{Äikesetorm}
Helena mõõtis äikesetormi ajal Maa magnetvälja väga täpse elektroonilise magnetomeetriga. Ta hakkas muretsema äikesetormi mõjudest mõõtmistele ja otsustas seda uurida. Äikese sähvatusest kuni heli saabumiseni läks aega $t = \SI{10}{\s}$. Ta uuris välja, et tavalise äikselöögi voolutugevus on $I = \SI{30000}{\ampere}$, õhu magneetiline läbitavus on $\mu = \SI{1.26e-6}{\tesla\metre\per\ampere}$ ja heli kiirus õhus on $v_{õ} = \SI{340}{\m\per\s}$. Leia mitu kraadi maksimaalselt muutub mõõteriista poolt mõõdetud põhja suund löögi hetkel. Tavakeskkonnas magneomeeter mõõdab Maa magnetvälja tugevuseks $B = \SI{50}{\micro \tesla}$ ning magnetväli moodustab vertikaalse sihiga nurga $\alpha = 17^{\circ}$. \\
\textit{Vihje:} Sirge juhe voolutugevusega $I$ tekitab endast kaugusele $r$ magnetvälja tugevusega $B = \frac{\mu I}{2\pi r}$. Tekkinud magnetvälja suund on risti juhtmega ja tangensiaalne vastavalt kruvireeglile.





\hint

\solu
Võime eeldada, et äike käitub löögi hetkel juhtmena, mis on sirge ja maaga risti, voolutugevusega $I = \SI{30000}{A}$. Leiame äikese kauguse $r = t\cdot v_{õ} = \SI{3400}{m}$. Seega äikese poolt tekitatud magnetvälja muutus on $\Delta B = \frac{\mu I}{2\pi r} = \SI{1.765}{\micro T}$. Kuna äike lööb maapinnaga risti, siis on äikese poolt tekitatud magnetväli horisontaalne maapinnaga. Maa magnetvälja horisontaalne komponent (mis näitab põhja suunda) on $B_{hor} = \sin{\alpha}B = \SI{14.62}{\micro T}$. Põhja suund on maksimaalselt mõjutatud siis, kui Maa magnetvälja horisontaalne komponent on risti äikese poolt tekitatud magnetvälja muuduga. Seega muutus on $\beta = \arcsin{\frac{\Delta B}{B_{hor}}} = 6.9^{\circ}$
\probend