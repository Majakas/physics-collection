\setAuthor{Jaan Kalda}
\setRound{lahtine}
\setYear{2021}
\setNumber{G 5}
\setDifficulty{5}
\setTopic{TODO}

\prob{Soolvesi}
Silindriline anum on täidetud kõrguseni $H=\SI{20}{cm}$ soolveega, mille tihedus on $\rho_s=\SI{1.25}{g/cm^3}$. Silindrisse visatakse magedast veest tehtud jääkuubikuid sellisel hulgal, et vedeliku kõrgus anumas tõuseb $h=\SI{10}{cm}$ võrra. Kui palju (ja millises suunas) muutub vedeliku tase silindris, kui kogu jää on ära sulanud ja mage vesi soolveega ära segunenud? Mageda vee tihedus  $\rho_v=\SI{1.00}{g/cm^3}$ ning jää tihedus  $\rho_j=\SI{0.90}{g/cm^3}$. Lugeda, et soolvee tiheduse erinevus mageda vee omast on võrdeline soola protsentuaalse sisaldusega soolvees. \\
\emph{Märkus:} kõiki arvandmeid ei pruugi vaja minna.



\hint

\solu
Et soolvee kihi paksus kasvab 50\% võrra, siis kasvab ka rõhk anuma põhjas 50\% võrra ning seega teeb seda ka kogu anumas oleva aine kaal. Seega kasvab anumas oleva aine mass 50\% võrra. See tähendab, et peale jää sulamist ja segunemist moodustab soola kontsentratsioon 2/3 esialgsest väärtusest. Seega on soolvee uueks tiheduseks $\rho'=(1+0.25\cdot \frac 23)\SI{}{g/cm^3}\approx \SI{1.17}{g/cm^3}$. Et silindris oleva aine kaal ei muutu lahustumise käigus, siis ei muutu ka rõhk anuma põhja juures ning seetõttu peab uus vedeliku taseme kõrgus olema $H'=(H+h)\rho/\rho'\approx \SI{32.1}{cm}$. Niisiis kerkib vedeliku tase  $\Delta h=\SI{2.1}{cm}$ võrra.
\probend