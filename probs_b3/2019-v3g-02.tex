\setAuthor{}
\setRound{lõppvoor}
\setYear{2019}
\setNumber{G 2}
\setDifficulty{2}
\setTopic{TODO}

\prob{Saunauks}
Juku istub saunas ja viskab saunakerisele vett. Tekkinud leilist läheb saunauks lahti. Arvutage, kui suur peaks olema hõõrdejõu moment $\tau$ saunaukse hingedes, et $V_v=\SI{200}{ml}$ vett korraga kerisele visates saunauks lahti ei läheks? Sauna mõõtmed on $300 \times 250 \times 240$ cm, sauna ukse mõõtmed $70 \times 190$ cm. Eeldada, et aurustumine toimub nii kiiresti, et õhk ei jõua läbi pilude saunast välja minna. Universaalne gaasikonstant $R=\SI{8,314}{J/(mol \cdot K)}$, vee molaarmass $\mu=\SI{18}{g/mol}$ ja vee tihedus $\rho = \SI{1}{g/cm^3}$. 

\hint

\solu
Saunakerisele visatud vee aurustamine lisas saunaruumi rõhu: 

\[ p = {\rho V_v  \over \mu} \cdot {RT\over V} = {200 \over 18} \cdot {8,\!314 \cdot 373 \over 3\cdot2,\!5\cdot2,\!4} \approx \SI{1914}{Pa}. \] 

kus me kasutasime teadmist, et auru temperatuur $T=\SI{373}{K}$. Saunaukse pindala $S_u = 0,\!7 \cdot 1,\!9 = \SI{1,33}{m^2}$ ning talle mõjub jõud $F = p S_u \approx \SI{2546}{N}$.
Selleks, et leili rõhk ust lahti ei teeks, peab ukse hingedele mõjuv hõõrdejõu moment olema vähemalt:

\[ \tau = { F l } = { 2546 \cdot {0,\!7 \over 2 } } \approx \SI{891}{N/m}.\]
\probend