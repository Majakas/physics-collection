\setAuthor{Kaarel Hänni}
\setRound{lahtine}
\setYear{2019}
\setNumber{G 4}
\setDifficulty{4}
\setTopic{Elektrostaatika}

\prob{Laetud tasand}
Ruumis on ühtlaselt laetud tasand ja paaritu arv võrdseid laenguid, mis ei asu tasandil. Tõesta, et tasandile mõjuv resultantjõud ei saa olla 0.


\hint
Ühe laengu poolt tasandile avaldatav jõud on Newtoni kolmanda seaduse kohaselt võrdne ja vastasmärgiline tasandi poolt laengule avaldatava jõuga.\solu
Ühe laengu poolt tasandile avaldatav jõud on Newtoni kolmanda seaduse kohaselt võrdne ja vastasmärgiline tasandi poolt laengule avaldatava jõuga. Seega on tasandile kokku mõjuv jõud 0 siis ja ainult siis, kui tasandi poolt laengutele avaldatud jõudude summa on 0. Tasandi elektriväli on konstantne ja risti tasandiga (ja tasandi eri pooltel erisuunaline). Seega on tasandi poolt laengutele avaldatud jõudude summa 0 siis ja ainult siis, kui kummalgi pool tasandit on võrdne arv laenguid. Laenguid on kokku paaritu arv, seega see on võimatu.\probend