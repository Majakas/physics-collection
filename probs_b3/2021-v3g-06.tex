\setAuthor{Krister Kasemaa}
\setRound{lõppvoor}
\setYear{2021}
\setNumber{G 6}
\setDifficulty{6}
\setTopic{TODO}

\prob{Keha keral}
Kerakujulisele, ühtlase massijaotusega planeedile, raadiusega $R$, massiga $M$ ja pöörlemisperioodiga $T$ asetatakse väikese massiga keha $m$. Gravitatsioonikonstant on G.\\
\osa Leia väikese keha kaal funktsioonina laiuskraadist $\theta$.\\
\osa Leia hõõrdeteguri $\mu$ väärtuste vahemik funktsioonina laiuskraadist $\theta$, mille puhul püsib väike keha planeedi peal staatilisena. \par
\emph{Märkus:} laiuskraadi $\theta$ mõõdetakse ekvaatorilt.


\hint

\solu
\osa Vaatleme keha liikumist ümber $\theta$ defineeritud ringjoone. Laiuskraadil $\theta$ avaldub väikese keha tiirlemisraadius kujul
\[
  r=R \: \cos \theta
\]
ja joonkiirus kujul
\[
  v= \omega r= \frac{2 \pi }{T} r = \frac{2 \pi R \: \cos \theta}{T}.
\]
Nüüd saab avaldada tsentrifugaaljõu:
\[
  F_{\mathrm{ kesk}}=\frac{m v^2}{r} =\frac{m \left(\frac{2 \pi R \: \cos \theta}{T}\right)^2}{R \: \cos \theta}=\frac{4 m \pi^2 R \: \cos \theta}{T^2}.
\]
Arvestades, et kaalule panustab ainult radiaalne tsentrifugaaljõu komponent
\[
  F_{\mathrm{\bot kesk}}= \cos \theta \; F_{\mathrm{ kesk}} = \frac{4 \pi^2 m R \: \cos ^2 \theta}{T^2},
\]
saab leida keha kaalu:
\[
  W=\frac{G m M}{R^2} - \frac{4 \pi^2 m R \: \cos ^2 \theta}{T^2} = m\bigg(\frac{G M T^2 - 4 \pi^2 R^3 \: \cos ^2 \theta }{R^2 T^2}\bigg).
\]
\osa Libisemist põhjustab tsentrifugaaljõu kera pinnaga tangentsiaalne komponent:
\[
  F_{\mathrm{kesk || }}= \sin(|\theta|) \; F_{\mathrm{ kesk}} = \frac{4 \pi^2 m R \: \cos \theta \sin(|\theta|)}{T^2}=\frac{2 \pi^2 m R \: \sin(|2\theta|)}{T^2} .
\]
Hõõrdeõud avaldub kujul
\[
  F_{\mathrm{h}}=\mu W = \mu \, m\left(\frac{G M T^2 - 4 \pi^2 R^3 \: \cos ^2 \theta }{R^2 T^2}\right).
\]
Seega, tasakaalu korral:
\[
  F_{\mathrm{kesk || }} \leq \mu W,
\]
\[
  \frac{2 \pi^2 m R \: \sin(|2\theta|)}{T^2} \leq
  \mu \, m\bigg(\frac{G M T^2 - 4 \pi^2 R^3 \: \cos ^2 \theta }{R^2 T^2}\bigg)
\]
ja järelikult
\[
  \mu \geq \frac{2 \pi^2 R^3 \: \sin(|2\theta|)}{G M T^2 - 4 \pi^2 R^3 \: \cos ^2 \theta }.
\]
\probend