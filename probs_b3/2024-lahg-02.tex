\setAuthor{Jarl Patrick Paide}
\setRound{lahtine}
\setYear{2024}
\setNumber{G 2}
\setDifficulty{2}
\setTopic{TODO}

\prob{Kaks kera}
Kaks sama tiheduse ja ühtlase massijaotusega kera peaaegu puutuvad üksteist (keskpunktide vaheline kaugus on kaks raadiust), aga hõõrdumist ei toimu. Kerad tiirlevad ümber üksteise gravitatsioonijõu tõttu nurkkiirusega $\omega$. Leia kerade tihedus.




\hint

\solu
Olgu kera tihedus $\rho$, raadius $R$ ja seega mass $M = \frac{4}{3}\pi R^3 \rho$. Kahe keha vaheline gravitatsioonijõud on $F = \frac{G M^2}{4R^2}$. Et kerad saaksid olla ringorbiidil nurkkiirusel $\omega$, on vaja neile rakendada tsentripetaaljõudu $F = M \omega^2 R$. Pannes vastavad jõud võrduma, saame $\frac{G M}{4 R^3} = \omega^2$ ja asendades sinna $M$ sisse, saame, et tihedus $\rho = \frac{3\omega^2}{\pi G}$.
\probend