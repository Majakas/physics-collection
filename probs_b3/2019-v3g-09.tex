\setAuthor{}
\setRound{lõppvoor}
\setYear{2019}
\setNumber{G 9}
\setDifficulty{9}
\setTopic{TODO}

\prob{Niit ja poolsfäär}
Millise kogulaengu $Q$ indutseerib peenikese juhtme abil maandatud kerale, mille raadius on $R$, peenest traadist rõngas raadiusega $r$, mis kannab laengut $q$, kui kera ja rõnga keskpunktide vahekaugus on $d$ ning kera keskpunkt asub rõnga sümmeetriateljel? Maandamisjuhtme mahtuvus on tühiselt väike. 

\hint

\solu
Et kera on maandatud, siis selle keskpunkti potentsiaal on 0. Superpositsiooniprintsiibi abil saame selle avaldada kui kera pinnale indutseeritud laengute $Q_i$ ja rõngal olevate laengute $q_j$ panuste summa: $0=\sum_i kQ_i/R+k\sum_jkq_j/l$, kus $l=\sqrt{r^2+d^2}$ on rõnga punktide kaugus kera keskpunktist. Summas saab konstandid sulgude ette tuua: $0=\frac kR \sum_iQ_i+\frac kl\sum_jq_j=kQ/R+kq/l$, seega $Q=-qR/\sqrt{r^2+d^2}$.
\probend