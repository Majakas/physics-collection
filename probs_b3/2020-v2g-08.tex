\setAuthor{}
\setRound{piirkonnavoor}
\setYear{2020}
\setNumber{G 8}
\setDifficulty{8}
\setTopic{TODO}

\prob{Dioodid}
Dioodide tootmisel fluktueeruvad nende parameetrid märkimisväärselt. Olgu meil kaks valgusdioodi, mille pinged erinevad samasuguse tugevusega voolu korral 2\% võrra; nende dioodide voolu-pinge tunnusjooned on toodud juuresoleval joonisel punase ja sinise kõverana. Nende dioodide toitmiseks kasutatakse konstantse voolu allikat, mille väljundvool püsib konstantselt võrdne $I_0=\SI{2.7}A$ seni kuni väljundpinge pole suurem kui $\SI 4V$. Dioodid ühendatakse paralleelühenduses vooluallika klemmide külge; mitme protsendi võrra erinevad nende tarbitavad võimsused? 

\hint

\solu
Lahendus: Kuivõrd dioodid on ühendatud paralleelselt, siis nende pinged on võrdsed  \pp{2}. Ilmselt kõige lihtsam moodus on katse-eksituse meetodil sellise pinge $V_0$ leidmine, mille korral dioodide voolude summa on võrdne $I_0$-ga, $I_1(V_0)+I_2(V_0)=I_0$  \pp{2}. Otsime lahendit piirkonnas, kus keskmine vool on pool vooluallika voolust, st 1.35 amprit  \pp{1}; seal piirkonnas on pinge umbes $V_0\approx \SI{3.6}V$ \pp{1} ja fikseeritud pinge $V_0$ juures on kahe voolutugevuse vahe $\Delta I = \SI {0.20}A$  \pp{1}. Seega $I_1=\SI{1.25}A$ [\pp{0.5} väärtuse eest, mis on vahemikus $\SI{1.22}A$ kuni $\SI{1.28}A$  ja rahuldab tingimust $I_1(V_0)+I_2(V_0)=\SI{2.7}A$] ning $I_2=\SI{1.45}A$ [\pp{0.5} väärtuse eest, mis on vahemikus $\SI{1.42}A$ kuni $\SI{1.48}A$ ja rahuldab tingimust $I_1(V_0)+I_2(V_0)=\SI{2.7}A$]. (Kui voolude väärtused on etteantud vahemikus, kuid summa erineb 2.7 amprist, siis arvväärtuste eest punkte ei saa.) Kuivõrd dioodide pinged on samad, siis võimsuste suhe on voolude suhe, st võimsuste suhteline erinevus on $2\Delta I/I_0\approx 15$\%  (\pp{1} avaldise ja \pp{1} väärtuse eest; kui lõppvastus on vahemikus 13\% kuni 18\%, siis antakse väärtuse eest täispunktid; kui lõppvastus on väiksem 10\%-st või suurem 22\%-st, siis arvväärtuse punkte ei anta; ülejäänud juhtumeil antakse 0.5 punkti). Märkus: aktsepteeritavad on kõik lähenemised, mis jõuavad õige tulemuseni $I_1$ ja $I_2$ jaoks tuginedes lahenduse alguses toodud kahele tingimusele pingete ja voolude jaoks.

\vspace{10pt}
\probend