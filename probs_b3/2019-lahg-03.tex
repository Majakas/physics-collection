\setAuthor{Hans Daniel Kaimre}
\setRound{lahtine}
\setYear{2019}
\setNumber{G 3}
\setDifficulty{2}
\setTopic{Optika}

\prob{Lääts}
Optiline süsteem koosneb punktvalgusallikast ja ekraanist, mille vahele on asetatud õhuke koondav lääts. Ekraani ja valgusallika vaheline kaugus on $L$ ning läätse asukohta saab vabalt liigutada. Millist tingimust peab rahuldama läätse fookuskaugus selleks, et ekraanile oleks võimalik tekitada tõelist kujutist? Missugune on optilise süsteemi suurendus juhul kui läätse fookuskaugus on võimalikult suur?\hint
Tõelise kujutise tekkimise tingimuse saab kirja panna läätse valemi kaudu funktsioonina läätse kaugusest ekraanist ja fookuskaugusest.\solu
Otsime kaugust $s$, mille korral tekib ekraanile reaalne kujutis, paneme kirja süsteemi jaoks läätse valemi:
$$
\frac{1}{s}+\frac{1}{L-s}=\frac{1}{f}\Rightarrow \frac{L}{s(L-s)}=\frac{1}{f} \Rightarrow s^2-LS+Lf=0.
$$
Tegu on tavalise ruutvõrrandiga, kust saame, et
$$
s_{1,2}=\frac{1}{2}\left(L\pm\sqrt{L(L-4f)}\right).
$$
Kuna $s$ on reaalne suurus, mitte imaginaarne, siis $L-4f \geq 0$, kust omakorda saame $f$ tingimuseks, et $f \leq L/4$. Seega kõige suurem võimalik fookuskaugus on $f = L/4$, mille korral saame $s_{1,2} = L/2$ ja suurenduse $M=s_2/s_1 = 1$ ehk kujutis on sama suur kui kujutist tekitav objekt.\probend