\setAuthor{Hans Daniel Kaimre}
\setRound{lahtine}
\setYear{2019}
\setNumber{G 3}
\setDifficulty{3}
\setTopic{TODO}

\prob{Lääts}
Optiline süsteem koosneb punktvalgusallikast ja ekraanist, mille vahele on asetatud õhuke koondav lääts. Ekraani ja valgusallika vaheline kaugus on $L$. Millist tingimust peab rahuldama läätse fookuskaugus, et ekraanile tekiks tõeline kujutis? Kui fookuskaugus on võimalikult suur, siis milline on optilise süsteemi suurendus?


\hint

\solu
Otsime kaugust $s$, mille korral tekib ekraanile reaalne kujutis, paneme kirja süsteemi jaoks läätse valemi:
$$\frac{1}{s}+\frac{1}{L-s}=\frac{1}{f}\Rightarrow \frac{L}{s(L-s)}=\frac{1}{f} \Rightarrow s^2-LS+Lf=0$$
Tegu on tavalise ruutvõrrandiga, kust saame, et $$s_{1,2}=\frac{1}{2}\big(L\pm\sqrt{L(L-4f)}\big)$$
Kuna $s$ on reaalne suurus, mitte imaginaarne, siis peab $L-4f \geq 0$, kust omakorda saame $f$ tingimuseks, et $f \leq L/4$. Seega kõige suurem võimalik fookuskaugus on $f = L/4$, mille korral saame $s_{1,2} = L/2$ ja suurenduse $M=s_2/s_1 = 1$ ehk kujutis on sama suur kui kujutist tekitav objekt.
\probend