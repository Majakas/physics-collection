\setAuthor{Jaan Kalda}
\setRound{lahtine}
\setYear{2019}
\setNumber{G 8}
\setDifficulty{8}
\setTopic{TODO}

\prob{Külm gaas}
Tihedalt suletud anumas on monomolekulaarne gaas molaarmassiga $\mu$ ja tihedusega $\rho$ temperatuuril $T_0$. Anum on silindriline ja tema telg vertikaalne. Anum liigub ülevalt alla algkiirusega $v$ ja peatub hetkeliselt; eeldada, et $v\gg \sqrt{RT_0/\mu}$, $R$ tähistab universaalset gaasikonstanti. Millist rõhku avaldab gaas anuma põhjale (a) vahetult peale anuma seinte peatumist; (b) peale seda, kui gaas jõuab termodünaamilisse tasakaalu. Gaasi soojusvahetust anuma seintega ignoreerida, eeldada, et põrkumisel seintega on molekulide eemaldumisnurk pinnanormaali suhtes võrdne langemisnurgaga.


\hint

\solu
Alghetkel lähenevad põhjale molekulid kiirusega $v$, seega lühikese ajavahemiku $t$ jooksul jõuavad põhjaga põrkuda molekulid ruumalast $vtS$, kus $S$ on põhja pindala, kogumassiga $\rho vtS$. Peale põrkumist lahkuvad nad kiirusega $v$ seetõttu said nad põhjalt koguimpulsi $\Delta p=2\rho vtS$. See vastab jõule $F=\Delta p/S$ ja rõhule $p=F/S=2\rho v$.

Pikema aja möödudes saavutavad molekulid soojusliku tasakaalu, st nende kiirusjaotus muutub isotroopseks ja algne kineetiline koguenergia muutub siseenergiaks: $U=V\rho v^2/2$, kus $V$ tähistab anuma ruumala. Teisest küljest, $U=\frac 32 \frac{\rho V}\mu RT$, millest $v^2=3 \frac{ RT}\mu$ ning $p=\frac{\rho}\mu RT=\rho v^2/3$.
\probend