\setAuthor{Jarl Patrick Paide}
\setRound{lõppvoor}
\setYear{2021}
\setNumber{G 2}
\setDifficulty{2}
\setTopic{TODO}

\prob{Pudel}
Pärast kuumal päeval õues treenimist tuli sportlane oma tühja õrnast plastmassist joogipudeliga jahedasse tuppa ning ta märkas, et ta pudel hakkas vaikselt \enquote{paukuma}. Lähemal uurimisel selgus, et kui pudeli sisene rõhk erineb välisest rõhust suuruse $\Delta p$ võrra, siis tekib pudeli kesta sisse üks lohukujuline mõlk juurde (selle tekkimine kostuski pauguna) ning hetkeks võrdsustuvad sise- ja välisrõhud. Sportlane mõõtis kahe paugu vaheliseks ajaks $t$. Leidke pudeli soojuskadude võimus vahetult pärast pudeli tuppa toomist. Tühja pudeli (seal hulgas pudelis oleva õhu) soojusmahutavus on $c$, õhu molaarruumala on $V_m$ ja universaalne gaasikonstant on $R$. Võib eeldada, et ajavahemiku $t$ jooksul soojuskadude võimus ei muutu.


\hint

\solu
Toa jahedam õhk jahutab pudelit võimsusega $P$ saades pudelilt aja $t$ jooksul energia $\Delta Q = Pt$. Pudeli temperatuur muutub selle aja jooksul $\Delta T = \Delta Q/c$ võrra. Ideaalse gaasi olekuvõrrandist $PV=nRT$ saame temperatuuri muutusest rõhu muutuse $\Delta p = \frac{R \Delta T}{V_m} $. Pannes seosed kokku saame, et $P = \frac{\Delta p c V_m }{Rt}$
\probend