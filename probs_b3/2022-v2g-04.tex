\setAuthor{Jarl Patrick Paide}
\setRound{piirkonnavoor}
\setYear{2022}
\setNumber{G 4}
\setDifficulty{4}
\setTopic{TODO}

\prob{Kaks tuba}
Majas asub kaks ruudukujulist tuba, millel on üks ühine sein, ülejäänud seinad on välisseinad. Kõik seinad on identsed ja iga seina soojusjuhtivustegur on $k$. Ühes toas on lisaks tavalisele küttele lisaks kütteallikas võimsusega $P$. Kui suur on tubade temperatuurierinevus? Võib eeldada, et soojusvahetus ei toimu läbi põranda ja lae.
\\\emph{Vihje}: Soojusvahetuse võimsus läbi seina avaldub kujul $N=k(T_1-T_2)$, kus $k$ on soojusjuhtivustegur ning $T_1$ ja $T_2$ seina kahe pinna temperatuurid.


\hint

\solu
Olgu mõlemas toas algne kütteallikas võimsusega $N$ \p{0,5}. Olgu toas, kus on lisaks kütteallikas võimsusega $P$ temperatuur $T_0$, teises toas temperatuur $T_1$ ja väljas temperatuur $T_2$. Süsteem on tasakaalus kui $T_0 > T_1 > T_2$ \p{0,5}. Paneme kirja võrrandi mõlema toa jaoks kus paremal pool on toast lahkuv soojus ja vasakul pool tuppa sisenev soojus.
\begin{align*}
  N+P&=3k(T_0 - T_2)+k(T_0-T_1), \quad \p2\\
  N+k(T_0-T_1)&=3k(T_1 - T_2). \quad \p2
\end{align*}
Siit saame avaldada temperatuurivahe $T_0-T_1=\frac{P}{5k}$ \p3.
\probend