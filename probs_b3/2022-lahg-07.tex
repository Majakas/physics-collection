\setAuthor{Jaan Kalda}
\setRound{lahtine}
\setYear{2022}
\setNumber{G 7}
\setDifficulty{7}
\setTopic{TODO}

\prob{Elektron ja kondensaator}
Plaatkondensaatori plaadid on ruudukujulised küljepikkusega $b$ ja plaatide vahekaugus $d$ on hulga väiksem kui $b$; plaatide vahele on rakendatud pinge $U$. Millist minimaalset algkiirust $v$ peab omama elektron massiga $m$ ja laenguga $e$, et see suudaks läbida plaatide vahelise ruumi ilma vastu plaate põrkamata? Elektron peab väljuma plaatidevahelisest ruumist sisenemispunkti suhtes vastasküljelt.




\hint

\solu
Lahendus. Optimaalne trajektoor on selline, mis riivab ühte plaati kondensaatori keskpunkti juures ja väljub vastasplaati riivates. Et $d\ll b$, siis on sisenemis- ja väljumisnurgad väiksed ning me võime lugeda, et elektroni plaadisihiline kiiruskomponent (mis püsib konstantsena, sest elekrivälja jõud on sellega risti) on võrdne $v$-ga. Elektroni liikumine on samasugune, nagu pallil Maa raskusväljas. Elektron viibib plaatide vahel aja $t=b/v$; et elektroni kiirendus $a=Ee/m$, siis saame välja kirjutada tingimuse, et elektron jõub ühe plaadi keskpunkti juurest startides liikuda plaatide ristsihis vahemaa $$d=\frac a2\left(\frac t2\right)^2=\frac{Eeb^2}{8v^2m}.$$ Arvestades, et $E=U/d$ saame siis avaldada $$v=\frac b{2d}\sqrt{\frac{Ue}{2m}}.$$
\probend