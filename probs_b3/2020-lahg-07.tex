\setAuthor{Krister Kasemaa}
\setRound{lahtine}
\setYear{2020}
\setNumber{G 7}
\setDifficulty{7}
\setTopic{TODO}

\prob{Poolsilinder basseinis}
\
	Basseinis laiusega $l$ takistab vedeliku ühelt poolt teisele poole voolamist poolsilinder massiga $m$ ja raadiusega $r$ (ja seega pikkusega $l$), kusjuures telg, piki mida silinder on poolitatud, on vastu basseini põhja. Hõõrdetegur poolisilindri ja basseini põhja vahel on $\mu$, vedeliku tihedus basseinis $\rho$, ja raskuskiirendus on $g$. Basseini üks pool on täidetud veega poolsilindri ülemise ääreni, kõrguseni $r$. Küsimusele vastates eeldage, et hõõrdumine poolsilindri külgmiste poolringide ja seina vahel on tühiselt väike. Mis peaks olema hõõrdeteguri $\mu$ vähim väärtus $\mu_0$, et poolislindri vabastades ei hakkaks see liikuma?
	
	
	
\hint

\solu
Vaatleme poolsilindrit ning selle kohal olevat veesammast ühtse kehana. Me saame seda teha, sest süsteem on tasakaalus ning vesi on paigal.
Paneme esmalt kirja jõudude tasakaalu vertikaalsihis.
Vertikaalsihis mõjub raskusjõud ning poolsilindri ja põranda vaheline toereaktsioon $N$.

Poolsilindri kohal oleva veesamba ruumala on
\begin{equation*}
	V_{\mathrm{vesi}}=l r^2-\pi \frac{r^2 l}{4}=l r^2\left(1-\frac{\pi}{4}\right).
\end{equation*}
Seega, vertikaalne jõudude tasakaal avaldub kujul
\begin{equation*}
	N=g(\rho V_{\mathrm{vesi}} + m) = g\left(\rho l r^2\left(1-\frac{\pi}{4}\right) + m\right).
\end{equation*}
Teades reaktsioonijõudu $N$ on lihtne leida hõõrdejõud poolsilindri ja põhja vahel:
\begin{equation*}
	F_{\mu}=\mu N= \mu g \left(\rho l r^2\left(1-\frac{\pi}{4}\right) + m\right).
\end{equation*} 
Horisontaalsuunas tasakaalustab hõõrdejõudu vee horisontaalsuunaline rõhumisjõud $F_{\mathrm{vedelik}}$. Poolsilindri ja selle kohal olevale veesamba süsteemile mõjub keskmiselt rõhk $p = \rho gr/2$, kusjuures kontaktpindala on $S = lr$. Järelikult on horisontaalsuunaline jõud
\begin{equation*}
	F_{\mathrm{vedelik}}=\frac{\rho g r}{2}r l=\frac{\rho g l r^2}{2} F_{\mu}.
\end{equation*}
Kombineerides mõlemad avaldised $F_\mu$ jaoks, saame
\begin{equation*}
	\mu_0=\frac{\rho l r^2}{2(\rho l r^2(1-\frac{\pi}{4})+m}.
\end{equation*}
\probend