\setAuthor{Kaido Reivelt}
\setRound{piirkonnavoor}
\setYear{2021}
\setNumber{G 4}
\setDifficulty{4}
\setTopic{TODO}

\prob{Batüskaaf}
Monika sõidab keset ookeanit batüskaafi ehk süvaveeliikuriga. Ühel hetkel
ütlesid üles batüskaafi mootorid, millega veepaakidest vett välja pumbata. See
vajus $h=\SI{10}{\kilo\meter}$ sügavusele ookeani põhja ja jäi sinna lebama.
Selleks, et batüskaaf pinna poole tõusma hakkaks, on selle paakidest vaja ookeanisse
välja pumbata $V=\SI{1}{\liter}$ vett, mille tulemusena jääb paakidesse vaakum.
Monikal on võimalik kasutada $d=\SI{1}{\centi\meter}$ läbimõõduga silindrilist
pumpa ja erinevaid lihtmehhanisme selle pumbaga töötamiseks. Kui kaua kulub Monikal
vee välja pumpamiseks aega, kui ta suudab rakendada jõudu $F=\SI{500}{\newton}$
ja teha tööd keskmise võimsusega $P=\SI{100}{\watt}$. Ookeani tihedus on
$\rho=\SI{1030}{\kilo\gram\per\meter\cubed}$ ning gravitatsioonikiirendus
$g=\SI{9.8}{\meter\per\second\squared}$. Õhurõhku võib ignoreerida.


\hint

\solu
Kuna kasutada on lihtmehhanismid, pole maksimaalne rakendatav jõud oluline, sest Monikal on neid kasutades võimalik avaldada ükskõik kui suuri jõude. Rõhu vahe batüskaafi väljas ja sees on
$$\Delta p=\rho gh. \quad \pp{2}$$

Rõhuvahe avaldab silindrile jõudu
$$F_p=\Delta p S = \rho gh S. \quad \pp{2}$$
Sellise jõuga tuleb tööd teha, et vett batüskaafist välja pumbata. Vee välja pumpamiseks vajalik töö avaldub kui
$$A=F_p\cdot \Delta x= \rho gh S \Delta x. \quad \pp{2}$$

Paneme tähele, et batüskaafist on tarvis välja pumbata kokku $V=\SI{1}{\liter}$ vett, järelikult peab kehtima $V=S\Delta x$ \pp{1}.

Ruumala $V$ batüskaafist välja pumpamiseks on vaja teha tööd $A=\rho g h V$. Kuna teame, et tööd saab teha keskmise võimsusega $P=\SI{100}{\W}$, saame et vee välja pumpamiseks kulub
$$t=\frac{A}{P}=\frac{\rho g h V}{P}=\SI{1009}{\s}. \quad \pp{1}$$
\probend