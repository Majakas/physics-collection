\setAuthor{Uku Andreas Reigo}
\setRound{lahtine}
\setYear{2024}
\setNumber{G 9}
\setDifficulty{9}
\setTopic{TODO}

\prob{Kondensaatorid}
Kaks ruudukujulise plaadiga õhkkondensaatorit on ühendatud jadamisi ning neile on rakendatud alalispinge $U$. Kondensaatorite plaatide küljepikkused on vastavalt $b$ ja $2b$ ning plaatidevaheline kaugus on $d$, kusjuures $d \ll b$. Kahe kondensaatori vaheline kaugus on $b$. Kondensaatoreid läbib plaadiga paralleelse algkiirusega $v_0$ osake laenguga $q$ ja massiga $m$. Milline on osakese nihe kondensaatori telje sihis, kui ta väljub teise kondensaatori piirkonnast? Eeldage, et osake läbib mõlemat kondensaatorit nende plaate puutumata. Raskuskiirenduse ja kondensaatori ääreefektidega arvestama ei pea. \\
\textit{Vihje:} Kondensaatori mahtuvus on $C=\frac{\varepsilon\varepsilon_0S}{d}$.

\begin{center}
\begin{circuitikz}[european]
\ctikzset{bipoles/cuteswitch/thickness=0.5}
\draw
(0,2) to[short] (0,6)
to[short] (2,6)
to[C] (2,4)
to[short, l^= $b$] (3,4)
to[short](4,4)
to[short] (4,4.85);
\draw
(4,5.15) to [short] (4,6)
to[short, l_ = $2b$] (6,6);

\draw
(0,2) to [battery1, l^ = $U$] (6,2)
to [short] (6,6);

\draw[line width = 1]
(3.25,5.15) to [short] (4.75,5.15);
\draw[line width = 1]
(3.25,4.85) to [short] (4.75,4.85);

\draw[-latex, dashed]
(0.25,5) node[below right] {$m,q$} to [short, l^=$v_0$]  (1.25,5) ;

\end{circuitikz}
\end{center}


\hint

\solu
Kondensaatori mahtuvusest $C = \frac{\varepsilon_0 b^2}{d}$ järeldub, et meil $C_1 = \frac{\varepsilon_0 b^2}{d}$ ja $C_2 = 4C_1$.
Jadamisi ühendatud kondensaatorite jaoks on $C_{tot} = \frac{1}{\frac{1}{C_1} + \frac{1}{4C_1}} = \frac{4C_1}{5}$, seega laeng kõigil plaatidel on $Q = CU = \frac{4UC_1}{5}$. Pinge esimesel kondensaatoril on $U_1 = \frac{Q}{C_1} = \frac{4U}{5}$ ning teisel kondensaatoril järelikult $U_2 = \frac{U}{5}$. Vastavalt, kuivõrd elektriväli kondensaatoris on $E=\frac{U}{d}$, rakendub osakesele esimese kondensaatori alas kiirendus $a_1 = \frac{U_1q}{dm} = \frac{4Uq}{5dm}$ ning teise kondensaatori alas $a_2 = \frac{U_2q}{dm} = \frac{Uq}{5dm}$.
Defineerime koordinaadid nii, et $x$-telg on osakese algses liikumissuunas ning $y$-telg on piki kondensaatori plaate ühendavat sirget.

Liikumisel huvitab meid 3 piirkonda:
\begin{itemize}
    \item $0 < x < b$: Osake on selles piirkonnas aja $T_1 = \frac{b}{v_0}$ ning talle rakendub $y$-telje suunaline kiirendus $a_1$. Seega keskmine y-telje suunaline kiirus on $\Bar{v_1}=\frac{a_1T_1}{2} = \frac{2Uqb}{5dmv_0}$. Kiirendus on positiivse $q$ korral alla
    \item $b<x<2b$: Osake on selles piirkonnas aja $T_2=\frac{b}{v_0}$ ning kiirendust ei rakendu. Osakese kiirus on $v_2 = 2\Bar{v_1} = \frac{4Uqb}{5dmv_0}$
    \item $2b<x<4b$: Osake on selles piirkonnas aja $T_3 = \frac{2b}{v_0}$ ning rakendub $y$-telje suunaline kiirendus $a_2$ vastassuunas esimese kondensaatoriga võrreldes. Osakese algkiirus on $v_2$ ning keskmine kiirus $\Bar{v_3} = v_2 - \frac{T_3a_2}{2} = \frac{4Uqb}{5dmv_0} - \frac{Uqb}{5dmv_0} = \frac{3Uqb}{5dmv_0}$ samas suunas, mis algselt.
\end{itemize}

Kõkkuvõtvalt on $y$-telje suunaline nihe $y = T_1\Bar{v_1} + T_2v_2 + T_3\Bar{v_3} = \frac{12Uqb^2}{5dmv_0^2}$.
\probend