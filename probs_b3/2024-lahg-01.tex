\setAuthor{Jarl Patrick Paide}
\setRound{lahtine}
\setYear{2024}
\setNumber{G 1}
\setDifficulty{1}
\setTopic{TODO}

\prob{Vabasukeldumine}
Vabasukeldumise sügavusrekordit püüdes peab sukelduja alguses sügavuse kasvamiseks palju vaeva nägema, kuna vee üleslükejõud surub vastu. Mingi hetk hakkab sukelduja vabalangema, sest sukelduja ruumala väheneb kokkusurutud kopsude arvelt. Leia kui sügavalt alates hakkab sukelduja vabalangema. Sukeldaja mass on $m_0 = \SI{75}{\kg}$, õhku täistõmmatud kopsude ruumala on $V_{k} = \SI{12}{\l}$ ja sellel hetkel on sukelduja koguruumala $V_0 = \SI{84}{\l}$. Eelda, et sukelduja temperatuur ei muutu sukelduse ajal ja rinnakorvi elastusjõud ei mõjuta kopsuruumala. Õhurõhk on $P_0 = \SI{101300}{\Pa}$, raskuskiirendus on $g = \SI{9.8}{\m\per\s\squared}$ ja vee tihedus on $\rho = \SI{1000}{\kg\per\m\cubed}$






\hint

\solu
Sukelduja vabalangemine hakkab pihta siis, kui üleslükkejõud muutub väiksemaks kui raskusjõud. Selleks peab sukelduja ruumala olema $V_1 = \frac{m_0}{\rho} = \SI{75}{l}$. Selle saavutamiseks peab sukelduja kopsu ruumal olema $V_{k2} = V_{k} - (V_0 - V_1) = \SI{3}{l}$. Meil on sukeldumise ajal isotermiline protsess, seega kops saavutab vajaliku ruumala rõhu $P_1 = \frac{P_0V_{k}}{V_{k2}} = \SI{405200}{Pa}$ juures. Selline rõhk saavutatakse sügavusel $H = \frac{P_1 - P_0}{\rho g} = \SI{31}{m}$
\probend