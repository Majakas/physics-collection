\setAuthor{Uku Andreas Reigo}
\setRound{lahtine}
\setYear{2023}
\setNumber{G 1}
\setDifficulty{1}
\setTopic{TODO}

\prob{Paviljon}
Mirtel varjus vihma eest paviljoni. Sellel seinteta paviljonil on ruudukujuline põrand küljepikkusega $a = \SI{5}{\meter}$, mida varjab postidele toetuv põrandaga ühesuurune ruudukujuline lame katus, mille kõrgus maapinnast $H=\SI{3}{\meter}$. Mirtel märkas, et 20\% põranda pindalast on siiski märg, sest sinna langevad vihmapiisad. Ta tegi anemomeetri abil kindlaks, et puhub konstantse suunaga tuul, mille kiirus $v_t = \SI{3}{\meter\per\second}$ on kõikjal üks ja sama. Mida on nende andmete põhjal võimalik öelda piiskade vertikaalkiiruse kohta: milline on selle suurim võimalik väärtus $v_{max}$ ja vähim võimalik väärtus $v_{min}$?
Minimaalsuse ja maksimaalsuse ranget tõestust pole vaja esitada.




\hint

\solu
Mõistame, et ülesande tingimustest tuletatav piiskade vertikaalne kiirus $v$ sõltub tuule suunast ruudukujulise paviljoni suhtes. Mõlemal korral vihmapiiskade horisontaalne nihe allpool katuseserva on $s = v_t T = \frac{v_t H}{v}$, kus $T$ on kukkumisaeg katuseservast põrandani. Esimesel juhul olgu tuule suund paviljoni mingi küljega risti:

\osa et 20\% põrandast märguks, peab piiskade horisontaalne nihe olema 20\% ruudu külje pikkusest.
\begin{align*}
    s = \frac{v_t H}{v} &= 0.2a \\
    \implies v &= \frac{v_t H}{0.2a} = \SI{9}{\meter\per\second}
\end{align*}

\osa Teine juht, tuule suund on piki ruudu diagonaali.

\begin{tikzpicture}[scale = 2]
    \draw (0,2) -- (2,0) --  (4,2)  -- (2,4) node[midway,above]{$a$}-- (0,2);
    \draw (1.5,3.5) -- (3,2) -- (1.5,0.5);
    \filldraw[fill=black!15](1.5,3.5)--(3,2)-- (1.5,0.5) -- (2,0) -- (4,2) -- (2,4) -- cycle;
    \draw (3,2) -- (4,2) node[midway,above]{$s$};
    \draw (3,2) -- (3.5,1.5) node[midway,xshift = -0.4cm, yshift = -0.4cm] {$s/\sqrt{2}$};
    \draw[-{Latex[length=5mm]}] (5.5,2) -- (4.5,2) node[midway,above]{$v_t$};
\end{tikzpicture}

Näeme, et kui kuiva ruudu külje pikkus on $b = a-\frac{s}{\sqrt{2}}$, ning kuiv ruut moodustab 80\% kogu ruudust, siis:

\begin{align*}
    b^2 &= (1-0.2)a^2\\
    \implies b &= \sqrt{0.8} a = a - \frac{s}{\sqrt{2}}\\
    \implies s &= \sqrt{2}(a-\sqrt{0.8}a) \\ 
    \implies v &= \frac{v_t H}{s} = \frac{v_t H}{\sqrt{2}(a-\sqrt{0.8}a)} \approx \SI{12}{\meter\per\second}
\end{align*}

Leiame, et minimaalne ja maksimaalne võimalik vihmapiisa kiirus on vastavalt $v_{min} = \SI{9}{\meter\per\second}$ ja $v_{max} = \SI{12}{\meter\per\second}$.
\probend