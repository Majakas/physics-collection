\setAuthor{}
\setRound{piirkonnavoor}
\setYear{2019}
\setNumber{G 1}
\setDifficulty{1}
\setTopic{TODO}

\prob{Rong}
Juku tahtis teada, mitu vagunit on rongis. Selleks seisis ta perroonile esimese vaguni esiotsaga kohakuti ning ootas, kuni rong ühtlase kiirendusega sõitmist alustas.
Stopperiga mõõtes sai ta teada, et esimene vagun läks temast mööda täpselt aja $t=\SI{10}{s}$ jooksul ning viimane vagun aja  $t_2 = \SI{1.83}{s}$ jooksul. Mitu vagunit oli rongis?
Vagunid on identsed. 

\hint

\solu
Olgu rongi kiirendus $a$. Kuna vagunid on identsed, siis ühe vaguni pikkus $s=\frac{at^2}{2}$. \pp{1} Olgu rongis $n$ vagunit, siis kõik vagunid mööduvad Jukust 
aja $t_n=\sqrt{\frac{2sn}{a}}$ jooksul. \pp{1} Kõik peale viimase vaguni mööduvad aja $t_{n-1}=\sqrt{\frac{2s(n-1)}{a}}$ jooksul. \pp{1} Seega 
$$t_2=t_n-t_{n-1}=\sqrt{\frac{2sn}{a}}-\sqrt{\frac{2s(n-1)}{a}}=t(\sqrt{n}-\sqrt{n-1}) \quad \pp{1}$$
Siit tuleb ruutvõrrand, mille lahendades saame:
$$n = \frac{1}{4}\bigg(\frac{t_2}{t}\bigg)^2 + \frac{1}{2} + \bigg(\frac{t}{4t_2}\bigg)^2 = 7.97 \approx 8 \quad \pp{1}$$
Üle kontrollides ainuke antud vahemikku sobiv $n=8$ \pp{1}.
\probend