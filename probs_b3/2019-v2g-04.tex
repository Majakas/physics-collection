\setAuthor{}
\setRound{piirkonnavoor}
\setYear{2019}
\setNumber{G 4}
\setDifficulty{4}
\setTopic{TODO}

\prob{Vaakum}
Mõlemast otsast õhukindlalt suletud klaastoru pikkus $\ell=\SI{1}{m}$ ja sisediameeter $d=\SI{1}{cm}$. Õhurõhk $p=\SI{101.3}{kPa}$.\
\osa Kui palju tööd tuleb minimaalselt teha vaakumi tekitamiseks selles torus?\\
\osa Horisontaalse vakumeeritud toru ühes otsas on teraskuulike, mis saab l	ibiseda toru sees praktiliselt ilma hõõrdumiseta ja mille diameeter on võrdne toru sisediameetriga. Õnnetuse tõttu puruneb toru ots, mille lähedal kuulike paikneb  ja õhu surve paneb kuulikese liikuma vakumeeritud osa suunas. Kui suure kiiruse saavutab kuulike jõudes toru teise otsa? Terase tihedus on $\SI{7.9}{g/cm^3}$.






\hint

\solu
\osa Toru tühjakspumpamine tähendab sisuliselt õhu väljasurumist torust, tehes tööd välisrõhu $p$ vastu. \pp{1} Õhk mõjub kuulikesele resultatiivse jõuga $pS$, kus toru ristlõikepindala $S=\pi(d/2)^2=\SI{0.79}{cm^2}$. \pp{1} Jõudu $pS$ tuleb rakendada teepikkusel $\ell$, nii et tehtud töö $A=pS\ell$. \pp{1}   $A=\SI{101300}{Pa}\cdot \SI{7.9e-5}{m^2}\cdot\SI{1}{m}\approx\SI{8.0}{J}$. \\
\osa Kuulikese mass: \[m=\rho V=\rho(4/3)\pi(d/2)^3=\pi \rho d^3/6= \pi\cdot \SI{7.9}{g/cm^3}\cdot (\SI{1}{cm})^3/6 =\SI{4.1}{g}\] \pp{1} Kuna kuulike liigub eeldatavasti hulga aeglasemalt kui on heli kiirus, siis talle mõjub praktiliselt konstantne kiirendav jõud $pS$, jällegi teepikkusel $\ell$. Järelikult eelmises punktis leitud töö $A$ annab ühtlasi kuulikese kineetilise energia toru teises otsas. \pp{1} Kuna $mv^2/2=A$, siis $v=\sqrt{2A/m}$. \pp{1}
\[
v=\sqrt{\frac{2\times \SI{8.0}{J}}{\SI{0.0041}{kg}}}\approx\SI{62}{m/s}\,\text{\pp{1}}
\]
Alternatiivselt võib kasutada ka tuntud valemit ühtlase kiirenduse jaoks, $v^2-v_0^2=2a\ell$, kus algkiirus $v_0=0$ ja kiirendus $a=pS/m$.
\probend