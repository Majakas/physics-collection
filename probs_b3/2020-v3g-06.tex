\setAuthor{}
\setRound{lõppvoor}
\setYear{2020}
\setNumber{G 6}
\setDifficulty{6}
\setTopic{TODO}

\prob{Palliviskenõlv}
Oleg viskab jalgpalliväljaku otsajoone tagant väljakule palli. Selgub, et sealt
jaksab ta visata maksimaalselt väljaku keskjooneni, mis asub kaugusel $L$. Ta
tahab ehitada otsajoone taha sellise nõlva, mille igast punktist jaksaks ta visata
maksimaalselt väljaku keskjooneni. Milline peaks olema selle nõlva kõrgusprofiil?

Võib eeldada, et maksimaalne kiirus, millega Oleg suudab palli visata, ei sõltu
viskamissuunast. Õhutakistust võib ignoreerida. Vastuse võib anda kujul $y=f(x)$
või $x=g(y)$.




\hint

\solu
%Allikas -- \url{https://www.usna.edu/Users/physics/mungan/_files/documents/Scholarship/Projectile.pdf}
\emph{Lahendus 1.}
Ilmselgelt viskab Oleg kaugeimale, kui ta viskab oma maksimaalse kiirusega $v$, mis ei sõltu viskesuunast. Olgu viskenurk horisontaalsihi suhtes $\theta$. Kõrguselt $h$ visates saame liikumise vertikaalsest komponendist lennuaja $t$ jaoks järgmise võrrandi:
\[h=\frac{gt^2}{2}-t\sin \theta v.\]
Horisontaalsuunas liikumisest saame viske pikkusest $\ell$ avaldada aja:
\[t=\frac{\ell}{v\cos \theta}.\]

Asendades selle sisse esimesse võrrandisse, saame
\begin{equation}\label{main}
    h(1+\cos 2\theta)=\frac{g\ell^2}{v^2}-\ell\sin 2\theta.
\end{equation}

Fikseerides $h$, me otsime $\ell=\ell(\theta)$ maksimumi. Maksimumi leidmiseks võtame ülemise võrrandi mõlemast poolest tuletise $\theta$ järgi ja kasutame ära, et maksimumis $\frac{d\ell}{d\theta}=0$, saades
\[\tan 2\theta=\frac{\ell}{h},\]
millest
\[cos2\theta=\frac{h}{\sqrt{h^2+\ell^2}}, \sin 2\theta=\frac{\ell}{\sqrt{h^2+\ell^2}}.\]

Asendades need sisse võrrandisse \ref{main}, saame

\[\ell=\frac{2v^2}{g}\left(h+\frac{v^2}{2g}\right).\]

$h=0$ juhust saame
\[L=\frac{v^2}{g}.\]

Asendades selle sisse avaldisse $\ell$ jaoks, saame
\[\ell=L\sqrt{1+2\frac{h}{L}}.\]

Seega peab nõlva kõrgusel $y=h$ olev punkt olema keskjoonest kaugusel $L\sqrt{1+2\frac{h}{L}}$. Seega keskjoone suhtes on nõlva võrrand
\[x=L\sqrt{1+2\frac{y}{L}}.\]

Nihutades $x$-koordinaadi $0$-punkti väljaku otsajoone juurde, saame
\[x=L\left(\sqrt{1+2\frac{y}{L}}-1\right).\]

\emph{Lahendus 2.}
\emph{Jaan Kalda}
Vaatleme suvalist viskamispunkti nõlval ja kasutame seda $\xi-\eta$~koordinaadistiku nullpunktina. Teatavasti on punktist $(0;0)$ kiirusega $v$ visates tabatava ruumipiirkonna piirjoon $\eta=\frac{v^2}{2g}-\frac{g\xi^2}{2v^2}$. See on parabool, mille tipp on kõrgusel $\frac{v^2}{2g}$ viskepunkti kohal. On ilmne, et väljaku keskpunkt lebab sellel joonel. Peegeldame seda parabooli horisontaaltasandi suhtes ja nihutame nii, et uue parabooli tipp oleks $\frac{v^2}{2g}$ sügavusel väljaku keskpunkti all: $y=-\frac{v^2}{2g}+\frac{gx^2}{2v^2}$, kus  $x-y$~koordinaadistiku nullpunkt on väljaku keskpunkt. Sümmeetria tõttu lebab viskamispunkt sellel uuel kõveral. Uue parabooli kuju ei sõltu viskamispunkti asukohast, seega asuvad kõik viskamispunktid sellel paraboolil.

Soovi korral võime valemi esitada kasutades väljaku poolpikkust $L=v^2/g$: $y=-\frac L2+\frac{x^2}{2L}$. Samuti võime nullpunkti nihutada väljaku serva, $y=x'(x'-2L)/2L$.
\probend