\setAuthor{Krister Kasemaa}
\setRound{lahtine}
\setYear{2018}
\setNumber{G 7}
\setDifficulty{5}
\setTopic{Dünaamika}

\prob{Kauss veega}
Kaalul olevasse kaussi hakatakse ühtlaselt pudelist kõrgusel $h$ vett valama. Vee valamine lõpetatakse hetkel, mil kaalu näit on $m$. Mis on kaalu lugem $M$ peale stabiliseerumist? Kas see on esialgsest näidust suurem, väiksem või võrdne? Õhutakistusega mitte arvestada.\hint
Kaalu näitu mõjutavad kaks asjaolu: ühest küljest suurendab kaalu näitu veesamba ajaühikus üleantud impulss ning teisest küljest suurendab peale valamise lõpetamist näitu veesamba lisandumine kaussi.\solu
Käsitleme esmalt jõudu, mida põhjustab veesamba impulsi muut kausile, jättes kõrvale veesamba massi, mis langeb kaussi peale valamise lõppemist. Veesammas mõjub kausi põhjale jõuga $F=\Delta p / \Delta t$, kus $\Delta p$ on kausile üleantud impulss. Olgu vee kiirus vahetult enne põhja vastu põrkumist $v$ ning ajavahemiku $\Delta t$ jooksul kaussi jõudev vee mass $\Delta m$. Eeldades, et vesi jääb kaussi jõudes koheselt seisma, on $\Delta p = \Delta m v$, ehk $F=\Delta m v / \Delta t = \dot{m} v$, kus $\dot{m} = \Delta m / \Delta t$ on vee massi lisandumise kiirus kausile. 

Vee kiiruse vahetult enne kaussi jõudmist leiame energia jäävuse seadusest kui $v=\sqrt{2gh}$, st
\[
F=\dot mv = \dot m\sqrt{2gh}.
\]
Vee valamine lõpetati hetkel, mil $m_{\mathrm{kausis}} g+F=m_\mathrm{skaalal}g$, millest leiame, et vee mass valamise hetkel kausis oli:
\begin{equation*}
m_{\mathrm{kausis}}=\frac{m_{\mathrm{skaalal}} g-F}{g}=m_{\mathrm{skaalal}}-\dot{m}\sqrt{\frac{2h}{g}}.
\end{equation*}
Käsitleme nüüd veesamba massi, mis lisandub kaussi peale valamise lõppemist. Veesamba mass on $m_{\mathrm{veesammas}}=\dot{m}t$, kus $t$ on veesamba kukkumise aeg $t = v / g$. Seega on valamise lõppedes kaussi jõudva vee mass
\begin{equation*}
m_{\mathrm{veesammas}}=\dot{m} \Delta t = \dot{m}\sqrt{\frac{2h}{g}}.
\end{equation*}
Kaussi jõudva vee mass $M$ on niisiis
\begin{equation*}
M=m_{\mathrm{kausis}}+m_{\mathrm{veesammas}}=m_{\mathrm{skaalal}}-\dot{m}\sqrt{\frac{2h}{g}}+\dot{m}\sqrt{\frac{2h}{g}}=m_{\mathrm{skaalal}}.
\end{equation*}
Teisisõnu kaalu lugem ei muutu.\probend