\setAuthor{Krister Kasemaa}
\setRound{lahtine}
\setYear{2018}
\setNumber{G 7}
\setDifficulty{7}
\setTopic{TODO}

\prob{Kauss veega}
Kaalul olevasse kaussi hakatakse ühtlaselt pudelist kõrgusel $h$ vett valama. Vee valamine lõpetatakse hetkel, mil kaalu näit on $m$. Mis on kaalu lugem $M$ peale stabiliserumist? Kas see on esialgsest näidust suurem, väiksem või võrdne? Õhutakistusega mitte arvestada.




\hint

\solu
Käsitleme esmalt jõudu mida põhjustab veesamba impulsi muut kausile, jättes kõrvale veesamba massi mis langeb kaussi peale valamise lõppemist. Veesammas mõjub kausi põhjale jõuga $F=\frac{dp}{dt}$. Olgu vee kiirus vahetult enne põhja vastu põrkumist $v$. Võime teha lihtsustuse, et kausi põhja vastu põrkudes jääb vesi seisma. Seega,
\[
F=\frac{\mathrm{d}{p}}{\mathrm{d}{t}}=\frac{\mathrm{d}{m}}{\mathrm{d}{t}} |v|.
\]\par 
\noindent Järgmiseks leiame vee kiiruse vahetult enne kaussi jõudmist.  Valamise algushetkel on vee kiirus $v_0\approx 0$. Langedes rakendab veesambale jõudu ainult gravitatsioon. Seega rakendame valemit $v^2-v_0^2=2a\Delta s$, millest avaldub, et $v=\sqrt{2hg}$
Järelikult $F=\frac{\mathrm{d}{m}}{\mathrm{d}{t}} |v|=\frac{\mathrm{d}{m}}{\mathrm{d}{t}}\sqrt{2gh}$
Vee valamine lõpetati hetkel, kui $m_{\mathrm{kausis}} g+F=m_\mathrm{skaalal}g$, millest avaldub, et vee mass valamise hetkel kausis oli:
\begin{equation*}
m_{\mathrm{kausis}}=\frac{m_{\mathrm{skaalal}} g-F}{g}=m_{\mathrm{skaalal}}-\frac{\mathrm{d}{m}}{\mathrm{d}{t}}\sqrt{\frac{2h}{g}}.
\end{equation*}
Käsitleme nüüd veesamba massi, mis lisandub kaussi peale valamise lõppemist. Veesamba mass on $m_{\mathrm{veesammas}}=\frac{\mathrm{d}{m}}{\mathrm{d}{t}}\Delta t$, kus $\Delta t$ on aeg mil kulub veesamba ühe elementaarelemendi jõudmiseks kaussi. Rakendame valemit$v=v_0+at$ ja leiame, et:
\begin{equation*}
\Delta t = \frac{v-v_0}{g}=\frac{v}{g}=\sqrt{\frac{2h}{g}}.
\end{equation*}
Millest järeldub, et valamise lõppedes on kaussi jõudva veesamba mass on
\begin{equation*}
m_{\mathrm{veesammas}}=\frac{\mathrm{d}{m}}{\mathrm{d}{t}} \Delta t = \frac{\mathrm{d}{m}}{\mathrm{d}{t}}\sqrt{\frac{2h}{g}}.
\end{equation*}
Nüüd saame leida vee massi $M$, mis jõuab kaussi:
\begin{equation*}
M=m_{\mathrm{kausis}}+m_{\mathrm{veesammas}}=m_{\mathrm{skaalal}}-\frac{\mathrm{d}{m}}{\mathrm{d}{t}}\sqrt{\frac{2h}{g}}+\frac{\mathrm{d}{m}}{\mathrm{d}{t}}\sqrt{\frac{2h}{g}}=m_{\mathrm{skaalal}}.
\end{equation*}
\probend