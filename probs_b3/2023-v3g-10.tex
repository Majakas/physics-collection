\setAuthor{Jaan Kalda}
\setRound{lõppvoor}
\setYear{2023}
\setNumber{G 10}
\setDifficulty{10}
\setTopic{TODO}

\prob{Uppuv pall}
Vees ujub vettinud ja vett täis tennisepall nii, et see on peaaegu üleni vee all ning ülemine serv puudutab veepinda. Kaldalt, $H=\SI 2\m$ kõrguselt vaadates näib, et pall on ellipsoidi kujuliselt lapik, kusjuures laiuse ja kõrguse suhe on $k=3$. Kui kaugel vaataja silmast on pall, kui vee murdumisnäitaja $n=\frac 43$? Ülesande lahendamisel võib teha mõistlikke lähendusi.


\hint

\solu
\par
Teeme joonise valguskiire käigu kohta silmast palli ülemise ja alumise servani. Need on veepinnal murduvad jooned, mis on peaaegu paralleelsed, kui vaadelda pallilähedast piirkonda, sest palli mõõtmed on hulga väiksemad kaugusest (mis on ilmselt suurem kõrgusest $H=\SI 2\m$). Alumiste sirgete osade vahekaugus $a$ on võrdne palli diameetriga, ülemiste sirgete osade vahekaugus $b$ vastab palli näivale kõrgusele. Ülaltvaates palli vasakusse ja paremasse serva tõmmatud sirged näiliselt ei murdu, seega pallin näiv laius on võrdne palli tegeliku laiusega. Seega on palli näiv lapikus $k=a/b=3$. Kui tähistada murdjoone ülemise osa kaldenurga veepinna suhtes $\beta$-ga ja alumise osa kaldenurga $\alpha$-ga, siis saame eelpooltoodud tingimusest johtuvalt seosed $a=d\sin\alpha$ ja $b=d\sin\beta$, kus $d$ tähistab murdjoonte murdepunktide kaugust. Seega saame võrrandi $\sin\alpha=k\sin\beta$ ning murdumisseadusest  $\cos\alpha=\cos\beta/n$. Võttes need avaldised ruutu ja liites vasakud ning paremad pooled saame $1-n^{-2}=(k^2-n^{-2})\sin^2\beta$, millest $\sin\beta=\sqrt{7/135}$. Silma kaugus pallist $L=H/\sin\beta=H\sqrt{135/7}\approx\SI{8.8}\m$.
\probend