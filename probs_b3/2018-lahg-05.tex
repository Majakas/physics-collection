\setAuthor{Andres Põldaru}
\setRound{lahtine}
\setYear{2018}
\setNumber{G 5}
\setDifficulty{5}
\setTopic{Varia}

\prob{Hiiglane}
Juku vanem vend on lineaarmõõtmete poolest tema kaks korda suuremaks skaleeritud identne koopia. Kas vanem vend hüppab kõrgemale kui Juku? Eeldage, et hüppeliigutus on mõlemal juhul täpselt sama ja et lihaste poolt tekitatav jõud sõltub ainult lihaste ristlõikepindalast. Hüppe kõrguse saamiseks lahutame pealae kõrgusest hüppaja pikkuse.\hint
Hüpet võib käsitleda kui protsessi, kus hüppaja seisab alguses paigal, laskub teatud vahemaa $h_1$ võrra alla ja hüppab seejärel üles kõrguseni $h_2$. Hüppe käigus kehtib energia jäävuse seadus, mida saab lihaste tehtud tööga siduda.\solu
Võrdleme mõlema hüppaja poolt tehtud tööd. Läbides väikese vahemaa $\Delta l$ teevad lihased töö
$$\Delta A = F \Delta l.$$
Jõud sõltub hüppaja kõrgusest $h$ ruutsõltuvuse järgi, sest lihaste pindala kasvab lineaarmõõtme ruuduga. Lisaks kasvab läbitud vahemaa võrdeliselt lineaarmõõtmega. Seega lihaste poolt tehtud töö on võrdeline hüppaja pikkuse kuubiga $A \propto h^3$.

Vaatleme nüüd, kui palju potentsiaalne energia muutub. Algselt seisab hüppaja vabalt, seejärel laskub alla vahemaa $h_1$ võrra ja hüppab üles. Kui hüppe kõrgus on $h_2$, siis potentsiaalsete energiate vahe hüppe madalaimas ja kõrgeimas punktis on $mg(h_1+h_2)$. Kuna alg- ja lõpphetkel on hüppaja paigal (kineetiline energia puudub), siis energia jäävuse seaduse järgi peab tehtud töö võrduma potentsiaalse energia muuduga:
$$A = mg(h_1+h_2) \quad\rightarrow\quad h_2 = \frac{A}{mg} - h_1.$$
Kuna nii mass kui tehtud töö $A$ sõltuvad lineaarmõõtme kuubist, siis jagatis $A/m$ on mõlema hüppaja jaoks sama. Kuna laskumise vahemaa $h_1$ on võrdeline hüppaja pikkusega, siis tuleb välja, et hüppe kõrgus on suuremal vennal hoopis väiksem.\probend