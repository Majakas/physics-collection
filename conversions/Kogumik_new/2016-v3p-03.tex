\ylDisplay{Mõõtepiirkond} % Ülesande nimi
{Tundmatu autor} % Autor
{lõppvoor} % Voor
{2016} % Aasta
{P 3} % Ülesande nr.
{2} % Raskustase
{
% Teema: Elektriõpetus

\ifStatement
Jüri soovis suurendada oma milliampermeetri mõõtepiirkonda. Tema mõõteriista takistus oli $R_A = 3,0$ $\Omega$ ning mõõtepiirkond $0 - 50$ $mA$. Jüri konstureeris uue mõõteriista, ühendades milliampermeetriga rööbiti takisti, mille tegemiseks oli ta kasutanud $s = 2,0$ $mm$ läbimõõduga $l = 20$ $m$ pikkust konstantaantraati, mille eritakistus $\rho = 0,50$ $\frac{\Omega \cdot mm^2}{m}$ . Uue mõõteriistaga voolutugevust mõõtes näitas milliampermeeter $I = 35$ $mA$. Kui suur oli tegelik vool uues mõõteriistas? Kui suur on uue mõõteriista mõõtepiirkond?
\fi

\ifHint
Kuna pinge on nii mõõteriista kui ka takisti klemmidel sama suurusega, siis on voolutugevuste ja takistuste vaheline suhe sama.
\fi

\ifSolution
Mõõteriist ja takisti on ühendatud rööbiti. Voolutugevus milliamprites avaldub seega:
\begin{center}
$I_A = \frac{U}{R_A}$.
\end{center}
Voolutugevuse takistis saame avaldada:
\begin{center}
$I_t = I - I_A = \frac{U}{R_t}$.
\end{center}
Kuna pinge on nii mõõteriista kui ka takisti klemmidel sama suurusega, siis
\begin{center}
$\frac{I - I_A}{I_A} = \frac{R_A}{R_t}$,
\end{center}
millest mõõdetud voolutugevuseks saame 
\begin{center}
$I = \frac{I_A (R_t + R_A)}{R_t}$.
\end{center}
Takisti takistuse arvutame seosest:
\begin{center}
$R_t = \rho \frac{4l}{\pi d^2} = 0,032$ $\Omega$.
\end{center}
Seega mõõdetud voolutugevus on:
\begin{center}
$I = \frac{35 mA(0,032 \Omega + 3 \Omega)}{0,032 \Omega} = 3316$ $mA$
\end{center}
Ampermeetri mõõtepiirkond on:
\begin{center}
$\frac{3316 mA}{35 mA} \cdot 50mA = 4737 mA = 4700 mA$.
\end{center}
\fi
}