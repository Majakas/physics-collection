\ylDisplay{Lennujaam} % Ülesande nimi
{Kristjan Kuppart} % Autor
{piirkonnavoor} % Voor
{2019} % Aasta
{P 4} % Ülesande nr.
{2} % Raskustase
{
% Teema: Mehaanika

\ifStatement
Jüri ja Mari astuvad lennujaama transportlindile, mis liigub kiirusega $u = 0,8$ m/s. Kuna Maril on igav, jookseb ta transportlindi lõppu, pöörab kohapeal ümber ning jookseb mööda linti tagasi Jürini. Kui kaugel transportlindi algusest nad kohtuvad, kui Maril kulus lindi algusest lõppu jooksmiseks $t = 40$ s?
\fi

\ifHint
Lihtsustuse mõttes võib antud liikumist vaadata hoopis seisvas taustsüsteemis.
\fi

\ifSolution
Vaatame ülesannet lindiga seotud taustsüsteemis. Sellises taustsüsteemis on on Jüri paigal ja Mari liigub mingi kiirusega $v$ mõlemas suunas. Kuna Mari kiirus nii mööda linti edasi kui tagasi jooksmisel on sama ning vahemaa, mis Mari läbib on samuti sama, on liikumise aeg mõlemal puhul sama. Maril kulub seega edasi-tagasi jooksmiseks kokku aeg $t_{kogu} = 2t = 80$ s Jüri on selle aja jooksul liikunud maapinna suhtes vahemaa $l = 2ut = 64$ m, mis ongi kaugus lindi algusest, kus nad kohtuvad.
\fi
}
