\ylDisplay{Prillid} % Ülesande nimi
{Tundmatu autor} % Autor
{lõppvoor} % Voor
{2017} % Aasta
{P 7} % Ülesande nr.
{3} % Raskustase
{
% Teema: Valgusõpetus
\ifStatement
Kui Jüri loeb raamatut prillide abil, mille fookuskaugus on $2/3$ meetrit, hoiab ta raamatut $25$ cm kaugusel silmadest. Kui kaugel silmadest peab ta sama raamatut hoidma, et lugeda seda ilma prillideta pingutades silmi nagu eelmisel lugemisel.
\fi
\ifHint
Silma ja prilliläätse koos kasutamisel nende optilised tugevused liituvad.
\fi
\ifSolution
Kehtib seos $D = \frac{1}{f}$ \\
Silma ja prilliläätse koos kasutamisel nende optilised tugevused liituvad. Prillidega raamatut lugedes kehtib seos 
\begin{center}
$\frac{1}{a} + \frac{1}{k} = D_p + D_s$ 
\end{center}
kus a on eseme kaugus, $k$ kujutise kaugus, $D_p$ prillide optiline tugevus, $D_s$ silma optiline tugevus. Ilma prillideta raamatut lugedes peab kujutise kaugus jääma samaks, muutub aga eseme kaugus 
\begin{center}
$\cfrac {1}{a_1} + \cfrac{1}{k} = D_s$, 
\end{center}
kus $a_1$ on eseme kaugus siis, kui prille ei kasutata. Eelnevatest seostest saame
\begin{center}
$\cfrac {1}{a_1} = \cfrac{1}{a} - \cfrac{1}{f_p}$
\end{center}
kus $f_p$ on prilliläätsede fookuskaugus ning $a_1 = 40$ cm.
\fi
}
