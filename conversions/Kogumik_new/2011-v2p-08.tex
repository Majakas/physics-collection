\ylDisplay{Radiaatorid} % Ülesande nimi
{Tundmatu autor} % Autor
{piirkonnavoor} % Voor
{2011} % Aasta
{P 8} % Ülesande nr.
{2} % Raskustase
{
% Teema: Soojusõpetus

\ifStatement
Kahte radiaatorit läbib ajauhikus võrdne hulk vett. Esimesse radiaatorisse siseneb vesi temperatuuriga $t_{1s} = 45^{\circ}$C ja väljub temperatuuriga $t_{1v} = 35^{\circ}$C. Teise radiaatorisse siseneb vesi temperatuuriga $t_{2s} = 40^{\circ}$C ja väljub temperatuuriga $t_{2v} = 25^{\circ}$C. Kumma radiaatori küttevõimsus on suurem ja mitu korda?
\fi

\ifHint
Kuna mõlemat radiaatorit läbib ajaühikus võrdne kogus vett, määrab võimsuste suhte sisenevate ja väljuvate temperatuuride muutude suhe.
\fi

\ifSolution
Veehulk massiga $m$ annab ära soojushulga $Q = cm(t_s - t_v)$, mis kütab ruumi. Võimsuse saamiseks tuleb antud soojushulk jagada ajaga, mis kulub selle veehulga sisenemiseks radiaatorisse.
\begin{center}
$N = c\frac{m}{t}(t_s - t_v)$.
\end{center}
Kuna mõlemat radiaatorit läbib ajaühikus võrdne kogus vett, määrab võimsuste suhte sisenevate ja väljuvate temperatuuride muutude suhe. Teise radiaatori võimsus on suurem
\begin{center}
$\frac{N_2}{N_1} = \frac{t_{2s} - t_{2v}}{t_{1s} - t_{1v}} = 1,5$ korda
\end{center}
\fi
}