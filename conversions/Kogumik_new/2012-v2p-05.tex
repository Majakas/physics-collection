\ylDisplay{Kolksatused} % Ülesande nimi
{Tundmatu autor} % Autor
{piirkonnavoor} % Voor
{2012} % Aasta
{P 5} % Ülesande nr.
{2} % Raskustase
{
% Teema: Mehaanika
\ifStatement
Raudteerööbas on $25$ $m$ pikk. Kui pika aja vältel tuleks lugeda vaguni ühe telje rataste kolksatusi, et nende arv võrduks vaguni kiirusega kilomeetrites tunnis?
\fi
Vaguni läbitud vahemaa on ühest küljest võrdne kiiruse ja aja korrutisega, kuid teisest küljest võrdne ühe rööpa pikkuse ja läbitud rööbaste arvu korrutisega.
\ifHint

\fi

\ifSolution

Vaguniratta kolksatus toimub ratta uleminekul ühelt rööpalt järgmisele rööpale. Teatud arvu kolksatuste jooksul läbib vagun vahemaa $n \cdot s$, kus $n$ on kolksatuste arv ja $s$ rööpa pikkus. Kuna $s = vt$, saab panna kirja seose $ns = vt$. Teisendades kiiruse $v = n \cdot \frac{1000}{3600}$, saab seose avaldada kujul $n \cdot 25 = (n \cdot \frac{1000}{3600})t$, millest $t = 90$ $s$.
\fi
}
