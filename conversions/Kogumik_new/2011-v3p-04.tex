\ylDisplay{Varras} % Ülesande nimi
{Tundmatu autor} % Autor
{lõppvoor} % Voor
{2011} % Aasta
{P 4} % Ülesande nr.
{2} % Raskustase
{
% Teema: Mehaanika

\ifStatement
Sirge ühtlane varras on ühest otsast jäigalt kinnitatud. Kui varda vabale otsale rakendatakse risti vardaga jõud $F_0$, murdub varras kinnituskohast. Nüüd asetatakse kaks korda pikema varda mõlemad otsad tugedele ning varda keskkohale hakatakse risti vardaga jõudu rakendama. Millise jõu $F$ korral ja kustkohast murdub varras seekord?
\fi

\ifHint
Sümmeetria tõttu on uues olukorras kumbki varda pool eraldi võetuna justkui esimeses situatsioonis: üks ots (varda keskkoht) jäigalt kinnitatud, teisele otsale (varda otspunkt) mõjub jõud $F/2$.
\fi

\ifSolution
Paneme tähele, et sümmeetria tõttu on uues olukorras kumbki varda pool eraldi võetuna justkui esimeses situatsioonis: üks ots (varda keskkoht) jäigalt kinnitatud, teisele otsale (varda otspunkt) mõjub jõud $F/2$. Seega murdub varras seekord keskelt. Murdumiseks vajalik jõud $F/2 =$ $F_0$, seega $F = 2F_0$.
\fi
}
