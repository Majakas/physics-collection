\ylDisplay{Kera vees} % Ülesande nimi
{Autor} % Autor
{lõppvoor} % Voor
{2015} % Aasta
{P 2} % Ülesande nr.
{1} % Raskustase
{
% Teema: Mehaanika

\ifStatement
 Kausis on vesi ja selles kera, mis puudutab põhja. Vett on kausis nii palju, et pool kerast on veest väljas. Kera mõjub põhjale jõuga, mis võrdub $\frac{1}{3}$ kera raskusjõust. Kui suur on kera aine tihedus? Vee tihedus on $\rho = 1,0$ $g/cm^3$.
\fi


\ifHint
Ülesande lahendamisel tuleb järgida põhimõtet, et kera poolt kausi põhjale mõjuv jõud on võrdne kera raskusjõu ja temale mõjuva üleslükkejõu vahega.
\fi

\ifSolution
Tähistan $\rho -$ kera aine tihedus, $\rho_v -$ vee tihedus.
\begin{center}
$\frac{1}{3}mg = mg- \rho_v g \frac{V}{2}$
\end{center}
\begin{center}
$\frac{1}{3}\rho V g = \rho V g - \rho_v g \frac{V}{2}$
\end{center}
Sellest saame kera aine tiheduseks
\begin{center}
$\rho = \frac{3}{4} \rho_v = 0.75$ $g/cm^3$.
\end{center}
\fi
}