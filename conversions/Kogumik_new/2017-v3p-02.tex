\ylDisplay{Tilkumine} % Ülesande nimi
{Tundmatu autor} % Autor
{lõppvoor} % Voor
{2017} % Aasta
{P 2} % Ülesande nr.
{2} % Raskustase
{
% Teema: Soojusõpetus

\ifStatement
Plastikanuma põhjas on $m_j = 100$ g jääd temperatuuriga $T_j = -10 ^{\circ}$C. Anuma kohal asuvad külma ja sooja vee kraanid, mis tilguvad. Külma vee kraanist tilgub iga ajavahemiku $t_k = 1$ s möödudes veetilk massiga $m_k = 0,3$ g ning temperatuuriga $T_k = 4 ^{\circ}$C. Sooja vee kraanist tilgub iga ajavahemiku $t_s = 2$ s möödudes veetilk massiga $m_s = 0,3$ g ning temperatuuriga $T_s = 40^{\circ}$C. Millise aja möödudes sulab ära kogu anumas olnud jää? Soojusvahetust väliskeskonnaga ning anuma soojusmahtuvust mitte arvestada. Jää sulamissoojus $\lambda = 330$ kJ/kg, jää erisoojus $c_j = 2100$ $J/(kg \cdot ^{\circ}C)$, vee erisoojus $c_v = 4200$ $J/(kg \cdot ^{\circ}C)$.
\fi

\ifHint
Kogu jää on sulanud siis, kui kraanidest tilkunud vesi on jahtunud $0 ^{\circ}C$-ni ning eraldunud soojus on läinud anumas oleva jää soojendamiseks $0 ^{\circ}C$-ni ning sulatamiseks.
\fi

\ifSolution
Kogu jää on sulanud siis, kui kraanidest tilkunud vesi on jahtunud $0 ^{\circ}C$-ni ning eraldunud soojus on läinud anumas oleva jää soojendamiseks $0 ^{\circ}C$-ni ning sulatamiseks.
Jää soojendamiseks vajaminev soojushulk $Q_1$.
\begin{center}
$Q_1 = c_j m_j (0 ^{\circ}C - T_j)2100$ J
\end{center}
Jää sulatamiseks vajaminev soojushulk $Q_2$
\begin{center}
$Q_2 = \lambda m_j = 33 000$ J
\end{center}
Kogu jää sulatamiseks vajaminev energia on seega $Q_{kogu}$
\begin{center}
$Q_{kogu} = Q_1 + Q_2 = 35 100$ J
\end{center}
Üks tilk külma vett annab $0 ^{\circ}C$-ni jahtudes soojushulga $Q_k$
\begin{center}
$Q_k = c_v m_k (T_k - 0 ^{\circ}C) = 5,04$ J
\end{center}
Üks tilk sooja vett annab $0 ^{\circ}C$-ni jahtudes soojushulga $Q_s$
\begin{center}
$Q_s = c_v m_s(T_s - 0 ^{\circ}C) = 50,4$ J
\end{center}
Kuna soe vesi liigub kaks korda aeglasemalt, siis iga sooja vee tilga kohta tilgub kaks külma vee tilka. Nimetame kolme tilka üheks tsükliks, milleks kulub $2$ sekundit. Ühes tsüklis eraldunud tilgad annavad jahtudes soojushulga $Q_t$.
\begin{center}
$Q_t = 2 \cdot Q_k + Q_s = 60,48$ J
\end{center}
Seega kogu vajamineva soojushulga $Q_{kogu}$ jaoks on vaja $n$ kahesekundilist tsüklit.
\begin{center}
$n = \frac{Q_{kogu}}{Q_t} = 580,36$.
\end{center}
Kuna külma enamus soojust tuleb sooja vee jahtumisest, siis peab tilkuma 581 sooja tilka, milleks kulub aeg $t$.
\begin{center}
$t = 581 \cdot 2 = 1162 s \approx 19$ min 22 s
\end{center}
\fi
}