\ylDisplay{Rongid} % Ülesande nimi
{EFO žürii} % Autor
{piirkonnavoor} % Voor
{2019} % Aasta
{P 1} % Ülesande nr.
{1} % Raskustase
{
% Teema: Mehaanika

\ifStatement
Kahe linna vaheline kaugus mööda raudteed on $900$ km. Linnast $A$ väljub kell $22$ kaubarong linna $B$ suunas. Kaubarong sõidab ühtlase kiirusega $90$ km/h. Kolm tundi pärast väljumist teeb rong ühes jaamas pooletunnise peatuse, et lasta endast mööda kiirrong. Pärast seda jätkab kaubarong sõitu endise kiirusega. Kaks tundi pärast kaubarongi väljumist väljub linnast $B$ kiirrong linna $A$ suunas. Kiirrong liigub peatusteta keskmise kiirusega $144$ km/h. Mis kell ja kui kaugel jaamast $A$ rongid kohtuvad?  
\fi

\ifHint
Esimese ning teise rongi liikumiste kiiruste ja aegade korrutiste summa annab kokku linnade vahelise teepikkuse. Sellest seosest on võimalik avaldada kogu aeg. 
\fi

\ifSolution
Tähistame kaubarongi kiiruse $v_1 = 90$ km/h ja kaubarongi peatuse mingis jaamas $\triangle t_1 = 0,5$ h; reisirongi kiiruse $v_2 = 144$ km/h ja selle kaubarongist hilisema väljumisaja $\triangle t_2 = 2$ h.
\newline
Koostame liikumisvõrrandi:
\begin{center}
$s = v_1(t - \triangle t_1) + v_2 (t - \triangle t_2)$.
\end{center}
Avaldame liikumisvõrrandist kohtumise aja
\begin{center}
$t = \frac{s + v_1 \triangle t_1 + v_2 \triangle t_2}{v_1 + v_2} = 5,27$ h.
\end{center}
Teisendades aja minutitesse, saame, et kahe rongi kohtumise ajaks kell $03:16$ öösel. 
\newline
Arvutame kohtumiskoha kauguse jaamas $A$ arvestades, et kaubarong liikus kohtumiskohani
\begin{center}
$t_1 = t - \triangle t_1 = 4,77$ h
\end{center}
ja kohtumiskoht oli jaamast A kaugusel
\begin{center}
$s_1 = 90 km/h \cdot 4,77 h = 429,3$ km.
\end{center}
\fi
}