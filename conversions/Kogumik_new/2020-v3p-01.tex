\ylDisplay{Päästerõngas} % Ülesande nimi
{EFO žürii} % Autor
{lõppvoor} % Voor
{2020} % Aasta
{P 1} % Ülesande nr.
{1} % Raskustase
{
% Teema: Mehaanika

\ifStatement
Juku tahtis kontrollida oma füüsikaalaste teadmiste õigsust. Ta leppis kokku isa ja tolle sõbraga, et nad sõidaksid paatidega suhteliselt pikal sirgel jõelõigul teineteisele vastu muutumatute kiirustega. Kohtumise hetkel visatakse kummastki paadist välja päästerõngas, sõidetakse edasi täpselt $3$ minutit, siis pööratakse ümber ja jätkatakse sõitu mootori sama võimsusega. Kumb paatidest jõuab enne päästerõngani? Esialgu vastuvoolu sõitva paadi kiiruseks on kohtumise hetkel $12$ km/h ja allavoolu sõitva paadi kiiruseks $20$ km/h.
\fi

\ifHint
Paatide kiirused ei ole tegelikult antud ülesandes üldse olulised. Lihtsustuse mõttes võib kujutada ka, et veevoolu ei eksisteeri ehk et paadid liiguvad seisvas vees.
\fi

\ifSolution
Tuleb aru saada, et ülesande lahendamiseks pole ülesandes antud paatide kiirused olulised. Vaatleme taustsüsteemi, mis on seotud vee vooluga jões. Kuna voolu mõjul liiguvad mõlemad paadid ja päästerõngad ühteviisi allavoolu, võime ette kujutada, et vesi jões ei voola ja paadid sõidavad seisvas vees. Sel juhul jäävad mõlemad vette visatud päästerõngad liikumatult oma kohale ning kuna paat eemaldub päästerõngast $3$ minutit, siis kulub samal paadil tagasi päästerõnga juurde sõitmiseks samuti $3$ minutit. Olukord jääb samaks ka siis, kui vesi jões voolab, sest siis liiguvad mõlemad paadid ja päästerõngad vee voolu kiirusega ühteviisi allavoolu. Seega mõlemad paadid jõuavad päästerõngaste juurde samaaegselt.
\fi
}
