\ylDisplay{Lääts} % Ülesande nimi
{Tundmatu autor} % Autor
{piirkonnavoor} % Voor
{2013} % Aasta
{P 1} % Ülesande nr.
{1} % Raskustase
{
% Teema: Valgusõpetus
\ifStatement
Olgu meil kumerlääts optilise tugevusega $D = 10$ dptr. Kui kaugele läätsest tekib Kuu kujutis? Kui kaugele läätsest tekib Päikese kujutis? Kuu orbiidi raadiuseks võtta $r = 3,8 \cdot 105$ km, Maa orbiidi raadiuseks võtta $R = 1,5 \cdot 108$ km.
\fi
\ifHint
Nii Kuu kui ka Päikese kaugus läätsest on väga suur, seega nendelt lähtunud kiired võib lugeda paralleelseteks
\fi
\ifSolution
Mõlemal juhul on ese läätsest väga kaugel ning sellelt lähtunud kiired võib lugeda paralleelseteks. Kujutised tekivad seega fokaaltasandil, mis asub läätsest kaugusel $f = 1/D = 10$ cm. \\
Lahendus 1: Läätse fookuskauguse leidmine. Läätse valemi teadmine. Taipamine, et kujutis tekib fokaaltasandisse.\\
Lahendus 2: Läätse fookuskauguse leidmine. Taipamine, et mõlemad kehad on väga kaugel. Taipamine, et mõlema keha igast punktist tulevad kiired on omavahel paralleelsed. Teadmine, et paralleelsed kiired tekitavad alati kujutise fokasaltaandis.
\fi
}
 
