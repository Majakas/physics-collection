\ylDisplay{Kontraktsioon} % Ülesande nimi
{EFO žürii} % Autor
{piirkonnavoor} % Voor
{2018} % Aasta
{P 5} % Ülesande nr.
{3} % Raskustase
{
% Teema: Mehaanika

\ifStatement
Omavahel segatakse vett ja piiritust nii, et tekkinud lahuse ruumala $V = 1 dm^3$ ning lahuses on massi järgi $p = 44,1\%$ piiritust. Leidke tekkinud lahuse tihedus $rho$? Arvestage, et lahuste kokkuvalamisel esineb $\gamma = 6 \%$-line kontraktsioon – saadud lahuse ruumala on $6 \%$ väiksem kui vee ja piirituse ruumalade summa enne kokku valamist. Vee tihedus $\rho_v = 1000$ $kg/m^3$ ning piirituse tihedus $\rho_p = 790$ $kg/m^3$.
\fi

\ifHint
Teades, piirituse massiprotsenti, saab esmalt leida vee ja piirituse masside suhte.
\fi

\ifSolution
Tähistame võetud vee massi $m_v$ ning piirituse massi $m_p$. Teades, piirituse massiprotsenti $p = 44,1 \%$, saame leida vee ja piirituse masside suhte.
\begin{center}
$\frac{m_p}{m_p + m_v} = 0,441$ $\Rightarrow$ $m_p = 0,789$ $m_v$ 
\end{center}
Teades lahuse kontraktsiooni $\gamma = 6 \%$, saame kirjutada seose
\begin{center}
$(V_v + V_p)0,94 = V$.
\end{center}
Avaldades vee ja piirituse ruumalad massi ja tiheduse kaudu, saame
\begin{center}
$\frac{m_v}{\rho_v} + \frac{m_p}{\rho_p} = 1,064$ V.
\end{center}
Masside suhtest saime, et $m_p = 0,789$ $m_v$. Asendades selle eelmisesse võrrandisse, saame leida vee ja piirituse massid.
\begin{center}
$\frac{m_v}{1 kg/dm^3} + \frac{0,789m_v}{0,79 kg/dm^3} = 1,064 \cdot 1 dm^3$ $\Rightarrow$ $m_v = 532$ g
\end{center}
\begin{center}
$m_p = 0,789m_v = 420$ g 
\end{center}
Vee ja piirituse lahuse kogumass on seega $m = m_v + m_p = 952$ g. Lahuse tiheduse on seega
\begin{center}
$rho = \frac{m}{V} = 0,952$ $kg/dm^3$.
\end{center}
\fi
}