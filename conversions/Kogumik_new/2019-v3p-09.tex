\ylDisplay{Mikrokuumuti} % Ülesande nimi
{Valter Kiisk} % Autor
{lõppvoor} % Voor
{2019} % Aasta
{P 9} % Ülesande nr.
{3} % Raskustase
{
% Teema: Elektriõpetus

\ifStatement
Mikrokuumuti kujutab endast miniatuurset küttekeha, mis on kantud elektrit mittejuhtiva alusmaterjali pinnale. Küttekeha takistus toatemperatuuril $(t_0 = 20$ $^{\circ}C)$ $R_0 = 50$ $\Omega$. Pärast pinge $U_1 = 1$ $V$ rakendamist saadi voolutugevuseks $I_1 = 12$ $mA$ ning kuumuti temperatuur saavutas väärtuse $t_1 = 420$ $^{\circ}C$. Küttekeha takistus selles temperatuurivahemikus sõltub lineaarselt temperatuurist, st $R_t = R_0 [1 + \alpha (t - t_0)]$, kus $\alpha$ on konstant. Eeldage, et soojuskadude võimsuse on võrdeliseks küttekeha temperatuuri ja toatemperatuuri vahega, st $P_t = k(t - t_0)$, kus $k$ on konstant. 
1) Leidke konstantide $\alpha$ ja $k$ väärtused. 
2) Hinnake võimalikult täpselt kuumuti temperatuur pingel $U = 0,7$ $V$.
\fi

\ifHint
Takistuse leidmiseks temperatuuril $t_1 = 420 ^{\circ}C$ tuleb kasutada Ohm'i seadust ning seejärel saab etteantud valmitest avaldada otsitavad tegurid.
\fi

\ifSolution
Temperatuuril $t_1 = 420 ^{\circ}C$ on küttekeha takistus $R_1 = U_1 / I_1 = 83,3 \Omega$, seega saame avaldada takistuse temperatuuriteguri $\alpha$:
\begin{center}
$\alpha = \frac{R_1 - R_0}{R_0(t_1 - t_0)} = 0,001 655 \frac{1}{K}$.
\end{center}
Samadel tingimustel kuumutusvõimsus $P_1 = U_1 I_1 = 0,012$ $W$, järelikult
\begin{center}
$k = \frac{P_1}{t_1 - t_0} = 3 \cdot 10 ^{-5} \frac{W}{K}$.
\end{center}
Pingel $U$ on kuumutusvõimsus $U^2 / R$ endiselt võrdne soojusjuhtivusest tingitud soojusvoo võimsusega $k(t - t_0)$:
\begin{center}
$\frac{U^2}{R} = \frac{U^2}{R_0 + \alpha R_0(t - t_0)} = k(t - t_0)$.
\end{center}
Siit tekib ruutvõrrand otsitava temperatuurivahe $t - t_0$ suhtes:
\begin{center}
$\alpha k R_0 (t - t_0)^2 + k R_0(t - t_0) - U^2 = 0$.
\end{center}
\begin{center}
$t - t_0 = \frac{- k R_0 \pm \sqrt{(k R_0)^2 + 4 \alpha k R_0 U^2}}{2 \alpha k R_0} = \frac{-1 \pm \sqrt{1 + \frac{4 \alpha U^2}{k R_0}}}{2 \alpha}$.
\end{center}
$t - t_0$ ei saa olla negatiivne, seega sobib vaid pluss-märgiga lahend. Võttes $U = 0,7$ $V$, saame
\begin{center}
$t = t_0 + \frac{-1 \pm \sqrt{1 + \frac{4 \alpha U^2}{k R_0}}}{2 \alpha} \approx 255 ^{\circ}C$.
\end{center}
\fi
}