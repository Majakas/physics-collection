\ylDisplay{Trepp} % Ülesande nimi
{Tundmatu autor} % Autor
{lõppvoor} % Voor
{2011} % Aasta
{P 1} % Ülesande nr.
{1} % Raskustase
{
% Teema: Mehaanika

\ifStatement
Juku kõndis vanaema Juulaga trepist üles. Juku kõndis trepist üles kiirusega $v_1$ (korrust/minutis). Kui ta jõudis viiendale korrusele, siis hakkas ta alla tulema kiirusega $v_2$. Juku ja vanaema kohtusid teisel korrusel. Mitmendale korrusele jõuaks Juku ajaga, mis vanaemal kulub viiendale korrusele minekuks? Juku liigub trepist alla kaks korda kiiremini kui üles. Maja esimene korrus asus maapinnal.
\fi

\ifHint
Kui esimene korrus asub maapinnal, siis teisele korrusele jõudmiseks tuleb tõusta ühe korruse võrra ja viiendale korrusele jõudmiseks on vaja tõusta $4$ korrust.
\fi

\ifSolution
Kui esimene korrus asub maapinnal, siis teisele korrusele jõudmiseks tuleb tõusta ühe korruse võrra ja viiendale korrusele jõudmiseks on vaja tõusta $4$ korrust. Olgu Juula kiirus $v_3$. Juulal ja Jukul kulus teise korruseni jõudmiseks sama aeg:
\begin{center}
$\frac{4}{v_1} + \frac{3}{2v_1} = \frac{1}{v_3}$,
\end{center}
kust leiame $v_1 = 5,5$ $v_3$. Kui Juula jõuab viiendale korrusele, siis oleks Juku tõusnud $\frac{4}{v_3} \times 5,5 v_3 =$ $22$ korrust, seega jõuaks ta $23$. korrusele.
\fi
}
