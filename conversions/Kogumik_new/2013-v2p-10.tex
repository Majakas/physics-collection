\ylDisplay{Vihm} % Ülesande nimi
{Tundmatu autor} % Autor
{piirkonnavoor} % Voor
{2013} % Aasta
{P 10} % Ülesande nr.
{3} % Raskustase
{
% Teema: Mehaanika
\ifStatement
Tuulevaikse ilmaga vihma käes seisev inimene saab märjaks $t = 2$ minutiga. Kui inimene jookseb kiirusega $v_2 = 18$ km/h, saab ta märjaks $t_2 = 0,5$ minutiga. Kui kiiresti saab inimene märjaks siis, kui kõnnib kiirusega $v_1 = 6$ km/h? Eeldage, et inimese keha on samas asendis seismisel, jooksmisel ja kõndimisel ning et inimest võib lähendada risttahukaga. Märjaks saamine tähendab seda, et inimesele langeb teatud kindel kogus vett.
\fi

\ifHint
Seisvale inimesele langeb teatud kogus vett, mille ruumala on võrdne horisontaalsele pinnale langeva vee kogusega. Liikuval inimesel lisandub juurde läbi veepiiskade täidetud õhukõndimisel vertikaalse pinna komponent.
\fi

\ifSolution
Seisvale inimesele langeva vee ruumala saame arvutada seosest $V = Svt$, kus $S$ on inimese horisontaalsuunaline pindala ja $v$ vihmapiiskade langemise kiirus. 
\newline
Kui inimene liigub, siis siseneb ta teatud kiirusega vihmapiiskasid täis õhku. Tema keha vertikaalse osa vastu tulnud vee ruumala saab arvutada seosest $V_v = S_v v 1t$, kus $S_v$ on inimese vertikaalsuunaline pindala ja $v_1$ inimese liikumise kiirus. 
\newline
Arvestades seda, on kiirusega $v_2$ liikuvale inimesele langeva vee ruumala võrdne $V = Svt_2 + S_v v_2 t_2$. 
\newline
Kuna seisvale ja liikuvale inimesele langeva vee ruumala peab olema sama, siis 
\newline
1) jooksva inimese korral kehtib seos $Svt = Svt_2 + S_v v_2 t_2$; 
\newline
2) kõndiva inimese korral seos $Svt = Svt1 + Svv1t1$. 
\newline
Avaldame esimesest seosest $Sv = \frac {1,5Sv}{9} $ ning asendades selle teise seosesse saame, et $t1 = 1$ min.
\fi
}