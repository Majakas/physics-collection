\ylDisplay{Paat vees} % Ülesande nimi
{Tundmatu autor} % Autor
{piirkonnavoor} % Voor
{2016} % Aasta
{P 9} % Ülesande nr.
{3} % Raskustase
{
% Teema: Mehaanika
\ifStatement
Mootorpaat sõidab jõe ääres asuvast külast $A$ ülesvoolu $s = 10$ km kaugusel olevasse teise külasse $B$. Ülesvoolu sõites kulub tal külla $B$ jõudmiseks $t_1 = 4$ h. Allavoolu tagasi sõites saab paadil $t_k = 24$ min pärast kütus otsa ning edasi kulgeb ta jõevoolu kiirusel. Küladevahelise maa läbimiseks kulub paadil tagasi tulles $t_2 = 2$ h. Kui kaugel oli paat külast $A$, kui tal kütus otsa sai?
\fi


\ifHint
Vastuvoolu liikumisel tuleb suhtelise kiiruse leidmiseks lahutada paadi kiirusest veevoolu kiirus ja pärivoolu liikudes tuleb paadi ja veevoolu kiirus liita.
\fi

\ifSolution
Olgu paadi kiirus seisvas vees $v$, veevoolu kiirus $u$. Paadil lõpeb kütus kaugusel $x$ linnast $A$.
\newline
Paadi ülesvoolu sõites kehtiv seos
\begin{center}
$ v - u = \frac {s}{t_1}$
\end{center}
Allavoolu sõites kuni kütuse lõppemiseni kehtib seos
\begin{center}
$ v + u = \frac{s-x}{t_k}$
\end{center}
ja pärast kütuse lõppemist
\begin{center}
$u = \frac {x}{t_2 - t_k}$
\end{center}
Lahendades kolmest võrrandist koosneva võrrandisüsteemi, saame
\begin{center}
$u = \frac{s(t_1 - t_2)}{t_1(t_2 + t_k)} = 3.75$ km/h
\end{center}
\begin{center}
$v = \frac {s(t_1 + t_2)}{t_1(t_2 + t_2)} = 6.25$ km/h
\end{center}
\begin{center}
$x = \frac{s(t_2-t_k)(t_1-t_k)}{t_1(t_2 + t_k)} = 6$ km/h
\end{center}
\fi
}