\ylDisplay{Takistid} % Ülesande nimi
{Tundmatu autor} % Autor
{piirkonnavoor} % Voor
{2006} % Aasta
{P 6} % Ülesande nr.
{2} % Raskustase
{
% Teema: Elektriõpetus

\ifStatement
On kaks takistit. Kui ühendada alalispinge-allikaga eraldi esimene takisti, siis sellel eraldub võimsus $P_1 = 10$ $W$. Kui ühendada eraldi teine takisti, siis takistil eraldub võimsus $P_2 = 15$ $W$. Kui suur summaarne võimsus eraldub pingeallikaga ühendatud takistitel, kui nad ühendada omavahel: a) rööbiti; b) jadamisi?
\fi

\ifHint
Arvestada tuleb, et rööpühendusel on pinge konstantne ning jadaühendusel on kogupinge osapingete summa.
\fi

\ifSolution
Olgu vooluallika pinge $U$, takistite takistused $R_1$ ja $R_2$. Vastavalt ülesande tingimustele
$P_1 = \frac{U^2}{R_1}$ ja $P_2 = \frac{U^2}{R_2}$
a) Rööplahenduse korral on mõlemal takistil pinge $U$, seega takistitel eraldub võimsus vastavalt 
$\frac{U^2}{R_1} = P_1$ ja $\frac{U^2}{R_2} = P_2$,
kokku saame $P = P_1 + P_2 = 25$ $W$.

b) Avaldame takistused võimsuste kaudu
$R_1 = \frac{U^2}{P_1}$, $R_2 = \frac{U^2}{P_2}$.
Jadaühenduses on kogutakistus $R = R_1 + R_2 $ ning koguvõimsus seega
$P = \frac{U^2}{R} = \frac{U^2}{R_1 + R_2} = \frac{U^2}{U^2 / P_1 + U^2/P-2} = \frac{1}{1/P_1 + 1/P_2} = 6$ $W$.
\fi
}