\ylDisplay{Pendel} % Ülesande nimi
{Tundmatu autor} % Autor
{piirkonnavoor} % Voor
{2011} % Aasta
{P 10} % Ülesande nr.
{3} % Raskustase
{
% Teema: Võnkumine

\ifStatement
Pendel pandi väikese amplituudiga võnkuma ning stopperiga registreeriti neid hetki, kui pendel läbis vasakult poolt tulles oma tasakaalupunkti. Kaks järjestikust sellist sundmust toimusid hetkedel  $t_1 = 3,19$ s ja $t_2 =  5,64$ s. Pendlil lasti mõnda aega segamatult võnkuda, seejärel saadi kaheks järjestikuseks näiduks $t_3 =  61,14$ s ja $t_4 =  63,54$ s. Leidke võimalikult täpselt pendli võnkeperiood.
\fi

\ifHint
Esialgse hinnangu perioodile saame leida $t_2 - t_1$ ja $t_4 - t_3$ keskmisest. Seda kasutades näeme, mitu võnget pidi toimuma ajahetkede $t_1$ ja $t_3$ vahel.
\fi

\ifSolution
Esialgse hinnangu perioodile, $\tau = 2,425$ s, saame $\tau _1 = t_2 - t_1$ ja $\tau _2 = t_4 - t_3$ keskmisest. Seda kasutades näeme, et $t_1$ ja $t_3$ vahel pidi toimuma täpselt $24$ võnget, samamoodi $t_2$ ja $t_4$ vahel. Saame kaks sõltumatut mõõtmist $24$ võnke kestuse kohta:
\begin{center}
$\tau'_1  = (t_3 - t_1)/24 = 2,4146$ s ja $\tau'_2  = (t_4-t_2)/24 =  2,4125$ s.
\end{center}
Nende keskmine annab meie hinnangu pendli perioodi kohta, $\tau' = 2,4135 s \approx 2,414$ s.
\fi
}