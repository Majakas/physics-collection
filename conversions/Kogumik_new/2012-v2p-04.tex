\ylDisplay{Bensiinikulu} % Ülesande nimi
{Tundmatu autor} % Autor
{piirkonnavoor} % Voor
{2012} % Aasta
{P 4} % Ülesande nr.
{2} % Raskustase
{
% Teema: Soojusõpetus

\ifStatement
Leidke kiirusel $v = 90$ km/h sõitva auto bensiinikulu liitrites $s = 100$ km kohta, kui mootoris kütuse põlemisel eralduv  võimsus on sellel kiirusel $P = 58$ kW. Bensiini põlemisel eralduv soojushulk ruumalaühiku kohta on $\rho = 35$ MJ/l.
\fi

\ifHint
Ülesande lahendamisel tuleb leida, millise ajaga läbib auto 100 km ja seejärel leida, millise hulga tööd teeb antud mootor selle ajaga.
\fi

\ifSolution
Autol kulub $s = 100$ $km$ läbimiseks aeg $t = \frac{s}{v}$. Selle ajaga kulutab mootor energiat $Q = Pt$. Vastava eneria saamiseks kulub $V = \frac{Q}{\rho}$ bensiini. Siit saab nüüd leida kütusekulu $100$ $km$ kohta.
\begin{center}
$V = \frac{Ps}{v \rho} = \frac{58 kW \cdot 100 km}{90 km/h \cdot 35 MJ/l} \approx 6.6$ $l$.
\end{center}
\fi
}