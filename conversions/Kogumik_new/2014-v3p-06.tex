\ylDisplay{Takistite võimsused} % Ülesande nimi
{Tundmatu autor} % Autor
{lõppvoor} % Voor
{2014} % Aasta
{P 6} % Ülesande nr.
{2} % Raskustase
{
% Teema: Elektriõpetus

\ifStatement
Vooluallikaga muutumatu pingega $U$ ühendati kaks takistit. Kui takistid olid ühendatud jadamisi, oli voolu võimsus vooluringis $2$ W. Kui takistid olid ühendatud rööbiti, oli voolu võimsus vooluringis $9$ W. Arvuta voolu võimsus nendel kahel juhul, kui sama vooluallikaga on ühendatud ainult üks või teine takisti.
\fi

\ifHint
Võimsus on võrdeline pinge ruudu ja pöördvõrdeline takistusega
\fi

\ifSolution
Voolu võimsus $N = \frac{U^2}{R}$.
Kogutakistus jadaühendusel $R = R_1 + R_2$.
Kogutakistus rööpühendusel $R = \frac{R_1 \cdot R_2}{R_1 + R_2}$.
Võimsus jadaühendusel $N_j = \frac{U^2}{R_1 + R_2} = 2$ W.
Võimsus rööpühendusel $N_r = \frac{U^2(R_1 + R_2)}{R_1 \cdot R_2} = 9$ W.
Leiame seostest kummagi takistuse väärtused $R_1 = \frac{U^2}{3} \Omega$ ja $R_1 = \frac{U^2}{6} \Omega$.
Siit voolu võimsused, kui on vooluringi ühendatud ainult üks takisti
$N_1 = \frac{U^2}{R_1} = 3$ W 
$N_1 = \frac{U^2}{R_1} = 6$ W
\fi
}