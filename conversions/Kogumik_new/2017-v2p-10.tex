\ylDisplay{Punker} % Ülesande nimi
{Mihkel Kree} % Autor
{piirkonnavoor} % Voor
{2017} % Aasta
{P 10} % Ülesande nr.
{3} % Raskustase
{
% Teema: Soojusõpetus

\ifStatement
Saarel toodab elektrit suletud punkrisse paigaldatud diiselgeneraator, mille soojuskadudest $P = 300$ W läheb punkri õhu soojendamiseks. Vältimaks punkri ülekuumenemist, paigaldakse ruumi ventilaator võimsusega $N = 5,0$ W, mis suunab õhku punkrist välja läbi toru sisediameetriga $d$. Kui suur peab olema toru läbimõõt $d$, et õhu temperatuur punkris ei ületaks $t_1 = 30$ $^{\circ}$C, samas kui välisõhu temperatuur on $t_0 = 20$ $^{\circ}$C? Soojuskadudega läbi punkri seinte mitte arvestada. Õhu tihedus $\rho = 1,2$ $kg/m^3$ ning erisoojus konstantsel rõhul $c_p = 1,0$ $\frac{kJ}{kg \cdot K}$.
\fi

\ifHint
Ventilaatori võimsus läheb toru läbivale õhule kineetilise energia andmiseks ning arvestame, et õhk liigub läbi ventilatsioonitoru kiirusega $v$. Tasakaalu korral peab ventilaatori kaudu väljuv õhk suutmata ära kanda kogu generaatorist eralduva soojusenergia.
\fi

\ifSolution
Liikugu õhk läbi ventilatsioonitoru kiirusega $v$. Ajaühikus $\triangle t$ läbib toru õhuhulk ruumalaga $\triangle V = S v \triangle t$, kus $S$ on toru ristlõikepindala. Selle õhuhulga mass $\triangle m = \rho S v \triangle t$. Ventilaatori võimsus läheb toru läbivale õhule kineetilise energia andmiseks, mille saame avaldada järgnevalt:
\begin{center}
$N = \frac{\triangle m v^2}{2 \triangle t} = \frac{1}{2} \rho S v^3$.
\end{center}
Tasakaalu korral peab ventilaatori kaudu väljuv õhk suutmata ära kanda kogu generaatorist eralduva soojusenergia, mille võimsuse saame avaldada järgnevalt:
\begin{center}
$P = \frac{\triangle m c_p (t_1 - t_0)}{ \triangle t} = \rho S v c_p(t_1 - t_0)$.
\end{center}
Nendest kahest seosest saame avaldada nõutava toru ristlõikepindala. Selleks peame taandame tundmatu kiiruse $v$. Seda on mugavam teha, kui võtta teine võrrand kuupi ning jagada võrrandid omavahel, millest saame:
\begin{center}
$S^2 = \cfrac{P^3}{2N \rho^2 c_{\rho}^3 (t_1 - t_0)^3}$
\end{center}
ehk
\begin{center}
$S = \sqrt{\frac{P^3}{{2N \rho^2 c_{\rho}^3 (t_1 - t_0)^3}}}$.
\end{center}
Siit saame omakorda nõutava toru diameetri $d = \sqrt{4S/\pi}$,
\begin{center}
$d = \sqrt[4]{\frac{8 P^3}{\pi ^2 N \rho ^2 c_p ^3(t_1 - t_0)^3}}$ ehk $d \approx 4,2$ cm.
\end{center}
\fi
}