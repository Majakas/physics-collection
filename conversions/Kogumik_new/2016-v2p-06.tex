\ylDisplay{Kera vees} % Ülesande nimi
{Tundmatu autor} % Autor
{piirkonnavoor} % Voor
{2016} % Aasta
{P 6} % Ülesande nr.
{3} % Raskustase
{
% Teema: Mehaanika
\ifStatement
Anuma põhjast lahti lastud õõnsusega kera massiga $m$ ja raadiusega $R$ tõusis ühtlase kiirusega vedeliku pinnale. Kui suur lisamass tuleks paigutada kera õõnsusesse, et see vajuks vedelikus põhja sama kiirusega kui enne tõusis. Vedelikus kerale mõjuv takistusjõud on võrdeline kera kiirusega. Vedeliku tihedus on  ja kera ruumala valem $V = \frac{4}{4} \cdot \pi \cdot R3$.
\fi


\ifHint
Kehale mõjuv takistusjõud on vedelikus tõustes ja vajudes sama, kuid vastupidise suunaga.
\fi

\ifSolution
Kerale mõjuv üleslükkejõud on 
\begin{center}
${F_ü} = \frac{4}{3} \rho g \pi R^3$
\end{center}
Ühtlase kiirusega vedelikus tõusvale kehale mõjuvad jõud:
\begin{center}
${F_ü} - mg - F_t = 0$
\end{center}
Ühtlase kiirusega vedelikus vajuvale kehale mõjuvad jõud:
\begin{center}
${F_ü} - (m + \triangle m)g + F_t = 0$
\end{center}
Kuna takistusjõud on mõlemal juhul sama, saame kirjutada
\begin{center}
${F_ü} - mg = (m + \triangle m)g - F_ü$,
\end{center}
millest
\begin{center}
$\triangle m = \frac{2{F_ü} - 2mg}{g}$
\end{center}
ehk
\begin{center}
$\triangle m = \frac{8}{3}\pi R^3 \rho - 2m$
\end{center}
\fi
}