\ylDisplay{Soojusvaheti} % Ülesande nimi
{Tundmatu autor} % Autor
{lõppvoor} % Voor
{2016} % Aasta
{P 6} % Ülesande nr.
{3} % Raskustase
{
% Teema: Soojusõpetus

\ifStatement
Tagasivoolu soojusvahetis jahutatakse sissetulevat naftat temperatuuriga $T_n = 90 ^{\circ}C$ temperatuurini $20^{\circ}$C. Jahutusvesi liigub soojusvahetis vastupidises suunas naftaga ja siseneb soojusvahetisse temperatuuriga $T_v = 10 ^{\circ}C$. Vesi liigub kiirusega $v_v = 6 m^3/min$ ja nafta kiirusega $v_n = 15 m^3/min$. Leidke, millise temperatuuriga väljub soojusvahetist vesi? Vee erisoojus $c_v = 4200 \frac{J}{kg \cdot ^{\circ}C}$ , nafta erisoojus $c_n = 1800 \frac{J}{kg \cdot ^{\circ}C}$. Vee tihedus $\rho_v = 1000 kg/m^3$ ja nafta tihedus $\rho_n = 850$ $kg/m^3$
\fi

\ifHint
Nafta jahtumisel eraldunud soojus läheb vee soojendamiseks.
\fi

\ifSolution
Nafta jahtumisel eraldunud soojus läheb vee soojendamiseks: $Q_{nafta} = Q_{vesi}$.
\begin{center}
$m_n c_n \triangle t_n = m_v c_v \triangle t_v$ $\Rightarrow$ 
\end{center}
\begin{center}
$\rho_n V_n c \triangle t_n = \rho_v V_v \triangle t_v$ $\Rightarrow$ $\triangle t_v = \frac{\rho_n V_n c \triangle t_n}{\rho_v V_v} \approx 64 ^{\circ}C$.
\end{center}
Seega väljub vesi soojusvahetist temperatuuriga
\begin{center}
$T = 64 ^{\circ}C + 10  ^{\circ}C = 74  ^{\circ}$C.
\end{center}
\fi
}