\ylDisplay{Takistite ühendused} % Ülesande nimi
{Tundmatu autor} % Autor
{piirkonnavoor} % Voor
{2008} % Aasta
{P 5} % Ülesande nr.
{2} % Raskustase
{
% Teema: Elektriõpetus

\ifStatement
Antud on kolm takistit väärtustega $R_1 = 1$ $\Omega$, $R_2 = 2$ $\Omega$ ja $R_3 = 3$ $\Omega$. Milliseid erinevaid kogutakistuse väärtusi võib saada neid omavahel kahe- või kolmekaupa kõikvõimalikel viisidel ühendades?
\fi

\ifHint
Kokku on võimalik takisteid ühendada 14 erinevat viisi.
\fi

\ifSolution
Võimalikud ühendused ja vastavad takistused ($\times$ tähistab jadaühenduse, $\mid \mid$ rööpühendust):
\begin{center}
$R_1 \mid\mid R_2 = \frac{2}{3} \Omega$,
$R_1 \mid\mid R_3 = \frac{3}{4} \Omega$,
$R_2 \mid\mid R_3 = \frac{6}{5} \Omega$,
$R_1 \mid\mid R_2 \mid\mid R_3 = \frac{6}{11} \Omega$.
\end{center}

\begin{center}
$(R_1 \mid\mid R_2) \times R_3 = \frac{11}{3} \Omega$,
$(R_1 \mid\mid R_3) \times R_2 = \frac{11}{4} \Omega$,
$(R_2 \mid\mid R_3) \times R_1 = \frac{11}{5} \Omega$.
\end{center}

\begin{center}
$(R_1 \times R_2) \mid\mid R_3 = \frac{3}{2} \Omega$,
$(R_2 \times R_3) \mid\mid R_1 = \frac{5}{6} \Omega$,
$(R_1 \times R_3) \mid\mid R_2 = \frac{4}{3} \Omega$.
\end{center}

Seega, kokkuvõttes saame järgmiste takistuste väärtused:
\begin{center}
$\frac{6}{11}$;
$\frac{2}{3}$;
$\frac{3}{4}$;
$\frac{5}{6}$;
$1\frac{1}{5}$;
$1\frac{1}{3}$;
$1\frac{1}{2}$;
$2\frac{1}{5}$;
$2\frac{3}{4}$;
$3$;
$3\frac{2}{3}$;
$4$;
$5$;
$6$
$\Omega$.
\end{center}
\fi
}