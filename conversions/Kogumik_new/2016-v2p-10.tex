\ylDisplay{Kütusekulu linnas} % Ülesande nimi
{Tundmatu autor} % Autor
{piirkonnavoor} % Voor
{2016} % Aasta
{P 10} % Ülesande nr.
{3} % Raskustase
{
% Teema: Mehaanika
\ifStatement
Oletame lihtsustavalt, et sõiduauto massiga $m = 1500$ kg peab linnaliikluses valgusfooride, sebrade jms tõttu peatuma iga vahemaa $L = 500$ m tagant. Peatumiste vahel sõidab auto kiirusega $v = 50$ $km/h$. Arvutage välja linnaliikluses sagedastest peatumistest tingitud auto keskmine kütusekulu (liitrit sajale kilomeetrile). Arvestage, et bensiini tihedus $\rho = 0,72$ $kg/dm^3$ ning kütteväärtus $M = 44$ MJ/kg, millest mootor muundab $\eta = 25$\% kasulikuks tööks. Õhutakistusega liikumisel ega kütusekuluga seismisel ärge arvestage. Märkus: Kineetilise energia valem $E_k = \frac{mv^2}{2}$ .\fi
\fi


\ifHint
Kogu mootori poolt tehtud töö läheb peatumise järgselt autole kineetilise energia.
\fi

\ifSolution
Kogu mootori poolt tehtud töö läheb peatumise järgselt autole kineetilise energia $E_k = \frac{mv^2}{2}$ andmiseks. Teisendame kiiruse $SI$ ühikutesse: $v = 13,9$ $m/s$. Kütuse massiga $E_k$ põletamisel vabaneb energia $E_p = M m_k$. Osa kütuselt tulevast energiast kulub auto kineetilise energia suurendamiseks.
\begin{center}
$\eta M m_k = \frac{mv^2}{2}$
\end{center}
Vahemaa $L$ läbimiseks kulub kütust 
\begin{center}
$m_k = \frac{Ek}{\eta M}$, 
\end{center}
millele vastab ruumala 
\begin{center}
$V =\frac{mk}{\rho}$.
\end{center}
Leitud kütuse ruumala $V$ kulub vahemaa $L = 500$ $m$ $= 0,5$ $km$ läbimiseks. See aga tähendab, et sajale kilomeetrile kulub 200 korda rohkem kütust. Niisiis, otsitav auto kütusekulu on
\begin{center}
$k = V \cdot \frac{200}{100km} = \frac{mv^2}{2 \eta M \rho} \cdot \frac{200}{100km} = 3.7 \frac{L}{100km}$
\end{center}
\fi
}