\ylDisplay{Glütseriin torus} % Ülesande nimi
{Autor} % Autor
{lõppvoor} % Voor
{2016} % Aasta
{P 7} % Ülesande nr.
{3} % Raskustase
{
% Teema: Mehaanika

\ifStatement
Purki, milles on vesi, on paigutatud vertikaalne lihvitud otstega toru, mida hoitakse kinni. Toru alumise otsa vastas on sileda pinnaga teeküünal. Teeküünla läbimõõt $D = 3,5$ cm, toru läbimõõt $d = 1,5$ cm. Teeküünla kõrgus $h_k = 1,2$ cm ja küünla tihedus $\rho_k = 0,9$ $g/cm^3$ . Torusse on valatud $h = 5$ cm kõrgune sammas glütseriini, mille tihedus on $\rho_g = 1,26$ $g/cm^3$ . Kui sügaval veepinnast $H$ peab olema toru alumine ots, et glütseriin sellest vette ei voolaks? Vee tihedus on $\rho_v = 1,0$ $g/cm^3$.
\fi

\ifHint
Glütseriin ei voola välja, kui teeküünal püsib paigal, ehk selle alla ja üles mõjuvad jõud on tasakaalus.
\fi

\ifSolution
Küünla alumise põhja pindala $S_{ka} = \frac{\pi d^2}{4}$
\newline
Toru ristlõikepindala $S_t = \frac{\pi d^2}{4}$
\newline
Küünla veega kokkupuutuva ülemise põhja pindala $S_{kü} = \frac{\pi}{4}(D^2 - d^2)$
\newline
Glütseriin ei voola välja, kui teeküünal püsib paigal, ehk selle alla ja üles mõjuvad jõud on tasakaalus. Jõudude tasakaal avaldub kujul $F_G + F_{vü} + mg = F_{va}$,
\newline
kus $F_G$ - glütseriini rõhumisjõud küünla ülemisele põhjale,
\newline
$F_{vü}$ - vee rõhumisjõud küünla ülemisele pinnale,
\newline
$mg$ - küünlale mõjuv raskusjõud,
\newline
$F_va$ - küünla alumisele põhjale mõjuv vee rõhumisjõud:
\begin{center}
$F_G = \rho_G g h S_t = \rho_G g h \frac{\pi d^2}{4}$
\end{center}
\begin{center}
$F_{vü} = \rho_v g H \frac{\pi}{4}(D^2 - d^2)$
\end{center}
\begin{center}
$mg = \frac{\pi D^2}{4}h_k g \rho_k$
\end{center}
\begin{center}
$F_va = \rho_v g (H + h_k)\frac{\pi D^2}{4}$
\end{center}
Tehes teisendused, saame 
\begin{center}
$H = \frac{\rho_{G} h d^2 + h_G D^2 (\rho_k - \rho_v)}{\rho_v d^2} = 5,65$ cm
\end{center}
\fi
}