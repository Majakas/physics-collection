\ylDisplay{Ringrada} % Ülesande nimi
{Tundmatu autor} % Autor
{lõppvoor} % Voor
{2014} % Aasta
{P 1} % Ülesande nr.
{1} % Raskustase
{
% Teema: Mehaanika
\ifStatement
Juhan, Kalle ning Lauri sõidavad ringrajal jalgratastega võidu. Kõik kolm stardivad korraga ühest kohast ning iga rattur sõidab muutumatu kiirusega. On teada, et Kalle teeb Juhanile ringi sisse siis, kui Kalle on just lõpetanud viienda ringi. Lauri teeb Kallele ringi sisse siis, kui Lauri on just lõpetanud kuuenda ringi. Mitu ringi oli Juhan sõitnud, kui Lauri temast esimest korda ringiga möödus?
\fi

\ifHint
Ülesande saab lahendada pelgalt loogilise arutluse teel ilma konkreetseid valemeid kasutamata.
\fi

\ifSolution
Kui Lauri lõpetas kuuenda ringi, pidi Kalle lõpetama viienda ringi ning seega Juhan neljanda. Näeme, et Lauri läbib kolm ringi selle ajaga, mis Juhanil kulub kahe läbimiseks, järelikult mööduti Juhanist esimest korda ringiga siis, kui ta oli lõpetanud oma teise ringi.
\fi
}

