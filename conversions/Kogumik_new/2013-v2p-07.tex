\ylDisplay{Elektriküünlad} % Ülesande nimi
{Tundmatu autor} % Autor
{piirkonnavoor} % Voor
{2013} % Aasta
{P 7} % Ülesande nr.
{3} % Raskustase
{
% Teema: Elektriõpetus

\ifStatement
Miku ehtis jõulude ajal kuuse elektriküünaldega, milles jadamisi oli ühendatud $20$ lampi nimepingega $12$ $V$ ja nimivõimsusega $15$ $W$. Paari päeva pärast põles uks lampidest läbi. Kuna Mikul samasugust lampi ei olnud, asendas ta läbipõlenud lambi ema õmblusmasina lambiga, mille nimivõimsus oli samuti $15$ $W$, kuid nimipinge $220$ $V$. Mitu korda muutus lambi vahetuse tõttu elektrivoolu võimsus teistes lampides ning millise eredusega põlesid pärast lambi vahetust teised lambid? Pinge elektriküünalde pistiku otstel oli $220$ $V$. Eeldada, et lambi takistus ei sõltu temperatuurist ja pingest.

\ifHint
Esmalt leia valgustite algne tegelik pinge ja võimsus, mis on väiksemad nimipingest ja -võimsusest.
\fi

\ifSolution
Esialgu on tegelik pinge iga lambi klemmidel $U_t = \frac{U}{n} = \frac{220V}{20} = 11$ $V$.
Kuna pinge lambil on väiksem nimipingest, on voolu võimsus lambis väiksem lambi nimivõimsusest. Lähtudes seosest $N = \frac{U^2}{R}$ leiame voolu võimsuse töötavates elektriküünalde lampides. $\frac{N_{t_1}}{N_1} = \frac{U^2 _{t_1} R}{RU^2 _1}$,  millest $N_{t_1} = N_1 \frac{U^2 _{t_1}}{U^2_1}$, $N_{t_1} = 12,6$ $W$. 
Leiame $12$ $V$ ja $220$ $V$ lampide takistused.
$R = \frac{U^2}{N}$, $R_1 = \frac{12^2 V^2}{15 W} = 9,6$ $\Omega$ ; $R_2 = \frac{220^2 V^2}{15 W} = 3227$ $\Omega$.
Pärast ühe lambi vahetust on jadamisi ühendatud lampide kogutakistus.
$R = n R_1 + R_2$, $R = 19 \cdot 9,6 \Omega + 3227 \Omega = 3410 \Omega$.
Voolutugevus lampides on $I = \frac{U}{R} = \frac{220 V}{3410 \Omega} = 0,065 \Omega$.
Voolu võimsus $12$ $V$ nimipingega lambis on sel juhul $N_{t_1} = I^2 R_1$, $N_{t_2} = (0,065A)^2 \cdot 9,6 \Omega = 0,04W$.
Voolu võimsus lampides vähenes seega $\frac{N_{t_1}}{N_{t_2}} = \frac{12,6 W}{0,04W} = 315 $ korda.
Kuna $12$ $V$ nimipingega lampides vähenes voolu võimsus 315 korda ja on $15$ $W$ asemel ainult $0,04$ $W$, siis ilmselt lampide hõõgniidid isegi ei hõõdu ning põleb ainult $220$ $V$ nimipingega lamp, kuna selles on vastav voolu võimsus.
\fi
}