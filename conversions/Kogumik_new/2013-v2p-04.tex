\ylDisplay{Veepudel} % Ülesande nimi
{Tundmatu autor} % Autor
{piirkonnavoor} % Voor
{2013} % Aasta
{P 4} % Ülesande nr.
{2} % Raskustase
{
% Teema: Soojusõpetus

\ifStatement
Külma ilmaga oli autosse ununenud $1,5$ liitrine täis veepudel. Auto juurde tulnud autojuht Koit ei uskunud oma silmi: temperatuur autos oli $-5 ^{\circ}$ C, aga vesi pudelis ei olnud kulmunud. Koidule tuli meelde, et ta oli kunagi kuulnud, et väga puhas vedelik võib olla vedelas olekus ka allpool tahkumistemperatuuri. Selle kontrollimiseks võttis ta pudeli ja raputas seda ning suhteliselt kiiresti muutus selles osa veest jääks. Mitu grammi jääd tekkis pudelisse? Vee erisoojus $c = 4200 J/kg^{\circ}C$ ja tihedus on $\rho = 1000 kg/m^3$, jää sulamissoojus $\lambda = 340$ kJ/kg.
\fi

\ifHint
Ülesande põhiline idee seisneb selles, et vesi jäätub temperatuuril $0 ^{\circ}$C. Osa vee jäätumisel eralduvast soojushulgast läheb allajahtunud vee soojendamiseks jäätumistemperatuurile, seega tahkumisel eralduv soojushulk ja allajahtunud vee soojenemiseks kuluv energiahulk peavad olema võrdsed.
\fi

\ifSolution
Vesi jäätub temperatuuril $0 ^{\circ}$C. Osa vee jäätumisel eralduvast soojushulgast läheb allajahtunud vee soojendamiseks jäätumistemperatuurile.
\newline
$Q_{tahkumine} = Q_{soojenemine} $.
\newline
Valemite teadmine ja nende seostamine $cm\triangle t = \lambda m_{jää}$
\newline
Tekkinud jää massi avaldamine $m_{jää} =  \frac{cm \triangle t}{\lambda}$.
\newline
Arvutus, ühikud ja õige vastus:
\newline
$m_{jää} = \frac{4200 J/kgK \cdot 1,5 kg \cdot 5 ^{\circ}C}{340 000 J/kg} = 93$ $g$.
\fi
}