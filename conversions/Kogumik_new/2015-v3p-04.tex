\ylDisplay{Kuup vedelikes} % Ülesande nimi
{Tundmatu autor} % Autor
{lõppvoor} % Voor
{2015} % Aasta
{P 4} % Ülesande nr.
{3} % Raskustase
{
% Teema: Mehaanika

\ifStatement
Anum on täidetud kahe mitteseguneva vedelikuga tihedustega $\rho_1$ ja $\rho_2$. Vedelikku lastakse kuup külje pikkusega $l$. Kui sügavale $x$ vajub kuup teise vedelikku, kui kuubi tihedus on $\rho$? On teada, et $\rho_1 < \rho < \rho_2$.
\fi

\ifHint
Kuup asub sellisel sügavusel, et esineks jõudude tasakaal, kus kuubi raskusjõu ja kuubi ülemisele pinnale mõjuva vedeliku rõhumisjõu summa on võrdne vedeliku rõhumisjõuga kuubi alumisele pinnale.
\fi

\ifSolution
Kuup asub sellisel sügavusel, et esineks jõudude tasakaal $mg + F_1 = F_2$, kus $F_1$ on vedeliku rõhumisjõud kuubi ülemisele pinnale ja $F_2$ vedeliku rõhumisjõud kuubi alumisele pinnale.
\begin{center}
$F_1 = \rho_1 g h l^2$ ja $F_2 = \rho_1 g (h + l - x) l^2 + \rho_2 g x l^2$, seega
\end{center}
\begin{center}
$\rho g l^3 + \rho_1 gh l^2 = \rho_1 g(h + l - x)l^2 = \rho_1 g (h +l -x)l^2 + \rho_2 g x l^2$, millest 
\end{center}
\begin{center}
$x = l\frac{\rho - \rho_1}{\rho_2 - \rho_1}$.
\end{center}
 \fi
}