\ylDisplay{Hõbetatud lääts} % Ülesande nimi
{Tundmatu autor} % Autor
{lõppvoor} % Voor
{2016} % Aasta
{P 1} % Ülesande nr.
{2} % Raskustase
{
% Teema: Valgusõpetus
\ifStatement
Tasakumera läätse kumera tahu kõverusraadius on $50$ $cm$. Selle läätse optiline tugevus on $1$ $dpt$. Läätse kumer pind kaetakse hõbeda kihiga ning sellest tekib peegelpind. Kui suureks kujuneb selle keha optiline tugevus pärast kumera pinna hõbedaga katmist?
\fi
\ifHint
Tekib optiline süsteem, kus liituvad kumerläätse, nõguspeegli ja uuesti kumerläätse optilised tugevused.
\fi
\ifSolution
Kui läätse kumer pind on hõbetatud, siis kehale langev valgus läbib kumerläätse, peegeldub nõguspeeglilt ja läbib uuesti kumerläätse. Seega süsteemi optiline tugevus on $D = D_1 + D_2 + D_1$. Nõguspeegli fookuskaugus on pool kõverusraadiusest, seega on see $25$ $cm$ ja optiline tugevus $4$ dpt. Kogu süsteemi optiline tugevus on seega $6$ dpt.
\fi
}
