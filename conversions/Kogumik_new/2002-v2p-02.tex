\ylDisplay{Vedelikud} % Ülesande nimi
{Tundmatu autor} % Autor
{piirkonnavoor} % Voor
{2002} % Aasta
{P 2} % Ülesande nr.
{1} % Raskustase
{
% Teema: Valgusõpetus
\ifStatement
Klaasis on kaks kihti erinevat läbipaistvat vedelikku, mille vahel on terav horisontaalne piirjoon. Kuidas valguskiire abil teha kindlaks, kummas vedelikus on valguse levimise kiirus suurem?
\fi

\ifHint
Ülesande lahendus on seotud optilise tiheduse ja murdumisseadusega.
\fi

\ifSolution
Kui valguskiir üleminekul ühest keskkonnast teise kaldub pinna ristsirge poole, on teise keskkonna optiline tihedus suurem esimese keskkonna optilisest tihedusest ja vastupidi. Valguse kiirus on väiksem selles keskkonnas, mille optiline tihedus on suurem. Järelduse võib esitada nii joonisena kui ka sõnaliselt.
\fi
}
 
 

