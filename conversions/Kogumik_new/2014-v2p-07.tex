\ylDisplay{Püstolkuulipilduja} % Ülesande nimi
{Tundmatu autor} % Autor
{piirkonnavoor} % Voor
{2014} % Aasta
{P 7} % Ülesande nr.
{2} % Raskustase
{
% Teema: Mehaanika
\ifStatement
Korraldati katse, milles mõõdeti püstolkuulipilduja kuulide keskmist kiirust ja laskude arvu minutis. Selleks sõitsid kahel paralleelsel raudteel vastassuunas rongid. Rong, millest tulistati, sõitis katse ajal muutumatu kiirusega $72$ km/h ja rong, mis oli märklauaks, muutumatu kiirusega $36$ km/h. Rongide kaugus teineteisest oli $60$ m. Esimene lask tehti hetkel, kui rongiga risti oleva püstolkuulipilduja toru ots oli täpselt vastakuti teisele rongile joonistatud märgiga. Kuuliaukude asukohtade mõõtmisel täheldati, et esimene kuul oli tabanud rongi $2,25$ m märgist eemal ja kõikide teiste kuuliaukude kaugus üksteisest oli $2,5$ m. Kui suur on püstolkuulipilduja kuuli lennukiirus ja mitu lasku teeb püstolkuulipilduja minutis?
\fi


\ifHint
Kuuli lennuaeg võrdub kõrvalekaldega märgist jagatud rongide suhtelise kiirusega.
\fi

\ifSolution
Kuuli lennuaeg võrdub kõrvalekaldega märgist jagatud rongide suhtelise kiirusega
\begin{center}
$t = \frac{2.25 m/s}{20m/s + 10 m/s} =$ $0.075$ s.
\end{center}
Kuuli kiirus $v = \frac{60m}{0.075s} =$ $800$ m/s.
Kahe järgneva lasu vaheline aeg
\begin{center}
$T = \frac{2.5m/s}{20m/s + 10m/s} = 0,833$ s.
\end{center}
Laskude sagedus $n = \frac{60 s/min}{0.833s} =$ $720$ lasku/min.
\fi
}