\ylDisplay{Jääst nõu} % Ülesande nimi
{Erkki Tempel} % Autor
{lõppvoor} % Voor
{2018} % Aasta
{P 10} % Ülesande nr.
{3} % Raskustase
{
% Teema: Soojusõpetus

\ifStatement
Vees temperatuuriga $0^{\circ}$C ujub jääst kuup massiga $m_j = 1,5$ kg, mille sees on tühimik ruumalaga $V = 12$ $cm^3$. Tühimikku valatakse hästi aeglaselt elavhõbedat temperatuuriga $t$. Täpselt sel hetkel, kui tühimik täitub elavhõbedaga, vajub jääst kuup põhja. Leidke tühimikku kallatud elavhõbeda temperatuur $t$. Jää tihedus $\rho_j = 900$ $kg/m^3$ , vee tihedus $\rho_v = 1000$ $kg/m^3$, elavhõbeda tihedus $\rho_{Hg} = 13 600$ $kg/m^3$ , elavhõbeda erisoojus $c = 140$ $J/(kg\cdot ^{\circ}C)$, jää sulamissoojus $\lambda = 330$ kJ/kg. Soojusvahetust väliskeskkonnaga mitte arvestada.
\fi

\ifHint
Teades, et uppumise korral on elavhõbedaga täidetud jääast kuubi keskmine tihedus võrdne vee tihedusega, saame leida lisatud elavhõbeda massi ning ruumala.
\fi

\ifSolution
Teades, et uppumise korral on elavhõbedaga täidetud jääast kuubi keskmine tihedus võrdne vee tihedusega, saame leida lisatud elavhõbeda mass $m_A$ ning ruumala $V_A$.
\begin{center}
$\frac{m_j + m_A}{V_j + V} = \frac{m_j + m_A}{\frac{m_j}{p_j} + V} = \rho_v$ $\Rightarrow$ $m_A = \frac{\rho_v}{\rho_j} m_j + V \rho_v - m_j$
\end{center}
\begin{center}
$m_A = 0,1797 kg$ 
$V_A = \frac{m_a}{\rho_A} = 13,14$ $cm^3$
\end{center}
Kuna jää ruumala on suurem kui jää sulamisel tekkinud vee ruumala, siis suurenes jää sulamisel tühimiku ruumala $\triangle V = 1,14$ $cm^3$ võrra. Sulanud jää mass on $m_s$, seega
\begin{center}
$\triangle V = \frac{m_s}{\rho_j} - \frac{m_s}{\rho_v}$ $\Rightarrow$ $m_s = \frac{\rho_j \rho_v \triangle V}{\rho_v - \rho_j} = 10,23$ $g$.
\end{center}
Teades sulanud jää massi ning elavhõbeda massi, saame soojusülekandest $Q_1 = Q_2$ leida elavhõbeda temperatuurimuutuse $\triangle t$
\begin{center}
$cm_{Hg} \triangle t = \lambda m_s$ $\Rightarrow$ $\triangle t = \frac{\lambda m_s}{cm_{Hg}} = 135$ $^{\circ}$C.
\end{center}
Elavhõbeda temperatuur on $T = 135$ $^{\circ}$C.
\fi
}