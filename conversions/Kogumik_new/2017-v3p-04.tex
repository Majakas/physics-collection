\ylDisplay{Juhtmed} % Ülesande nimi
{Tundmatu autor} % Autor
{lõppvoor} % Voor
{2017} % Aasta
{P 4} % Ülesande nr.
{2} % Raskustase
{
% Teema: Elektriõpetus

\ifStatement
Saarevaht saab elamiseks vajaliku elektrienergia katusele paigaldatud päikeseelementidest, mille energia salvestatakse majas asuvasse akusse. Majast kaugusel $l = 50$ $m$ paikneb veinikelder, kuhu saarevaht kogub kaldale uhutud rummipudeleid. Et ka öisel ajal oleks võimalik pudelite etikette selgelt välja lugeda, otsustab saarevaht keldrisse paigaldada hõõglambipirni nimipingega $U = 12$ $V$ ja nimivõimsusega $P = 25$ $W$ ning vedada majast keldrini toitejuhtmed, milleks on saarevahil paraku vaid kasutada valvesüsteemides kasutatav peenike vaskjuhe eritakistusega $\rho = 1,7 \cdot 10^{-8} \cdot m$ ja ristlõikepindalaga $S = 0,20$ $mm^2$. Arvutage, millise võimsusega hakkab põlema keldrisse paigaldatud hõõglamp, kui pinge aku klemmidel on $E = 13$ $V$. Pirni takistuse sõltuvust temperatuurist ei ole vaja arvestada. 
\fi

\ifHint
Kuna tegemist on võrdlemisi pika ja peenikese juhtmega, siis peame me arvesse võtma juhtme takistust, mis on jadamisi lambi takistusega.
\fi

\ifSolution
Lambi takistuse $r_L$ saame avaldada selle nimipinge $U$ ja nimivõimsuse seosest $P = \frac{U^2}{r_L}$, kust $r_L = \frac{U^2}{P} = 5,76$ $\Omega$. Keldrini veetud kahe juhtsoone kogutakistus on $r_J =\frac{ 2 \rho l}{S} = 8,50$ $\Omega$. Juhtmed ja lamp on ühendatud jadamisi ning ahelat kogutakistusega $r_K = r_L + r_J$ läbib seega vool voolutugevusega $I = \frac{\varepsilon}{r_K}$, millest saame avaldada lambi põlemise võimsuse:
\begin{center}
$P_L = I^2 r_L = \frac{\varepsilon ^2 r_L}{(r_L + r_J)^2} = 4,8$ $W$. 
\end{center}
\fi
}