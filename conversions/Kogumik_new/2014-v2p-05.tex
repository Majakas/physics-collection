\ylDisplay{Parv} % Ülesande nimi
{Tundmatu autor} % Autor
{piirkonnavoor} % Voor
{2014} % Aasta
{P 5} % Ülesande nr.
{2} % Raskustase
{
% Teema: Mehaanika
\ifStatement
Tiigivees asetseb männipuidust parv pindalaga $S = 3$ $m^2$ . Parvel lamab inimene massiga $m = 70$ kg. Arvutage parve paksus, kui on teada, et parve ülemine äär on täpselt tasa veepinnaga. Vee ning männipuidu tihedused on vastavalt $\rho_v = 1000$ $kg/m^3$ ja $\rho_m = 400$ $kg/m3$.
\fi

\ifHint
Inimesele ja parvele mõjuvate raskusjõudude summa peab olema võrdne parvele mõjuva üleslükkejõuga.
\fi

\ifSolution
Ujumise tingimusest saame, et $m_i g + m_p g = \rho_v g V$
\newline
Parve mass tiheduse kaudu avaldub kujul $(m_i + \rho_p Sh)g = \rho_v g Sh$
\newline
Avaldades sealt $h$ saame
\begin{center}
$h= \frac{m}{(\rho_v -  \rho_p)S} = \frac{70kg}{(1000 kg/m^3 - 400 kg/m^3) \cdot 3 m^2} =$ $3,9$ cm
\end{center}
\fi
}