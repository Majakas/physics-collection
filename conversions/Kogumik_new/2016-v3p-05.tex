\ylDisplay{Kajalood} % Ülesande nimi
{Tundmatu autor} % Autor
{lõppvoor} % Voor
{2016} % Aasta
{P 5} % Ülesande nr.
{3} % Raskustase
{
% Teema: Võnkumine

\ifStatement
Vertikaalsuunas sukelduva allveelaeva kajalood kiirgab veekogu põhja suunas lühikesi heliimpulsse kestvusega $t_0$ sekundit. Põhjast tagasipeegeldunud registreeritud heliimpulsside kestvus on aga $t$ sekundit. Kui suur on allveelaeva sukeldumise kiirus $u$, kui heli levimise kiirus vees on $v$?
\fi

\ifHint
Heliimpulsi poolt läbitud vahemaa laeva suhtes saab leida, kui lahutada seisvast allveelaevast kiiratud heliimpulssi läbitud vahemaast sama ajaga allveelaeva poolt läbitud vahemaa.
\fi


\ifSolution
Seisvast allveelaevast kiiratud heliimpulss läbib ajavahemiku $t_0$ jooksul vahemaa $s = vt_0$. Kui allveelaev sukeldub, läbib allveelaev sama ajavahemiku jooksul vahemaa $s_1=ut_0$, seega heliimpulsi poolt läbitud vahemaaks kujuneb
\begin{center}
$s_2 = s - s_1 = vt_0 - ut_0$
\end{center}
Põhjast tagasipeegelunud heliimpulss liigub allveelaeva suhtes kiiruega $v + u$, seega registreerib vastuvõtja heliimpulsi ajavahemiku $t$ jooksul, mis võrdub 
\begin{center}
$t = \frac{vt_0 - ut_0}{v + u}$
\end{center}
Siit saame laeva laskumiskiiruseks
\begin{center}
$u = \frac{v(t_0 - t)}{t_0 + t}$
\end{center}
\fi
}
