\ylDisplay{Kohtumispaik} % Ülesande nimi
{EFO žürii} % Autor
{lõppvoor} % Voor
{2020} % Aasta
{P 2} % Ülesande nr.
{2} % Raskustase
{
% Teema: Mehaanika

\ifStatement
Kahe linna vaheline kaugus $s = 180$ km. Linnast $A$ väljub auto linna $B$ suunas. Auto keskmine kiirus terve teekonna ulatuses $v_1 = 75$ km/h. Auto jõuab linna $B$ ning $1,5$ tunni möödudes väljub linnast $B$ tagasi linna $A$ suunas. Ka tagasiteel on auto keskmine kiirus $75$ km/h. Pool tundi pärast esimese auto väljumist linnast $A$ väljub linnast $B$ teine auto linna $A$ suunas. See auto liigub terve tee keskmise kiirusega $v_2 = 80$ km/h. Viibinud $2$ tundi linnas $A$ suundub teine auto tagasi koju liikudes jälle keskmise kiirusega $v_2 = 80$ $km/h$. Kui kaugel linnast $A$ autod kohtuvad autod tagasiteel? 
\fi

\ifHint
Ülesande lahendamise lähtuda sellest, et kohtumise hetkeks on mõlemad autod olnud liikvel koos pausidega võrdse aja. Mõlema auto liikumisvõrranditest saab avaldada aja ning panna need võrduma.
\fi

\ifSolution
Kirjutame üles autode liikumisvõrrandid. Kohutmiskoht asub linnast $A$ kaugusel $x$, sinna jõudmiseks on mõlemad autod teel olnud aja $t$.
\begin{center}
$\frac{s}{v_1} + t_3 + \frac{s - x}{v_1} = t_5 + \frac{s}{v_2} + t_4 + \frac{x}{v_2}$.
\end{center}
Asendades tähised arvudega ja lahendades võrrandi, saame, et autod kohutvad $x = 60$ km linnast $A$.
\fi
}
