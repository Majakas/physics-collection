\ylDisplay{Võidusõiduautod} % Ülesande nimi
{Tundmatu autor} % Autor
{piirkonnavoor} % Voor
{2012} % Aasta
{P 6} % Ülesande nr.
{1} % Raskustase
{
% Teema: Mehaanika
\ifStatement
Võidusõiduauto keskmine kiirus ringrajal peetud treeningu jooksul oli $v = 50$ $m/s$. Kui arvutati peale esimese ja teise kõikide ulejäänud sõidetud ringide keskmine kiirus, leiti, et see oli täpselt sama, $50$ $m/s$. Esimese ringi läbimiseks kulus aega $t_1 = 107$ $s$. Kui palju aega kulus teise ringi läbimiseks? Ühe ringi pikkus oli $l = 5150$ $m$.
\fi

\ifHint
Keskmine kiirus on võrdne kogu teepikkuse ja kogu aja jagatisega.
\fi

\ifSolution
Paneme tähele, et võidusõiduauto keskmine kiirus esimese kahe ringi jooksul peab samuti olema v$ = 50$ $m/s$, ehk siis 
\begin{center}
$v = \frac{2l}{(t_1 + t_2)}$. Sellest saame, et $t_2 = \frac{2l}{v} - t_1 $, $t2 = 99$ $s$.
\end{center}
\fi
}
