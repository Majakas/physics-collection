\ylDisplay{Kaks matkaselli} % Ülesande nimi
{Koit Timpmann} % Autor
{lõppvoor} % Voor
{2018} % Aasta
{P 2} % Ülesande nr.
{2} % Raskustase
{
% Teema: Mehaanika

\ifStatement
Kaks matkaselli pidid jõudma võimalikult kiiresti ja üheaegselt $s = 40$ km kaugusel olevasse laagrisse. Kuna neil oli kahe peale ainult üks äbarik jalgratas, otsustasid nad, et sõidavad jalgrattaga vaheldumisi. Kui kaua „igavles” ratas tee ääres, kui matkasellid jooksid kiirusega $v_1 = 8$ km/h ja sõitsid rattaga kiirusega $v_2 = 15$ km/h?
\fi


\ifHint
Et jõuda laagrisse üheaegselt, peab kumbki matkasell pool teed läbima joostes ja pool jalgrattal.
\fi

\ifSolution
Et jõuda laagrisse üheaegselt, peab kumbki matkasell pool teed läbima joostes ja pool jalgrattal. Seega on aeg, mis kulub tee läbimiseks 
\begin{center}
$t = \frac{s}{2v_1} + \frac{s}{2v_2} = 230$ min.
\end{center}
Ratas seisis tee ääres
\begin{center}
$t_{ratas} = t - t_2$, kus $t_2 = \frac{s}{v_2} = 160$ min
\end{center}
Ratas "igavles" tee ääres $t_{ratas}$ = 230min - 160min = 70min
\fi
}