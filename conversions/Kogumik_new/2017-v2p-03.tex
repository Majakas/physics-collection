\ylDisplay{Ujumine kanalis} % Ülesande nimi
{EFO žürii} % Autor
{piirkonnavoor} % Voor
{2017} % Aasta
{P 3} % Ülesande nr.
{2} % Raskustase
{
% Teema: Mehaanika

\ifStatement
Voolava veega kanalis on esimene pool kaks korda laiem kui teine pool. Kanali pikkus on $l$ ning sügavus on igal pool sama. Pärivoolu ujudes läbib Juku kanali esimese poole ajaga $t_1$ ning teise poole ajaga $t_2$. Milline on Juku ujumiskiirus $v$ seisvas vees?
\fi

\ifHint
Kuna kanal on teises pooles kaks korda kitsam, kuid sama sügav, siis voolab seal vesi kaks korda kiiremini.
\fi

\ifSolution
Olgu veevoolu kiirus kanali esimeses pooles $u$. Kuna kanal on teises pooles kaks korda kitsam, kuid sama sügav, siis voolab seal vesi kaks korda kiiremini, seega on veevoolu kiirus kanali teises pooles $2u$. Juku kiirus kanali esimeses pooles on $v + u$ ning teses pooles $v + 2u$. Kuna me teame, et Juku läbis kanali esimese poole ajaga $t_1$ ning teise poole ajaga $_2$, saame kirjutada seosed
\begin{center}
$t_1 = \frac{0,5l}{v + u}$ ja $t_2 = \frac{0,5 l}{v + 2u}$. 
\end{center}
Avaldades esimesest võrrandist $u$ ning asendades teise võrrandisse, saame
\begin{center}
$u = \frac{0,5l - t_1 v}{t_1}$.
\end{center}
\begin{center}
$v = \frac{l(2_t2 - t_1)}{2t_1t_2} = \frac{l}{t_1} - \frac{l}{2t_2}$.
\end{center}
\fi
}