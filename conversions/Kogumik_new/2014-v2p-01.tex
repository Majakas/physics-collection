\ylDisplay{Jõe ületamine} % Ülesande nimi
{Tundmatu autor} % Autor
{piirkonnavoor} % Voor
{2014} % Aasta
{P 1} % Ülesande nr.
{1} % Raskustase
{
% Teema: Mehaanika
\ifStatement
Paat sõitis üle jõe, mille laius oli $d = 120$ m, nii, et paadi siht oli kogu aeg risti jõega. Kui suur pidi olema paadi keskmine kiirus jõevoolu suhtes, kui on teada, et paadi maabumiskoht teisel kaldal asetses $s = 12$ m lähtekohast allavoolu? Vee voolukiirus jões oli $u = 0,8$ m/s.
\fi



\ifHint
Paadi liikumine koosneb kahest kompnendist: risti liikumine üle jõe ja allavoolu liikumine.
\fi

\ifSolution
Paat võtab osa kahest liikumisest: sõidab risti jõge ning liigub allavoolu. Jõe laius on $s = 120$ m; allavooolu liikumine $l = 12$ m.
\newline
Sõidu aeg $t = \frac{l}{v_j}$
\newline
Paadi kiirus $v_p = \frac {s}{t} = \frac {s v_j}{l}$
\newline
Pannes arvandmed asemele, saame
\begin{center}
$v_p = \frac{120 m \cdot 0.8 m/s}{12 m}$ $=8$ m/s
\end{center}
\fi
}