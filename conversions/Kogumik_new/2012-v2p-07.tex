\ylDisplay{Tünn} % Ülesande nimi
{Tundmatu autor} % Autor
{piirkonnavoor} % Voor
{2012} % Aasta
{P 7} % Ülesande nr.
{2} % Raskustase
{
% Teema: Mehaanika
\ifStatement
Vees ujuva tühja plekktünni ruumalast on $\frac{1}{10}$ vee sees. Pärast tünni täitmist tundmatu vedelikuga jääb tünn vee peale ujuma, kuid nüüd on vee sees $\frac{9}{10}$ tünni ruumalast. Kui suur on tünni valatud vedeliku tihedus? Vee tihedus on $1000 \frac{kg}{m3}$.
\fi


\ifHint
Tühjale tünnile ja vedelikuga täidetud tünnile mõjuvad raskusjõud peavad olema võrdsed nendele tünnidele mõjuva üleslükke jõuga. Vedeliku tihedus on avaldatav vedeliku massist, mis lisatakse tünni.
\fi

\ifSolution
Tühja tünni korral kehtib seos $mg = \frac {\rho_v V g}{10}$
\newline
Vedelikku täis tünni korral kehtib seos $(m + \rho V)g = \frac{9 \rho_v V g}{10}$
\newline
Taandades ruumala $V$ ja $g$ saame $\frac{\rho_v}{10} + \rho = \frac{9\rho_v}{10}$, millest $\rho = \frac{8}{10}\rho_v$.
\newline
Vastus $\rho = 800$ $\frac{kg}{m^3}$.
\fi
}