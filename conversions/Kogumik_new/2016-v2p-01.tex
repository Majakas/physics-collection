\ylDisplay{Kärbes peeglis} % Ülesande nimi
{Tundmatu autor} % Autor
{piirkonnavoor} % Voor
{2016} % Aasta
{P 1} % Ülesande nr.
{1} % Raskustase
{
% Teema: Valgusõpetus
\ifStatement
Kärbes lendab tasapeegli poole risti peegli pinnaga kiirusega $v$. Peegel liigub kulgevalt kärbse liikumisega samas suunas. Kui suure kiirusega $v'$ peaks liikuma peegel, et kärbse kujutis peeglis jääks liikumatuks?
\fi

\ifHint
Kui mingi ajavahemiku vältel liigub peegel kärbsest eemale kauguse $a$ võrra, eemaldub kärbse kujutis peeglis kärbsest kauguse $2a$ võrra.
\fi

\ifSolution
Kui mingi ajavahemiku vältel liigub peegel kärbsest eemale kauguse $a$ võrra, eemaldub kärbse kujutis peeglis kärbsest kauguse $2a$ võrra. Kui sama ajavahemiku vältel liigub ka kärbes peegliga samas suunas kauguse $2a$ võrra, tuleb kärbse kujutis kärbsele lähemale kauguse $2a$ võrra. Järelikult, et kärbse kujutis peeglis jääks liikumatuks peab peegel liikuma kärbsest eemale kärbse kiirusest kaks korda väiksema kiirusega ehk $v'=v/2$.
\fi
}