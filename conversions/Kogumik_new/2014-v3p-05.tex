\ylDisplay{Jääst klaas} % Ülesande nimi
{Tundmatu autor} % Autor
{lõppvoor} % Voor
{2014} % Aasta
{P 5} % Ülesande nr.
{3} % Raskustase
{
% Teema: Soojusõpetus

\ifStatement
Jääst klaasi massiga $m_k = 20$ g ning temperatuuriga $t_0 = 0^{\circ}$C kallatakse $m_A=200$ g vedelikku $A$ temperatuuriga  $t_A = 25^{\circ}$C. Mitme protsendine vedeliku $A$ vesilahus tekib klaasis pärast soojusvahetuse lõppemist? Jää sulamissoojus $\lambda_{jaa} = 330$ kJ/kg, vedeliku $A$ erisoojus $c_A = 2400 J/kg \cdot ^{\circ}C$.
\fi

\ifHint
Jääst klaas saab eneriat vedeliku $A$ jahtumisel eraldunud energia arvelt. Vabanenud energia läheb jääst klaasi sulamiseks.
\fi

\ifSolution
Jääst klaas saab eneriat vedeliku $A$ jahtumisel eraldunud energia arvelt.
\begin{center}
$Q = c_A m_A \triangle T = 12$ kJ
\end{center}
Vabanenud energia läheb jääst klaasi sulamiseks, seega sulanud vee mass on
\begin{center}
$m_v = \frac{Q}{\lambda_{jaa}} = 36,4$ g
\end{center}
Vedeliku $A$ protsent saadud lahuses on seega $p = \frac{m_a}{m_a + m_v} = 84,6 \%$
\fi
}