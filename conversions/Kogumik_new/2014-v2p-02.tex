\ylDisplay{Kujutis kumerläätsega} % Ülesande nimi
{Tundmatu autor} % Autor
{piirkonnavoor} % Voor
{2014} % Aasta
{P 2} % Ülesande nr.
{1} % Raskustase
{
% Teema: Valgusõpetus
\ifStatement
Kumerläätsega tekitatakse valgusallika kujutis. Kui valgusallikas asub punktis $A$, tekib kujutis punktis $B$. Kui aga valgusallikas paigutada punkti $B$ tekib kujutis punktis $C$. Kas punkt $C$ langeb kokku punktiga $A$? Põhjendage. Valgusallika asukoha muutmisega ei muutu läätse asukoht.
\fi
\ifHint
Ülesande lahendamiseks tuleb kirjeldada, kuidas tekivad kujutised kui ese asetada kuumerläätsel kaugemale kui läätse fookuskaugus ning kuidas siis kui asub läätsele lähemal kui fookuskaugus.
\fi
\ifSolution
Kasutades kiirte pööratavuse printsiipi võib kumerläätsega tekitatud tõelise kujutise korral eseme ja kujutise asukohad vahetada. Kumerläätsega tekitatud näiva kujutise puhul seda teha ei saa. Seega, kui ese asub läätsest kaugemal kui läätse fookuskaugus, langevad punktid $C$ ja $A$ kokku. Kui ese asub läätsele lähemal kui fookuskaugus, punktid $C$ ja $A$ kokku ei lange.
\fi
}
