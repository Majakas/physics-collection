\ylDisplay{Veejuga} % Ülesande nimi
{Jonatan Kalmus} % Autor
{lõppvoor} % Voor
{2018} % Aasta
{P 8} % Ülesande nr.
{3} % Raskustase
{
% Teema: Mehaanika

\ifStatement
Vertikaalsest kraanist voolab vesi välja algkiirusega $v_0$. Leidke, millisel kaugusel $h$ kraanist on veejoa läbimõõt poole väiksem kui kraanist väljudes. Raskuskiirendus on $g$.
\fi


\ifHint
Voolamisel kehtib massi jäävus ehk aja $\triangle t$ jooksul läbib pindu erinevaid veejoa ristlõikeid sama kogus vett. Setõttu peab muutuma veevoolu kiirus. Veevoolu kiiruse muutumise tingib potentsiaalse energia muutumine veejoa kõrguse vähenemisel.
\fi

\ifSolution
Olgu veejoa läbimõõt kraanist väljudes $D$ ning otsitaval kõrgusel $h$ seega $D/2$. Veejoa ristlõikepindalad on vastavalt 
\begin{center}
$S_0 = \pi \frac{D^2}{4}$ ning $Sh = \pi \frac{D^2}{16}$. 
\end{center}
Voolamisel kehtib massi jäävus ehk aja $\triangle t$ jooksul läbib pindu $S_0$ ja $S_h$ sama kogus vett, sest vastasel juhul hakkaks vesi kas pindade vahele kuhjuma või kaoks seal ajapikku ära, mis ei ole taolise voolamise puhul võimalik. Olgu kõrgusel $h$ veejoa kiirus $v_h$. Kuna vee tihedus jääb konstanseks, saab massi jäävuse asemel kirja panna ruumala jäävuse ehk pindu läbivad veekogused on võrdsed
\begin{center}
$S_0 v_0 \triangle t = S_h v_h \triangle t$. 
\end{center}
Siit 
\begin{center}
$v_h = v_0 \frac{S_0}{S_h} = 4v_0$.
\end{center}
Nüüd vaatleme väikest veekogust $\triangle m $. Kraanist väljudes on sellel kineetiline energia
\begin{center}
$E_{K0} = \frac{\triangle m v_0 ^2}{2}$
\end{center}
ning otsitaval kõrgusel $h$ kraanist all pool
\begin{center}
$E_{Kh} = \frac{\triangle mv_h ^2}{2}$.
\end{center}
Potentsiaalse energia erinevus nende kahe kõrguse vahel on 
\begin{center}
$E_{P0} = \triangle mgh$.
\end{center}
Kuna sisehõõrde ning õhutakistusega arvestama ei pea, saame kirja panna energia jäävuse:
\begin{center}
$E_{K0} + E_{K0} = E_{Kh}$
\end{center}
Asendades:
\begin{center}
$\frac{\triangle mv_0 ^2}{2} + \triangle mgh = \frac{\triangle mv_h ^2}{2}$
\end{center}
Siit saame ära taandada masssi ning avaldada otsitava kõrguse h:
\begin{center}
$h = \frac{v_h ^2 - v_0 ^2}{2g}$
\end{center}
Asendades eelnevalt leitud $v_h$:
\begin{center}
$h = \frac{(4v_0)v_0 ^2 - v_0 ^2}{2g} = 7.5 \frac{ v_0 ^2}{g}$
\end{center}
\fi
}