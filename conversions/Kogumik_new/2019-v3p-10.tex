\ylDisplay{Ühendatud anumad} % Ülesande nimi
{Kristjan Kuppart} % Autor
{lõppvoor} % Voor
{2019} % Aasta
{P 10} % Ülesande nr.
{3} % Raskustase
{
% Teema: Mehaanika

\ifStatement
Pooleldi veega täidetud omavahel ühendatud anumate ühte harusse valatakse $h = 4$ $cm$ õli. Seejärel asetatakse sinna harusse ujuma ühtlase ristlõikepindalaga puidust klots kõrgusega $L = 10$ $cm$ ja tihedusega $\rho k = 0,5$ $g/cm^3$ . Kui palju tõuseb veetase teises harus võrreldes algse olukorraga, kui klotsi ja haru pindalade suhe $S_h / S_k = 2$? Õli tihedus $\rho_õ = 0,9$ $g/cm^3$ ja vee tihedus $\rho_v = 1$ $g/cm^3$ . Harude ristlõikepindalad on võrdsed ning õli ühest harust teise ei voola.
\fi

\ifHint
Süsteem on tasakaalus, kui teisse harusse lisandunud veesamba rõhk tasakaalustab õli ja puuklotsi lisamiseset tuleneva rõhu.
\fi

\ifSolution
Süsteem on tasakaalus, kui teisse harusse lisandunud veesamba rõhk tasakaalustab õli ja puuklotsi lisamiseset tuleneva rõhu
\begin{center}
$\triangle p = \frac{\rho_o gh + mg}{S_a}$,
\end{center}
kus $m$ on klotsi mass. Samuti peab arvestama, et kuna harude pindalad on võrdsed, siis väheneb veesamba kõrgus ja seetõttu ka veesamba rõhk esimeses harus sama palju, kui teises tõuseb. Kokkuvõttes saame võrrandi: 
\begin{center}
$\rho_v g \triangle h_v = \triangle p - \rho_v \triangle h_v$
\end{center}
\begin{center}
$\triangle h_v = \frac{\triangle p}{2 \rho _v g} = \frac{1}{2\rho_v}(\rho_o h + \rho_k L\frac{S_k}{S_a}) = 2,15$ cm
\end{center}
\fi
}