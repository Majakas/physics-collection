\ylDisplay{Tanker} % Ülesande nimi
{Koit Timpmann} % Autor
{lõppvoor} % Voor
{2018} % Aasta
{P 1} % Ülesande nr.
{2} % Raskustase
{
% Teema: Mehaanika

\ifStatement
Vaiksel merel lähenes sadamale $L = 300$ m pikkune tanker, mis sõitis ühtlase kiirusega $v_1$ sirgjoonelisel kursil. Laevale sõitis vastu samal sihil piirivalve kaater, mis liikus kiirusega $v_2 = 90$ km/h. Kaater sõitis laeva ninast sabani, pööras ümber ja sõitis sama teed tagasi. Kaatril kulus laeva kõrval edasi-tagasi sõitmiseks $t = 25$ s. Kui suure kiirusega $v_1$ sõitis tanker? Ümberpööramiseks kulunud aega ei ole vaja arvestada. 
\fi

\ifHint
Kui kaater liigub tankriga vastassuunas, siis tuleb kaatri ja tankri kiirused liita ning kui kaater liigub samas suunas, siis tuleb vastavad kiirused lahutada, et leida kaatri suhteline kiirus tenkri suhtes.
\fi

\ifSolution
Kaatril kulub edasi-tagasi sõitmiseks 
\begin{center}
$\frac{L}{v_2 + v_1} + \frac{L}{v_2 - v_1} = t$.
\end{center}
Liites murrud kokku saame 
\begin{center}
$\frac{2Lv_2}{v_2 ^2 - v_1 ^2} =$ $t$, 
\end{center}
millest
\begin{center}
$2Lv_2 = tv_1 ^2 - tv_1 ^2$ ja $v_1 = \sqrt{ v_2 ^2 - \frac{2Lv_2}{t}} = 5$ m/s $= 18$ km/h.
\end{center}
\fi
}