\ylDisplay{Ujuv anum} % Ülesande nimi
{Tundmatu autor} % Autor
{lõppvoor} % Voor
{2013} % Aasta
{P 1} % Ülesande nr.
{2} % Raskustase
{
% Teema: Mehaanika

\ifStatement
Risttahukakujulisse anumasse põhja pindalaga $S_1$ asetatakse ujuma väiksem risttahukakujuline anum põhjapindalaga $S_2$. Selle tulemusel tõusis veetase suures anumas kõrguse $\triangle h$ võrra. Siis hakati väiksemasse anumasse vett valama. Milline on minimaalne kaugus väiksemas anumas oleva vee pinna ja väikese anuma ääre vahel nii, et see veel ei upuks? 
\fi

\ifHint
Väiksem anum upub, kui ta vajub piisavalt sügavale, et vesi saaks hakata sisse voolama. Väikema anuma mass pluss anumas oleva vee mass peab olema võrdne üleslükkejõuga, et täita tasakaalutingimust.
\fi

\ifSolution
Anuma jaoks ilma veeta : $mg = \rho g V$, kus $m$ on anuma mass, $\rho$ on vee tihedus ja $V$ on anuma vee alla jääva osa ruumala. Kuna vesi on kokkusurumatu, siis selle sama ruumala võrra surutakse vett ka välja:
\begin{center}
$V = \triangle h (S_1 - S_2).$
\end{center}
Avaldame anuma massi:
\begin{center}
$m = \rho \triangle h (S_1 - S_2)$.
\end{center}
Väiksem anum upub, kui ta vajub piisavalt sügavale, et vesi saaks hakata sisse voolama. Tasakaalutingimuse saab kirja panna nii: 
\begin{center}
$mg + \rho V_s g = \rho V g$,
\end{center}
kus $V_s$ on anumas oleva vee ruumala ja V on kkogu anuma ruumala. Piirjuhul on meil $V_s = S_2 h $ ja $V = S_2 H$, kus $h$ on veetaseme kõrgus anumas ja $H$ on anuma kõrgus. Otsitav minimaalne kogus avaldub:
\begin{center}
$\triangle l = H - h$.
\end{center}
Eelnevat arvesse võttes saame: 
\begin{center}
$\rho (V - V_s) = m$
\end{center}
\begin{center}
$H - h = \frac{m}{\rho S_2}$
\end{center}
\begin{center}
$\triangle l = \frac{(S_1 - S_2)}{S_2} \triangle h$
\end{center}
\fi
}