\ylDisplay{Keha tihedus} % Ülesande nimi
{Autor} % Autor
{lõppvoor} % Voor
{2017} % Aasta
{P 5} % Ülesande nr.
{2} % Raskustase
{
% Teema: Mehaanika

\ifStatement
Dünamomeetri otsas rippuv keha sukeldati täielikult veeanumasse, mille ristlõike pindala on $S = 120$ $cm^2$. Selle tulemusena suurenes rõhk anuma põhjale $\triangle p = 500$ Pa võrra. Vette sukeldatud keha korral oli dünamomeetri näit $F = 9$ N. Leidke keha keskmine tihedus? Vee tihedus on $\rho = 1000$ $kg/m^3$.
\fi

\ifHint
Vette sukeldatud kehale mõjuva üleslükkejõu tõttu väheneb ka dünamomeeteri näit. Keha sukeldamisel vette surub keha välja oma ruumalaga võrdse koguse vett, mille tulemusel tõuseb veetase anumas ning suureneb rõhk anuma põhjale.
\fi

\ifSolution
Vette sukeldatud kehale mõjub üleslükkejõud, mille tulemusel dünamomeeteri näit väheneb. Üleslükkejõu valem on ${F_ü} = \rho_v g V$. 
\newline
Keha sukeldamisel vette surub keha välja oma ruumalaga võrdse koguse vett, mille tulemusel tõuseb veetase anumas ning suureneb rõhk anuma põhjale $\triangle p = \rho_v g \triangle h$.
\newline
Kui keha on sukeldatud vette on dünamomeetri näit võrdne kehale mõjuva raskusjõu ja üleslükkejõu vahega $F = mg - {F_ü}$, seega $mg = F + {F_ü}$.
\newline
Avaldame rõhu muutusest vee taseme tõusu $\triangle h$ ning keha ruumala V.
\begin{center}
$\triangle h = \frac{\triangle p}{\rho_v g}$ 
\end{center}
\begin{center}
$V = \frac{\triangle p}{\rho_v g} \approx 0,00061 m^3$
\end{center}
Kehale mõjuv üleslükkejõud on ${F_ü}$ $= \rho_v g V \approx 5,9$ N
\newline
Seega $m g = 9 N + 5,9 N = 15,9$ N, millest saame leida keha massi
\begin{center}
$m = F/g \approx 1.6$ kg
\end{center}
Keha aine tihedus on seega
\begin{center}
$\rho = \frac{m}{V} \approx 2600 kg/m^3$
\end{center}
\fi
}
