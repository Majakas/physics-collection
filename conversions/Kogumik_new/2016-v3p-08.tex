\ylDisplay{Pidurdus} % Ülesande nimi
{Autor} % Autor
{lõppvoor} % Voor
{2016} % Aasta
{P 8} % Ülesande nr.
{3} % Raskustase
{
% Teema: Mehaanika

\ifStatement
Auto sõidab teel, mille kõrguse muut teepikkuse kohta $k = $ $1/30$ . Ühesuguse algkiiruse ning pidurdusjõu korral jääb auto ülesmäge liikudes seisma $s_1 = 25$ m, allamäge liikudes aga $s_2 = 30$ m jooksul. Mis on auto algkiiruse $v$ väärtus?
\fi

\ifHint
Auto kineetiline energia kulub pidurdusjõu ületamiseks ning potentsiaalse energia muuduks. Kõrguse muut ning auto poolt läbitud teepikkus on omavahel seotud avaldisega $\triangle h_1 = ks_1$.
\fi

\ifSolution
Olgu auto mass $m$ ning pidurdusjõud $F$. Auto kineetiline energia kulub pidurdusjõu ületamiseks ning potentsiaalse energia muuduks. Ülesmäge sõites
\begin{center}
$\frac{mv^2}{2} = Fs_1 + mg\triangle h_1$.
\end{center}
Kõrguse muut ning auto poolt läbitud teepikkus on omavahel seotud avaldisega $\triangle h_1 = ks_1$, seega
\begin{center}
$\frac{mv^2}{2} = (F + mgk)s_1$.
\end{center}
Allamäge sõites kehtib analoogiliselt
\begin{center}
$\frac{mv^2}{2} = (F - mgk) s_2$.
\end{center}
Vasakute poolte võrdusest järeldub paremate poolte võrdsus
\begin{center}
$(F + mgk)s_1 = (F - mgk)s_2 \Rightarrow F = \frac{s_2 + s_1}{s_2 - s_1}mgk$.
\end{center}
Kiiruse jaoks saame energia jäävusest avaldise
\begin{center}
$v = \sqrt{2gks_1 (\frac{s_2 + s_1}{s_2 - s_1} + 1)} = \sqrt{4gk\frac{s_1 s_2}{s_2 - s_1}} = 14$ m/s = 50,4 km/h.
\end{center}
\fi
}