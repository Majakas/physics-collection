\ylDisplay{Veeboiler} % Ülesande nimi
{Tundmatu autor} % Autor
{piirkonnavoor} % Voor
{2016} % Aasta
{P 5} % Ülesande nr.
{2} % Raskustase
{
% Teema: Soojusõpetus

\ifStatement
Kui majja ei tule eraldi soojaveetoru, siis üks võimalus sooja vee saamiseks on elektrilise läbivooluboileri kasutamine. Boileri võimsus $N = 5,0$ kW ja kasutegur $\eta = 80\%$. Külmaveekraanist tuleva vee temperatuur $t_0 = 14$ $^{\circ}C,$ dušist väljuva vee temperatuur $t = 40 ^{\circ}C$ Mitu liitrit $40 ^{\circ}$ vett võib maksimaalselt dušist väljuda ühes minutis? Vee erisoojus on $c = 4200 J kg \cdot K$ ja tihedus $\rho = 1000 kg/m^3$.
\fi

\ifHint
Selleks, et leida millise koguse vett suudab boiler soojendada ühes minutis, tuleb leida millise kasuliku hulga energiat suudab boiler ühe minuti jooksul anda.
\fi


\ifSolution
Selleks, et soojendada kogus vett massiga $m$ algtemperatuurilt $t_0$ lõpptemperatuurile $t$, kulub soojushulk $Q = cm(t - t_0)$, kus $c$ on erisoojus. Boileri küttekehast eraldub minutis soojushulk $Q_1 = N \cdot 60 s$, millest veele läheb soojushulk $\eta Q_1 = Q$. Seega suudab boiler ühes minutis soojendada veekoguse, mille mass on 
\begin{center}
$m_1 = \frac{\eta N \cdot 60 s}{c(t - t_0)}$.
\end{center}
Kasutades massi ja ruumala seost $m = \rho V$, saame ühes minutis soojendatava vee ruumalaks
\begin{center}
$ V_1 = \frac{\eta N \cdot 60 s}{\rho c (t - t_0)}$.
\end{center}
Teame, et ühes kuupmeetris on $1000$ liitrit ja seega on ühes minutis soojendatava vee ruumala liitrites
\begin{center}
$V_1 = \frac{5000 W \cdot 0,8 \cdot 60 s}{1000 kg/m^3 \cdot 4200\frac{J}{kg \cdot K} \cdot (40 ^{\circ} C - 14 ^{\circ}C) } \cdot 1000 \frac{L}{m^3} \approx 2,2$ L.
\end{center}
\fi
}
