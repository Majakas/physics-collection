\ylDisplay{Jää sulamine} % Ülesande nimi
{Koit Timpmann} % Autor
{lõppvoor} % Voor
{2018} % Aasta
{P 6} % Ülesande nr.
{3} % Raskustase
{
% Teema: Soojusõpetus

\ifStatement
Jäätükk massiga $m = 100$ g ja temperatuuriga $t_0 = 0^{\circ}$C ümbritseti soojusisolatsiooni kihiga ja paigutati hüdraulilise pressi alla, kus sellele jäätükile avaldati rõhku $p = 550$ atm ($1$ atm on rõhk, mis on võrdne õhurõhuga normaaltingimustel). Leidke selles protsessis tekkiva vee mass, kui on teada, et jää sulamistemperatuuri alanemine on võrdeline jääle avaldatud rõhuga ning rõhu suurenemisel $\triangle p = 138$ atm alaneb jää sulamistemperatuur $\triangle t = 1^{\circ}$C võrra. Jää erisoojus $c = 2100 J/(kg \cdot ^{\circ}C)$ ja sulamissoojus on $\lambda = 330$ kJ/kg.
\fi

\ifHint
Esmalt tuleb leida jää sulamistemperatuur antud rõhul. Teiseks saame leida tekkiva vee massi võttes arvesse, et jää saab sulada vabanenud soojushulga arvelt.
\fi

\ifSolution
Leiame jää sulamistemperatuuri rõhul
\begin{center}
$t_1 = \frac{p}{\cfrac{\triangle p}{\triangle t}} = -4 ^{\circ}$C.
\end{center}
Jää saab sulada vabanenud soojushulga arvelt, seega
\begin{center}
$\lambda m_{sulanud} = cm_j (t_0 - t_1)$ $\Rightarrow$ $m_{sulanud} = 2,5$ g.
\end{center}
\fi
}