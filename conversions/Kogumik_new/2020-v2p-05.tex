\ylDisplay{Saun} % Ülesande nimi
{Richard Luhtaru} % Autor
{piirkonnavoor} % Voor
{2020} % Aasta
{P 5} % Ülesande nr.
{2} % Raskustase
{
% Teema: Soojusõpetus

\ifStatement
Juhan ja Peeter on saunas ning Juhan viskab kuumale kivikerisele külma vett temperatuuriga $10^{\circ}C$. Peeter väidab, et Juhan jahutab kerise niimoodi ära ja ütleb Juhanile, et ta viskaks külma vee asemel kuuma vett temperatuuriga $60 ^{\circ}C$. Juhan aga väidab vastu, et külma ja kuuma vee kasutamisel ei ole erilist vahet (kerise jahtumise erinevus on väiksem kui $10\%$). Kui palju väheneb kerise temperatuur kummalgi juhul, kui visata sinna $V = 200$ $cm^3$ vett? Kas Juhanil on õigus? Vee tihedus $\rho = 1000$ $kg/m^3$ , erisoojus $c_v = 4200$  $\frac{J}{kg \cdot ^{\circ}C}$ ja aurustumissoojus $L = 2300$ $kJ/kg$. Kerisekivide erisoojus $c_k = 700$ $\frac{J}{kg \cdot ^{\circ}C}$ ja kogumass $M = 100$ $kg$. Võib eeldada, et keris on piisavalt kuum ja kogu vesi aurustub ära.
\fi

\ifHint
Kerisekivide jahtumsiel eralduv soojushulk kulub vee soojendamiseks ja aurustamiseks. Sellest võrdsusest saame avaldada kerisekivide temperatuuride muudu.
\fi

\ifSolution
Vee mass on $m = \rho V$ ja seega vee soojendamiseks ja aurustamiseks kuluv soojushulk on
\begin{center}
$ Q = cv \rho V (T_k + T_0) + L \rho V$,
\end{center}
kus $T_k = 100^{\circ}C$ on keemistemperatuur ja $T_0$ on vee algtemperatuur. Kui kerise temperatuur väheneb $\triangle T$ võrra ($\triangle T$ on temperatuuri muutuse absoluutväärtus), siis kerise poolt antud soojushulk on
\begin{center}
$Q = c_k M \triangle T$.
\end{center}
Võrdsustades seosed saame
\begin{center}
$\triangle T = \frac{\rho V (c_v (T_k - T_0) + L)}{c_k M}$.
\end{center}
Asendades sisse antud väärtused, saame 
\begin{center}
$\triangle T_{külm} \approx 7,65 ^{\circ}$C
\end{center}
\begin{center}
$\triangle T_{kuum} \approx 7,05 ^{\circ}$C
\end{center}
Kuna $\frac{7,65 - 7,05}{7,05} \approx 0,085$ on väiksem kuna $10\%$, siis Juhanil on õigus.
\fi
}