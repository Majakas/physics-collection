\ylDisplay{Film} % Ülesande nimi
{EFO žürii} % Autor
{piirkonnavoor} % Voor
{2019} % Aasta
{P 10} % Ülesande nr.
{3} % Raskustase
{
% Teema: Võnkumine

\ifStatement
Filmis näidatakse, kuidas poiss sõidab jalgrattaga. Kui poiss hakkab sõitma, veerevad rattad õiget pidi. Kiiruse kasvades paistavad rattad pöörlevat tagurpidi. Veel suurema kiiruse $v = v_0$ puhul näib, nagu ei pöörleks rattad üldse. Leidke kiirus $v_0$, kui on teada, et ratta ümbermõõt on $p = 2,5$ m ning rattal on $N = 36$ kodarat. Filmis vahetuvad kaadrid sagedusega $f = 24$ Hz (kaadrit sekundis).
\fi

\ifHint
Ratas näib seisvat, kui järgmise kaadri ajaks on järgmine kodar jõudnud sama koha peale, kus eelmise kaadri ajal oli eelmine kodar.
\fi

\ifSolution
Ratas näib seisvat, kui järgmise kaadri ajaks on järgmine kodar jõudnud sama koha peale, kus eelmise kaadri ajal oli eelmine kodar. Kahe kaadri vahelise ajavahemiku $t_1 = \frac{1}{f}$ jooksul pöördub ratas ühe kodara võrra edasi ning $N$ korda pikema aja $t_N = \frac{N}{f}$ jooksul teeb ta täispöörde. Täispöördega liigub ratas edasi vahemaa $s$, seega on ratta kiirus
\begin{center}
$v_0 = \frac{sf}{N}$,
\end{center}
ehk
\begin{center}
$v_0 = \frac{2,5 \cdot 24}{36} \approx 1,7 m/s =$ $6$ km/h.
\end{center}
Pilt kordub kui jalgratta kiirus on $v = nv_0$, kus $n$ on täisarv.
\fi
}