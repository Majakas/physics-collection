\ylDisplay{Bussid} % Ülesande nimi
{Tundmatu autor} % Autor
{lõppvoor} % Voor
{2013} % Aasta
{P 3} % Ülesande nr.
{2} % Raskustase
{
% Teema: Mehaanika

\ifStatement
Jalgrattur Rein tegi maantee ääres trenni. Mööda teed tulid talle vastu bussid, mis lahkusid algpeatusest iga $15$ minuti tagant. Vähemalt mitu bussi tuli Reinule treeningu jooksul vastu, kui ta sõitis $30$ $km/h$ ja läbis 120 km? Busside sõidukiirus oli $90$ km/h.
\fi

\ifHint
Leia kahe järjestikuse bussi vahekaugus ning jalgratturi suhteline kiirus temale vastu liikuvate busside suhtes ning sealt saad leida, millise ajavahemiku tagant tuli talle buss vastu.
\fi

\ifSolution
Kahe järjestikuse bussi vahekaugus on $90$ km/h $\cdot$ $0,25$ h = $22,5$ km. Jalgratturi taustsüsteemis sõitsid bussid talle vastu kiirusega $120$ km/h, ehk Rein nägi bussi iga $\frac{22,5}{120} = \frac{3}{16}$ tunni järel. Trenn kestis $4h$, seega treeningu jooksul tuli Reinule vastu $\frac{4 \cdot 16}{3} \approx 21,33$ $\Rightarrow$ vähemalt $21$ bussi.
\fi
}