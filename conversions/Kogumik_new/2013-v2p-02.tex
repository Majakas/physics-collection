\ylDisplay{Palli soojenemine} % Ülesande nimi
{Tundmatu autor} % Autor
{piirkonnavoor} % Voor
{2013} % Aasta
{P 2} % Ülesande nr.
{1} % Raskustase
{
% Teema: Soojusõpetus

\ifStatement
Kui palju soojeneb $100$ grammine kummist pall (erisoojusega $c = 1400 J/kg^{\circ}C$), mis kukub lauale $4$ meetri kõrguselt ning põrkab tagasi $230$ $cm$ kõrgusele? Eeldada, et pall saab $90\%$ soojushulgast.
\fi

\ifHint
Energia, mis põrkel vabaneb ja muutub soojuseks, saab leida  potentisaalsete energiate vahest alguses ja pärast põrget.
\fi

\ifSolution
Energia, mis põrkel vabaneb, saame potentisaalsete energiate vahest alguses ja pärast põrget:
\newline
$E = mgh_1 - mgh_2 = 1,666$ J.
\newline
$90\%$ sellest energiast läheb palli soojendamiseks $Q_{pall} = 1,50$ J.
\newline
Palli soojenemise saame leida valemist $Q = cm\triangle T$.
\newline
$\triangle T = \frac{Q}{cm} = 0.0107 ^{\circ}C \approx 0,01 ^{\circ}$C.
\fi
}