\setAuthor{Erkki Tempel}
\setRound{piirkonnavoor}
\setYear{2014}
\setNumber{G 1}
\setDifficulty{1}
\setTopic{Dünaamika}

\prob{Lendav pudel}
Pooleliitrises pudelis, mille põhja on tehtud \SI{0.4}{V} väike auk pindalaga $S$, on $m$ grammi vett. Pudelil keeratakse kork pealt ning pudel visatakse õhku algkiirusega $v$. Kui kiiresti voolab vesi pudeli põhjas olevast august välja siis, kui pudel veel üles liigub? Kui kiiresti voolab vesi august välja sel hetkel, kui pudel alla kukub? Põhjendage.

\hint
Nii pudelile kui veele mõjuvad täpselt samad jõud ning neid visatakse sama algkiirusega.

\solu
Mõlemat juhtu, millal pudel liigub üles ning pudel liigub alla, võib vaadelda kui vabalangemist. Kuna pudelile ja veele mõjuvad jõud on vabalangemise korral samasugused, siis vesi ei voola pudelist välja kummalgi juhul. Seega vee väljavoolu kiirus on \SI{0}{m/s}.

\probeng{Flying bottle}
A hole with an area of $S$ has been made in the bottom of a half-liter bottle. The bottle has $m$ grams of water in it. The lid of the bottle is removed and the bottle is thrown in the air with a starting speed of $v$. How fast is the water flowing out of the hole when the bottle is still moving up? How fast is it flowing out when the bottle is falling down? Explain.

\hinteng
The exact same forces apply for both the bottle and the water, they are also thrown with the same initial speed.

\solueng
Both of the cases, when the bottle moves up and the bottle moves down, can be looked at as a free fall. Because the forces applied to the bottle and the water are the same for free fall then the water does not flow out of the bottle for either case. Thus, the speed of the water flowing out is $\SI{0}{m/s}$.
\probend