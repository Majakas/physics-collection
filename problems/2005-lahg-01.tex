\setAuthor{Tundmatu autor}
\setRound{lahtine}
\setYear{2005}
\setNumber{G 1}
\setDifficulty{1}
\setTopic{Vedelike mehaanika}

\prob{Vedelik}
Ühendatud silindrilistesse anumatesse diameetritega $d_1$ ja $d_2$ on valatud vedelik tihedusega $\rho$. Kui palju tõuseb vedeliku tase anumates, kui ühte anumasse pannakse ujuma vedeliku tihedusest väiksema tihedusega keha massiga $m$?

\hint
Kuna anumad on ühendatud, on rõhud mõlemas anumas samal kõrgusel samad ja anumate vedeliku tasemed võrdsed.

\solu
Kuna anumad on ühendatud ning vedelikes antakse rõhk igas suunas võrdselt edasi, on mõlemas anumas rõhud samal kõrgusel võrdsed. Järelikult võib antud anumaid vaadelda ühe anumana, mille pindala on
\[
S=\pi\left(\frac{d_{1}}{2}\right)^{2}+\pi\left(\frac{d_{2}}{2}\right)^{2}=\frac{\pi\left(d_{1}^{2}+d_{2}^{2}\right)}{4}.
\]
Väljasurutud vedeliku mass on võrdne lisatava ujuva keha massiga. Seega on lisanduva vedeliku ruumala $\Delta V = m/\rho$ ning kõrguse muut avaldub kui
\[
\Delta h=\frac{\Delta V}{S}=\frac{4 m}{\pi \rho\left(d_{1}^{2}+d_{2}^{2}\right)}.
\]
\probend