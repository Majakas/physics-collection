\setAuthor{Siim Ainsaar}
\setRound{lõppvoor}
\setYear{2008}
\setNumber{G 2}
\setDifficulty{2}
\setTopic{Dünaamika}

\prob{Pingpong}
Pingpongipall kukutatakse kõrguselt $h$ horisontaalsele lauale. Igal põrkel kahaneb palli energia $k$ korda. Leidke palli lahtilaskmisest seismajäämiseni kuluv aeg $t$. Vabalangemise kiirendus on $g$.

\hint
Palli koguenergia vastab geomeetrilisele jadale, sest iga järgmise põrke energia on eelnevast $k$ korda väiksem. Saame sarnase jada, kui avaldame kahe järjestikuse põrke vahelise aja summaarse energia kaudu.

\solu
Kui pall tõuseb kahe järjestikuse põrke vahel (pärast i-ndat põrget) kõrgusele $h_i$, saame nende põrgete vahelise ajavahemiku $t_i$:
\[
h_{i}=\frac{g\left(\frac{t_{i}}{2}\right)^{2}}{2} \Longrightarrow t_{i}=2 \sqrt{\frac{2 h_{i}}{g}}.
\]
Igas lennu haripunktis on palli kiirus ja ka kineetiline energia null ning koguenergia $E_i$ potentsiaalne. Kui palli mass on $m$, siis $E_i = mgh_i$ ja on võrdeline haripunkti kõrgusega $h_i$. Seega ka haripunkti kõrgus kahaneb pärast igat põrget $k$ korda: $h_{i+1} = \frac{h_i}{k}$. Ilmselt selles seoses $h_0 = h$.

Kukkumise aeg enne esimest põrget:
\[
t_0 = \frac{2h}{g}.
\]
Nii saamegi koguaja:
\[
\begin{aligned} 
	t &=t_{0}+t_{1}+t_{2}+\cdots=\sqrt{\frac{2 h}{g}}+2 \sqrt{\frac{2 h}{g k}} + 2 \sqrt{\frac{2 h}{g k^{2}}}+2 \sqrt{\frac{2 h}{g k^{3}}}+\cdots=\\ &=\sqrt{\frac{2 h}{g}}+2 \sqrt{\frac{2 h}{g k}}\left[1+\left(\frac{1}{\sqrt{k}}\right)^{1}+\left(\frac{1}{\sqrt{k}}\right)^{2}+\left(\frac{1}{\sqrt{k}}\right)^{3}+\ldots\right]=\\ &=\sqrt{\frac{2 h}{g}}+\sqrt{\frac{2 h}{g k}} \frac{1}{1-\frac{1}{\sqrt{k}}}=\sqrt{\frac{2 h}{g}}+2 \sqrt{\frac{2 h}{g}} \frac{1}{\sqrt{k}-1}=\\ &=\sqrt{\frac{2 h}{g}}\left(1+\frac{2}{\sqrt{k}-1}\right)=\frac{\sqrt{k}+1}{\sqrt{k}-1} \sqrt{\frac{2 h}{g}.} 
\end{aligned}
\]
\probend