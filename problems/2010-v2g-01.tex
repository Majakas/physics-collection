\ylDisplay{Eiffeli torn} % Ülesande nimi
{Aigar Vaigu} % Autor
{piirkonnavoor} % Voor
{2010} % Aasta
{G 1} % Ülesande nr.
{1} % Raskustase
{
% Teema: Dünaamika
\ifStatement
Eiffeli torni ülemiselt vaateplatvormilt (kõrgus maapinnast
$h=\SI{273}{m}$) lastakse kukkuda raudkuulil. Täpselt $t=3$ sekundi pärast kukutatakse
veel üks raudkuul. Kui suur on raudkuulide suurim
kiiruste vahe langemisel? Kui suur on ajavahemik kuulide maapinnale jõudmiste
vahel? Raskuskiirendus $g=\SI{9.8}{m/s^2}$. Katse käigus ükski külastaja viga ei saanud.
\fi


\ifHint
Kukkumise käigus kiirenevad mõlemad kuulid sama kiirusega. Seega on nende suhteline kiirus muutumatu.
\fi


\ifSolution
Kontrollime, kui kaua kukub raudkuul ülemiselt vaateplatvormilt $h=\SI{273}{m}$ maapinnale.
\[h=\frac{gt^2}{2} \Rightarrow t=\sqrt{\frac{2h}{g}}\approx \SI{7.5}{s}.\]
Alates hetkest, kui mõlemad kuulid langevad, on nende suhteline kiirus muutumatu, sest mõlemad kuulid on siis vabalt langevas taustsüsteemis paigal.
Leiame esimese kuuli kiiruse teise kuuli kukutamise hetkel.
\[v=gt\approx \SI{29.4}{m/s}.\]
Ajavahemik kuulide maapinnale jõudmisel on sama mis kuulide kukutamiselgi, ehk $t=\SI{3}{s}$.
\fi
}