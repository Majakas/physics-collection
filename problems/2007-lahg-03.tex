\ylDisplay{Allveelaev} % Ülesande nimi
{Tundmatu autor} % Autor
{lahtine} % Voor
{2007} % Aasta
{G 3} % Ülesande nr.
{1} % Raskustase
{
% Teema: Gaasid
\ifStatement
Uppunud allveelaevadest on inimesed mõnikord pääsenud avades esialgu alumised ventiilid (mida mööda vesi sisse tungib), seejärel ülemise luugi ning siis ise koos õhumulliga veepinnale tõustes. Kui suur osa $k$ laeva ruumalast polnud täidetud veega peale ventiilide avamist, kui laev asus sugavusel $h = \SI{42}{m}$? Merevee tihedus $\rho = \SI{1,03e3}{kg/m^3}$. Õhu rõhk laevas alghetkel $p_0 = \SI{0,1}{MPa}$. Võite lugeda, et vee sisse laskmise käigus õhu temperatuur laevas ei muutu (tänu soojusvahetusele ümbritseva veega).
\fi


\ifHint
Ideaalse gaasi olekuvõrrandist on võimalik leida seos esialgse ja pärastise õhu ruumala vahel.
\fi


\ifSolution
Kuna temperatuur on jääv, kehtib seos $pV = \const$. Olgu laeva ruumala $V$ ning õhu ruumala laevas peale ventiilide avamist $v$. Siis
\[
p_{0} V=\left(p_{0}+\rho g h\right) v,
\]
\[
k=\frac{v}{V}=\frac{p_{0}}{p_{0}+\rho g h}=\frac{\num{0,1e6}}{\num{0,1e6}+\num{1,03e3} \cdot 9,8 \cdot 42} \approx \num{0,19}.
\]
\fi
}