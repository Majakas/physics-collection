\setAuthor{Tundmatu autor}
\setRound{piirkonnavoor}
\setYear{2011}
\setNumber{G 3}
\setDifficulty{1}
\setTopic{Termodünaamika}

\prob{Küttesüsteem}
Küttesüsteem täidetakse $t_1=\SI{10}{\degreeCelsius}$ temperatuuriga veega. Kui palju peab paisupaagis olema vaba ruumi, et kütmisel avatud paisupaagist vesi välja ei voolaks? Küttesüsteemis on $V_1=\SI{250}{}$ liitrit vett ja tööolukorras on selle keskmine temperatuur $t_2=\SI{63}{\degreeCelsius}$. Vee ruumpaisumistegur on $\beta=3\times 10^{-4}K^{-1}$. Vedeliku ruumala mingil temperatuuril avaldub kujul $V=V_{0}(1+\beta t)$, kus $t$ on vedeliku temperatuur Celsiuse kraadides, ning $V_{0}$ on vedeliku ruumala temperatuuril \SI{0}{\degreeCelsius}.

\hint
Küttesüsteemis oleva vee ruumala esialgses olukorras ja töörežiimis on avaldatavad ülesandes antud valemiga.

\solu
Paisumisel lisanduva ruumala jaoks peab olema paisupaagis piisavalt lisaruumi. Vajalik ruumala on
\[
V-V_{1}=V_{0}\left(1+\beta t_{2}\right)-V_{0}\left(1+\beta t_{1}\right)=\frac{V_{1} \beta}{1+\beta t_{1}}\left(t_{2}-t_{1}\right).
\]
Vajalik vaba ruum paisupaagis on seega $V - V_1 \approx \SI{4,0}{l}$.
\probend