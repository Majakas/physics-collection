\setAuthor{Oleg Košik}
\setRound{lõppvoor}
\setYear{2006}
\setNumber{G 7}
\setDifficulty{6}
\setTopic{Termodünaamika}

\prob{Tuba}
Külma tõttu läks küttesüsteem rikki ja toatemperatuur hakkas langema. Ühel hetkel pandi tööle ajas muutumatu võimsusega töötav soojapuhur ning toatemperatuur hakkas taas tõusma. Graafikul on toodud toatemperatuuri sõltuvus ajast. Leidke toatemperatuur pika aja möödumisel. Protsessi vältel välistingimused ei muutunud. Seinte ja toas olevate esemete soojusmahtuvusega mitte arvestada. Soojusvahetuse kiirus väliskeskkonnaga ei ole võrdeline temperatuuride vahega.

\begin{center}
	\includegraphics[width=\linewidth]{2006-v3g-07-yl}
\end{center}

\hint
Temperatuuri kasvu või langemise kiirus on võrdeline tuppa siseneva summaarse soojusliku võimsusega. Sellele vastab graafiku puutuja tõus. Soojapuhuri sisse lülitamise ajahetkel oli graafiku tõusu muut võrdeline soojapuhuri võimsusega.

\solu
Temperatuuri kasvu või langemise kiirus on võrdeline tuppa siseneva summaarse soojusliku võimsusega. Sellele vastab graafiku puutuja tõus. Ajahetkeni $t = \SI{1100}{min}$ läheb soojus toast välja, peale seda lisandub kaotatavale võimsusele soojapuhuri võimsus. Soojapuhuri võimsusele vastab puutuja tõusu muut mingil temperatuuril. Näiteks hetke $t = \SI{1100}{min}$ jaoks saame, et puutuja tõusu muut on (ligikaudu) $8/1000 + \num{10,2}/400 \approx \SI{0,036}{\degreeCelsius/min}$. Graafiku abil leiame nüüd temperatuuri, mille korral soojuskadude võimsus võrdub soojapuhuri võimsusega. Selle jaoks võib kasutada joonlauda tõusuga \SI{0,036}{\degreeCelsius/min} ning määrata punkt graafikus, mis puutub antud sirget. Graafiku esimeses osas on selline punkt umbes temperatuuril $T = \SI{20}{\degreeCelsius}$. Seega toatemperatuur pika aja möödumisel on $T = \SI{20}{\degreeCelsius}$.
\probend