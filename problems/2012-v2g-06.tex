\ylDisplay{Kuumaõhupall} % Ülesande nimi
{Andreas Valdmann} % Autor
{piirkonnavoor} % Voor
{2012} % Aasta
{G 6} % Ülesande nr.
{5} % Raskustase
{
% Teema: Gaasid
\ifStatement
Mis temperatuurile tuleb kuumaõhupalli sees õhk kütta, et õhupall lendu tõuseks?
Välisõhu temperatuur $t=\SI{20}{\celsius}$, õhupalli ruumala $V=\SI{3000}{m^3}$ ja ei
muutu. Õhupalli kesta ja laadungi kogumass $m=\SI{700}{kg}$ ja õhu tihedus
20 kraadi juures $\rho_{20}=\SI{1,2}{kg/m^3}$.
\fi


\ifHint
Õhupalli lendu tõusmise piiril peab kuumaõhupalli keskmine tihedus olema võrdne välisõhu tihedusega.
\fi


\ifSolution
Õhupallile mõjub ühes suunas üleslükkejõud, mis võrdub väljasurutud külma õhu 
kaaluga $\rho_{20} V g$. Vastassuunas mõjub raskusjõud nii kestale ja laadungile ($mg$) kui ka kuumale õhule õhupalli sees ($\rho_k V g$, kus $\rho_k$ on kuuma õhu tihedus). Õhupall tõuseb lendu, kui üleslükkejõud saab võrdseks raskusjõuga:
\[ \rho_{20} V g=mg+\rho_k V g,\]
millest avaldub
\[ \rho_{20}=\rho_k-\frac{m}{V}.\]
Õhu tihedus väheneb temperatuuri kasvamisel vastavalt ideaalse gaasi olekuvõrrandile
\[ pV=\frac{m}{M}RT. \]
Avaldades õhu tiheduse, saame
\[ \rho=\frac{m}{V}=\frac{pM}{RT}.\]
Õhupall on alt lahti ja õhurõhk palli sees on võrdne välisõhu rõhuga. Seetõttu  on ülaltoodud võrrandis $p$, $M$ ja $R$ konstantsed ning kahe erineva temperatuuri jaoks kirjapandud võrrandeid läbi jagades näeme, et
\[ \frac{\rho_{20}}{\rho_k}=\frac{T_k}{T_{20}}.\]
Avaldades $T_k$ ja kasutades lendutõusmise tingimust, saame:
\[ T_k=\frac{\rho_{20} T_{20}}{\rho_k}=\frac{\rho_{20} 
T_{20}}{\rho_{20}-\frac{m}{V}}=\frac{T_{20}}{1-\frac{m}{\rho_{20}V}}.\]
Ülaltoodud valemis tuleb kasutada absoluutset temperatuuri (kelvinites). Null 
kraadi Celsiuse skaalas on 273 K ja välisõhu temperatuur on seega 
\SI{20}{\celsius}=\SI{20}{K} + \SI{273}{K}=\SI{293}K. Nüüd saamegi välja arvutada temperatuuri  õhupalli sees, milleks on $T_k=\SI{364}K=\SI{91}\celsius$.
\fi


\ifEngStatement
% Problem name: hot air balloon
To what temperature should the air in a hot air balloon be heated so that the balloon would fly up? The outside air temperature is $t=\SI{20}{\celsius}$, the balloon's volume is $V=\SI{3000}{m^3}$ and it does not change. The total mass of the balloon's shell and load is $m=\SI{700}{kg}$ and the air density at 20 degrees is $\rho_{20}=\SI{1,2}{kg/m^3}$.
\fi


\ifEngHint
At the verge of the balloon's flight into air the average density of the hot air balloon must be equal to the density of the outer air.
\fi


\ifEngSolution
The balloon is applied with a buoyancy force from one direction, this force is equal to the weight of the cold air $\rho_{20} V g$ that is pressed out. From the opposite direction the gravity force is applied to both the shell and the load ($mg$) and also the hot air inside the balloon ($\rho_k V g$ where $\rho_k$ is the density of the hot air). The balloon starts to fly if the buoyancy force gets equal to the gravity force:
\[ \rho_{20} V g=mg+\rho_k V g,\] 
from which we can express
\[ \rho_{20}=\rho_k-\frac{m}{V}.\] 
According to the ideal gas law the air density decreases with the increase in temperature 
\[ pV=\frac{m}{M}RT. \] 
Expressing the air density we get
\[ \rho=\frac{m}{V}=\frac{pM}{RT}.\] 
The balloon is open at the bottom and the air pressure inside the ball is equal to the pressure of the outer air. Because of this $p$, $M$ and $R$ are constant in the equation above and dividing the equations which are written for two different temperatures we get
\[ \frac{\rho_{20}}{\rho_k}=\frac{T_k}{T_{20}}.\] 
Expressing $T_k$ and using the condition of hovering we get:
\[ T_k=\frac{\rho_{20} T_{20}}{\rho_k}=\frac{\rho_{20} 
T_{20}}{\rho_{20}-\frac{m}{V}}=\frac{T_{20}}{1-\frac{m}{\rho_{20}V}}.\] 
In the equation above we have to use absolute temperature (in kelvins). Zero degrees in Celsius is 273 K and the temperature of the outer air is therefore $\SI{20}{\celsius}=\SI{20}{K} + \SI{273}{K}=\SI{293}K$. Now we can calculate the temperature inside the balloon which is $T_k=\SI{364}K=\SI{91}\celsius$.
\fi
}