\setAuthor{Mihkel Kree}
\setRound{piirkonnavoor}
\setYear{2009}
\setNumber{G 2}
\setDifficulty{2}
\setTopic{Dünaamika}

\prob{Mürsk}
Kahurist välja lennanud mürsk (massiga $M$) laguneb oma lennutrajektoori kõrgeimas punktis mingi sisemise vedrumehanismi abil kaheks võrdseks pooleks (kumbki massiga $M/2$) nii, et üks osadest kukub mürsu senist trajektoori pidi liikudes täpselt kahurini tagasi. Kui kaugele kahurist maandub teine pool? Lagunemispunkti projektsioon maapinnale asub kahurist kaugusel $L$.

\hint
Vedrumehanismi vallandumisel muutub osa vedrudesse salvestatud potentsiaalsest energiast kineetiliseks energiaks --- seega kineetilise energia jäävus ei kehti. See-eest kehtib impulsi jäävuse seadus, sest lagunemise käigus ei mõju mürsule väliseid jõude (eeldusel, et mürsk laguneb hetkeliselt).

\solu
Tähistame mürsu kiiruse lagunemishetkel $v$-ga. Vahetult pärast lagunemist peab ühe poole kiirus olema samuti $v$, kuid vastassuunaline. Olgu teise osa kiirus sel hetkel $u$. Impulsi jäävuse tõttu
\[
Mv = \frac{M}{2}u - \frac{M}{2}v,
\]
millest $u=3v$. Vahetult pärast lagunemist kuulidel vertikaalne kiiruskomponent puudub, seetõttu võtab kummagi tüki langemine võrdselt aega. Teine
tükk maandub kahurist kaugusele $L + 3L = 4L$.
\probend