\setAuthor{Aigar Vaigu}
\setRound{piirkonnavoor}
\setYear{2010}
\setNumber{G 4}
\setDifficulty{3}
\setTopic{Dünaamika}

\prob{Vedru}
Raske tellis kukub poole meetri kõrguselt jäigale lühikesele vedrule. Põrge
on elastne ja tellis lendab peaaegu algsele kõrgusele tagasi. Kui
kõrgele maast kerkib vedru pärast põrget?

\hint
Tellise eemaldumise hetkel on vedru alumine ots paigal, aga ülemine ots liigub tellisega sama kiirusega üles.

\solu
Hetkel, mil tellis vedrust eemalduma hakkab, liigub vedru ülemine ots koos tellisega kiirusega $v$ üles. Tellise kiirus on piisav, et kerkida tagasi esialgsele kõrgusele $H\propto v^2$. Eeldame, et vedru on pikisuunas ühtlane, sellisel juhul liigub vedru massikese ülemise ja alumise otsa keskmise kiirusega ülesse. Niisiis, $v\idx{vedru}=v/2$. Seega tõuseb vedru massikese kõrgusele $h\propto v\idx{vedru}^2=v^2/4$, ehk
\[h=\frac{H}{4}.\]
\probend