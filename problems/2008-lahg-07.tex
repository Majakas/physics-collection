\setAuthor{Tundmatu autor}
\setRound{lahtine}
\setYear{2008}
\setNumber{G 7}
\setDifficulty{5}
\setTopic{Elektrostaatika}

\prob{Elektronkiir}
Kitsas elektronkiir läbib vaakumis tasaparalleelsete plaatide vahelise pilu ja langeb seejärel fluorestseeruvale ekraanile, mis asub plaatide ekraanipoolsemast servast kaugusel $l = \SI{15}{cm}$. Kui plaatidele antakse pinge $U = \SI{50}{V}$, nihkub helendav punkt ekraanil endisest asukohast kaugusele $s = \SI{21}{mm}$. Plaatidevaheline kaugus $d = \SI{18}{mm}$, plaatide mõõtmed elektronide liikumise suunas on $b = \SI{6}{cm}$. Milline on elektronide algkiirus plaatide vahele sattumisel? Elektroni laengu ja massi suhe $e/m \approx \SI{1,76e11}{C/kg}$.

\hint
Selleks, et siduda elektronide kiirust ekraani-sihilise nihkega tasub ülesanne jagada kaheks eraldi osaks: elektronide viibimine plaatide vahel ning plaatide ja ekraani vahelises ruumis. Esimeses osas mõjub elektronile ühtlane kiirendus, teises osas liigub elektron sirgjooneliselt.

\solu
Olgu $v_0$ elektronide algkiirus plaatide vahele sattumisel. Aeg, mille jooksul üks elektron viibib plaatide vahel, on $t = \frac{b}{v_o}$.

Plaatide vahel on elektriväli tugevusega $E = \frac{U}{d}$. Newtoni II seadusest $eE = ma$ leiame, et elektron liigub plaatide vahel kiirendusega $a = \frac{eU}{md}$. Läbides plaatide vahelise tee, kalduvad elektronid vahemaa $s_0$ esialgsest trajektoorist kõrvale, kus
\[
s_{0}=\frac{a t^{2}}{2}=\frac{U e b^{2}}{2 d m v_{0}^{2}}.
\]
Elektronide liikumise kiirus $v$ plaatide vahelisest ruumist väljudes koosneb kahest komponendist:\\
-- paralleelsest ekraaniga $v_{y}=a t=\frac{e U b}{m d v_{0}}$,\\
-- risti ekraaniga $v_x = v_0$.\\
Seega veedavad elektronid plaatidest ekraanini aja $t' = \frac{l}{v_x} = \frac{l}{v_0}$. Selle aja jooksul lisandub täiendav ekraaniga paralleelne nihe
\[
s^{\prime}=v_{y} t^{\prime}=\frac{e U b l}{m d v_{0}^{2}}.
\]
Kogu nihe on niisiis
\[
s=s_{0}+s^{\prime}=\frac{e U b}{m d v_{0}^{2}}\left(\frac{b}{2}+l\right).
\]
Seega
\[
v_{0}=\sqrt{\frac{e}{m} \frac{U b}{d s}\left(\frac{b}{2}+l\right)} \approx \SI{1,58e7}{m/s}.
\]
Sooritame kontrolli, kas elektroni nihe plaatide vahel on väiksem, kui plaatide vaheline kaugus:
\[
s_{0}=\frac{U e b^{2}}{2 d m v_{0}^{2}}=\frac{bs}{2l + b} = \SI{3.5}{mm} < d.
\]
\probend