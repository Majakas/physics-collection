\ylDisplay{Kaks kuuli} % Ülesande nimi
{Jaan Kalda} % Autor
{lõppvoor} % Voor
{2018} % Aasta
{G 10} % Ülesande nr.
{9} % Raskustase
{
% Teema: Elektrostaatika
\ifStatement
Kaks ühesugust metallkuuli raadiusega $R$ ja massiga $m$ on ühendatud peenikese terastraadiga pikkusega $L\gg R$. Piirkonnas $x\ge 0$ on elektriväli tugevusega $E$, mis on suunatud piki $x$-telge; piirkonnas $x< 0$ elektriväli puudub. Alghetkel on kuulid paigal ja üksteisest kaugusel $L$, nii et traat on pingul ning paralleelne $x$-teljega; ühe kuuli keskpunkt asub punktis $x=R$ ning teine kuul --- piirkonnas $x<0$. Visandage kvalitatiivselt kuulide kiiruse graafik sõltuvuses ajast (kvantitatiivset ajaskaalat ei ole vaja) ning leidke nende kiirus punkti $x=2L$ läbimisel. Terastraadi mahtuvus lugeda tühiselt väikeseks.
\fi


\ifHint
Elektriväli indutseerib kuulidele vastasmärgilised laengud, mis tagavad, et kuulide potentsiaalid on võrdsed.
\fi


\ifSolution
Elektriväli indutseerib kuulidele vastasmärgilised laengud, mis tagavad, et kuulide potentsiaalid on võrdsed. Olgu esimese kuuli koordinaat $z$; siis väline väli tekitab kuulide vahel potentsiaalide vahe $Ez$; see peab võrduma kuulidele indutseeritud laengute $\pm q$ tekitatud potentsiaalide vahega $kq/R-(-q)k/R=2kq/R$. Seega $q=EzR/2k$. Esimesele kuulile mõjub elektriväljas jõud $Eq=E^2zR/2k$; näeme, et see on võrdeline kaugusega ja toimib sarnaselt vedruga. Seega teeb summaarne jõud tööd $E^2z^2R/4k$. Kui ka teine kuul siseneb elektrivälja, siis muutub summaarne jõud hetkeliselt nulliks ning kuulid jätkavad liikumist konstantse kiirusega, mis on leitav energia jäävuse seadusest: $E^2z^2R/4k=2mv^2/2$, millest $v=\frac {Ez}{2}\sqrt{\frac{R}{km}}$. Seni, kui kuulid kiirenevad, on liikumisvõrrandiks $\ddot z=E^2z^2R/4km$, millest nii $z$ kui $v=\dot z$ on eksponentsiaalsed funktsioonid ajast. Seega on kiiruse graafikuks eksponentisaalselt kasvav kõver, mis läheb hüppeliselt üle kontsantseks funktsiooniks (hetkel, kui ka teine kuul siseneb elektrivälja).
\fi


\ifEngStatement
% Problem name: Two balls
Two identical metal balls of radius $R$ and mass $m$ are connected to each other with a thin steel wire of length $L\gg R$. In the region $x\ge 0$ there is an electric field of strength $E$ and it is directed along the $x$-axis. In the region $x< 0$ there is no electric field. At the initial moment the balls are still and are at the distance $L$ from each other so that the wire is taut and parallel to the $x$-axis. The center of one ball is at a point $x=R$, the other ball is in the region $x<0$. Qualitatively sketch a graph depicting the dependence of the balls’ speed on time (a quantitative time scale is not needed) and find their speed while passing through the point $x=2L$. The capacity of the steel wire is negligibly small.
\fi


\ifEngHint
The electric field induces opposite charges on the balls that ensure that the potentials of the balls are equal.
\fi


\ifEngSolution
The electric field induces the balls with charges opposite in sign that ensure that the potential of the balls are equal. Let the coordinate of the first ball be $z$; then the outer field creates a potential difference $Ez$ between the balls; this has to be equal to the potential difference $kq/R-(-q)k/R=2kq/R$ created by the induced charges $\pm q$ on the balls. Therefore $q=EzR/2k$. The first ball is applied with a force $Eq=E^2zR/2k$ in the electric field; we see that it is proportional to the distance and that it works similarly to a spring. Therefore the total force does the work $E^2z^2R/4k$. If the other ball also enters the electric field then the total force changes momentarily to zero and the balls continue to move with a constant velocity that can be found from the conservation of energy: $E^2z^2R/4k=2mv^2/2$ where $v=\frac {Ez}{2}\sqrt{\frac{R}{km}}$. Until the balls accelerate the equation of motion is $\ddot z=E^2z^2R/4km$, where both $z$ and $v=\dot z$ are exponential functions of time. Therefore the graph of velocity is an exponentially growing curve that goes into a constant function with a jump (at the moment where the second ball enters the electric field).
\fi
}