\ylDisplay{Viiul} % Ülesande nimi
{Jaan Toots} % Autor
{lõppvoor} % Voor
{2014} % Aasta
{G 2} % Ülesande nr.
{4} % Raskustase
{
% Teema: Kinemaatika
\ifStatement
Viiulikeelt pikkusega $L$ kaugusel $\frac{3}{7}L$ ühest otsast alla vajutades ning lühemal osal poognaga tõmmates kõlab mingi põhisagedusega heli. Samal kaugusel $\frac{3}{7}L$ keelt ainult puudutades (alla vajutamata), on kõlav heli erinev. Milline on nende kahe põhisageduse suhe?
\fi


\ifHint
Tekkival seisulainel peavad olema sõlmed mõlemas keele võnkuva osa otspunktis. Viiuli keelt puudutades peab ka keele puutepunkt sõlmpunkt olema.
\fi


\ifSolution
Tekkival seisulainel peavad olema sõlmed mõlemas keele võnkuva osa otspunktis, seega võngub alla vajutades osa pikkusega $\frac{3}{7}L$, millele vastab lainepikkus $\lambda_0=\frac{6}{7}L$. Puudutades võngub kogu keel ning on kolm tingimust: sõlmpunktid on mõlemas otsas ning lisaks puudutatavas punktis. Seega peab sellest punktist mõlemale poole mahtuma täisarv poollainepikkusi. Võnkuvate osade suhe on $\frac{3/7}{1-3/7}$ ehk $\frac{3}{4}$. $3$ ja $4$ on ühistegurita. Seega peab jääma võnkuvatele pooltele vastavalt $3$ ja $4$ poollainepikkust. Vaadeldes pikkusega $\frac{3}{7}L$ keele poolt, taipame et $\lambda=\frac{2L}{7}$ ning
\[
\frac{\nu}{\nu_0}=\frac{\lambda_0}{\lambda}=3.
\]
\fi


\ifEngStatement
% Problem name: Violin
If pressing down on a violin string with a length $L$ at a distance $\frac{3}{7}L$ from one end and dragging along the shorter part with a violin bow, a sound with fundamental frequency will be heard. When just touching the string at the same distance $\frac{3}{7}L$ (without pressing down) the sound heard is different. What is the ratio of these two fundamental frequencies?
\fi


\ifEngHint
On the emerging standing wave there has to be nodes on both end points of the swinging string. When touching the string of the violin, the touching point of the string also has to be a node.
\fi


\ifEngSolution
The arising standing wave has to have nodes at both ends of the oscillating part of the string, therefore when sinking down the part with the length $\frac{3}{7}L$ oscillates and it has a corresponding wavelength $\lambda_0=\frac{6}{7}L$. When touching the whole string oscillates and there are three conditions: nodes are at both ends and additionally in the part that is touched. Therefore from this point whole number of half wavelengths has to fit to both sides. The ratio of the oscillating parts is $\frac{\frac{3}{7}}{1-\frac{3}{7}}$ meaning $\frac{3}{4}$. $3$ and $4$ are without a common divisor. Therefore respectively $3$ and $4$ half wavelengths have to stay at the oscillating parts. When observing the side of the string with the length $\frac{3}{7}L$ we see that $\lambda=\frac{2L}{7}$ and $\frac{\nu}{\nu_0}=\frac{\lambda_0}{\lambda}=3$.
\fi
}