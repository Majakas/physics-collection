\setAuthor{Urmo Visk}
\setRound{lõppvoor}
\setYear{2006}
\setNumber{G 5}
\setDifficulty{4}
\setTopic{Varia}

\prob{Kuu}
Peegeldusteguriks nimetatakse pinnalt peegeldunud ja pinnale langenud valgusvõimsuste suhet. Säriaeg on ajavahemik, mille vältel langeb fotoaparaadis objektiivi läbinud valgus filmilindile. Päikeselisel sügispäeval on mingi objekti pildistamisel optimaalne säriaeg $t_1 = 1/\SI{8000}{s}$. Sama objekti pildistamisel öösel, kui paistab täiskuu, on optimaalne säriaeg $t_2 = \SI{160}{s}$. Mõlema pildi tegemisel on erinev vaid säriaeg. Hinnake Kuu pinna keskmist peegeldustegurit. Kuu kaugus Maast $R = \SI{384000}{km}$ ja Kuu raadius $r = \SI{1740}{km}$. Kvaliteetse pildi saamiseks peab filmile langev valgusenergia päeval ja öösel olema sama väärtusega ehk fotografeerimisel võib valgustatuse ja optimaalse säriaja lugeda pöördvõrdeliseks.

\hint
Lihtsuse mõttes võib eeldada, et Kuu pinnale jõudev summaarne valgusvoog Päikeselt peegeldub ühtlaselt poolsfäärile, mille keskpunktiks on Kuu ja pinna peal asub Maa.

\solu
Olgu valgustatus (valgusvoog pinnaühiku kohta) päeval $I_1$ ja öösel $I_2$. Kuna säriaeg ja valgustatus on pöördvõrdelised, siis kehtib seos
\[
t_1/t_2 = I_2/I_1.
\]
Öösel valgustab Maad Kuult peegeldunud valgus. Kuna eeldati, et Kuu ja Maa on Päikesest võrdsel kaugusel, siis on valgustatus Kuul ka $I_1$. Kui Kuu pind oleks Maad ja Kuud (ning ühtlasi Päikest) ühendava teljega risti, siis oleks Kuu poolt tagasi (Maa poole) suunatud valgusvoo tihedus $I_k = kI_1$, kus $k$ on Kuu pinna keskmine peegeldustegur (albeedo). Kuu näiv keskpunkt seda ka on, aga servad mitte. Ometigi, visuaalne kogemus ütleb, et Kuu ketas näib kõikjal enam-vähem ühe heledune. Hinnanguliselt asendagem mõtteliselt Kuu \enquote{pannkoogiga}, st kettaga, mille raadius on võrdne Kuu raadiusega.

Kuu pinnalt peegeldub Päikeselt sumaarne valgusvoog $E = I_1 \pi r^2k$ ning see jaotub ühtlaselt mõttelise poolsfääri peale, mille keskpunktis asub Kuu ja pinna peal Maa. Seega jõuab Maale valgusvoog
\[
I_2 = \frac{E}{2\pi R^2} = I_1 \frac{r^2k}{2R^2}.
\]
Seega
\[
\frac{I_2}{I_1} = \frac{t_1}{t_2} = \frac{r^2k}{2R^2},
\]
ehk
\[
k = \frac{t_1}{t_2} \frac{2R^2}{r^2} \approx \SI{8}{\%}.
\]
\probend