\ylDisplay{Paisupaak} % Ülesande nimi
{Ardi Loot} % Autor
{lahtine} % Voor
{2016} % Aasta
{G 7} % Ülesande nr.
{6} % Raskustase
{
% Teema: Gaasid
\ifStatement
Selleks, et vältida küttesüsteemis vee paisumise tulemusena tekkivat
ülerõhku, lisatakse süsteemi paisupaak. See koosneb silindrist ruumalaga
$V,$ mis on jaotatud vabalt liikuva õhukese vaheseinaga kaheks osaks. Üks
neist osadest täidetakse suruõhuga ($T_{0}=\SI{20}{\celsius}$)
rõhuni $p_{0}$, võttes enda alla kogu silindri ruumala. Seejärel ühendatakse silindri teine osa küttesüsteemiga temperatuuril $T_{1}=T_{0}$ ning süsteem täidetakse
veega, kuni saavutatakse rõhk $p_{1}=\SI{300}{kPa}$ ning vee koguruumala süsteemis $V_{s}=\SI{100}{L}$.
Täitmise lõppedes on paisupaagist $\beta=10\%$ täidetud veega. Talvel suureneb kütmise tõttu süsteemis oleva vee ruumala $\alpha=\SI{1}{\%}$ võrra, ning selle tulemusena tõuseb süsteemi rõhk $p_{2}$-ni, kusjuures paisupaagis olev õhk soojeneb temperatuurini $T_{2}=\SI{40}{\celsius}$. Leidke, kui suur peab olema paisupaagis
õhu algne rõhk $p_{0}$ ja minimaalne paisupaagi ruumala $V,$
et vee paisumise tulemusena tekkiv lisarõhk $\Delta p=p_{2}-p_{1}$
poleks suurem kui \SI{50}{kPa}. 
\fi


\ifHint
Antud ülesandes on kolm tundmatut: paisupaagis oleva õhu moolide arv, paagi esialgne rõhk ning ruumala. Lisaks on tekstis kirjeldatud kolme tingimust, mida paak täitma peab; igaühe jaoks saab kirja panna ideaalse gaasi olekuvõrrandi.
\fi


\ifSolution
Juhul, kui paisupaak pole veel küttesüsteemiga ühendatud, on terve
paisupaak täidetud õhuga. Seega saab kirja panna ideaalse gaasi olekuvõrrandi:

\begin{equation}
p_{0}V=nRT_{0},\label{eq:2016-lahg-07-paisupaak-eq1}
\end{equation}

\noindent kus $n$ on paagis oleva õhu moolide arv ja $R$ universaalne
gaasikonstant. Teisalt on nõutud, et juhul, kui süsteem täidetakse
rõhuni $p_{1}$, peab olema paisupaagist osa $\beta$ täidetud veega;
seega osa $\gamma=\SI{100}{\%}-\beta$ on täidetud õhuga. Kuna vahesein
on vabalt liikuv, siis peavad silindris asuva vee ja õhu rõhud olema
võrdsed. Seega saab kirja panna teise olekuvõrrandi:

\begin{equation}
p_{1}\gamma V=nRT_{0}.\label{eq:2016-lahg-07-paisupaak-eq2}
\end{equation}


\noindent Lahendades võrranditest \eqref{eq:2016-lahg-07-paisupaak-eq1} ja \eqref{eq:2016-lahg-07-paisupaak-eq2} tekkinud süsteemi, saame

\[
p_{0}=\gamma p_{1}=\SI{270}{kPa}.
\]


\noindent Paisupaagi minimaalse ruumala $V$ leiame tingimusest, et
$\Delta p\leq\SI{50}{kPa}.$ On selge, et kui vee ruumala suureneb
$\Delta V=\alpha V_{s}=\SI{1}{L}$ võrra, siis paagis oleva õhu ruumala
väheneb sama palju. Seega saame kirja panna

\begin{equation}
p_{2}\left(\gamma V-\Delta V\right)=nRT_{2}.\label{eq:2016-lahg-07-paisupaak-eq3}
\end{equation}


\noindent Lahendades võrranditest \eqref{eq:2016-lahg-07-paisupaak-eq1} ja \eqref{eq:2016-lahg-07-paisupaak-eq3}
tekkinud süsteemi, saame

\[
p_{2}=p_{0}\frac{V}{\gamma V-\Delta V}\cdot\frac{T_{2}}{T_{0}}
\]


\noindent ning tingimus paisupaagi ruumala jaoks avaldub seega

\[
V\geq\frac{\left(p_{1}+\Delta p\right)T_{0}}{\left(p_{1}+\Delta p\right)T_{0}-p_{1}T_{2}}\cdot\frac{\Delta V}{\gamma}\approx\SI{13.2}{L}.
\]
\fi


\ifEngStatement
% Problem name: Expansion vessel
An expansion vessel is added to a heating system to avoid the overpressure forming in the system due to the expansion of water. The vessel consists of a cylinder of volume $V,$ that is divided into two parts by a freely moving thin partition. One of these parts is filled with compressed air ($T_{0}=\SI{20}{\celsius}$) to a pressure $p_{0}$, taking under it all of the cylinder’s volume. Next, the other part of the cylinder is connected to the heating system at a temperature $T_{1}=T_{0}$ and the system is filled with water until a pressure $p_{1}=\SI{300}{kPa}$ is reached and the total volume of the water in the system is $V_{s}=\SI{100}{L}$. After this $\beta=10\%$ of the expansion vessel is filled with water. In winter the volume of the water in the system increases by $\alpha=\SI{1}{\%}$ due to the heating and in result the pressure of the system increases to $p_{2}$, moreover the air in the expansion vessel is warmed to a temperature $T_{2}=\SI{40}{\celsius}$. Find how big must be the initial air pressure $p_{0}$ in the expansion vessel and the minimal volume of the vessel $V$, so that the extra pressure $\Delta p=p_{2}-p_{1}$ formed in result of the water’s expansion would not be bigger than 50 kPa.
\fi


\ifEngHint
There are three unknowns in the given problem: the mole number of the air inside the expansion vessel, the initial pressure and volume of the vessel. In addition there are three conditions described in the text that the vessel has to meet; for each of them the ideal gas law can be used.
\fi


\ifEngSolution
In the case where the expansion vessel has not yet been connected to the heating system the whole vessel is filled with air. Therefore we can write down the ideal gas law:
\begin{equation}
p_{0}V=nRT_{0},\label{eq:2016-lahg-07-paisupaak-eq1}
\end{equation} 
where $n$ is the number of moles that the air inside the vessel has and $R$ the universal gas constant. On the other hand it is required that in the case where the system is filled to the pressure $p_{1}$ a fraction $\beta$ of the expansion vessel has to be filled with water; therefore the part $\gamma=\SI{100}{\%}-\beta$ is filled with air. Because the partition wall is freely moving the pressures of the air and water inside the cylinder have to be equal. Thus, we can write down a second equation based on the ideal gas law:
\begin{equation}
p_{1}\gamma V=nRT_{0}.\label{eq:2016-lahg-07-paisupaak-eq2}
\end{equation} 
Solving the equation system of the equations (1) and (2) we get
\[
p_{0}=\gamma p_{1}=\SI{270}{kPa}.
\] 
We find the minimal volume $V$ of the expansion vessel from the condition that $\Delta p\leq\SI{50}{kPa}.$. It is clear that if the water’s volume increases by $\Delta V=\alpha V_{s}=\SI{1}{L}$ then the volume of the air inside the vessel decreases by the same amount. Therefore we can write down
\begin{equation}
p_{2}\left(\gamma V-\Delta V\right)=nRT_{2}.\label{eq:2016-lahg-07-paisupaak-eq3}
\end{equation} 
Solving the equation system consisting of equations (1) and (3) we get
\[
p_{2}=p_{0}\frac{V}{\gamma V-\Delta V}\cdot\frac{T_{2}}{T_{0}}
\] 
and the condition for the expansion vessel’s volume is therefore expressed as 
\[
V\geq\frac{\left(p_{1}+\Delta p\right)T_{0}}{\left(p_{1}+\Delta p\right)T_{0}-p_{1}T_{2}}\cdot\frac{\Delta V}{\gamma}\approx\SI{13.2}{L}.
\]
\fi
}