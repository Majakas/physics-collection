\setAuthor{Ants Remm}
\setRound{lahtine}
\setYear{2012}
\setNumber{G 6}
\setDifficulty{5}
\setTopic{Dünaamika}

\prob{Kloori molekul}
Kloori molekul, mis liigub kiirusega $v = \SI{600}{m/s}$, neelab
footoni lainepikkusega $\lambda = \SI{350}{nm}$ ning jaguneb kaheks aatomiks.
Ühe aatomi kiiruseks
mõõdetakse $ u = \SI{1600}{m/s}$, mis on risti molekuli esialgse
kiirusega. Leidke kloori molekuli seoseenergia, kui kõik osakesed olid
minimaalse siseenergiaga seisundis. Plancki konstant on $h =
\SI{6,6e-34}{J.s}$, valguse kiirus on $c = \SI{3,0e8}{m/s}$, kloori aatomnumber on 35 ning Avogadro arv on $N_A
= \SI{6,0e23}{\text{mol}^{-1}}$. Footoni energia avaldub valemiga $E =
\frac{h c}{\lambda}$. Eeldada, et footoni impulss on tühine võrreldes Kloori impulsiga.

\hint
Jagunemise käigus peab säilima summaarne impulss ning energia. Osakeste kiirusi on mugavam vaadelda komponentide kaupa. Selleks võib võtta, et $x$-telg on molekuli esialgse suunaga paralleelne ning $y$-telg sellega risti.

\solu
Siin ülesandes saab lähtuda energia jäävusest: kloori molekuli seoseenergia $E_s$ on võrdne vahega, kus footoni energiast $\frac{hc}{\lambda}$ ja esialgsest mehaanilisest energiast $E_0 = 2m\idx{Cl} \frac{v^2}{2} $ on maha lahutatud kloori aatomite mehaaniline energia pärast jagunemist
\[
E_1 = m\idx{Cl} \frac{v_1^2}{2} + m\idx{Cl} \frac{v_2^2}{2}.
\]
Olgu $x$-telg paralleelne molekuli esialgse liikumisega. Teame, et üks aatomitest liikus pärast jagunemist risti $x$-teljega. See tähendab, et $v_{1x} = 0$ ning $v_{1y} = u$. Impulsi jäävuse seadusest saame ka teise aatomi kiiruse komponendid $v_{2x} = 2 v$ ja $v_{2y} = -u$. Nüüd saab panna kirja energia jäävuse:
$$
	\frac{hc}{\lambda} + m\idx{Cl} v^2 = E_s + m\idx{Cl} \frac{u^2}{2} + m\idx{Cl} \frac{4v^2 + u^2}{2} = E_s + 2 m\idx{Cl} v^2 + m\idx{Cl} u^2.
$$
Sealt saab avaldada $E_s$, arvestades, et $m\idx{Cl} = \frac{\mu\idx{Cl}}{N_A}$, kus $\mu\idx{Cl} = \SI{35e-3}{kg\per mol}$.
$$ E_s = \frac{hc}{\lambda} - m\idx{Cl} v^2 - m\idx{Cl} u^2 = \frac{hc}{\lambda} - \frac{\mu\idx{Cl}}{N_A} (v^2 + u^2) = \SI{4,0e-19}{J} = \SI{2,5}{eV}.
$$

\probeng{Chlorine’s molecule}
Chlorine’s molecule, which is moving with a velocity $v = \SI{600}{m/s}$, absorbs a photon with a wavelength $\lambda = \SI{350}{nm}$ and is divided into two atoms. The first atom’s velocity is measured to be $ u = \SI{1600}{m/s}$, which perpendicular to the molecule’s initial velocity. Find the chlorine molecule’s binding energy if all the particles were at a state with minimal inner energy. Planck constant is $h =
\SI{6,6e-34}{J.s}$, speed of light is $c = \SI{3,0e8}{m/s}$, chlorine’s atomic number is 35 and Avogadro constant is $N_A
= \SI{6,0e23}{\text{mol}^{-1}}$. Photon’s energy is defined as $E =
\frac{h c}{\lambda}$. It is presumed that a photon’s momentum is negligible compared to chlorine’s momentum.

\hinteng
During the division the total momentum and energy must remain. The velocities of the particles are easier to observe by components. For that you can assume that the $x$-axis is parallel to the initial direction of the molecule and the $y$-axis perpendicular to it.

\solueng
In this problem we can use the conservation of energy: the binding energy $E_s$ of chlorine’s molecule is equal to the difference where the mechanical energy $E_1 = m\idx{Cl} \frac{v_1^2}{2} + m\idx{Cl} \frac{v_2^2}{2}$ of chlorine molecules after division is subtracted from photon energy $\frac{hc}{\lambda}$ and initial mechanical energy $E_0 = 2m\idx{Cl} \frac{v^2}{2} $.\\\    
Let the $x$-axis be parallel to the initial movement of the molecule. We know that one of the atoms moved perpendicularly to the $x$-axis after the division. This means that $v_{1x} = 0$ and $v_{1y} = u$. From the conservation of momentum we get the components of the other atom’s velocity $v_{2x} = 2 v$ and $v_{2y} = -u$. Now we can write down the conservation of energy:
$$
	\frac{hc}{\lambda} + m\idx{Cl} v^2  = E_s + m\idx{Cl} \frac{u^2}{2} + m\idx{Cl} \frac{4v^2 + u^2}{2} = E_s + 2 m\idx{Cl} v^2 + m\idx{Cl} u^2.
$$ 
From this we can express $E_s$ considering that $m\idx{Cl} = \frac{\mu\idx{Cl}}{N_A}$ where $\mu\idx{Cl} = \SI{35e-3}{kg\per mol}$.
$$ E_s = \frac{hc}{\lambda} - m\idx{Cl} v^2 - m\idx{Cl} u^2 = \frac{hc}{\lambda} - \frac{\mu\idx{Cl}}{N_A} (v^2 + u^2) = \SI{4,0e-19}{J} = \SI{2,5}{eV}.
$$
\probend