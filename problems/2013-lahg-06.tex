\setAuthor{Eero Vaher}
\setRound{lahtine}
\setYear{2013}
\setNumber{G 6}
\setDifficulty{5}
\setTopic{Magnetism}

\prob{Tiirlev kuulike}
Olgu meil positiivselt laetud kuulike massiga $m$. On teada, et kui kuulike
liiguks kiirusega $v$ sellega ristuvas magnetväljas induktsiooniga $B$,
siis oleks selle trajektooriks ringjoon raadiusega~$r$. Kui suure laenguga $q$ peab olema teine sama massiga kuulike, et esimene
kuulike liiguks teise kuulikese elektriväljas sama kiirusega samal trajektooril? Eeldage, et
kahe kuulikese süsteemile ei mõju väliseid jõude. Kõiki liikumisi vaadeldakse
laboratoorses taustsüsteemis.
\pagebreak

\hint
Magnetväljas ringliikumisel käitub Lorentzi jõud kesktõmbejõuna. Analoogselt on teisel juhul kesktõmbejõud kuloniline jõud kahe kuuli vahel, kusjuures peab tähelepanu pöörama sellele, et kuulid tiirlevad ümber ühise massikeskme.

\solu
Olgu positiivse laenguga kuulikese laeng $Q$. Kuulikesele mõjuks kirjeldatud magnetväljas Lorentzi jõud suurusega $F_L=QvB$. See jõud oleks kuulikesele mõjuvaks kesktõmbejõuks $F_k=\frac{mv^2}{r}$. Saame $QvB=\frac{mv^2}{r}$. Kaks isoleeritud võrdse massiga kuulikest tiirlevad ümber ühise masskeskme, seega on nende omavaheline kaugus $d=2r$. Kuulikesi ringorbiidil hoidvaks kesktõmbejõuks on kuloniline jõud suurusega 
\[
F_C=\frac{kqQ}{d^2}=\frac{kqQ}{4r^2}.
\]
Kuna teineteise ümber tiirlevad laengud peavad olema vastasmärgilised, siis $|F_C|=-F_C$. Võrdsustades kesktõmbejõu ning kulonilise jõu suuruse, saame
\[
\frac{mv^2}{r}=-\frac{kqQ}{4r^2}
\]
ehk
\[
QvB=-\frac{kqQ}{4r^2}.
\]
Lõpptulemuseks saame 
\[
q=-\frac{4vBr^2}{k}.
\]

\probeng{Revolving ball}
Let there be a positively charged ball of mass $m$. It is known that if the ball would move with a speed $v$ in a perpendicular magnetic field of induction $B$ then the ball’s trajectory would be a circle of radius $r$. How big should be the charge $q$ of another ball with the same mass so that the first ball would move with the same speed on the same trajectory in the electric field of the second ball? Assume that no external forces are applied to this system of two balls. All the movements are observed in a laboratory frame of reference.

\hinteng
In the case of circular motion in the magnetic field the Lorentz force acts as centripetal force. Similarly in the other case the centripetal force is a Coulombic force between the two balls, moreover you should notice that the balls revolve around a coinciding center of mass.

\solueng
Let the charge of the positively charged ball be $Q$. In the described magnetic field the ball would be affected by a Lorentz force $F_L=QvB$. This force would be the centripetal force $F_k=\frac{mv^2}{r}$ applied to the ball. We get $QvB=\frac{mv^2}{r}$. Two isolated balls with equal masses rotate around a common center of mass, therefore the distance between them is $d=2r$. The centripetal force holding the balls in the circular orbit is a coulombic force of value $F_C=\frac{kqQ}{d^2}=\frac{kqQ}{4r^2}$. Because the charges rotating around each other have to have the opposite signs then $|F_C|=-F_C$. Equating the centripetal force and coulombic force we get $\frac{mv^2}{r}=-\frac{kqQ}{4r^2}$, meaning $QvB=-\frac{kqQ}{4r^2}$. As the final result we get $q=-\frac{4vBr^2}{k}$.
\probend