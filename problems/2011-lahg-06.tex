\setAuthor{Jaan Kalda}
\setRound{lahtine}
\setYear{2011}
\setNumber{G 6}
\setDifficulty{7}
\setTopic{Magnetism}

\prob{Laengud}
Kaks osakest laenguga $q$ stardivad koordinaatide alguspunktist kiirusega $v$:
üks $x$-telje sihis, teine $y$-telje sihis.
Liikumine toimub homogeenses $z$-telje sihilises magnetväljas induktsiooniga
$B$; osakeste omavahelise
elektrostaatilise vastasmõjuga ärge arvestage. Milline on osakeste vahelise
kauguse maksimaalväärtus $l_{\max}$ edasise liikumise käigus?

\hint
Lorentzi jõu tõttu liiguvad mõlemad osakesed mööda ringjoont, kusjuures mõlema osakese kiirusvektorite pöörlemise nurkkiirused on samad.

\solu
Kuna Lorentzi jõud mõjub alati risti liikumissuunaga, liiguvad laengud mööda ringjooni, mille raadiuse leiame Newtoni teisest seadusest:
\[
qvB=m\frac{v^2}{r} \qquad \Rightarrow \qquad R=mv/qB,
\]
kusjuures ühe ringjoone keskpunkt on punktis $(0,R)$ ja teisel --- $(-R,0)$.
Nende kiirusvektorid on alghetkel risti ja kuivõrd need pöörlevad ühesuguse kiirusega, siis jäävad risti ka edasise liikumise käigus,
kusjuures suhtelise kiiruse vektor $\vec w= \vec v_1-\vec v_2$ moodustab kummagi kiirusvektoriga 45-kraadilise nurga. Vahekaugus on
maksimaalne, kui $\vec w$ on risti laenguid ühendava sirgega, st laenguid ühendav sirge moodustab laengu asukohast tõmmatud
puutujaga (st laengu kiirusvektoriga) 45-kraadilise nurga; on lihtne näha, et see juhtub hetkel, mil laengud on punktides $(0,2R)$ ja $(-2R,0)$, mis
annab maksimaalseks vahekauguseks $l=2\sqrt 2 R =2\sqrt 2mv/qB$.

\vspace{0.5\baselineskip}

{\em Alternatiivne lahendus}\\
Esitame laengute asukohad ajalises sõltuvuses kompleksarvudena komplekstasandil:
$$z_1=R\mathrm{i} - R\mathrm{i}e^{\mathrm{i}\omega t}\;\mbox{ja}\; z_2= -R + Re^{\mathrm{i}\omega t},$$ kus
$\omega$ on tsüklotronsagedus. Nende vahekaugus
$$l=|z_1-z_2|=|R(1+\mathrm{i})(1-e^{\mathrm{i}\omega t})|=R\sqrt 2|1-e^{\mathrm{i}\omega t}|$$ on maksimaalne, kui $e^{\mathrm{i}\omega t}=-1$, mil $l=2\sqrt{2}R=2\sqrt{2}mv/qB$.
\probend