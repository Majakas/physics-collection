\ylDisplay{Lihvimisketas} % Ülesande nimi
{Ott Krikmann} % Autor
{piirkonnavoor} % Voor
{2005} % Aasta
{G 3} % Ülesande nr.
{4} % Raskustase
{
% Teema: Termodünaamika
\ifStatement
Detaili lihvitakse horisontaalselt pöörleva lihvimiskettaga, mille raadius on $r = \SI{20}{cm}$. Ülekuumenemise vältimiseks jahutatakse seda veega. Aja $t = \SI{1}{s}$ jooksul eraldub ketta ühelt ruutmeetrilt ($s = \SI{1}{m^2}$) keskmiselt $q = \SI{10}{kJ}$ suurune soojushulk, mille neelab jahutusvesi. Jahutusvett, algtemperatuuriga $t_1 = \SI{10}{\celsius}$, juhitakse ketta tsentrisse vooga $w = \SI{10}{cm^3/s}$. Vee erisoojus $c = \SI{4200}{J/(kg.K)}$. Leidke üle kettaääre voolava vee keskmine temperatuur $t_2$.
\fi


\ifHint
Kehtib soojusbilanss lihvimise käigus eralduva soojuse ning sisse- ja väljavoolava vee soojusvoo vahel. Mugavuse mõttes võib vaadelda ajavahemikku $\Delta t$ ning selle jooksul lihvil eralduvat ning vee poolt äraantavat soojushulka.
\fi


\ifSolution
Kettal aja $t = \SI{1}{s}$ jooksul eraldub soojushulk
\[
Q = \frac{\pi r^2 q}{s}.
\]
Sama aja jooksul voolab vesi massiga
\[
m = wt\rho.
\]
Et jahutusvesi kannab kogu eralduva soojuse, siis võib koostada soojusbalansi võrrandi aja $t = \SI{1}{s}$ jaoks:
\[
\frac{\pi r^{2} q}{s}=w t \rho c\left(t_{2}-t_{1}\right).
\]
Siit võrrandist avaldame $t_2$:
\[
t_2 = t_1 + \frac{\pi r^2q}{swtc\rho} \approx \SI{40}{\celsius}.
\]
\fi
}