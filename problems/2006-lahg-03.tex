\ylDisplay{Keha} % Ülesande nimi
{Tundmatu autor} % Autor
{lahtine} % Voor
{2006} % Aasta
{G 3} % Ülesande nr.
{2} % Raskustase
{
% Teema: Dünaamika
\ifStatement
Vertikaalselt ülesse visatud keha läbib kaks korda kõrgusel $h$ asuvat punkti. Ajavahemik nende kahe läbimiste vahel on $\Delta t$. Leida keha algkiirus $v_0$ ja aeg $\tau$ keha liikumise algusest kuni algpunkti tagasi jõudmiseni.
\fi


\ifHint
Ülesse visatud keha vertikaalne koordinaat avaldub vastavalt liikumisvõrrandile kui $h = v_0t - \frac{gt^2}{2}$. Kuna tegu on ruutvõrrandiga, leidub fikseeritud $h$ jaoks kaks ajahetke, mil keha sellel kõrgusel on.
\fi


\ifSolution
Ülesse visatud keha koordinaadi leiame võrrandist: 
\[
h = v_0t - \frac{gt^2}{2}.
\]
Iga antud $v_0$ ja $h$ jaoks annab see võrrand kaks $t$ väärtust:
\[
t_{1,2}=\frac{v_{0} \pm \sqrt{v_{0}^{2}-2 g h}}{g}.
\]
Nende kahe väärtuse vahe on ajavahemik, mis möödub kõrguse $h$ kahe läbimise vahel keha poolt:
\begin{equation} \label{2006-lahg-03:eq1}
\Delta t=t_{2}-t_{1}=\frac{2 \sqrt{v_{0}^{2}-2 g h}}{g}.
\end{equation}
Siit saame, et
\[
v_{0}=\sqrt{2 g h+\frac{g^{2} \Delta t^{2}}{4}}.
\]
Kui valemis (\ref{2006-lahg-03:eq1}) võtta $h = 0$, siis saame ajavahemiku, mis möödub liikumise algusest kuni jõudmiseni tagasi algpunkti: $\tau = 2v_0/g$. Asendades siia varem saadud $v_0$ väärtuse, saame:
\[
\tau=2 \sqrt{\frac{2 h}{g}+\frac{\Delta t^{2}}{4}}.
\]
\fi
}