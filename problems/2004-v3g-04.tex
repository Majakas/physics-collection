\setAuthor{Tundmatu autor}
\setRound{lõppvoor}
\setYear{2004}
\setNumber{G 4}
\setDifficulty{1}
\setTopic{Teema}

\prob{Mootorratas}
Inimene massiga $m_{1}=\SI{75}{kg}$ sõidab mootorrattaga, mille mass on $m_{2}=\SI{150}{kg}$. Mootorratturi ja mootorratta ühine masskese asub kõrgusel $h=\SI{0,6}{m}$ maapinnast ja horisontaalsihis kaugusel $l=\SI{0,5}{m}$ tagumise ratta teljest. Millise kiirendusega sõites kerkib esiratas maapinnast lahti? Kui suur peab olema hõõrdetegur tagaratta ja maapinna vahel, et niimoodi saaks sõita?

\hint

\solu
Kuna mootorratas ei pöörle, siis mootorrattale mõjuvate jõumomentide summa mistahes punkti, näiteks masskeskme, suhtes peab olema null ehk $F_{h} h=N l$, kus $N$ on tagarattale mõjuv maapinna normaalreaktsioon ja $F_{h}-$ tagarattale mõjuv hõõrdejõud. Kuna $N$ kompenseerib mootorratta kaalu, $F_{h}$ aga annab kiirenduse $a$, siis $N=m g$ ja $F_{h}=m a$, kus $m=m_{1}+m_{2}$. Seega $a=\lg / h \approx\SI{8,2}{m/s^2}$. Kui
tagumine ratas on libisemise piiril, siis $\mu=F_{h} / N=l / h \approx \num{0,83}$.
\probend