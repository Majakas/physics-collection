\ylDisplay{Rattur} % Ülesande nimi
{Ardi Loot} % Autor
{lahtine} % Voor
{2016} % Aasta
{G 6} % Ülesande nr.
{5} % Raskustase
{
% Teema: Dünaamika
\ifStatement
Rattur massiga $m=\SI{100}{kg}$ sõidab ilma väntamata alla mäenõlvalt langemisnurgaga
$\theta_{1}=\SI{4.8}{\degree}$ (nurk horisondi ja mäenõlva vahel)
ja märkab, et piisavalt pika nõlva korral on tema lõppkiiruseks $v_{1}=\SI{50}{km/h}$.
Kaks korda väiksema nõlva korral $(\theta_{2}=\SI{2.4}{\degree})$
on ratturi lõppkiirus aga $\Delta v=\SI{15}{km/h}$ võrra väiksem.
Leidke, kui suur peab olema ratturi väntamise võimsus, et horisontaalsel
teel hoida kiirust $v=\SI{20}{km/h}.$ Kui suur osa võimsusest kulub
tuuletakistuse ületamiseks? Eeldage, et tegemist on tuulevaikse ilmaga
ja raskuskiirendus $g=\SI{9.8}{m/s^{2}}$. 

\emph{Märkus.} Arvestada tuleks nii kiirusest sõltumatu hõõrdejõuga kui ka tuuletakistusega,
mis on võrdeline kiiruse ruuduga.
\fi


\ifHint
Ratturile mõjuvad laskumisel kolm jõudu: mäest allaviiv raskusjõud, takistav hõõrdejõud ning tuuletakistus. Rattur on saavutanud lõppkiiruse, kui need jõud on tasakaalustunud.
\fi


\ifSolution
Ratturile mõjuvad laskumisel kolm jõudu: mäest allaviiv raskusjõud
($F_{a}=mg\sin(\theta)$) ning takistavad hõõrdejõud ($F_{h}\cos(\theta)$, väikese nurga tõttu võib $\cos(\theta)$ ära jätta)
ja tuuletakistus ($F_{t}=cv^{2},$ kus $c$ on kordaja). Rattur on
saavutanud lõppkiiruse, kui need jõud on tasakaalustunud. Kuna lõppkiirus
on teada kahe eri langemisnurga korral, on võimalik kirja panna võrrandisüsteem
hõõrdejõu $F_{h}$ ja tuuletakistuskordaja $c$ leidmiseks ($v_{2}=v_{1}-\Delta v$) 
\[
\begin{cases}
mg\sin(\theta_{1})&=F_{h}\cos(\theta_1)+cv_{1}^{2}\\
mg\sin(\theta_{2})&=F_{h}\cos(\theta_2)+cv_{2}^{2}.
\end{cases}
\]
Lahendades süsteemi saab avaldada:
\begin{align*}
F_{h} & = mg\cdot\frac{v_{1}^{2}\sin(\theta_{2})-v_{2}^{2}\sin(\theta_{1})}{v_{1}^{2}\cos(\theta_2)-v_{2}^{2}\cos(\theta_1)}\approx\SI{1.7}{N},\\[7pt] 
c & = mg\cdot\frac{\sin(\theta_{1})\cos(\theta_2)-\sin(\theta_{2})\cos(\theta_1)}{v_{1}^{2}\cos(\theta_2)-v_{2}^{2}\cos(\theta_1)}\approx\SI{0.42}{kg/m}.
\end{align*}

Kasutades saadud tulemusi on lihtne arvutada ratturile mõjuvad
takistusjõud ja sellele ületamiseks kuluv võimsus horisontaalsel teel
kiirusega $v$:
\[
F=F_{h}+cv^{2}\approx\SI{14.5}{N},
\]
\[
P=Fv=mgv\cdot\frac{\sin(\theta_{1})\left(v^{2}\cos(\theta_2)-v_{2}^{2}\right)-\sin(\theta_{2})\left(v^{2}\cos(\theta_1)-v_{1}^{2}\right)}{v_{1}^{2}\cos(\theta_2)-v_{2}^{2}\cos(\theta_1)},
\]
\[
P\approx\SI{80.8}{W}.
\]
Tuuletakistuse ületamiseks kulub $cv^{2}/F=\SI{88.5}{\percent}$
koguvõimsusest.
\fi


\ifEngStatement
% Problem name: Cyclist
A cyclist with a mass $m=\SI{100}{kg}$ rides without pedaling down a hill with an angle of inclination of $\theta_{1}=\SI{4.8}{\degree}$ (angle between the horizontal and the hill) and notices that if the hill is long enough his final speed would be $v_{1}=\SI{50}{km/h}$. However, if the mountain is two times smaller $(\theta_{2}=\SI{2.4}{\degree})$ his final speed would be smaller by $\Delta v=\SI{15}{km/h}$. Find how big must be the cyclist pedaling power so he would maintain a speed of $v=\SI{20}{km/h}.$ on a horizontal road. What fraction of the power does it take to overcome the air resistance? Assume that there is a windless weather and that the gravitational acceleration is $g=\SI{9.8}{m/s^{2}}$. \\
\emph{Note.} You should take into account the friction force that does not depend on the speed and also the wind resistance which is proportional to speed squared.
\fi


\ifEngHint
During the descent three forces are applied to the cyclist: the gravity force helping the descent, hindering friction force and the air resistance. The cyclist has achieved the final speed when these forces have balanced.
\fi


\ifEngSolution
Three forces are applied to the cyclist while descending: gravity force ($F_{a}=mg\sin(\theta)$) and hindering friction forces ($F_{h}\cos(\theta)$, due to small angle $\cos(\theta)$ can be left out) and wind resistance ($F_{t}=cv^{2},$ where $c$ is a coefficient). The cyclist has reached his final speed when these forces have balanced. Because the initial speed is known for two different angles of descent it is possible to write down a system of equations to find the friction $F_{h}$ and the coefficient $c$ of wind resistance ($v_{2}=v_{1}-\Delta v$)
\[
\begin{cases}
mg\sin(\theta_{1})&=F_{h}\cos(\theta_1)+cv_{1}^{2}\\
mg\sin(\theta_{2})&=F_{h}\cos(\theta_2)+cv_{2}^{2}.
\end{cases}
\] 
Solving this system we can express:
\begin{align*}
F_{h} & = mg\cdot\frac{v_{1}^{2}\sin(\theta_{2})-v_{2}^{2}\sin(\theta_{1})}{v_{1}^{2}\cos(\theta_2)-v_{2}^{2}\cos(\theta_1)}\approx\SI{1.7}{N},\\[7pt] 
c & = mg\cdot\frac{\sin(\theta_{1})\cos(\theta_2)-\sin(\theta_{2})\cos(\theta_1)}{v_{1}^{2}\cos(\theta_2)-v_{2}^{2}\cos(\theta_1)}\approx\SI{0.42}{kg/m}.
\end{align*} 
Using the results it is easy to calculate the drag forces applied to the cyclist and the power to overcome them on a horizontal road with the speed $v$:
\[
F=F_{h}+cv^{2}\approx\SI{14.5}{N},
\] 
\[
P=Fv=mgv\cdot\frac{\sin(\theta_{1})\left(v^{2}\cos(\theta_2)-v_{2}^{2}\right)-\sin(\theta_{2})\left(v^{2}\cos(\theta_1)-v_{1}^{2}\right)}{v_{1}^{2}\cos(\theta_2)-v_{2}^{2}\cos(\theta_1)},
\]
\[
P\approx\SI{80.8}{W}.
\]
It takes $cv^{2}/F=\SI{88.5}{\percent}$ of the total power to overcome the wind resistance.
\fi
}