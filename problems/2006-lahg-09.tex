\ylDisplay{Laetud klotsid} % Ülesande nimi
{Tundmatu autor} % Autor
{lahtine} % Voor
{2006} % Aasta
{G 9} % Ülesande nr.
{8} % Raskustase
{
% Teema: Elektrostaatika
\ifStatement
Horisontaalsel siledal dielektrilisel pinnal asuvad kaks laetud klotsi massidega $m$ ja samanimeliste laengutega $q$. Alghetkel on vahemaa nende vahel $l$. Mis tingimusel hakkavad klotsid liikuma ja kui suur on vahemaa $L$ nende vahel, kui liikumine lõppeb? Hõõrdetegur klotside ja pinna vahel on $\mu$. Klotside mõõtmeid ja liikuvate laengute elektromagnetkiirgust mitte arvestada. Pinna dielektriline läbitavus on 1.
\fi


\ifHint
Klotside läbitav vahemaa on leitav energia jäävuse seadusest. Nimelt on klotsid paigal nii alg- kui ka lõppasendis ning ainus viis soojuse eraldumiseks on hõõrdejõu kaudu, mis on omakorda avaldatav alg- ja lõppasendi potentsiaalsete energiate vahest.
\fi


\ifSolution
Klots hakkab liikuma, kui sellele mõjuv elektrostaatiline jõud ületab maksimaalsehõõrdejõu:
\[
\frac{kq^2}{l^2} > \mu mg.
\]
Klotside läbitava vahemaa leiame energia jäävuse seadusest. Punktis, milles klots peatub, on elektrostaatilise välja potentsiaalne energia väiksem, kui algpunktis. Potentsiaalsete energiate vahe muundub liikumise käigus klotside kineetiliseks energiaks, mis, omakorda, hõõrdejõu töö tulemusena muundub soojuseks.

Kui kuulide vahemaa liikuma hakkamisel oli $l$ ning seisma jäämise hetkel $L$, siis muutus elektrostaatilises potentsiaalses energias on
\[
\Delta E = \frac{kq^2}{l} - \frac{kq^2}{L}.
\]
Hõõrdejõud $\mu mg$ mõjub kummagile klotsile vahemaa $(L-l)/2$ jooksul, seega kogu
hõõrdejõu töö on
\[
A = \mu mg (L - l).
\]
Et need energia muudud on võrdsed, saame vahemaa $L$ jaoks lihtsa võrrandi:
\[
\begin{aligned}
k q^{2}\left(\frac{1}{l}-\frac{1}{L}\right)=\mu m g(L-l) &\Rightarrow \frac{k q^{2}}{\mu m g} \frac{L-l}{l L}=L-l \Rightarrow\\
&\Rightarrow L=\frac{k q^{2}}{\mu m g l}
\end{aligned}
\]
\fi
}