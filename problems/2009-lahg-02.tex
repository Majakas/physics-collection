\ylDisplay{Lääts} % Ülesande nimi
{Tundmatu autor} % Autor
{lahtine} % Voor
{2009} % Aasta
{G 2} % Ülesande nr.
{2} % Raskustase
{
% Teema: Geomeetriline-optika
\ifStatement
Lääts tekitab esemest $d = \SI{24}{cm}$ kaugusele ekraanile kujutise, mis on esemest \num{3} korda suurem. Leidke läätse fookuskaugus.
\fi


\ifHint
Tasub teha selge joonis ning kasutada sarnaseid kolmnurki.
\fi


\ifSolution
Kuna kujutis tekib ekraanile, siis on kujutis tegelik ning tegemist on koondava läätsega.

\begin{center}
	\includegraphics[width=0.9\linewidth]{2009-lahg-02-lah}
\end{center}

Olgu $a$ kaugus esemest läätseni, $k$ kaugus kujutisest läätseni ning $f$ läätse fookuskaugus. Et kujutis on esemest \num{3} korda suurem, siis sarnastest kolmnurkadest $ABO$ ja $A'B'O$
\[
\frac{k}{a}=3 \quad \Rightarrow \quad k=3 a.
\]
Kujutis tekib kugusele $d = \SI{24}{cm}$, seega
\[
a+k=4 a=\SI{24}{cm} \quad \Rightarrow \quad a=\SI{6}{cm}, k=\SI{18}{cm}.
\]
Nüüd läätse valemist
\[
\frac{1}{a}+\frac{1}{k}=\frac{1}{f}
\]
leiame, et $f = \SI{4,5}{cm}$.

\emph{Märkus}. Läätse valemi asemel võib fookuskauguse leidmiseks kasutada sarnaseid kolmnurki $A'B'F$ ja $OKF$. Saame
\[
\frac{k-f}{f}=3 \quad \Rightarrow \quad f=\frac{k}{4}=\SI{4,5}{cm}.
\]
\fi
}