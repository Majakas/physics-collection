\setAuthor{Jonatan Kalmus}
\setRound{lõppvoor}
\setYear{2018}
\setNumber{G 6}
\setDifficulty{7}
\setTopic{Elektrostaatika}

\prob{Pendel}
Elektriliselt isoleeritud metallkuul massiga $M$ ja laenguga $Q>0$ ripub vertikaalse vedru otsas jäikusega $k$ tasakaaluasendis gravitatsiooniväljas $g$. Nüüd tekitatakse vertikaalne elektriväli tugevusega $E$, mis on esialgu suunatud alla ning edaspidi alati kuuli liikumise suunas. Eeldada, et elektriväli muutub hetkeliselt. Leida kuuli kaugus algsest asukohast ajahetkel $t=7\pi \sqrt{\frac{M}{k}}$.

\hint
Elektrivälja lisamisel nihkub kuuli tasakaaluasend kas üles või alla, sõltuvalt elektrivälja suunast. Seega on igat poolperioodi mõistlik eraldi vaadata, sest selle raames liigub kuul ümber fikseeritud tasakaaluasendi.

\solu
Kuna liikumine toimub vaid vertikaalsihis, võtame $x$-teljeks vertikaalsihi suunaga alla. Algsest jõudude tasakaalust saame
$$Mg = kx_0,$$
kust kuuli algkõrgus on $x_0 = \frac{Mg}{k}$. Teame, et vedru otsa asetatud kuuli pool võnkeperioodi on 
$$\tau = \pi \sqrt{\frac{M}{k}}.$$
On selge, et see ei sõltu gravitatsioonivälja tugevusest. Kuna nii gravitatsioonivälja kui ka elektrivälja poolt tekitatud jõud käituvad analoogselt, võime järeldada, et võnkeperiood ei sõltu ka elektrivälja tugevusest ning jääb konstantseks. Seega muudab elektrivälja jõud vaid kuuli tasakaaluasendit iga kord, kui kuul muudab oma liikumissuunda (sest ülesandes on öeldud, et elektriväli on alati kuuli liikumise suunas) ehk iga poole võnkeperioodi $\tau$ järel. Alla liikumisel on elektriväli suunatud alla ning saame kirja panna uue jõudude tasakaalu
$$Mg + Eq = kx_\downarrow.$$
Seega on tasakaaluasend alla liikumisel
$$x_\downarrow = \frac{Mg}{k} + \frac{Eq}{k} = x_0 + \Delta x.$$
Analoogselt saame leida tasakaaluasendi üles liikumisel, kuid nüüd on elekriväli suunatud üles:
$$x_\uparrow = \frac{Mg}{k} - \frac{Eq}{k} = x_0 - \Delta x.$$
On selge, et algselt elektrivälja sisselülitamisel, kui elektriväli on alla suunatud, muutub tasakaaluasend hetkeliselt vastavaks, kuid kuuli asukoht jääb algselt samaks. Seega pole kuul enam tasakaalus, vaid uuest tasakaaluasendist kaugusel $\Delta x$ ning hakkab võnkuma amplituudiga $A_1 = \Delta x$ tasakaaluasendi $x_\downarrow$ ümber. Aja $\tau$ järel on kuul jõudnud kaugusele $x_0 + 2\Delta x$. Kuna kuul hakkab nüüd üles liikuma, muutub ka kuuli tasakaaluasend koos elektriväljaga vastavaks ning üles minnes võngub kuul juba amplituudiga $A_2 = 3 \Delta x$ tasakaaluasendi $x_\uparrow$ ümber.
On selge, et iga tasakaaluasendi vahetumisega kasvab kuuli amplituud edaspidi $2 \Delta x$ võrra. Tasakaaluasend vahetub aga iga ajavahemiku $\tau$ järel.
Nüüd leiame kuuli asukoha ülesandes antud ajahetkel, milleks on $7 \tau$. Selle aja jooksul liigub kuul korra algasendist alla, ning teeb siis $3$ täisvõnget, jõudes nende järel uuesti alla. Pärast algasendist alla jõudmist on kuuli asukoht $x_0 + 2 \Delta x $. Seejärel liigub kuul üles asendisse $x_0 - 4 \Delta x$ ja uuesti alla asendisse $x_0 + 6 \Delta x$. Kauguse aboluutväärtus algasendist suureneb seega iga $\tau$ järel $2 \Delta x$ võrra. $7 \tau$ järel on kaugus tasakaaluasendist seega $7 \cdot 2 \Delta x = 14 \Delta x$.
Nagu enne selgeks tegime, on kuul seega küsitud ajahetkel $7 \tau$ algasendist $x_0$ $14 \Delta x = 14 \frac{Eq}{k}$ võrra all pool.

\probeng{Pendulum}
An electrically insulated metal ball of mass $M$ and charge $Q>0$ is hanging at the end of a vertical spring of stiffness $k$ in a gravitational field $g$, the ball is in its equilibrium position. Now a vertical electrical field of strength $E$ is created. At the first moment it is directed downwards and afterwards always at the same direction as the ball’s velocity. Assume that the electrical field changes instantaneously. Find the ball’s distance from its initial position at a moment of time $t=7\pi \sqrt{\frac{M}{k}}$.

\hinteng
When adding the electric field the equilibrium position of the ball shifts either up or down depending on the direction of the electric field. Thus, it is reasonable to look each half-period separately because within this framework the ball moves around a fixed equilibrium position.

\solueng
Because the movement is only vertical we set the $x$-axis to be the vertical direction downwards. From the initial force balance we get
$$Mg = kx_0,$$
where the initial height of the ball is $x_0 = \frac{Mg}{k}$. We know that the ball placed to the end of the spring has an oscillation period
$$\tau = \pi \sqrt{\frac{M}{k}}.$$ 
It is clear that the period does not depend on the strength of the gravitational field. Because forces generated by both the gravitational field and an electric field act in analogous manner then we can conclude that the oscillation period also does not depend on the strength of the electric field and stays constant. Therefore the electric field only changes the equilibrium state of the ball every time the ball changes its movement direction (because in the problem it is said that the electric field is always to the direction of the ball’s movement) meaning after each half period of the oscillation period $\tau$. During downwards motion the electric field is directed down and we can write down a new force balance
$$Mg + Eq = kx_\downarrow.$$ 
Therefore the equilibrium balance during downwards motion is 
$$x_\downarrow = \frac{Mg}{k} + \frac{Eq}{k} = x_0 + \Delta x.$$ 
Similarly we can find the equilibrium position when moving upwards but now the electric field is directed up:
$$x_\uparrow = \frac{Mg}{k} - \frac{Eq}{k} = x_0 - \Delta x.$$ 
It is clear that initially turning the electric field on, when the electric field is directed downwards, the equilibrium position changes momentarily correspondingly but the location of the ball initially stays the same. Therefore the ball is not in equilibrium anymore but at a distance $\Delta x$ from the new equilibrium position and starts to oscillate with an amplitude $A_1 = \Delta x$ around the equilibrium position $x_\downarrow$. After the time $\tau$ the ball has reached a distance $x_0 + 2\Delta x$. Because the ball now starts to move upwards then the equilibrium position of the ball with the electric field changes correspondingly and when going up the ball oscillates with an amplitude $A_2 = 3 \Delta x$ around the equilibrium position $x_\uparrow$. It is clear that after each change of the equilibrium position the amplitude of the ball henceforth increases by $2 \Delta x$. The equilibrium position, however, changes after each time period $\tau$. Now we find the position of the ball at the moment of time that is given in the problem, which would be $7 \tau$. During this time the ball moves down from the initial position once and then makes $3$ full oscillations and after them arriving down again. After reaching down from the initial position the position of the ball is $x_0 + 2 \Delta x $. Next the ball moves up to the position $x_0 - 4 \Delta x$ and again down to the position $x_0 + 6 \Delta x$. The absolute value of the distance from the initial position increases therefore by $2 \Delta x$ after each $\tau$. After $7 \tau$ the distance from the equilibrium position is therefore $7 \cdot 2 \Delta x = 14 \Delta x$. Like it was made clear before, the ball at the desired moment of time $7 \tau$ is therefore downwards from the initial position $x_0$ by $14 \Delta x = 14 \frac{Eq}{k}$.
\probend