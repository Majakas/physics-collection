\setAuthor{Siim Ainsaar}
\setRound{lõppvoor}
\setYear{2009}
\setNumber{G 9}
\setDifficulty{8}
\setTopic{Elektrostaatika}

\prob{Kosmoseprügi}
Kaks ühesugust elektriliselt laetud
kuuli, mis on ühendatud ideaalse nööriga, hõljuvad vabalt kosmoses. Kummagi
kera laeng on $q$ ja mass $M$, nööri pikkus on $l$.
Ootamatult lendab nööriga risti selle keskkoha pihta kosmoseprügi tükk massiga $m$ ja kiirusega
$v$ ning jääb nööri külge kinni. Millisele vähimale kaugusele $d$ lähenevad teineteisele kuulid?
Eeldada, et kuulikeste diameetrid on väiksemad kui otsitav kaugus $d$.

\hint
Hetkel, mil kuulide vahekaugus on minimaalne, peab massikeskme taustsüsteemis mõlemad kuulid ja kosmoseprügi paigal olema. Vastasel korral ei oleks kuulide vahekaugus minimaalne, sest kuulid liiguksid üksteise suhtes.

\solu
Massikeskme taustsüsteemis on kerade kaugus minimaalne hetkel, kui süsteem on
paigal. $d$ saame energia jäävuse seadusest, mis kehtib, kuna nöör puutehetkel
mutrile (sirgena) jõudu ei avalda ega muuda nii põrget plastseks.

Süsteemi masskeskme liikumiskiirus satelliidi süsteemis
\[ v_c = \frac{m v}{m+2M} \text, \]
mutri algkiirus masskeskme süsteemis
\[ w = v - v_c = v \left(1 - \frac{m}{m+2M}\right) = v \frac{2M}{m+2M} \text, \]
tehiskaaslase oma
\[ W = v_c = \frac{m v}{m+2M} \text. \]

Energia jäävus masskeskme taustsüsteemis on
\[
\frac{m w^2}{2} + \frac{2M W^2}{2} + \frac{k q^2}{l} = \frac{k q^2}{d} \text,
\]
kust
\begin{align*}
d &= \frac{k q^2}{\frac{m w^2}{2} + M W^2 + \frac{k q^2}{l}} =
\frac{k q^2}{\frac{m v^2}{2} \left(\frac{2M}{m+2M}\right)^2 + M v^2
	\left(\frac{m}{m+2M}\right)^2 + \frac{k q^2}{l}} = \\
&= \frac{1}{\frac{1}{l} + \frac{m M v^2}{k q^2 (m+2M)}}
\text.
\end{align*}

\vspace{0.5\baselineskip}

\emph{Alternatiivne lahendus}\\
Hetkel, kui keradevaheline kaugus on minimaalne, on satelliidi osad üksteise
suhtes paigal. Seega liigub süsteem sel hetkel nagu jäik keha. Võtame
inertsiaalse taustsüsteemi, kus tehiskaaslane oli enne kokkupõrget paigal, ja
tähistame süsteemi kiiruse minimaalse kauguse saavutamise hetkel kui $v_1$.

Impulsi jäävusest
\[ m v = (m+2M) v_1 \implies v_1 = \frac{m v}{m+2M}\text. \]

Kehtib ka energia jäävus.
\begin{align*}
\frac{m v^2}{2} + \frac{k q^2}{l} &= \frac{(m+2M) v_1^2}{2} + \frac{k q^2}{d}
\text, \\
\frac{m v^2}{2} + \frac{k q^2}{l} &= \frac{m^2 v^2}{2 (m+2M)} + \frac{k q^2}{d}
\text, \\
d &= \frac{k q^2}{\frac{m v^2}{2} + \frac{k q^2}{l} - \frac{m^2 v^2}{2 (m+2M)}}
= \frac{1}{\frac{1}{l} + \frac{m M v^2}{k q^2 (m+2M)}} \text.
\end{align*}
\probend