\setAuthor{Tundmatu autor}
\setRound{lahtine}
\setYear{2007}
\setNumber{G 1}
\setDifficulty{1}
\setTopic{Dünaamika}

\prob{Pallid}
Juku istub puu otsas ja laseb algkiiruseta lahti tema käes oleva palli. All seisab Juhan, kes samal hetkel viskab Juku pihta vertikaalselt üles täpselt samasuguse palli. Pärast pallide põrget jõuab Juku pall täpselt tema kõrgusele tagasi. Kas pall tabab Juhanit enne või pärast seda, kui Juku pall jõuab Jukuni? Lugeda, et pallide põrge on absoluutselt elastne.

\hint
Kuna pallid on samasuguse massiga ja tegu on elastse kokkupõrkega, vahetavad pallid oma kiirusvektorid. Seega võime sama hästi öelda, et pallid lähevad üksteisest vabalt läbi.

\solu
Ühesuguse massiga pallide elastse kokkupõrke tulemusena vahetavad nad oma
kiirusevektorid (järeldub lihtsalt impulsi ja energia jäävusest massikeskme süsteemis vaadatuna). Seega võime sama hästi öelda, et pallid lähevad üksteisest vabalt läbi, kusjuures ühe palli algkiirus ja teise palli lõppkiirus on võrdsed nulliga. Kui Juku istub kõrgusel $h$, siis saame mõlema vabalt liikuva palli jaoks lennuajaks $t = \sqrt{2h/g}$. Seega tabavad pallid viskajaid üheaegselt. 

\vspace{0.5\baselineskip}

\emph{Alternatiivne lahendus}\\
Kui Juku visatud pall jõuab täpselt tagasi oma algkõrgusele, siis peab tema kiiruse absoluutväärtus vahetult kokkupõrke eel ja vahetult kokkupõrke järel olema sama. Elastse kokkupõrke korral kehtib energia jäävus, järelikult võrdub ka Juhani palli kiiruse absoluutväärtus vahetult kokkupõrke eel kiiruse absoluutväärtusega vahetult kokkupõrke järel. Seega liigub kumbki pall tuldud suunas tagasi nii, et liikumise ajagraafik on peegelsümmeetriline põrkehetke suhtes. Seega, kui nad startisid samaaegselt, siis nad ka finišeerivad samaaegselt.
\probend