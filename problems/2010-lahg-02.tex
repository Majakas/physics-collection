\setAuthor{Oleg Košik}
\setRound{lahtine}
\setYear{2010}
\setNumber{G 2}
\setDifficulty{2}
\setTopic{Termodünaamika}

\prob{Rauatükk}
Anumasse, milles oli $V=\SI{1}{l}$ vett temperatuuril $t_1=\SI{20}{\degreeCelsius}$,
visati rauatükk massiga $m=\SI{100}{g}$ temperatuuril $t_0=\SI{500}{\degreeCelsius}$. Osa
veest aurustus. Mõne aja pärast mõõdeti vee temperatuuriks
$t_2=\SI{24}{\degreeCelsius}$. Kui palju vett aurustus? Vee erisoojus
$c_1=\SI{4200}{J/(kg\cdot\degreeCelsius)}$, aurustumissoojus $L=\SI{2,26 e6}{J/kg}$ ja
tihedus $\rho=\SI{1000}{kg/m^3}$; raua
erisoojus $c_2=\SI{460}{J/(kg\cdot\degreeCelsius)}$. Anum on tühise soojusmahtuvusega ning väliskeskkonnast
hästi isoleeritud.

\hint
Energia jäävuse seaduse kohaselt peab rauatükist eraldunud soojusenergia minema vee soojendamiseks ja aurustumiseks.

\solu
Algne vee mass on $M=\rho V=\SI{1000}{g}$. Olgu väljaaurustunud vee mass $m_0$. Selle soojendamiseks keemistemperatuurini $t=\SI{100}{\degreeCelsius}$ ning aurustamiseks läheb vaja energiat
\[
Q_1=m_0(c_1(t-t_1)+L).
\]
Ülejäänud vee soojendamiseks temperatuurini $t_2$ läheb energiat
\[
Q_2=(M-m_0)c_1(t_2-t_1).
\]
Raua jahtumisel eraldub energia $Q_3=mc_2(t_0-t_2)$. Energia jäävuse seaduse kohaselt $Q_1+Q_2=Q_3$, ehk
\[
m_0(c_1(t-t_1)+L)+(M-m_0)c_1(t_2-t_1)=mc_2(t_0-t_2).
\]
Siit
\[
m_0=\frac{mc_2(t_0-t_2)-Mc_1(t_2-t_1)}{c_1(t-t_2)+L}\approx \SI{2}{g}.
\]
\probend