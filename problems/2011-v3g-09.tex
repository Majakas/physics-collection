\setAuthor{Andres Laan}
\setRound{lõppvoor}
\setYear{2011}
\setNumber{G 9}
\setDifficulty{9}
\setTopic{Elektriahelad}

\prob{Närvirakk}
Närviraku membraani võib vaadelda kui õhukest kilet mahtuvusega $C$, mida läbivad ioonkanalid, mis võimaldavad laengutel liikuda läbi
membraani. Närviraku elektrilise tasakaalu seisukohast on olulisteks ioonideks
naatrium ja kaalium. Kui naatriumioon (laenguga $+e$) läbib ioonkanali (sisenedes närvirakku), siis sooritavad keemilised jõud töö $e\mathcal{E}_{\mathrm{Na}}$, st võib öelda, et
naatriumioonidele mõjub ioonkanalis elektromotoorjõud $\mathcal{E}_{\mathrm{Na}}$. Kaaliumioonide
puhul on kanali läbimise protsess täpselt samasugune, kuid efektiivne elektromotoorjõud on sel puhul $\mathcal{E}_{\mathrm{K}}$ ($\neq \mathcal{E}_{\mathrm{Na}}$). Peale keemiliste jõudude töö toimivad
laengu liikumisel ioonkanalis ka hõõrdejõud, mida saab kirjeldada elektrilise
takistuse abil: naatriumioonide jaoks on membraani elektriline takistus $R_{\mathrm{Na}}$ ja kaaliumioonide jaoks $R_{\mathrm{K}}$. Millise laengu omandab närviraku membraan
elektrilise tasakaalu saabudes? 

\hint
Ülesande peamisi raskusi on korrektselt määrata ülesande tekstiga ekvivalentne elektriskeem. Peale elektriskeemi määramist taandub ülesanne stabiilse režiimi leidmisele. See on tehtav näiteks Kirchhoffi seadustega.

\solu
% Joonis
Kuna laengud saavad voolata üle membraani kolme eri teed mööda ja kondensaatorile kogunev laeng põhjustab kõigile kolmele teele ühiselt mõjuva elektrostaatilise
pinge U, siis on meil närviraku mudeldamiseks sobiv skeem, kus meil on rööbiti kolm
vooluteed: kondensaatori voolutee, kaaliumi voolutee ja naatriumi voolutee

% Joonis

Kui saabub tasakaal, ei lähe voolu läbi kondensaatori. Selleks peab kaaliumi ja naatriumi voolu summa olema elektriliselt neutraalne. Kaaliumi vool on $(\mathcal{E}_{\mathrm{K}} - U)/R_\mathrm{K}$.
Naatriumi vool on $(\mathcal{E}_{\mathrm{Na}} - U)/R_{\mathrm{Na}}$. Võrrutades nende voolude summa nulliga saame
pinge avaldiseks
\[
U=\frac{R_{\mathrm{K}} \mathcal{E}_{\mathrm{Na}}+R_{\mathrm{Na}} \mathcal{E}_{\mathrm{K}}}{R_{\mathrm{Na}}+R_{\mathrm{K}}}.
\]
Membraani kogulaeng on siis
\[
q=C U=C \frac{R_{\mathrm{K}} \mathcal{E}_{\mathrm{Na}}+R_{\mathrm{Na}} \mathcal{E}_{\mathrm{K}}}{R_{\mathrm{Na}}+R_{\mathrm{K}}}.
\]
\probend