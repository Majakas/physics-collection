\setAuthor{Siim Ainsaar}
\setRound{piirkonnavoor}
\setYear{2009}
\setNumber{G 10}
\setDifficulty{8}
\setTopic{Taevamehaanika}

\prob{Kuukaabel}
Oletame, et Maa ja Kuu on ühendatud sirge homogeense mõlema suhtes radiaalse kaabliga.\\
\osa Mitu korda on Maa poolt kaablile avaldatav ras\-kus\-jõud suurem
Kuu-pool\-sest?\\
\osa Maa pinnal asuv kaabli kinnitus sellele vertikaalsihis jõudu ei avalda.
Kui kõrgel Kuu kohal asub punkt, kus pisut liiga nõrk kaabel katkeks?

Lugegem taevakehad paigalseisvaiks.
Maa raadius $r_M=\SI{6370}{km}$, Kuu raadius $r_K=\SI{1740}{km}$,
Maa mass $m_M=\SI{5,97e24}{kg}$, Kuu mass $m_K=\SI{7,35e22}{kg}$, taevakehade keskmete vahekaugus $D=\SI{3,80e5}{km}$.
%, gravitatsioonikonstant $G = \unit{6,67}{\newton \cdot
%\squaren\meter \per \squaren\kilogram}$.
%Kaabli joontihedus olgu $\lambda=\unit{7850}{\kilo\gram \per \meter}$.

\emph{Abivalem}. Kui kaablit tõmbaks Maa üksi ning otspunktide kaugused Maa
tsentrist oleksid $a$ ja $b$, mõjuks sellele raskusjõud $G m_M \lambda
\left( \tfrac1a - \tfrac1b \right)$,
kus $G$ on gravitatsioonikonstant ning $\lambda$ kaabli joontihedus (ühikuga \si{kg/m}).

\hint
\osa Kuu ja Maa poolt avaldatavad jõud on otseselt leitavad abivalemi kaudu.\\
\osa Kaabli katkemispunktis on pinge maksimaalne. Pinge on leitav jõudude tasakaalust.

\solu
\osa
Maa avaldab kaablile jõudu
\[ 
F_M = G m_M \lambda \left( \frac{1}{r_M} - \frac{1}{D - r_K} \right),
\]
analoogiliselt Kuu,
\[ 
F_K = G m_K \lambda \left( \frac{1}{r_K} - \frac{1}{D - r_M} \right).
\]
Suhe on seega
\[ \frac{F_M}{F_K} = \frac{ m_M \left( \frac{1}{r_M} - \frac{1}{D -
		r_K} \right) }{ m_K \left( \frac{1}{r_K} - \frac{1}{D - r_M} \right) } \approx
\num{21,9}.\]

\osa
Leiame kaablit pingutava jõu $T(x)$ Kuu keskmest mingil kaugusel $x$. Sellest
kaugusest Maa-poolset kaabliosa mõjutavad kolm jõudu: kaabli pinge $T(x)$ ning
Maa ja Kuu poolt avaldatavad raskusjõud. ($x$-st Kuu-poolset osa mõjutab ka
otsa Kuu küljes hoidev jõud, selle arvutamiseks pole tarvidust.) Nimetatud jõud on
tasakaalus, mistõttu
\[ 
T(x) = G\lambda m_M \left( \frac{1}{r_M} - \frac{1}{D-x} \right) -
G\lambda m_K \left( \frac{1}{x} - \frac{1}{D-r_M} \right) \text. 
\]

Kaabel katkeks sealt, kus pinge on tugevaim, seega lahendame
ekst\-ree\-mum\-üles\-an\-de
ja leiame $T(x)$ maksimumi. Seal tuletis $T'(x)=0$. Kui saame ainult ühe mõistliku lahendi, pole
ekstreemumi liigi määramiseks teist tuletist vajagi: teame, et kosmoses on $T$
suurem kui kaabli otstes (taevakehapinnalt kaabliosade eralduspunkti
eemaldades kasvab lähema keha tõmme alumisele, \enquote{tekkivale} kaablipoolele ilmselt kiiremini
kui kahaneb teise keha tõmme ülemisele; osade suhtelised massid muutuvad
oluliselt erineva kiirusega), tänu millele (vähemalt) üks maksimum eksisteerib.
\[ 
T'(x) = - \frac{ G\lambda m_M }{ (D-x)^2 } + \frac{ G\lambda m_K }{ x^2 } = 0
\implies x = \frac{D}{1 \pm \sqrt{\frac{m_M}{m_K}}}.
\]
Miinusmärgiga lahend on negatiivsena mittefüüsikaline, seega otsitavaks
kõr\-gu\-seks osutub
\[ 
h = \frac{D}{1 + \sqrt{\frac{m_M}{m_K}}} - r_K \approx
\SI{36200}{km}.
\]
\probend