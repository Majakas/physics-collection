\setAuthor{Jaan Kalda}
\setRound{lõppvoor}
\setYear{2008}
\setNumber{G 4}
\setDifficulty{5}
\setTopic{Varia}

\prob{Maja}
Fotol kujutatud maja alumise korruse kõrgus (mõõdetuna esimese korruse akna alumisest servast teise korruse akna alumise servani) on 3 meetrit. Kui kõrgel veepinnast on maja (täpsemalt, tema vundamendi ülemine serv)?

\begin{center}
	\includegraphics[height=\textheight]{2008-v3g-04-yl}
\end{center}

\hint
Maja teatud punkt ja tema peegelkujutis mere pinnalt paiknevad sümmeetriliselt mere tasandiga võrreldes. See võimaldab vee tasandi leidmist.

\solu
Maja teatud punkt $P$ ja tema peegelkujutis mere pinnalt $P'$ paiknevad sümmeetriliselt mere tasandiga. Vaatleme mõttelist sirget $PP'$. Tema lõikepunkt merega $O$ paikneb mõlemast otsast võrdsel kaugusel ning tänu sellele saame me jooniselt punkti $O$ kergelt määrata kui lõigu $PP'$ keskpunkti. Maja kõrgus merepinnalt vastab vundamendi kaugusele punktist $O$, vt joonist. Mõõtes jooniselt akende vahekauguse $|AB| = \SI{9,5}{mm}$ ja $|OQ| = \SI{58,5}{mm}$ saame
\[
H = \SI{3}{m} \cdot \frac{|OQ|}{|AB|} = \SI{18,5}{m}.
\]

\begin{center}
	\includegraphics[height=\textheight]{2008-v3g-04-lah}
\end{center}
\probend