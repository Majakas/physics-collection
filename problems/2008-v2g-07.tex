\setAuthor{Jaan Kalda}
\setRound{piirkonnavoor}
\setYear{2008}
\setNumber{G 7}
\setDifficulty{5}
\setTopic{Elektrostaatika}

\prob{Kuup}
Õhukesest elektrit mittejuhtivast materjalist on valmistatud kuup küljepikkusega $a$. Kuubil on elektrilaeng ühtlase pindtihedusega $\sigma$ (pindtihedus on laeng pinnaühiku kohta). Ühe tahu keskkohta lõigatakse väike ruudukujuline auk mõõtmetega $b \times b$ ($b \ll a$). Leida elektrivälja tugevus kuubi keskpunktis.

\hint
Kasulikuks võib osutada superpositsiooniprintsiip, mille kohaselt võib välja lõigatud ruutu tekitatud välja leida kui $+$ ja $-$ laenguga ruutude väljade summana.

\solu
Kui auku ei oleks, oleks väljatugevus sümmeetria tõttu 0. Antud olukord on ekvivalente auguta kuubi ja negatiivse pindlaenguga $b\times b$ ruudu superpositsiooniga. Negatiivne ruut moodustab laengu $q = -\sigma b^2$ ning tekitab kuubi keskel väljatugevuse
\[
E=\frac{\sigma b^{2}}{4\pi \epsilon_{0} \left(\frac{a}{2}\right)^{2}} = \frac{\sigma b^{2}}{\pi \epsilon_{0} a^{2}}.
\]
\probend