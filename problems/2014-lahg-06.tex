\ylDisplay{Klots vedelikes} % Ülesande nimi
{Erkki Tempel} % Autor
{lahtine} % Voor
{2014} % Aasta
{G 6} % Ülesande nr.
{5} % Raskustase
{
% Teema: Vedelike-mehaanika
\ifStatement
Silindrilises anumas põhja pindalaga $S$ on kaks mittesegunevat vedelikku tihedustega $\rho_1$ ja $\rho_2$. Anumasse asetatakse kuubikujuline klots ruumalaga $V$ ning tihedusega $\rho_k$ ($\rho_1>\rho_k>\rho_2$). Klots on täielikult vedelike sees ega puuduta anuma põhja.\\
\osa Kui suur osa klotsist asub alumises vedelikus?\\
\osa Kui palju muutub kahe vedeliku eralduspinna kõrgus pärast klotsi asetamist anumasse?
\fi


\ifHint
Kuna klots on tasakaalus, peab kuubile mõjuv summaarne üleslükkejõud olema võrdne gravitatsioonijõuga. Selle jaoks on mugav võtta alumisse vedelikku jäänud kuubi osa ruumala tundmatuna.
\fi


\ifSolution
Kuna klotsi tihedus on ülemisest vedelikust suurem ning alumisest vedelikust väiksem, jääb klots kahe vedeliku piirpinnale ujuma. Klotsile mõjub sellisel juhul klotsi raskusjõud $F_r = m\idx{klots}g=\rho_k Vg$, ning vedelike üleslükkejõud
\[ F_y=F_{y1}+F_{y2}=\rho_1gV_x + \rho_2g(V-V_x), \]
kus $V_x$ on alumises vedelikus oleva klotsi ruumala ning $V-V_x$ on ülemises vedelikus oleva klotsi ruumala.
Raskusjõud ja üleslükkejõud on võrdsed, seega saame võrrandi
\[ \rho_k Vg = \rho_1gV_x + \rho_2g(V-V_x). \]
Sellest saab avaldada 
\[ V_x = \frac{V(\rho_k-\rho_2)}{\rho_1-\rho_2}. \]
Alumise vedeliku nivoo tõuseb ruumala $V_x$ võrra. Kuna anuma põhjapindala on $S$, siis tõuseb vedelike eraldusnivoo $\Delta h$ võrra, kusjuures
\[ \Delta h = \frac{V_x}{S}. \]
Asendades siia $V_x$, saame
\[ \Delta h = \frac{V(\rho_k-\rho_2)}{S(\rho_1-\rho_2)}. \]
\fi


\ifEngStatement
% Problem name: Block in liquids
In a cylindrical vessel with a bottom area $S$ there are two non-miscible liquids with densities $\rho_1$ and $\rho_2$. A cube-shaped block of volume $V$ and density $\rho_k$ ($\rho_1>\rho_k>\rho_2$) is placed inside the vessel. The block is completely inside the liquids and does not touch the bottom of the vessel.\\
\osa How big part of the block is located in the bottom liquid?\\
\osa How much does the height of the separating surface of the two liquids change after placing the block inside the vessel?
\fi


\ifEngHint
Because the block is in equilibrium then the total buoyancy force applied to the cube has to be equal to the gravity force. For that it is convenient to consider the part of the cube left in the bottom liquid to be unknown.
\fi


\ifEngSolution
Because the block’s density is bigger than the upper liquid’s density and smaller than the bottom liquid’s density the block will stay swimming at the separating surface between the two liquids. In this case the block is applied with the block’s gravity force $F_g = m\idx{block}g=\rho_k Vg$ and the buoyancy force of the liquids 
\[ F_y=F_{y1}+F_{y2}=\rho_1gV_x + \rho_2g(V-V_x), \]
where $V_x$ is the volume of the block in the bottom liquid and $V-V_x$ the volume of the block in the upper liquid. The gravity force and the buoyancy forces are even, therefore we get an equation 
\[ \rho_k Vg = \rho_1gV_x + \rho_2g(V-V_x). \]
From this we can express
\[ V_x = \frac{V(\rho_k-\rho_2)}{\rho_1-\rho_2}. \]
The level of the bottom liquid rises by the volume $V_x$. Because the area of the vessel’s base is $S$ then the level of the interface between the liquids rises by $\Delta h$, moreover
\[ \Delta h = \frac{V_x}{S}. \]
Replacing $V_x$ here we get
\[ \Delta h = \frac{V(\rho_k-\rho_2)}{S(\rho_1-\rho_2)}. \]
\fi
}