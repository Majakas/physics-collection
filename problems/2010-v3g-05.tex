\ylDisplay{Toru} % Ülesande nimi
{Aigar Vaigu} % Autor
{lõppvoor} % Voor
{2010} % Aasta
{G 5} % Ülesande nr.
{6} % Raskustase
{
% Teema: Staatika
\ifStatement
Kareda horisontaalselt kinnitatud toru (raadius $R$) peal tasakaalustatakse risttahukakujulist prussi. Leidke prussi paksus $L$, mille
korral prussi asend torul on stabiilne.

\emph{Märkus.} Võivad olla kasulikud väikeste nurkade korral kehtivad lähendused
$\sin\alpha\approx \alpha$ ja $\cos\alpha\approx 1-\alpha^2/2$, kus nurgad on radiaanides.
\fi


\ifHint
Prussi asend on stabiilne, kui väikse kõrvalekalde korral muutub prussi potentsiaalne energia suuremaks. Teisisõnu peab massikese kõrgemale tõusma.
\fi


\ifSolution
Prussi võnkumine torul on stabiilne, kui prussi
kõrvalekallutamisel väikese nurga $\alpha$ võrra prussi masskese
tõuseb kõrgemale, kui alguses. Esialgne prussi massikeskme kõrgus on $R+L/2$.
Masskeskme kõrgus kõrvalekallutamisel on
\[\left(R+{L \over 2}\right)\cos\alpha+R\alpha\sin\alpha,\]
ning peab kehtima
\[\left(R+{L \over 2}\right)\cos\alpha+R\alpha\sin\alpha >R+{L \over 2}.\]
Kuna kõrvalekalde nurk on väike, siis võime arvestada, et
$\sin\alpha\approx \alpha$ ja $\cos\alpha\approx 1-\alpha^2/2$. Lihtsustades ning avaldades $L$-i, saame, et võnkumised on väikeste kõrvalekallete korral stabiilsed, kui
\[L<2R.\]
\fi
}