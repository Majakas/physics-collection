\setAuthor{Aigar Vaigu}
\setRound{lõppvoor}
\setYear{2010}
\setNumber{G 5}
\setDifficulty{6}
\setTopic{Staatika}

\prob{Toru}
Kareda horisontaalselt kinnitatud toru (raadius $R$) peal tasakaalustatakse risttahukakujulist prussi. Leidke prussi paksus $L$, mille
korral prussi asend torul on stabiilne.

\emph{Märkus:} kasulikud võivad olla väikeste nurkade korral kehtivad lähendused
$\sin\alpha\approx \alpha$ ja $\cos\alpha\approx 1-\alpha^2/2$, kus nurgad on radiaanides.

\hint
Prussi asend on stabiilne, kui väikse kõrvalekalde korral muutub prussi potentsiaalne energia suuremaks. Teisisõnu peab massikese kõrgemale tõusma.

\solu
Prussi võnkumine torul on stabiilne, kui prussi
kõrvalekallutamisel väikese nurga $\alpha$ võrra prussi massikese
tõuseb kõrgemale kui alguses. Esialgne prussi massikeskme kõrgus on $R+L/2$.
Massikeskme kõrgus kõrvalekallutamisel on
\[\left(R+{L \over 2}\right)\cos\alpha+R\alpha\sin\alpha,\]
ning peab kehtima
\[\left(R+{L \over 2}\right)\cos\alpha+R\alpha\sin\alpha >R+{L \over 2}.\]
Kuna kõrvalekalde nurk on väike, siis võime arvestada, et
$\sin\alpha\approx \alpha$ ja $\cos\alpha\approx 1-\alpha^2/2$. Lihtsustades ning avaldades $L$-i, saame, et võnkumised on väikeste kõrvalekallete korral stabiilsed, kui
\[L<2R.\]
\probend