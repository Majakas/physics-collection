\setAuthor{Urmo Visk}
\setRound{piirkonnavoor}
\setYear{2010}
\setNumber{G 2}
\setDifficulty{2}
\setTopic{Termodünaamika}

\prob{Jõhvikad}
Keevasse vette kallatakse külmutatud jõhvikaid. Vee temperatuur langes väärtuseni $t=\SI{89}{\degreeCelsius}$.
Mitu korda oli vee mass suurem jõhvikate massist? Kuna jõhvikad olid väikesed ja sulasid väga kiiresti, siis
võib vee soojusvahetuse ümbritseva keskkonnaga arvestamata jätta. Jõhvikate algtemperatuur oli
$t_2=\SI{-18}{\degreeCelsius}$. Jää erisoojus $c_j=\SI{2100}{J/(kg.C)}$, vee erisoojus $c_v=\SI{4200}{J/(kg.\degreeCelsius)}$, jää sulamissoojus $L=\SI{330}{kJ/kg}$. Jõhvikate suure veesisalduse tõttu võib neid käsitleda jääna.

\hint
Energia jäävuse seaduse kohaselt peab jõhvikate soojendamiseks kuluv soojushulk tulema vee jahtumise arvelt.

\solu
Vee algtemperatuur oli $t_1=\SI{100}{\degreeCelsius}$. Olgu vee ja jõhvikate massid vastavalt $M$ ja $m$. Jõhvikate soojendamiseks kuluv soojushulk tuleb vee jahtumise arvelt. Vee jahtumisel eralduv soojushulk oli
\[
Q_j=Mc_V(t_1-t).
\]
Jõhvikate soojendamise käigus tuli 1) soojendada külmunud jõhvikad sulamistemperatuurini, 2) sulatada külmunud jõhvikad ja 3) soojendada sulanud jõhvikad vee temperatuurini.
Leiame igas etapis kulunud soojushulga:
\[
\begin{aligned}
Q_{s1}&=mc_j(0-t_2)=-mc_jt_2,\\
Q_{s2}&=mL,\\
Q_{s3}&=mc_v(t-0)=mc_vt.
\end{aligned}
\]
Liidame jõhvikate soojendamiseks kulunud soojushulgad ja võrdsustame saadud summa vee jahtumisel eraldunud soojushulgaga. Saadud võrrandist avaldame vee ja jõhvikate masside suhte:
\[
-mc_jt_2+mL+mc_vt=Mc_V(t_1-t),
\]
kust
\[
\frac{M}{m}=\frac{-c_jt_2+L+c_vt}{c_V(t_1-t)}.
\]
Arvuliseks vastuseks saame \num{16}.
\probend