\ylDisplay{Kondensaator} % Ülesande nimi
{Aigar Vaigu} % Autor
{piirkonnavoor} % Voor
{2010} % Aasta
{G 7} % Ülesande nr.
{5} % Raskustase
{
% Teema: Elektriahelad
\ifStatement
Muudetava mahtuvusega kondensaator on ühendatud patareiga,
mille klemmide pinge on $U$. Kondensaatori mahtuvust muudetakse laadimisel nii, et
kondensaatori laadimise vool $I$ on konstantne. Leidke patarei võimsus ja kondensaatori laadimisel energia salvestamise kiirus.
Põhjendage võimalikku erinevust.
\fi


\ifHint
Patarei tehtud töö ajaühikus $\Delta t$ on $UI\Delta t$. Kondensaatorisse salvestava energia muutumise kiirus on leitav võttes kondensaatori siseenergiast $\frac{CU^2}{2}$ aja järgi tuletise.
\fi


\ifSolution
Patarei pinge on $U=\text{const}$ ja vool ahelas, vastavalt ülesande tingimustele, on $I=\text{const}$. Patarei võimsus on
\[P_p=UI. \]
Energia kondensaatoris on
\[E=\frac{CU^2}{2}, \]
kus kondensaatori mahtuvus on $C=q/U$ ja laeng kondensaatoris $q$.
Kondensaatorisse energia salvestamise kiirus on energia muutumise kiirus kondensaatoris ehk energia tuletis aja järgi,
\[P_C=\frac{\D E}{\D t}=\frac{\D C}{\D t}\frac{U^2}{2}=\frac{\D (CU)}{\D t}\frac{U}{2}=\frac{\D q}{\D t}\frac{U}{2}.\]
Arvestades, et laengu muutumise kiirus $\D q/\D t$ on vool $I$, saame energia salvestumise kiiruseks kondensaatorisse
\[P_C=\frac{UI}{2}.\]
Näeme, et patareist \enquote{väljub} energiat kaks korda kiiremini, kui seda salvestub kondensaatorisse. Energia, mis ajaühikus kondensaatorisse ei jõua läheb välisjõudude, mis muudavad kondensaatori mahtuvust selliselt, et $I=\text{const}$, vastu töö tegemiseks.
\fi
}