\ylDisplay{Pendel} % Ülesande nimi
{Taavi Pungas} % Autor
{piirkonnavoor} % Voor
{2011} % Aasta
{G 7} % Ülesande nr.
{3} % Raskustase
{
% Teema: Dünaamika
\ifStatement
Pendel pandi väikese amplituudiga võnkuma ning stopperiga registreeriti neid hetki, kui pendel läbis vasakult poolt tulles oma tasakaalupunkti. Kaks järjestikust sellist sündmust toimusid hetkedel $t_1=\SI{3,19}{s}$ ja $t_2=\SI{5,64}{s}$. Pendlil lasti mõnda aega segamatult võnkuda, seejärel saadi kaheks järjestikuseks näiduks $t_3=\SI{61,14}{s}$ ja $t_4=\SI{63,54}{s}$. Leidke võimalikult täpselt pendli võnkeperiood ning hinnata selle mõõtemääramatust.
\fi


\ifHint
Pendli perioodi leidmiseks on võimalik teha esialgne jäme hinnang, mis põhineb järjestikustel mõõtmistel. Täpsema hinnangu jaoks võib kasutada esialgset jämedat hinnangut ja pikemat ajavahemikku, et määrata täpselt mitu võnget antud ajavahemiku sisse mahub.
\fi


\ifSolution
Esialgse hinnangu perioodile, $\tau = \SI{2,425}{s}$, saame $\tau_1 = t_2 - t_1$ ja $\tau_2 = t_4 - t_3$
keskmisest. Seda kasutades näeme, et $t_1$ ja $t_3$ vahel pidi toimuma täpselt \num{24} võnget,
samamoodi $t_2$ ja $t_4$ vahel. Saame kaks sõltumatut mõõtmist \num{24} võnke kestuse kohta:
$\tau_1' = (t_3-t_1)/\num{24} = \SI{2,4146}{s}$ ja $\tau_2'= (t_4-t_2)/\num{24} = \SI{2,4125}{s}$. Nende keskmine annab
meie hinnangu pendli perioodi kohta, $\tau' = \SI{2,4135}{s} \approx \SI{2,414}{s}$.
\fi
}