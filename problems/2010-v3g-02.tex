\ylDisplay{Destillaator} % Ülesande nimi
{Koit Timpmann} % Autor
{lõppvoor} % Voor
{2010} % Aasta
{G 2} % Ülesande nr.
{4} % Raskustase
{
% Teema: Termodünaamika
\ifStatement
Destillaator toodab tunnis $V=\SI{2}{l}$ puhast vett. Sisenev aur ja kondenseerunud vesi on samal temperatuuril. Auru kondenseerumisel vabanenud soojusest kulub $\eta=\SI{95}{\%}$ jahutusvee soojendamiseks. Jahutussüsteem kujutab endast pikka toru, milles voolab jahutusvesi. Toru ristlõikepindala on $S=\SI{0,8}{cm^2}$. Destillaatorisse siseneva ja sealt väljuva jahutusvee temperatuurid erinevad $\Delta T=\SI{30}{\celsius}$ võrra. Kui kiiresti peab vesi voolama jahutussüsteemis? Vee aurustumissoojus $L=\SI{2300}{kJ/kg}$, vee erisoojus $c = \SI{4,2}{kJ/(kg.K)}$ ja vee tihedus $\rho=\SI{1000}{kg/m^3}$.
\fi


\ifHint
Süsteemis kehtib energia jäävuse seadus. Nimelt ühe tunni jooksul läheb \SI{95}{\%} kahe liitri aurustumisel eraldunud energiast tundmatu massiga jahutusvee soojendamiseks $\Delta T$ võrra. Jahutusvee kiirus on leitav toru massivoo ja ristlõike pindala kaudu.
\fi


\ifSolution
Kahe liitri vee mass on $m=\SI 2{kg}$. Kondenseerudes eraldub soojushulk $Q=Lm$.
95\% eraldunud soojushulgast laheb jahutusvee soojendamiseks. Seosest $\eta Lm=cM\Delta T$ saame jahutusvee massi
\[M=\frac {\eta L m}{c\Delta T}.\]
Jahutusvee massi saame avaldada tiheduse ja ruumala kaudu ning ruumala omakorda toru ristlõikepindala, voolu kiiruse ja aja kaudu:
\[M=\rho V=\rho Sl=\rho Svt.\]
Viies kokku need kaks võrrandit, saame avaldada kiiruse:
\[v=\frac {\eta Lm}{c\Delta T \rho S t}\approx \SI{0,12}{m/s}.\]
\fi
}