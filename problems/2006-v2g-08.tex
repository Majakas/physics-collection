\setAuthor{Aigar Vaigu}
\setRound{piirkonnavoor}
\setYear{2006}
\setNumber{G 8}
\setDifficulty{4}
\setTopic{Elektrostaatika}

\prob{Tolmukübe}
Tolmukübe massiga $m = \SI{1e-9}{g}$ on kondensaatori horisontaalsete plaatide vahel tasakaalus. Laengu pindtihedus kondensaatori plaatidel $\sigma = \SI{2,6e-5}{C/m^2}$. Kui suur on tolmukübeme elektrilaeng? Millise kiirendusega hakkaks tolmukübe langema, kui kondensaatori polaarsus muuta vastupidiseks? Eeldada, et elektriväli kondensaatori plaatide vahel on homogeenne. Õhutakistust mitte arvestada. Elektriline konstant $\varepsilon_0 = \SI{8,85e-12}{F/m}$, õhu dielektriline läbitavus $\varepsilon \approx 1$.

\hint
Tolmukübemel kehtib jõudude tasakaal raskusjõu ja elektrostaatilise jõu vahel.

\solu
Olgu plaatide pindala $S$, plaatide vaheline kauguse $d$, kondensaatori mahtuvus ja pinge vastavalt $C$ ja $U$. Avaldame elektrivälja tugevuse kondensaatori plaatide vahel:
\[
C=\frac{\varepsilon \varepsilon_{0} S}{d} \quad \text{ja} \quad C=\frac{q}{U},
\]
kust
\[
E=\frac{q}{\varepsilon \varepsilon_{0} S}.
\]
Arvestades, et pindtihedus $\sigma = q/S$, saame
\[
E = \frac{\sigma}{\varepsilon\varepsilon_0}.
\]
Tolmukübemele mõjub raskusjõud $mg$ ja elektrostaatiline jõud $Eq$. Tasakaalu korral $mg = Eq$. Tolmukübeme laeng on seega
\[
q=\frac{m g}{E}=m g \frac{\varepsilon \varepsilon_{0}}{\sigma} \approx \SI{0,33e-17}{C}.
\]
Kui polaarsust muuta, siis mõjub tolmukübemele jõud
\[
F = mg + Eq = 2mg.
\]
Kiirendus, millega tolmukübe hakkab langema on $a = 2g$.
\probend