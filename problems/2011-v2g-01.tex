\ylDisplay{Kokkupõrge} % Ülesande nimi
{Tundmatu autor} % Autor
{piirkonnavoor} % Voor
{2011} % Aasta
{G 1} % Ülesande nr.
{1} % Raskustase
{
% Teema: Dünaamika
\ifStatement
Kaks autot massidega $m=\SI{1,5}{}$ tonni teevad laupkokkupõrke, mille võib lugeda täielikult plastseks. Kui suur energia kulus purustuste tekitamiseks, kui:\\
\osa mõlema auto kiirus oli $v_a=\SI{50}{\kilo\metre\per\hour}$; \\
\osa üks auto seisis paigal ja teise auto kiirus oli $v_b=\SI{100}{\kilo\metre\per\hour}$?\\
(Võib arvestada, et autode lohisemisel pärast põrget olulist kahju ei teki.)
\fi


\ifHint
\osa Energia jäävuse seaduse kohaselt kulub purustuse tekitamiseks esialgse ja pärastise kineetiliste energiate vahe.\\
\osa Taustsüsteemide vahetamine lihtsustab olukorda oluliselt.
\fi


\ifSolution
\osa Autode kiirused on võrdsed ja vastassuunalised. Seetõttu on koguimpulss võrdne nulliga ja autod jäävad pärast kokkupõrget paigale. Kogu esialgne kineetiline energia kulub purustuste tekitamiseks. Selleks on $2 \frac{m v_{a}^{2}}{2}=m v_{a}^{2}$. Autode kiirused on \SI{50}{km/h} = \SI{13,9}{m/s} ja koguenergia on \SI{289}{kJ}.

\osa Minnes üle massikeskme taustsüsteemi näeme, et olukord taandub eelmiseks olukorraks, seega kokkupõrke koguenergia on \SI{289}{kJ}.

\fi
}