\ylDisplay{Rõngas} % Ülesande nimi
{Taavi Pungas} % Autor
{piirkonnavoor} % Voor
{2014} % Aasta
{G 6} % Ülesande nr.
{4} % Raskustase
{
% Teema: Staatika
\ifStatement
Lae külge on nööriga, mille pikkus on $L$, kinnitatud kerge plastmassrõngas raadiusega $R$, mille küljes on omakorda raske metallist mutter. Mutrit saab mööda rõngast libistada. Rõnga ja mutri vaheline hõõrdetegur on $\mu$. Juku tahab mutrit mööda rõngast nihutades saavutada olukorda, kus mutri ja lae vahekaugus $h$ oleks võimalikult väike, aga süsteem püsiks veel ilma välise sekkumiseta tasakaalus. Leidke vähim vahekaugus $h\idx{min}$, mille Juku võib saavutada. Eeldage, et rõnga mass on mutri omaga võrreldes tühiselt väike.
\fi


\ifHint
Selleks, et süsteem oleks tasakaalus, peab mutter asuma täpselt nööri kinnituspunkti all.
\fi


\ifSolution
Et süsteem oleks tasakaalus, peab mutter asuma täpselt nööri kinnituspunkti all. Teiseks: maksimaalse mutri kõrguse korral on rõnga kaldenurk mutri asukohas $\alpha$, kus $\tan \alpha = \mu$ - siis on mutter täpselt libisemise piiril. Sel juhul on ka mutrini tõmmatud raadiuse ja vertikaali vahel nurk $\alpha$. Tekkinud kolmnurgast näeme, et $h=L+2R\cos \alpha = L+ \frac{2R}{\sqrt{1+\mu^2}}$.
\fi


\ifEngStatement
% Problem name: Ring
A plastic ring of radius $L$ is attached to the ceiling with a rope of length $R$. A heavy metal nut is in turn attached to the ring. The nut can be slided along the ring. The coefficient of friction between the nut and the ring is $\mu$. While sliding the nut along the ring Juku wants to achieve a situation where the distance $h$ between the nut and the ceiling would be as small as possible but that the system would still be in equilibrium. Find the smallest distance $h\idx{min}$ that Juku can achieve. Assume that the mass of the ring is insignificantly small compared to the mass of the nut.
\fi


\ifEngHint
For the system to be in equilibrium the nut has to be located exactly below the attachment point of the rope.
\fi


\ifEngSolution
For the system to be in equilibrium the nut has to be located exactly below the rope’s attachment point. Secondly: in the case of the nut’s maximal height the circle’s angle of inclination in the nut’s location is $\alpha$, where $\tan \alpha = \mu$ – then the nut is exactly on the verge of slipping. In this case the angle between the radius drawn up to the nut and the vertical is also $\alpha$. From the formed triangle we see that $h=L+2R\cos \alpha = L+ \frac{2R}{\sqrt{1+\mu^2}}$.
\fi
}