\setAuthor{Joonas Kalda}
\setRound{piirkonnavoor}
\setYear{2016}
\setNumber{G 4}
\setDifficulty{3}
\setTopic{Dünaamika}

\prob{Kelk}
\begin{wrapfigure}[2]{r}{0.3\textwidth}
	\vspace{-12pt}
	\begin{resizebox}{\linewidth}{!}{
	\begin{tikzpicture}
	\coordinate (C) at (0.7,-0.1);
	\draw (0,0) -- ++(180-15:1.5);
	\draw (0,0) to [out = -15, in = 180] (C);
	\draw (C) to ++(0:1.5);
	\draw (180-15:1.5) -- ++(-15:0.2) -- ++(90-15:0.1) -- ++(180-15:0.2) -- (180-15:1.5);
	\end{tikzpicture}}
	\end{resizebox}
\end{wrapfigure}

Juku tahab kelguga ületada jääga kaetud jõge. Ta stardib lumega kaetud kaldalt, mis on horisondiga $\alpha = 15^{\circ}$ nurga all. Jõe laius $l = \SI{10}{m}$, kelgu ja lume vaheline hõõrdetegur $\mu_1 = \SI{0.20}{}$, kelgu ja jää vaheline hõõrdetegur $\mu_2 = \SI{0.10}{}$. Kui kõrgele veepinnast peab kallas ulatuma, et Juku libiseks teise kaldani?

\hint
Piirjuhul läheb kogu Juku potentsiaalne energia hõõrdejõu poolt eraldatud soojusenergiaks.

\solu
Olgu jõe ületamiseks vajalik kõrgus $h$ ja Juku mass koos kelguga $m$. Startides on Jukul potentsiaalne energia $E\idx{pot} = mgh$, mis muundub kelgu liikumisel soojuseks hõõrdejõu kaudu. Kaldal tuleb Jukul läbida distants $s = h/\sin(\alpha)$. Hõõrdejõu väärtus kaldpinnal on $F_h=\mu_1 mg \cos(\alpha)$. Hõõrdejõu poolt tehtud töö on $A_1 = F_h s = {\mu_1}mgh/\tan(\alpha)$. Jää peal tuleb läbida distants $l$ ja hõõrdejõu poolt tehtud töö on $A_2 = {\mu_2}mgl$. Piirjuhul läheb kogu potentsiaalne energia soojusenergiaks:
$$E\idx{pot} = A_1 + A_2,$$
$$h =\frac{\mu_1 h}{\tan(\alpha)} + {\mu_2} l, $$
$$h \cdot \left(1 - \frac{\mu_1}{\tan(\alpha)}\right) = {\mu_2} l,$$
$$h = \frac{\mu_2 l \tan(\alpha)}{\tan(\alpha) - {\mu_1}} \approx \SI{3.9}{m}.$$

\probeng{Sledge}
\begin{wrapfigure}[2]{r}{0.3\textwidth}
	\vspace{-12pt}
	\begin{resizebox}{\linewidth}{!}{
	\begin{tikzpicture}
	\coordinate (C) at (0.7,-0.1);
	\draw (0,0) -- ++(180-15:1.5);
	\draw (0,0) to [out = -15, in = 180] (C);
	\draw (C) to ++(0:1.5);
	\draw (180-15:1.5) -- ++(-15:0.2) -- ++(90-15:0.1) -- ++(180-15:0.2) -- (180-15:1.5);
	\end{tikzpicture}}
	\end{resizebox}
\end{wrapfigure}
Juku wants to cross an ice covered river on a sledge. He starts from a shore covered with snow, its slope has an angle of inclination $\alpha = 15^{\circ}$. The width of the river is $l = \SI{10}{m}$, coefficient of friction between the sledge and the snow is $\mu_1 = \SI{0.20}{}$ and between the sledge and the ice $\mu_2 = \SI{0.10}{}$. What must be the height of the shore with respect to the water’s surface so that Juku would slide to the other shore?

\hinteng
At a limit case Juku’s total potential energy goes to the heat energy caused by the friction force.

\solueng
Let the necessary height to cross the river be $h$ and the mass of Juku with the sledge $m$. At the moment of starting the slide Juku’s potential energy is $E\idx{pot} = mgh$ which due to friction turns into heat during the sledge’s movement. At the shore Juku needs to cover the distance $s = h/\sin(\alpha)$. The value of friction force on the inclined surface is $F_h=\mu_1 mg \cos(\alpha)$. The work done by friction force is $A_1 = F_h s = {\mu_1}mgh/\tan(\alpha)$. Distance $l$ has to be covered on the ice and the work done by friction is $A_2 = {\mu_2}mgl$. At limit case the total potential energy goes into thermal energy: 
$$E\idx{pot} = A_1 + A_2,$$
$$h =\frac{\mu_1 h}{\tan(\alpha)} + {\mu_2}   l, $$
$$h \cdot \left(1 - \frac{\mu_1}{\tan(\alpha)}\right) = {\mu_2}  l,$$
$$h = \frac{\mu_2 l \tan(\alpha)}{\tan(\alpha) - {\mu_1}} \approx \SI{3.9}{m}.$$
\probend