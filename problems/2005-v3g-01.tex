\setAuthor{Aigar Vaigu}
\setRound{lõppvoor}
\setYear{2005}
\setNumber{G 1}
\setDifficulty{1}
\setTopic{Dünaamika}

\prob{Kivi}
Sirgjooneliselt ja jääva kiirusega $v = \SI{4}{m/s}$ tõusva õhupalli gondlis on poiss. Mingil hetkel kukutab poiss gondlist alla kivi ning seejärel viskab kivile järgi palli, millega proovib langevat kivi tabada. Milline võib olla suurim ajavahemik kivi lahtilaskmise ja palli viskamise vahel, et see oleks veel võimalik? Maapinnal seistes suudaks poiss visata palli vertikaalselt üles kuni $h = \SI{20}{m}$ kõrgusele. Võib eeldada, et õhupall asub piisavalt kõrgel selleks, et pall tabaks kivi enne maapinnale kukkumist. Õhutakistus lugeda tühiseks. Raskuskiirendus $g = \SI{9,8}{m/s^2}$.

\hint
Maksimaalse viivituse korral on palli kiirus vaevu kivi omast suurem. Selles on võimalik veenduda liikudes vabalt langevasse taustsüsteemi. Seal liiguvad vabalt langevad kehad konstantse kiirusega ning selleks, et pall ja kivi kokku põrkaksid, peaks nende suhteline kiirus olema negatiivne.

\solu
Läheme üle vabalt langevasse taustsüsteemi. Selles süsteemis liiguvad vabalt langevad kehad konstantse kiirusega. Kivi saavutab langevas süsteemis palli viskamise hetkeks, $\Delta t$, õhupalli suhtes suhtelise kiiruse $u = g\Delta t$; see ei tohi olla suurem, kui maksimaalne viskekiirus $v\idx{max}$. Seega
\[
\Delta t \leq \frac{v\idx{max}}{g}.
\]
Avaldame palli viskekiiruse energia jäävuse seadusest:
\[
\frac{m v_{\max }^{2}}{2}=m g h \quad \Rightarrow \quad v_{\max }=\sqrt{2 g h}.
\]
Maksimaalne viivituse aeg on seega
\[
\Delta t_{\max }=\frac{v_{\max }}{g}=\sqrt{\frac{2 h}{g}} \approx \SI{2}{s}.
\]

\probeng{Stone}
A boy is in the basket of a balloon rising upwards at a constant speed of $v=4\SI{4}{m/s}$. At a certain point, the boy lets a stone fall from the gondola, and then he throws a ball at the stone to try and hit it. What is the longest time the boy can wait between throwing the stone and throwing the ball? Standing on the ground, the boy could throw the ball vertically up to a height of $h = \SI{20}{m}$. It can be assumed that the balloon is high enough for the ball to hit the stone before it reaches the ground. Air resistance is negligible. Take the gravitational acceleration to be $g=\SI{9.8}{m/s^2}$.  
\probend
