\setAuthor{Oleg Košik}
\setRound{piirkonnavoor}
\setYear{2012}
\setNumber{G 2}
\setDifficulty{3}
\setTopic{Termodünaamika}

\prob{Küttesüsteem}
Talvel siseneb koolimaja küttesüsteemi vesi algtemperatuuriga $t_0=\SI{60}{\degreeCelsius}$
ning väljub sealt temperatuuriga $t_1=\SI{40}{\degreeCelsius}$. Koolimaja soojuskadude
võimsus on $N=\SI{100}{kW}$. Kooli siseneva ja sealt väljuva veetoru sisediameeter on
$D=\SI{100}{mm}$. Leidke veevoolu kiirus neis torudes. Vee erisoojus
$c=\SI{4200}{J/(kg\cdot\degreeCelsius}$, tihedus $\rho=\SI{1000}{kg/m^3}$).

\hint
Aja $\Delta t$ jooksul peab koolimajja siseneva ja väljuva vee soojushulkade vahe olema võrdne soojuskadudega läbi seinte.

\solu
Mingi ajavahemiku $\Delta t$ jooksul kaotab koolimaja väliskeskkonda soojust $Q_1=N\Delta t$, sama palju soojust peavad andma selle aja jooksul talle radiaatorid. Toru ristlõike pindala on $S=\frac{\pi D^2}{4}$. Aja $\Delta t$ jooksul küttesüsteemi siseneva ja ühtlasi sellest väljuva vee ruumala on seega $V=Sv\Delta t$, kus $v$ on otsitav veevoolu kiirus, ning mass $m=\rho V=\rho Sv\Delta t$. Radiaatorites eraldub soojushulk $Q_2=mc(t_0-t_1)$. Kuna $Q_1=Q_2$, saame võrrandi
\[
N\Delta t=\rho \frac{\pi D^2}{4}v\Delta t c (t_0-t_1),
\]
millest
\[
v=\frac{4N}{\pi D^2\rho c (t_0-t_1)}=0,15\;\hbox{m/s}.
\]

\probeng{Heating system}
Water of initial temperature $t_0=\SI{60}{\degreeCelsius}$ enters a school house’s heating system during winter. The water exits the system with a temperature $t_1=\SI{40}{\degreeCelsius}$. The power of the school house’s heat losses is $N=\SI{100}{kW}$. The diameter of the tube entering and exiting the school house is $D=\SI{100}{mm}$. Find the speed of the water in the tubes. The specific heat of water is $c=\SI{4200}{J/(kg\cdot\degreeCelsius)}$ and the density $\rho=\SI{1000}{kg/m^3}$.

\hinteng
During the time $\Delta t$ the difference between the heat of the water entering and the one exiting the school house has to be equal to the heat losses through the walls.

\solueng
During some time period $\Delta t$ the school house loses the heat $Q_1=N\Delta t$ into the external environment, during this time the radiators have to give the same amount of heat to the house. The area of the tube's cross section is $S=\frac{\pi D^2}{4}$. The volume of the water that enters and also the water that leaves the heating system during the time $\Delta t$ is therefore $V=Sv\Delta t$, where $v$ is the speed of the sought water flow, and the mass $m=\rho V=\rho Sv\Delta t$. The heat $Q_2=mc(t_0-t_1)$ is dissipated by the radiators. Because $Q_1=Q_2$, we get an equation
\[
N\Delta t=\rho \frac{\pi D^2}{4}v\Delta t c (t_0-t_1),
\] 
where 
\[
v=\frac{4N}{\pi D^2\rho c (t_0-t_1)}=0,15\;\hbox{m/s}.
\]
\probend