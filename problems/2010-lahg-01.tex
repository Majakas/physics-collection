\setAuthor{Valter Kiisk}
\setRound{lahtine}
\setYear{2010}
\setNumber{G 1}
\setDifficulty{2}
\setTopic{Dünaamika}

\prob{Kerad}
On antud kolm väliselt identset ja ühesuguse massiga kera. On teada, et üks
neist keradest on homogeenne, teine on seest õõnes ja kolmas on seest vedel.
Kuidas saab lihtsate võrdlevate mehaanikakatsetega kindlaks teha, milline on iga
kera sisemus? Abivahendeid võib vabalt valida, aga kerasid vigastada ei tohi.

\hint
Õõnes ja homogeenne kera erinevad nende intertsimomentide poolest. Vedelikku sisaldaval keral toimub sees paratamatult hõõrdumine ning seega energia kadu vedeliku erinevate kihtide vahel.

\solu
Õõnes ja homogeenne kera eristuvad selle poolest, et esimese inertsimoment on suurem, sest
mass on koondunud pöörlemistsentrist kaugemale, st sama nurkkiirusega pööreldes on pöörlemisega seotud kineetiline enregia suurem. 
%Piirjuhul, sfääri korral, on inertsimoment $(2/3)mR^2$ samas kui kera inertsimoment on $(2/5)mR^2$. 
Niisiis võrdse kineetilise energia omandamisel (näiteks sama kaldpinda mööda alla veeredes) saavutab 
õõnes kera väiksema kiiruse (sest pöörlemisega seotud energia on suurem). 

Vedelikku sisaldava kera korral kulub aga osa liikumise energiast paratamatult
vedeliku sisehõõrdumise ületamiseks, seetõttu mehaanilise energia jäävus on katsetes
rikutud (näiteks lükkame kerad veerema; vedelikku sisaldav kera pidurdub iseenesest).
\probend