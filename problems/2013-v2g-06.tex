\setAuthor{Eero Vaher}
\setRound{piirkonnavoor}
\setYear{2013}
\setNumber{G 6}
\setDifficulty{4}
\setTopic{Taevamehaanika}

\prob{Päikese tihedus}
Leidke Päikese keskmine tihedus $\varrho$. Maa
tiirlemisperiood on $T=1$~aasta, gravitatsioonikonstant 
$G=\SI{6.7e-11}{N m^2/kg^2}$, Maa kaugus Päikesest $R=\SI{1.5e11}{m}$, Päikese nurkläbimõõt Maalt vaadatuna on
$\alpha=\ang{0,54}$ (see on nurk, mis moodustub kahe kiire vahel, mis on tõmmatud
vaatleja silma juurest Päikese diameetri otspunktide juurde).

\hint
Maa jõudude tasakaalust on võimalik avaldada Päikese mass ning Päikese nurkdiameetrist Päikese raadius.

\solu
Päikese raadius on $r=R\sin\alpha/2$ ja ruumala $V=4\pi r^3/3$.
Maa joonkiirus oma orbiidil ümber Päikese on $v=\frac{2\pi R}{T}$. Et Maa püsiks oma orbiidil, peab sellele mõjuma kesktõmbejõud $F=\frac{mv^2}{R}$, kus $m$ on Maa mass. See kesktõmbejõud on teadagi Maa ja Päikese vaheline gravitatsioonijõud $G\frac{mM}{R^2}$, kus $M$ on Päikese mass, seega
\[
\frac{mv^2}{R}=G\frac{mM}{R^2}.
\]
Saame
\[
M=\frac{Rv^2}{G}.
\]
Päikese tihedus avaldub seosest $\varrho=\frac{M}{V}$, $$\varrho=\frac{Rv^2}{G} \frac{3}{4\pi r^3}=\frac{24 \pi R^3}{G T^2 \sin^3 \alpha}=\SI{1,4e3}{kg/m^3}.$$

\probeng{Density of the Sun}
Find the average density $\varrho$ of the Sun. The orbital period of the Earth is $T=1$ year, gravitational constant $G=\SI{6.7e-11}{N m^2/kg^2}$, the distance between the Earth and the Sun $R=\SI{1.5e11}{m}$, the angular diameter of the Sun observed from the Earth is $\alpha=\ang{0,54}$ (the angle that forms between the two lines that are drawn from the eye of the observer to the ends of the Sun’s diameter).

\hinteng
From the Earth’s force balance it is possible to express the Sun’s mass and from the Sun’s angular diameter the Sun’s radius.

\solueng
The Sun’s radius is $r=\frac{R\sin\alpha}{2}$ and volume $V=\frac{4\pi r^3}{3}$. The Earth’s speed on its orbit around the Sun is $v=\frac{2\pi R}{T}$. For the Earth to stay on its orbit a centripetal force $F=\frac{mv^2}{R}$ has to be applied to it where $m$ is the Earth’s mass. This centripetal force is the gravity force $G\frac{mM}{R^2}$ between the Earth and the Sun where $M$ is the Sun’s mass, therefore $\frac{mv^2}{R}=G\frac{mM}{R^2}$. We get $M=\frac{Rv^2}{G}$. The Sun’s density is expressed from the relation $\varrho=\frac{M}{V}$. 
$$\varrho=\frac{Rv^2}{G} \frac{3}{4\pi r^3}=\frac{24 \pi R^3}{G T^2 \sin^3 \alpha}=\SI{1,4e3}{kg/m^3}.$$
\probend