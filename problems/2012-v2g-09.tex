\ylDisplay{Robin Hood} % Ülesande nimi
{Madis Ollikainen} % Autor
{piirkonnavoor} % Voor
{2012} % Aasta
{G 9} % Ülesande nr.
{7} % Raskustase
{
% Teema: Dünaamika
\ifStatement
Robin Hood on täpsuslaskmisvõistlustel, kus tal tuleb tabada märklauda, mis asub
$L=\SI{200}{m}$ kaugusel. Millise nurga $\alpha$ all horisontaalsihi suhtes
peab Robin vibust laskma, et tabada täpselt märklaua keskpunkti? Vibu vinnamisel
teeb ta tööd $A=\SI{500}{J}$ ning vibu kasutegur on $\eta=0,17$. Noole mass on
$m=\SI{54}{g}$ ja see lastakse lendu märklaua keskpunktist $h=\SI{70}{cm}$ võrra
kõrgemalt. Õhutakistusega ärge arvestage. Raskuskiirenduseks lugege
$g=\SI{9,8}{m/s^2}$.
\fi


\ifHint
Noole algkiirus on leitav lähteandmetest. Edasi peab nool tabama punkti koordinaatidega $(\SI{200}{m};\SI{-0,7}{m})$ alguspunkti suhtes. Selle jaoks saab kirja panna liikumisvõrrandi nii $x$- kui ka $y$-koordinaadi jaoks.
\fi


\ifSolution
Kõigepealt leiame noole algkiiruse. Selle leiame noolele antud kineetilise energia kaudu.
$$ E_{K}=\frac{m v^2}{2},$$
$$E_{K}=\eta A \Rightarrow$$
$$v=\sqrt{\frac{2\eta A}{m}}=\sqrt{\frac{2\cdot0,17\cdot 500}{0,054}}\approx \SI{56,1}{m/s}.$$

Paneme tähele, et noole kiiruse horisontaalne komponent on
$$v\idx{horisontaal}=v\cdot \cos\alpha.$$

ja vertikaalne komponent on
$$v\idx{vertikaal}=v\cdot \sin\alpha.$$

Kui nool lendab aja $t$, siis 

$$v \cos\alpha\cdot t=L,$$

$$v \sin\alpha\cdot t-\frac{g t^2}{2}=-h. $$

(Kuna nool lastakse lendu märklaua keskpunktist h võrra kõrgemalt.)

Nüüd avaldame ülemisest võrrandist $\sin\alpha$:
$$\cos\alpha=\frac{L}{vt} \Rightarrow \sin\alpha=\sqrt{1-\left(\frac{L}{vt}\right)^2}=\sqrt{\frac{v^2t^2-L^2}{v^2t^2}}.$$


Asendame selle alumisse võrrandisse
$$vt\cdot \sqrt{\frac{v^2t^2-L^2}{v^2t^2}}=\frac{g t^2}{2}-h \Rightarrow$$
$$v^2t^2-L^2=h^2 - hgt^2+\frac{g^2 t^4}{4} \Rightarrow$$
$$\frac{g^2}{4}\cdot t^4 - \left(hg+v^2\right)\cdot t^2 + \left(h^2+L^2\right)=0.$$
Lahendame ruutvõrrandi $t^2$ suhtes:
$$t^2_{1,2}=\frac{\left(hg+v^2\right)\pm\sqrt{\left(hg+v^2\right)^2-4\cdot \frac{g^2}{4}\left(h^2+L^2\right)}}{2\cdot \frac{g^2}{4}} \Rightarrow$$
$$t^2_{1}=\SI{117}{s^2} \Rightarrow t_{1}=\pm\sqrt{116}=\pm \SI{10,8}s,$$
$$t^2_{2}=\SI{14,3}{s^2} \Rightarrow t_{2}=\pm \sqrt{14,3}\SI{}s=\pm \SI{3,77}s.$$
On selge, et negatiivne aeg ei oma antud juhul füüsikalist tähendust. Tuleb välja, et Robin võib noolt lasta kahe erineva nurga, 
$\alpha_{1}$ ja $\alpha_{2}$ all:
$$\alpha_{1}= \arccos\frac{L}{vt_{1}}=\arccos\frac{200}{56\cdot10,8}\approx 71^\circ ,$$
$$\alpha_{2}=\arccos\frac{L}{vt_{2}}=\arccos\frac{200}{56\cdot3,78}\approx 19^\circ .$$
\fi


\ifEngStatement
% Problem name: Robin Hood
Robin Hood is in an archery competition where he has to hit a target at a distance $L=\SI{200}{m}$. At what angle $\alpha$ with respect to the horizontal direction has to Robin shoot with his bow so he would hit the target exactly in the middle? While straining the bow Robin’s work is $A=\SI{500}{J}$ and the bow’s coefficient of efficiency is $\eta=0,17$. The arrow’s mass is $m=\SI{54}{g}$ and it is shot $h=\SI{70}{cm}$ higher than the target’s center is. Do not account for air resistance. Gravitational acceleration is $g=\SI{9,8}{m/s^2}$.
\fi


\ifEngHint
The initial speed of the arrow can be found with the primary data. Next the arrow must hit the point with the coordinates $(\SI{200}{m};\SI{-0,7}{m})$ with respect to the initial point. For that you can write down equation of motion for both the $x$- and $y$-coordinate.
\fi


\ifEngSolution
First we find the initial velocity of the arrow. We can find this with the kinetic energy that is given to the arrow.
$$ E_{K}=\frac{m v^2}{2},$$ 
$$E_{K}=\eta A \Rightarrow$$
$$v=\sqrt{\frac{2\eta A}{m}}=\sqrt{\frac{2\cdot0,17\cdot 500}{0,054}}\approx \SI{56,1}{m/s}.$$
Let us notice that the horizontal component of the arrow’s velocity is 
$$v\idx{horizonta}=v\cdot \cos\alpha.$$ 
and the vertical component
$$v\idx{vertical}=v\cdot \sin\alpha.$$ 
If the arrow flies with time $t$ then 
$$v \cos\alpha\cdot t=L,$$ 
$$v \sin\alpha\cdot t-\frac{g t^2}{2}=-h. $$
(Because the arrow is shot by the distance h higher from the center of the target.) 
Now we express $\sin\alpha$ from the upper equation:
$$\cos\alpha=\frac{L}{vt} \Rightarrow \sin\alpha=\sqrt{1-\left(\frac{L}{vt}\right)^2}=\sqrt{\frac{v^2t^2-L^2}{v^2t^2}}.$$ 
We replace it into the bottom equation
$$vt\cdot \sqrt{\frac{v^2t^2-L^2}{v^2t^2}}=\frac{g t^2}{2}-h \Rightarrow$$ 
$$v^2t^2-L^2=h^2 - hgt^2+\frac{g^2 t^4}{4}  \Rightarrow$$
$$\frac{g^2}{4}\cdot t^4 - \left(hg+v^2\right)\cdot t^2 + \left(h^2+L^2\right)=0.$$
We solve a quadratic equation with respect to $t^2$:
$$t^2_{1,2}=\frac{\left(hg+v^2\right)\pm\sqrt{\left(hg+v^2\right)^2-4\cdot \frac{g^2}{4}\left(h^2+L^2\right)}}{2\cdot \frac{g^2}{4}} \Rightarrow$$ 
$$t^2_{1}=\SI{117}{s^2}  \Rightarrow t_{1}=\pm\sqrt{116}=\pm \SI{10,8}s,$$
$$t^2_{2}=\SI{14,3}{s^2} \Rightarrow t_{2}=\pm \sqrt{14,3}\SI{}s=\pm \SI{3,77}s.$$
It is clear that a negative time does not hold a physical meaning in this case. It turns out that Robin can shoot the arrow at two different angles $\alpha_{1}$ and $\alpha_{2}$:
$$\alpha_{1}= \arccos\frac{L}{vt_{1}}=\arccos\frac{200}{56\cdot10,8}\approx 71^\circ ,$$ 
$$\alpha_{2}=\arccos\frac{L}{vt_{2}}=\arccos\frac{200}{56\cdot3,78}\approx 19^\circ .$$
\fi
}