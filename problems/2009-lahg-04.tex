\setAuthor{Tundmatu autor}
\setRound{lahtine}
\setYear{2009}
\setNumber{G 4}
\setDifficulty{4}
\setTopic{Elektrostaatika}

\prob{Laetud rõngas}
Peenikesest traadist rõngas raadiusega $R$ on ühtlaselt laetud negatiivse laenguga $Q$. Elektron massiga $m$ ja laenguga $e$ läheneb rõngale mööda sirget, mis on risti rõnga tasandiga ning läbib rõnga keskpunkti. Millist tingimust peab rahuldama elektroni kiirus punktis, mis asub kaugusel $d = R\sqrt 3$ rõnga keskpunktist, et elektron saaks rõngast läbi lennata?

\hint
Põhimõtteliselt on võimalik leida elektroni kiirendus rõnga poolt tekitatud elektriväljas ning seda integreerida, aga märgatavalt lihtsam on rakendada energia jäävuse seadust ning kasutada rõnga poolt tekitatud potentsiaalset energiat. Punktlaengu $q$ poolt tekitatud potentsiaal kaugusel $r$ on $k\frac{q}{r}$.

\solu
Punktis, mis asub rõnga teljel kaugusel $d$ rõnga keskpunktist on rõnga poolt tekitatud välja potentsiaal
\[
\phi_{1}=k \frac{Q}{\sqrt{R^{2}+d^{2}}}=k \frac{Q}{2 R}.
\]
Rõnga keskpunktis on rõnga poolt tekitatud välja potentsiaal
\[
\phi_{2}=k \frac{Q}{R}.
\]
Et elektron saaks rõngast läbi lennata, peab tema kineetiline energia olema piisav
potentsiaalide vahe $U = \phi_2 - \phi_1$ läbimiseks. Piirjuhul saame
\[
\frac{m v^{2}}{2}=e\left(k \frac{Q}{R}-k \frac{Q}{2 R}\right)=k \frac{e Q}{2 R}.
\]
Siit
\[
v_{\min }=\sqrt{k \frac{e Q}{m R}}.
\]
\probend