\ylDisplay{Kadunud rahakott} % Ülesande nimi
{Eero Vaher} % Autor
{lahtine} % Voor
{2012} % Aasta
{G 2} % Ülesanne nr.
{2} % Raskustase
{
% Teema: Dünaamika
\ifStatement
Suusahüppemäe hoovõturada asub nõlval, mille tõusunurk on $\alpha$. Hoovõturaja alumine
ots on horisontaalne. Suusahüppaja alustas hoovõttu kõrguselt $h$,
kuid kohe hoovõtu alguses kukkus tal rahakott taskust välja. Kui kaugele
äratõukepunktist (mööda
horisontaali) rahakott lendab, kui see liigub ilma takistuseta?
\fi


\ifHint
Hoovõturaja alumises otsas on rahakott omandanud teatud horisontaalse kiiruse. Edasi hakkab rahakott liikuma mööda paraboolset trajektoori, kuni see taas vastu mäe nõlva kukub.
\fi


\ifSolution
Hoovõturaja alumise otsani jõudes on rahakott omandanud kineetilise energia $E=\frac{mv^2}{2}=mgh$, kusjuures algkiirus $v=\sqrt{2gh}$ on horisontaalne. Kui valime koordinaatide alguspunktiks hoovõturaja alumise otsa, määravad rahakoti lennutrajektoori võrrandid $x=vt$ ja $y=-gt^2 / 2$. Rahakott maandub siis, kui selle trajektoor ja nõlva kirjeldav joon $y=-x\tan\alpha$ lõikuvad. Seega 
$$-gt^2 / 2=-vt\tan\alpha,$$ 
$$t=\frac{2v \tan \alpha}{g},$$ 
ehk
$$x=4h \tan\alpha.$$
\fi


\ifEngStatement
% Problem name: Lost wallet
A Ski jumping hill’s track is situated on a slope with an angle of $\alpha$. The track’s take-off platform is horizontal. The ski jumper began the in-run at a height of $h$ relative to the take-off platform, but right at the start, his wallet fell out of his pocket. How far from the take-off will the wallet land (along the horizontal axis) if the wallet is moving without friction?
\fi


\ifEngHint
At the bottom section of the jumping track the wallet has obtained a certain horizontal speed. After that the wallet starts to move along a parabolic trajectory until it falls against the mountain’s surface again.
\fi


\ifEngSolution
Reaching the bottom part of the acceleration track the wallet has obtained kinetic energy $E=\frac{mv^2}{2}=mgh$ where the initial speed $v=\sqrt{2gh}$ is horizontal. If we choose the bottom end of the track as the origin then the equations $x=vt$ and $y=-gt^2 / 2$ determine the wallet’s flight trajectory. The wallet lands when its trajectory and the line $y=-x\tan\alpha$ describing the hill intersect. Thus
$$-gt^2 / 2=-vt\tan\alpha,$$
$$t=\frac{2v \tan \alpha}{g},$$
then $$x=4h \tan\alpha.$$
\fi
}