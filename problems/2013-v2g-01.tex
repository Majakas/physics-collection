\setAuthor{Koit Timpmann}
\setRound{piirkonnavoor}
\setYear{2013}
\setNumber{G 1}
\setDifficulty{1}
\setTopic{Kinemaatika}

\prob{Rong}
Kaubarong läbis kahe jaama vahelise teelõigu keskmise kiirusega \SI{36}{km/h}.
Kogu sõiduajast 2/5 vältel liikus rong ühtlaselt kiirenevalt, siis 2/5 vältel
liikus jääva kiirusega ning viimase 1/5 vältel pidurdas ühtlaselt aeglustuvalt.
Kui suur oli rongi maksimaalne kiirus kahe jaama vahelisel teel?

\hint
Kiirendamise ning pidurdamise käigus on rongi keskmine kiirus $v\idx{max}/2$.

\solu
Olgu rongi maksimaalne kiirus $v$ ning kogu sõiduaeg $t$. Kiirendamise jooksul on keskmine kiirus $v/2$ ning sellele kulub aega $\frac{2t}{5}$. Pidurdamine võtab aega $\frac{t}{5}$ ning ka selle jooksul on keskmine kiirus $v/2$. Kogu sõidu keskmine kiirus on seega 
$$v_k = \frac{\frac{2t}{5} \frac{v}{2}+\frac{2t}{5} v + \frac{t}{5} \frac{v}{2}}{t} = \frac{7}{10} v.$$ 
Kokku,
\[
v=\frac{10}{7}v_k \approx \SI{51}{km/h}.
\]

\probeng{Train}
A freight train passed through a road section between two stations with an average speed 36 km/h. Through 2/5 of the driving time the train was evenly accelerating, then through the next 2/5 it drove with a constant speed and through the last 1/5 it slowed down evenly. How big was the train’s maximal speed on the section between the two stations?

\hinteng
During acceleration and braking the average speed of the train is $v\idx{max}/2$.

\solueng
Let the maximal velocity of the train be $v$ and the total driving time $t$. During acceleration the average velocity is $v/2$ and the time of the acceleration is $\frac{2t}{5}$. The braking takes the time $\frac{t}{5}$ and during this the average velocity is also $v/2$. The average velocity of the whole drive is therefore
$$v_k = \frac{\frac{2t}{5} \frac{v}{2}+\frac{2t}{5} v + \frac{t}{5} \frac{v}{2}}{t} = \frac{7}{10} v.$$
From here $v=\frac{10}{7}v_k \approx \SI{51}{km/h}$.
\probend