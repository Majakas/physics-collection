\ylDisplay{Veetoru} % Ülesande nimi
{Taavi Pungas} % Autor
{lõppvoor} % Voor
{2011} % Aasta
{G 4} % Ülesande nr.
{4} % Raskustase
{
% Teema: Vedelike-mehaanika
\ifStatement
% kuidas kirjutada const?
Kaks erineva diameetriga horisontaalset toru on otsapidi
kokku ühendatud nii, et nende teljed ühtivad. Mööda esimest toru voolab vesi
kiirusega $v_1$. Kummagi veetoru külge on ühendatud väike vertikaalne toruke,
vedelikusamba kõrgused neis on vastavalt $h_1$ ja $h_2$ (toru teljest mõõtes). Leidke
horisontaalsete torude diameetrite suhe. Hõõrdumist mitte arvestada. 

\emph{Vihje}. 
Vedeliku horisontaalsel voolamisel kehtib Bernoulli seadus kujul $\frac{\rho v^2}{2}+p=\const$, kus $p$ on hüdrostaatiline rõhk, $ρ$ vedeliku tihedus ning $v$ vedeliku kiirus.
\fi


\ifHint
Torudes oleva vee rõhk on avaldatav vedelikusammaste kõrguste kaudu. Lisaks kehtib mõlemas torus Bernoulli seadus ning vee pidevustingimus, st sama aja jooksul läbib mõlemat toru sama kogus vett.
\fi


\ifSolution
Bernoulli seadusest saame seose 
\[
\frac{\rho v_1^2}{2}+p_1=\frac{\rho v_2^2}{2}+p_2.
\]
Rõhu torudes leiame veesamba kõrguse järgi, $p_1=p_0 + \rho g h_1$ ja $p_2=p_0 + \rho g h_2$, kus $p_0$ on atmosfäärirõhk.
Asendades $p_1$ ja $p_2$ esimeses seoses, saame vee kiiruse teises torus:
\[
v_2=\sqrt{v_1^2+2g(h_1-h_2)}.
\]
Et torud on ühendatud, peab läbi nende voolama sama aja jooksul sama kogus vett, $v_1 S_1 = v_2 S_2$.
Kuna toru ristlõikepindala on $S=\pi d^2 / 4$, saame $v_1 d_1^2 = v_2 d_2^2$.

Kokku,
\[
\frac{d_1}{d_2}=\sqrt{\frac{v_2}{v_1}}=\left(1+\frac{2g(h_1-h_2)}{v_1^2}\right)^{1/4}.
\]
\fi
}