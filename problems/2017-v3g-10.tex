\setAuthor{Jaan Kalda}
\setRound{lõppvoor}
\setYear{2017}
\setNumber{G 10}
\setDifficulty{10}
\setTopic{Magnetism}

\prob{Elektronid}
Ruumipiirkonnas $x>-a$ ($a>0$) on homogeenne $z$-telje sihiline magnetväli induktsiooniga $B$. 
Koordinaatide alguspunktis on elektronide allikas, mis kiirgab elektrone võrdsel arvul
kõikidesse suundadesse (üle ruuminurga $4\pi$). Kõikide elektronide kiirus on $v$. Tasandis $x=-a$ on ekraan. Kui elektronid 
laenguga $e$ ja massiga $m$ põrkuvad vastu 
ekraani, siis on kokkupõrkepunktis näha helendust. Leidke helenduva laigu $y$-telje sihiline
läbimõõt tasandil $z = 0$ eeldusel, et vähemalt osa elektronidest jõuavad ekraanini.
Samal tasandil leida, kus kohas on laigu helenduse intensiivsus
kõige suurem. Milline on selle laigu $z$-telje sihiline pikkus tasandil $y=0$? 

\hint
Elektronid hakkavad liikuma mööda $z$-teljelist heeliksit, kusjuures heeliksi raadius on võrdeline $x-y$ suunalise kiirusega. Seega liiguvad kõik elektronid $x-y$ tasandis mööda ringjoont.

\solu
Maksimaalse $y$-telje sihilise läbimõõdu leidmiseks vaatleme osakesi, mis liiguvad $x-y$-tasandis. Sellisel juhul on elektroni trajektooriks ringjoon, sest magnetvälja poolt mõjuv jõud $F=evB$ on konstantse suurusega ja kiirusega kogu aeg risti. Raadiuse saame leida, pannes magnetvälja poolt mõjuva jõu võrduma kesktõmbejõuga:
$$\frac{mv^2}{R}=evB \hence R = \frac{mv}{Be}.$$
Kõik selles tasandis olevad elektronid liiguvad sama raadiusega ringjoonel ja eri nurkade all liikuva elektronide trajektoorid saab leida seda ringjoont lihtsalt pöörates ümber punkti O. Kõige kaugemale $y$-telje positiivses suunas jõuab elektron siis, kui ekraani tabades on ta algpunktist võimalikult kaugel ehk läbinud täpselt pool ringjoone kaart, vaata joonist:
\begin{center}
	\begin{tikzpicture}
	\def\aEl{1};
	\def\rEl{1.5};
	\coordinate (P1) at (-\aEl,{sqrt(4*\rEl*\rEl-\aEl*\aEl)});
	\coordinate (P2) at (-\aEl*0.5,{0.5*sqrt(4*\rEl*\rEl-\aEl*\aEl)});
	\node[label=below right:O, fill=black, circle,inner sep=1pt] at (0,0){};
	\path let \p1=(P1) in node [label=left:b] at (-\aEl, \y1/2){};
	\node[label=right:Q, fill=black, circle,inner sep=1pt] at (P2){};
	\node[label=below:a] at (-\aEl/2,0){};
	\node [label=above:$y$] at (-\aEl, 2.5*\rEl){};
	\draw[->] (-\aEl,-\rEl/2) -- (-\aEl, 2.5*\rEl);
	\draw (0,0) -- (P1);
	\draw (0,0) -- (-\aEl,0);
	%\draw (0,0) arc ({-90+asin(\aEl/(2*\rEl))}:{360-(-90+asin(\aEl/(2*\rEl)))}:\rEl);
	\draw (0,0) arc ({-90+asin(\aEl/(2*\rEl))}:90+asin(\aEl/(2*\rEl)):\rEl);
	%\draw [red](P1) ++ (-90:0.4) arc (-90:-90+\angKatus:0.4);
	\end{tikzpicture}
\end{center}

Kuna ekraan on kaugusel $a$ elektronide allikast, siis Pythagorase teoreemi järgi saame laigu maksimaalseks $y$ väärtuseks $$b=\sqrt{4R^2-a^2} = \sqrt{4\left(\frac{mv}{Be}\right)^2-a^2}.$$
Siit näeme, et väiksemale $R$-ile vastab väiksem läbimõõt ja maksimaalne $y$-telje sihiline läbimõõt on tõesti siis, kui osake liigub $x-y$-tasandis, nagu ka alguses väitsime.

Ringjoone kaart ümber punkti $O$ pöörates näeme, et minimaalne $y$-suunalise väärtuse korral on ringjoon ekraanile puutujaks, vaata joonist:
\begin{center}
	\begin{tikzpicture}
	\def\aEl{1};
	\def\rEl{1.5};
	\coordinate (P1) at (\rEl-\aEl,{-sqrt(\rEl*\rEl-(\rEl-\aEl)*(\rEl-\aEl))});
	\node[label=above right:O, fill=black, circle,inner sep=1pt] at (0,0){};
	\node[label=below right:Q, fill=black, circle,inner sep=1pt] at (P1){};
	\path let \p1=(P1) in node[label=below:T, fill=black, circle,inner sep=1pt] at (0,\y1){};
	%\path let \p1=(P1) in node [label=left:b] at (-\aEl, \y1/2){};
	\node[label=above:a] at (-\aEl/2,0){};
	\node [label=above:$y$] at (-\aEl, 0.5*\rEl){};
	\draw[->] (-\aEl,-2*\rEl) -- (-\aEl, 0.5*\rEl);
	\draw (0,0) -- (P1);
	\draw (P1) -- ($(P1)-(1.0*\rEl,0)$);
	\draw (0,0) -- (-\aEl,0);
	%\draw (0,0) arc ({180-acos((\rEl-\aEl)/\rEl)}:{360}:\rEl);
	\draw (P1) ++ (-180:\rEl) arc (180:-180:\rEl);
	%\draw (0,0) arc ({180-acos((\rEl-\aEl)/\rEl)}:180:\rEl);
	%\draw [red](P1) ++ (-90:0.4) arc (-90:-90+\angKatus:0.4);
	\end{tikzpicture}
\end{center}
Kuna lõigu QT pikkus on $R-a$, siis lõigu OQ $y$-suunalise komponendi saab leida Pythagorase teoreemi abil
$$y\idx{min}=-\sqrt{R^2-(R-a)^2} = -\sqrt{2aR-a^2}=-\sqrt{\frac{2amv}{Be}-a^2}.$$
Seega laigu $y$-suunaline läbimõõt on
$$L=\sqrt{\frac{2amv}{Be}-a^2} + \sqrt{4\left(\frac{mv}{Be}\right)^2-a^2}.$$

Selle laigu peal on heledus kõige suurem maksimaalse $y$ väärtuse korral, sest see vastab elektroni $y$-telje sihilise kõrvalekalde $\Delta_y$ ekstreemumile. 
Kui tähistame elektroni stardinurga $\alpha$ abil, siis 
$$\frac{\D \Delta_y}{\D \alpha}=0,$$
st väikeses stardinurga vahemikus $\Delta\alpha$ saabuvad kõik elektronid peaaegu täpselt samasse sihtpunkti.

Leiame $z$-telje sihilise läbimõõdu tasandis $y=0$. Kõige kaugemale $z$-teljel jõuavad need on osakesed, mille 
heeliksikujuline trajektoor puudutab tasandit $x=-a$ (``ülaltvaates'' $x-y$-tasandile puudutab ringjoonekujuline trajektoor raadiusega $r=\frac a2$ 
joont $x=-a$) ja mis stardivad $y-z$-tasandis, st algkiirusega, mille $x$-projektsioon $v_x=0$.
Edasi leiame $$\frac a2=\frac {mv_y}{Be}\hence v_y=\frac{aBe}{2m}\hence v_z=\sqrt{v^2-\left(\frac{aBe}{2m}\right)^2}.$$
Sellise algkiirusega elektronide teekond kestab pool tsüklotronperioodist. Perioodi saab leida seosest
\[
T=\frac{2\pi R}{v} = \frac{2\pi m}{Be}.
\]
Pool perioodi on seega
\[
t=\frac{T}{2}=\frac{\pi m}{Be}
\]
ning laigu $z$-telje sihiliseks 
mõõtmeks saame
$$2\cdot tv_z=\pi\sqrt{\left(\frac{2mv}{Be}\right)^2-a^2}.$$

\probeng{Electrons}
At a region of space $x>-a$ ($a>0$) there is a homogenous $z$-directional magnetic field of induction $B$. At the origin there is a source of electrons that radiates electrons equally to all directions (over a solid angle $4\pi$). The speed of all the electrons is $v$. At the plane $x=-a$ there is a screen. If the electrons of charge $e$ and mass $m$ hit the screen, then there is light seen at the point of the collision. Find the $y$-directional diameter of the glowing spot on the plane $z = 0$, assuming that at least some of the electrons reach the screen. At the same plane, find where is the biggest luminous intensity of the spot. What is the $z$-directional length of the spot on the plane $y=0$?

\hinteng
The electrons start to move along an $z$-axial helix, moreover the radius of the helix is equal to the $x-y$ directional velocity. Therefore all the electrons move along a circle on the $x-y$ plane.

\solueng
To find the maximal $y$-directional diameter we observe the particles that move on $x-y$ plane. In this case the trajectory of the electron is a circle because the force $F=evB$ applied by the magnetic field has a constant value and is always perpendicular to the velocity. We can find the radius if we equate the force applied by the magnetic field to the centripetal force:
$$\frac{mv^2}{R}=evB \hence R = \frac{mv}{Be}.$$
All the electrons in that plane move on a circle with the same radius and we can find the trajectories of the electrons moving at different angles by just turning this circle around the point O. An electron reaches the furthest on the positive direction of the $y$-axis if when hitting the screen it is as far as possible from the origin, meaning it has covered exactly half of a circle’s arc, see figure:
\begin{center}
	\begin{tikzpicture}
	\def\aEl{1};
	\def\rEl{1.5};
	\coordinate (P1) at (-\aEl,{sqrt(4*\rEl*\rEl-\aEl*\aEl)});
	\coordinate (P2) at (-\aEl*0.5,{0.5*sqrt(4*\rEl*\rEl-\aEl*\aEl)});
	\node[label=below right:O, fill=black, circle,inner sep=1pt] at (0,0){};
	\path let \p1=(P1) in node [label=left:b] at (-\aEl, \y1/2){};
	\node[label=right:Q, fill=black, circle,inner sep=1pt] at (P2){};
	\node[label=below:a] at (-\aEl/2,0){};
	\node [label=above:$y$] at (-\aEl, 2.5*\rEl){};
	\draw[->] (-\aEl,-\rEl/2) -- (-\aEl, 2.5*\rEl);
	\draw (0,0) -- (P1);
	\draw (0,0) -- (-\aEl,0);
	%\draw (0,0) arc ({-90+asin(\aEl/(2*\rEl))}:{360-(-90+asin(\aEl/(2*\rEl)))}:\rEl);
	\draw (0,0) arc ({-90+asin(\aEl/(2*\rEl))}:90+asin(\aEl/(2*\rEl)):\rEl);
	%\draw [red](P1) ++ (-90:0.4) arc (-90:-90+\angKatus:0.4);
	\end{tikzpicture}
\end{center}
Because the screen is at the distance $a$ from the source of the electrons then according to the Pythagorean theorem we get the maximal $y$ value of the spot to be 
$$b=\sqrt{4R^2-a^2} = \sqrt{4\left(\frac{mv}{Be}\right)^2-a^2}.$$
From here we see that a smaller diameter corresponds to a smaller $R$ and the maximal $y$-directional diameter is indeed when the particle moves on the $x-y$ plane like we claimed in the beginning.\\
Turning the circle’s arc along the point $O$ we see that for the minimal $y$-directional value the circle is a tangent to the screen, see figure:
\begin{center}
	\begin{tikzpicture}
	\def\aEl{1};
	\def\rEl{1.5};
	\coordinate (P1) at (\rEl-\aEl,{-sqrt(\rEl*\rEl-(\rEl-\aEl)*(\rEl-\aEl))});
	\node[label=above right:O, fill=black, circle,inner sep=1pt] at (0,0){};
	\node[label=below right:Q, fill=black, circle,inner sep=1pt] at (P1){};
	\path let \p1=(P1) in node[label=below:T, fill=black, circle,inner sep=1pt] at (0,\y1){};
	%\path let \p1=(P1) in node [label=left:b] at (-\aEl, \y1/2){};
	\node[label=above:a] at (-\aEl/2,0){};
	\node [label=above:$y$] at (-\aEl, 0.5*\rEl){};
	\draw[->] (-\aEl,-2*\rEl) -- (-\aEl, 0.5*\rEl);
	\draw (0,0) -- (P1);
	\draw (P1) -- ($(P1)-(1.0*\rEl,0)$);
	\draw (0,0) -- (-\aEl,0);
	%\draw (0,0) arc ({180-acos((\rEl-\aEl)/\rEl)}:{360}:\rEl);
	\draw (P1) ++ (-180:\rEl) arc (180:-180:\rEl);
	%\draw (0,0) arc ({180-acos((\rEl-\aEl)/\rEl)}:180:\rEl);
	%\draw [red](P1) ++ (-90:0.4) arc (-90:-90+\angKatus:0.4);
	\end{tikzpicture}
\end{center}
Because the length of the segment QT is $R-a$ then we can find the segment’s CQ $y$-directional component with the Pythagorean theorem
$$y\idx{min}=-\sqrt{R^2-(R-a)^2} = -\sqrt{2aR-a^2}=-\sqrt{\frac{2amv}{Be}-a^2}.$$
Therefore the spot’s $y$-directional diameter is
$$L=\sqrt{\frac{2amv}{Be}-a^2} + \sqrt{4\left(\frac{mv}{Be}\right)^2-a^2}.$$
The brightness is biggest on this spot for the maximal value of $y$ because it corresponds to the electron’s $y$-directional deflection’s $\Delta_y$ extremum. If we mark the electron’s starting angle with $\alpha$ then 
$$\frac{\D \Delta_y}{\D \alpha}=0,$$
meaning in a small starting angle’s interval $\Delta\alpha$ all the electrons arrive to almost exactly the same destination point.\\
Let us find the $z$-directional diameter in the plane $y=0$. The particles that reach the furthest on the $z$-axis have a helical trajectory touching the plane $x=-a$ (from the “top view” of the $x-y$ plane a circle-shaped trajectory of radius $r=\frac a2$ touches the line $x=-a$) and start from the $y-z$ plane, meaning with an initial velocity which has a $x$-projection of $v_x=0$. Next we find
$$\frac a2=\frac {mv_y}{Be}\hence v_y=\frac{aBe}{2m}\hence v_z=\sqrt{v^2-\left(\frac{aBe}{2m}\right)^2}.$$
An electron with this initial velocity travels for half the cyclotron’s period. We can find the period from the relation $T=\frac{2\pi R}{v} = \frac{2\pi m}{Be}$. A half period is therefore $t=\frac{T}{2}=\frac{\pi m}{Be}$ and we get the spot’s $z$-directional dimension to be 
$$2\cdot tv_z=\pi\sqrt{\left(\frac{2mv}{Be}\right)^2-a^2}.$$
\probend