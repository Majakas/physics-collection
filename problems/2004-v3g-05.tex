\setAuthor{Tundmatu autor}
\setRound{lõppvoor}
\setYear{2004}
\setNumber{G 5}
\setDifficulty{1}
\setTopic{Teema}

\prob{Mäed}
Kahe teravatipulise lumise mäe vahel orus on keskpäeval päike seniidis. Läänepoolse mäe kõrgus oru põhjast on $H=\SI{2000}{m}$ ja idapoolse mäe kõrgus oru põhjast on $h=\SI{1500}{m}$. Mägede tippude vaheline kaugus horisontaalsihis on $l=\SI{5}{km}$ ja oru põhi on mõlemast tipust võrdsel kaugusel. Mitu tundi pärast otsese päikesevalguse kadumist toimub orus järsk hämardumine? Eeldada, et taevas on pilvitu.

\hint

\solu
Lumised mägede tipud toimivad hajusate valguskiirguritena seni, kuni otsene päikesevalgus paistab mägede tippudele. Seega läheb orus pimedaks, kui päike ei paista üle läänepoolse mäe enam idapoolse mäe tippu. Selleks peab Maa pöörama mägede tippe ühendava ja läänepoolse mäe tippu oru põhja keskpunktiga ühendava sirgete vahelise nurga $\alpha$ võrra:
$$
\alpha=\arctan \frac{H}{l / 2}-\arctan \frac{H-h}{l}=\ang{32,9}.
$$
Kuna ööpäeva pikkus on 24 tundi, siis pöörleb Maa ühe tunniga
$$
\frac{\ang{360} \cdot \SI{1}{h}}{\SI{24}{h}}=\ang{15}
$$
võrra. Seega pöörab Maa nurga $\alpha$ vôrra ajaga
$$
\frac{32,9^{\circ}}{15^{\circ}} \cdot 1 \mathrm{~h}=2 \mathrm{~h} 12 \mathrm{~min}
$$
\probend