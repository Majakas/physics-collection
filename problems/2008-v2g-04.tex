\setAuthor{Urmo Visk}
\setRound{piirkonnavoor}
\setYear{2008}
\setNumber{G 4}
\setDifficulty{1}
\setTopic{Termodünaamika}

\prob{Jääkuul}
Õhukeste seintega jääst kera sees on õhk. Algselt on jääkera külmkapis temperatuuril $t_0 = \SI{-9}{\degreeCelsius}$ ning õhurõhk tema sees võrdub välisrõhuga $p_0 = \SI{105}{kPa}$. Kera tõstetakse külmikust välja tuppa, kus see hakkab soojenema. Kera sein on nii õhuke, et maksimaalne ülerõhk (st. sise- ja välisrõhkude vahe), mida ta purunemata talub, on $\Delta p = \num{0,2}p_0$. Mis juhtub enne: kas kera hakkab sulama või ta puruneb ülerõhu tõttu? Kuuli soojenemine lugeda nii aeglaseks, et igal ajahetkel võib lugeda õhu temperatuuri tema sees ning seinte sise- ja välispinna temperatuurid võrdseks.

\hint
Jääkera sees tekib ülerõhk, sest õhu soojenedes rõhk tõuseb. Seega on kõige kriitilisem moment vee sulamise temperatuuril.

\solu
Rõhk kuuli sees kasvab seetõttu, et õhk kuulis soojeneb. Ülesande teksti põhjal võime eeldada, et õhu temperatuur kuuli sees on võrdne tema seinte temperatuuriga. Meie ülesandeks on kontrollida, kui palju on rõhk kasvanud selleks hetkeks, kui seinad hakkavad sulama, st on saavutanud temperatuuri $t_1 = \SI{0}{\degreeCelsius}$. 

Eeldame, et kera soojuspaisumine on tühine. Siis on õhu ruumala keras konstantne. Isohoorilises protsessis kehtib seos 
\[
\frac{p_1}{T_1} = \frac{p_0}{T_0}.
\]
Tähistame indeksiga \enquote{0} gaasi omadusi külmikus ja indeksiga \enquote{1} omadusi temperatuuril, mille juures seinad hakkavad sulama. Niisiis 
\[
p_1 = p_0 \frac{T_1}{T_0}.
\] 
Kasutades seda tulemust saame avaldada rõhu suhtelise muutuse 
\[
\frac{\Delta p}{p_{0}}=\frac{p_{1}-p_{0}}{p_{0}}=\left(\frac{T_{1}}{T_{0}}-1\right).
\]
Leiame selle avaldise numbrilise väärtuse: 
\[
\frac{\Delta p}{p_0} \approx \num{0,034} = \SI{3,4}{\%}.
\] 
See on selgelt väiksem kui kuuli seinte purunemispiir, st kuul hakkab enne sulama (kuid puruneb ilmselt ülerõhu tõttu enne lõplikku ära sulamist). 

\emph{Märkus:} alternatiivse ja võrdväärse lahendusena võib leida, millise õhutemperatuuri juures saavutaks suhteline ülerõhk väärtuse \SI{20}{\%} (selleks tuleb \SI{317}{K} ehk \SI{44}{\degreeCelsius}) ja võrrelda seda jää sulamistemperatuuriga.
\probend