\ylDisplay{Kuulid} % Ülesande nimi
{Jaan Kalda} % Autor
{lahtine} % Voor
{2016} % Aasta
{G 10} % Ülesande nr.
{9} % Raskustase
{
% Teema: Elektrostaatika
\ifStatement
Kaks metallkuulikest raadiusega $R$ on ühendatud peenikese metalltraadi abil ja asuvad homogeenses 
elektriväljas tugevusega $E$. Metalltraadi pikkus on $l$, kusjuures $l\gg R$. 
Süsteem on tasakaalus. Leidke mehaaniline pinge $T$ traadis.
\fi


\ifHint
Metallkeradele indutseeritakse elektrivälja poolt võrdsed ja vastasmärgilised laengud $\pm q$; kuivõrd traat on peenike,
siis võime ignoreerida sellel olevaid laenguid. Kuna metallkuulid käituvad nagu elektriline dipool, on süsteem tasakaalus siis, kui metalltraat on paralleelne elektriväljaga. Et traat on juhtivast materjalist, siis on süsteem ekvipotentsiaalne.
\fi


\ifSolution
Metallkeradele indutseeritakse elektrivälja poolt võrdsed ja vastasmärgilised laengud $\pm q$; kuivõrd traat on peenike,
siis võime ignoreerida sellel olevaid laenguid. Et traat on juhtivast materjalist, on süsteem ekvipotentsiaalne,
ning võrdne süsteemi sümmmeetriatasandis lõpmatuses paiknevate punktide potentsiaaliga:
$$E\frac l2=\frac q{4\pi\varepsilon_0R}\;\;\Rightarrow\;\; q=2lE\pi\varepsilon_0R.$$
Seetõttu mõjub kummalegi kuulile elektrostaatiline jõud $qE$, mille tasakaalustab niidi pinge:
$$T=2lE^2\pi\varepsilon_0R.$$
\fi


\ifEngStatement
% Problem name: Balls
Two metal balls of radius $R$ are connected to each other with a thin metal wire. The balls are in a homogeneous electric field of strength $E$. The length of the metal wire is $l$, furthermore $l\gg R$. The system is at equilibrium. Find the mechanical tension $T$ in the wire.
\fi


\ifEngHint
Equal and opposite charges $\pm q$ are induced on the metal plates by the electric field; since the wire is thin we can ignore the charges on it. Because the metal balls act like an electric dipole the system is in equilibrium if the metal wire is parallel to the electric field. Because the wire is made of conductive material the system is equipotential.
\fi


\ifEngSolution
The metal balls are induced by the electric field with charges $\pm q$ that are equal and opposite in sign; since the wire is thin then we can ignore the charges on it. Because the wire is made of conducting material then the system is equipotential and equal to the potential of the points that are located at the infinity of the system’s symmetry plane:
$$E\frac l2=\frac q{4\pi\varepsilon_0R}\;\;\Rightarrow\;\; q=2lE\pi\varepsilon_0R.$$ 
Therefore both of the balls are applied with an electrostatic force $qE$ that is balanced by the tension:
$$T=2lE^2\pi\varepsilon_0R.$$
\fi
}