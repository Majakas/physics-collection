\ylDisplay{Kile} % Ülesande nimi
{Jaan Kalda} % Autor
{lahtine} % Voor
{2008} % Aasta
{G 8} % Ülesande nr.
{7} % Raskustase
{
% Teema: Laineoptika
\ifStatement
Selleks, et vähendada peegeldusi optilistelt klaasidelt, kaetakse nende pinnad õhukese läbipaistva kilega. Leida, millise paksusega peaks olema selline kile, kui klaasi murdumisnäitaja on $n_0 = \num{1,5}$ ja kile oma $n_1 = \num{1,3}$. Eeldada, et kile on optimeeritud risti langeva rohelise valguse jaoks lainepikkusega $\lambda = \SI{530}{nm}$.
\fi


\ifHint
Valgus peegeldub tagasi kile ülemiselt ja alumiselt pinnalt. Peegeldus on minimaalne, kui vastavad kiired liituvad vastasfaasides. Selleks, et määrata mitu täislainepikkust optiliste teede vahesse mahub, tuleb vaadelda, kuidas minimeerida peegeldunud valguse hulka punase ja sinise valguse jaoks. 
\fi


\ifSolution
Valgus peegeldub tagasi kile ülemiselt ja alumiselt pinnalt. Risti langeva valguse puhul on nende vaheline optiliste teede pikkuse erinevus $2n_1d$, kus $d$ on (otsitav) kile paksus. Peegeldus on minimaalne, kui need kiired liituvad vastasfaasis, st
\[
2n_1d = \left( N + \frac 12 \right)\lambda.
\]

Kui tahetakse, et peegeldunud valgus oleks nõrk ka punase ja sinise valguse jaoks, siis tuleb täisarvu $N$ väärtus võtta võimalikult väike. Tõepoolest, olgu $\lambda - \Delta \lambda $ selline lainepikkus, mille puhul peegeldunud kiired liituvad samas faasis, st peegeldunud valgus on maksimaalse intensiivsusega. Sellisel juhul $2n_1d = N(\lambda + \Delta \lambda )$. Kahest võrdusest saame $\Delta \lambda = \lambda /2N$. Et antud juhul oleme huvitatud võimalikult suurest $\Delta \lambda$ väärtusest, siis tuleb valida $N = 1$. Niisiis $d = \lambda /4n_1 \approx \SI{102}{nm}$.
\fi
}