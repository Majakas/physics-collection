\ylDisplay{Korsten} % Ülesande nimi
{Jaan Kalda} % Autor
{piirkonnavoor} % Voor
{2009} % Aasta
{G 9} % Ülesande nr.
{7} % Raskustase
{
% Teema: Gaasid
\ifStatement
Hinnake, milline oleks suitsu kiirus korstnast väljumisel, kui õhutakistusega (sh turbulentsest liikumisest tingitud takistusega) korstnas ning ahjulõõrides võiks mitte arvestada.
Korstna kõrgus (mõõdetuna korstnajala juurest, kuhu siseneb ahjust tulev soe õhk) on $h=\SI{10}{m}$ ja õhu keskmine temperatuur korstnas $t=\SI{80}{\celsius}$. Lugeda, et ahju uks ja korstna jalg on samal kõrgusel. Välisõhu temperatuur on $t_0=\SI{0}{\celsius}$.
\fi


\ifHint
Korstna sees voolav õhk tekitab alarõhu, mis on avaldatav Bernoulli võrrandist. Rõhu langud korstna sees ja väljas peavad olema võrdsed.
\fi


\ifSolution
Ajusuusse siseneva õhu rõhk on võrdne õhurõhuga ahjusuu kõrgusel $p_1$ ning korstnast väljuv õhk on võrdne õhurõhuga korstnasuu kõrgusel $p_2=p_1-\rho_0 gh$;
$\rho_0$ tähistab välisõhu tihedust. Seega kehtib Bernoulli seaduse kohaselt seos 
\[
p_0=p_0-\rho_0 gh +\rho gh+\rho \frac{v^2}{2},
\]
kus $\rho_0=T\rho_0/T_0$ on õhu tihedus korstnas ning $v$ on otsitav kiirus. Seega
\[
v=\sqrt {2(\frac T{T_0}-1)gh}\approx \SI{3,5}{m/s}.
\]

\emph{Märkus}. ahjusuhu voolava õhu kiiruse võib lugeda tühiselt väikseks, sest voolava õhuga piirkonna ristlõike pindala on ilmselt hulga suurem korstna ristlõikepindalast.


Bernoulli seaduse võib tuletada ka energia jäävuse seadusest. Samuti lugeda õigeks impulsibalansi abil saadav tulemus (mis tuleb $\sqrt 2$ korda väiksem), sest nõuti vaid hindamist.
\fi
}