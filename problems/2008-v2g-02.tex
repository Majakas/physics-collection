\ylDisplay{Autod} % Ülesande nimi
{Jaan Kalda} % Autor
{piirkonnavoor} % Voor
{2008} % Aasta
{G 2} % Ülesande nr.
{3} % Raskustase
{
% Teema: Kinemaatika
\ifStatement
Juuresolev joonis on tehtud kõrgelt otse alla pildistatud foto põhjal, millel on jäädvustatud kaks autot (tähistatud punktidega $A$ ja $B$), mis lähenevad ristmikule jäävate kiirustega $v_A = \SI{40}{km/h}$ ja $v_B = \SI{60}{km/h}$. Kasutades joonist ja sellel antud mõõtkava, leidke autode edasisel liikumisel nende vaheline minimaalne kaugus.

\begin{center}
	\includegraphics[width=0.9\linewidth]{2008-v2g-02-yl}
\end{center}
\fi


\ifHint
Ülesannet on mugavam vaadelda emma-kumma autoga seotud taustsüsteemis.
\fi


\ifSolution
Kanname joonisele autode $A$ ja $B$ kiirusvektorid suvalises mõõtkavas (st vektorite moodulid suhtuvad nagu 40:60). Leiame nende vektorite vahe, see on autode suhteline kiirus. Tõmmates ühe auto juurest selle vektori sihilise sirge leiame tema trajektoori teise autoga seotud süsteemis. Teise auto kaugus sellest sirgest annabki vastuse. Mõõtkava arvestamine ja mõistlik numbriline tulemus annab \SI{60}{m}.
\begin{center}
	\includegraphics[width=0.7\linewidth]{2008-v2g-02-lah}
\end{center}
\fi
}