\ylDisplay{Solenoid} % Ülesande nimi
{Andres Laan} % Autor
{lõppvoor} % Voor
{2011} % Aasta
{G 5} % Ülesande nr.
{4} % Raskustase
{
% Teema: Magnetism
\ifStatement
Õhksüdamikuga solenoidis (pikas silindrilises poolis) on vool
$I$. Solenoidi sisemuses liigub elektron, mille trajektoor kujutab endast sellist
kruvijoont, mille keerdude arv on võrdne solenoidi keerdude arvuga. Leidke
selle elektroni kiiruse teljesihiline komponent. Võib eeldada, et elektroni kiiruse teljega risti olev komponent on piisavalt väike, et kokkupõrkeid solenoidi
seinaga ei toimu. Elektroni mass on $m$ ja laeng $e$.

\emph{Vihje}. Solenoidi sees on
homogeenne magnetväli induktsiooniga $B = \mu_0nI$, kus $n$ on solenoidi traadi
keerdude arv pikkusühiku kohta, $I$ selles olev vool ja $\mu_0$ vaakumi magnetiline
läbitavus.
\fi


\ifHint
Elektron sooritab solenoidi teljega risti tasandis ringliikumist. Antud ringliikumise periood on leitav kasutades Lorentzi ja tsentrifugaaljõu tasakaalutingimust.
\fi


\ifSolution
Väljatugevus solenoidi sees on $B=\mu nI$, kus $n$ on solenoidi traadi keerete arv pikkusühiku kohta, $I$ seda läbiv vool ja $\mu$ vaakumi magnetiline läbitavus. Väli on suunatud pikki solenoidi telge. Kui selle välja suunaga on risti mingisugunegi kiirus $v$, siis ühe pöörde tegemiseks kulub aeg $T=2\pi m/eB$ (tuletatav Lorentzi ja tsentrifugaaljõu tasakaalust), kus $m$ ja $e$ on vastavalt elektroni mass ja laeng. Olgu elektronil ka solenoidi telje sihiline kiirus $v$. Ühikulises ajas läbib ta distantsi $1/v$. Selle aja sees teeb ta $1/(vT)$ pööret. Nende pöörete arv ülesande püstituse kohaselt peab olema $n$. Seega
\[
n=\frac{1}{vT}=\frac{eB}{2v\pi m}=\frac{e\mu nI}{2v\pi m}.
\]
Siit saame $v=e\mu I/(2\pi m)$.

Telje sihiline komponent kiirusel on üheselt määratud. Teljega risti olev kiiruse komponent peab olema nullist suurem.
\fi
}