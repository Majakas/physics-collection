\setAuthor{Tundmatu autor}
\setRound{lahtine}
\setYear{2011}
\setNumber{G 7}
\setDifficulty{6}
\setTopic{Kinemaatika}

\prob{Hävituslennuk}
Ühel ilusal augustipäeval käis Mati paraadil vaatamas NATO hävituslennukeid, mis
tegid rahva kohal demonstratsioonlende. Diktor ütles valjuhääldist, et lennuk
lendab horisontaalselt üle rahva kiirusega $v=\SI{1350}{km/h}$. Matit huvitas aga, kui
kõrgel lennuk lendab. Vajalike mõõtetulemuste saamiseks seisis ta nii, et tema
ja läheneva lennukiga ühele joonele jäi täpselt üks 9 meetri pikkune elektripost
ning Mati ise asus teise posti juures; postide vahekaugus oli \SI{50}{m}. Mati käivitas oma
mobiiltelefoni stopperi just siis, kui lennuk posti ülemise otsa tagant nähtavale ilmus, ning 
peatas hetkel, kui käis kõva pauk ja hakkas kostuma lennuki müra. Ta sai
stopperi näiduks \SI{32,04}{s}. Kodus
mõõtis ta üle ka enda silma kõrguse maapinnast: $l=\SI{1,68}{m}$.
Kui kõrgel lendas lennuk? Heli kiirus õhus on umbes
$u=\SI{330}{m/s}$.\\ 
\textit{Vihje:} kui lennuk lendab
ülehelikiirusel, siis levib tema taga
koonusekujuline lööklaine front, kusjuures koonuse tipus on lennuk ja selle
koonuse telglõike
tipunurk on $\alpha=2\arcsin\left(\frac{u}{v}\right)$.

\hint
Ülesande geomeetria ning lööklaine koonuse nurga kaudu on võimalik avaldada lennuki koordinaadid stopperi käivitamise ja peatamise hetkedel. Leitud nihke kaudu on lennukiirus lihtsasti leitav.

\solu
Olgu hetkel, mil Mati käivitab stopperi, lennuki horisontaalsuunaline kaugus
temast $x_{1}$ . Tähistame lennuki lennukõrguse $H$, posti pikkuse $h$ ja
kauguse
$L$. Tekib kaks kujutletavat täisnurkset sarnast kolmnurka, mille ühise nurga
tipus asub Mati. Nendest saame (kasutades eeldust, et ilmselt $l\ll H$):
\[\frac{H-l}{x_{1}}\approx\frac{H}{x_{1}}=\frac{h-l}{L}. \]
Kui lennuk on jõudnud üle Mati pea, jõuab temani lööklaine hetkel, mil lennuki horisontaalkaugus $x_{2}$ Matist on
\[ x_{2}=\frac{H}{\tan\left(\frac{\alpha}{2}\right)}=\frac{H}{\tan\left(\arcsin\left(\frac{u}{v}\right)\right)}=\frac{H}{u}\sqrt{v^2-u^2}.
\]
Olgu $\tau$ mõõdetud aeg. Stopperi käivitamise hetkest seiskamiseni liikus lennuk vahemaa
\[ x_1+x_2=H\left(\frac{\sqrt{v^2-u^2}}{u} +\frac{L}{h-l}\right)=v\tau
\Rightarrow
H=\frac{v\tau}{\left(\frac{\sqrt{v^2-u^2}}{u} +\frac{L}{h-l}\right)}\approx
\SI{1630}{m}.
\]
\probend