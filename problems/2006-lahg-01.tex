\setAuthor{Tundmatu autor}
\setRound{lahtine}
\setYear{2006}
\setNumber{G 1}
\setDifficulty{1}
\setTopic{Gaasid}

\prob{Jalgpall}
Kui suure rõhuni $p_N$ võib pumbata jalgpalli palli kolbpumbaga, kui pumpa surutakse kokku $N = 40$ korda? Iga pumpamiskäigu jooksul võtab pump atmosfäärist õhu koguse ruumalaga $v = \SI{150}{cm^3}$. Atmosfääri rõhk $p_0 = \SI{0,1}{MPa}$, palli ruumala $V = \SI{3}{l}$. Lugeda, et õhu temperatuur pallis võrdub välistemperatuuriga.

\hint
Pumpamise käigus kehtib ideealse gaasi olekuvõrrand.

\solu
Iga pumpamiskäigu alguses täidab atmosfääri õhk rõhuga  $p_0$ pumba siseruumi ruumalaga $v$. Pumpamiskäigu lõpus on see õhk pallis, kus ta ruumala on $V$ ja osarõhk $p$. Viimase saame leida Boyle-Mariotte’i seadusest:
\[
p = \frac{p_0v}{V}.
\]
Pärast $N$ pumpamiskäiku on rõhk pallis võrdne osarõhkude summaga:
\[
p_{N}=N p=\frac{N v p_{0}}{V}=\frac{40 \cdot 150 \cdot \num{0,1}}{3000}=\SI{0,2}{MPa}.
\]
\probend