\ylDisplay{Pendel} % Ülesande nimi
{Mihkel Heidelberg} % Autor
{piirkonnavoor} % Voor
{2008} % Aasta
{G 1} % Ülesande nr.
{1} % Raskustase
{
% Teema: Staatika
\ifStatement
Otsast kinnitatud varras saab pöörelda ümber horisontaaltelje ühes tasandis. Varda otsa on kinnitatud koormis massiga $m$. Varda pikkus on $l$. Varda kinnitusele mõjub hõõrdest tingitud pidurdav jõumoment $M$. Millistes nurkade vahemikes võib olla varras paigal (vt joonist)? Arvestada, et $mgl > M$.
\begin{center}
	\includegraphics[width=0.3\linewidth]{2008-v2g-01-yl}
\end{center}
\fi


\ifHint
Kangi kriitilise nurga korral kehtib varda jaoks jõumomentide tasakaal.
\fi


\ifSolution
Koormisele mõjub raskusjõu moment $mgl\sin \alpha$. Kang püsib paigal, kui see on väiksem hõõrdejõu momendist $M$, seega $mgl\sin \alpha < M$, millest $\sin \alpha < \frac{M}{mgl}$.
\fi
}