\ylDisplay{Värinaalarm} % Ülesande nimi
{Jaan Kalda} % Autor
{lahtine} % Voor
{2011} % Aasta
{G 9} % Ülesande nr.
{9} % Raskustase
{
% Teema: Dünaamika
\ifStatement
Uurime mobiiltelefoni liikumist nõrgalt kaldus pinnal värinaalarmi töötamise
ajal lihtsustatud mudeli abil.
Kujutagu lauale asetatud mobiil risttahukat massiga $M$, mille sees liigub
üles-alla väike keha massiga $m$.
Liikugu see keha ajahetkedel $t=0, 2\tau, 4\tau, \ldots$ vahemaa $x$ võrra
hetkeliselt üles ning ajahetkedel $t=\tau, 3\tau, 5\tau, \ldots$ algasendisse
tagasi.
Olgu mobiiltelefoni ja laua vaheline hõõrdetegur $\mu$ ning laua kaldenurk
$\alpha \ll 1$. Mobiiltelefoni ja laua vahelised põrked lugege absoluutselt
plastseiks.
Millise keskmise kiirusega $u$ hakkab mobiiltelefon mööda lauda liikuma?
\fi


\ifHint
Väikse keha hetkelise liikumise käigus püsib mobiil+keha massikese paigal. Seega liigub mobiil iga $\tau$ tagant sarnaselt kehaga hetkeliselt ülesse või alla. Iga kord kui mobiil ülesse liigub, nihkub ta gravitatsiooni tõttu ka veidike laua sihis edasi.
\fi


\ifSolution
Kuivõrd väike keha liigub alla hetkeliselt, siis süsteemi mobiil+keha masskese püsib paigal, mistõttu mobiil kerkib lauapinnast kõrgusele $h=x\frac mM$.
Edasi hakkab mobiil raskusjõu toimel langema; lauapinnale jõudmiseks kuluv aeg $t=\sqrt{2h/g}$. Hõõrdejõud peatab mobiili ilma libisemata, kui $\mu>\tan\alpha$; et
$\alpha \ll 1$, siis võime eeldada, et see nii ka juhtub. Kui $t=\sqrt{2h/g}<\tau$, siis jõuab mobiil liikuda langemise jooksul lauapinna sihis vahemaa $\delta=h\sin\alpha\approx h\alpha$,
mis annab keskmiseks kiiruseks $$u=\frac{xm\alpha}{2\tau M}.$$ Kui $\sqrt{2h/g}<\tau$, siis ei jõua mobiil lõpuni langeda, vaid väikese keha kerkimine surub mobiili ennatlikult vastu lauda tagasi.
Mobiil jõuab langeda vahemaa $H=g\tau^2/2$ võrra, mis annab keskmiseks kiiruseks
$$u=\frac{H\alpha}{2\tau}=\frac 14 g\tau\alpha.$$
Kokkuvõtlikult võib vastuse anda kujul
$$u=\min\left(\frac{xm\alpha}{2\tau M}, \frac 14 g\tau\alpha\right).$$
\fi
}