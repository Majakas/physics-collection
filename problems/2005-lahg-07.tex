\ylDisplay{Anum} % Ülesande nimi
{Tundmatu autor} % Autor
{lahtine} % Voor
{2005} % Aasta
{G 7} % Ülesande nr.
{5} % Raskustase
{
% Teema: Dünaamika
\ifStatement
Siledal pinnal asub kerge ristkülikuline anum, mis on täidetud vedelikuga tihedusega $\rho_0$, vedeliku ruumala on $V_0$. Anuma põhja sattunud põrnikas ruumalaga $V$ ja tihedusega $\rho$ hakkab anuma põhja suhtes roomama kiirusega $u$. Millise kiirusega hakkab anum pinnal liikuma? Anuma mass on tühine, veetase jääb kogu aeg horisontaalseks. Eeldada, et pinna ja anuma vahel hõõre puudub.
\fi


\ifHint
Kui põrnikas (massiga $\rho V$) roomab mööda anuma põhja, siis selle peale liigub ka põrnikat ümbritsev vedelik. Põrnika liikumist võib mugavuse mõttes ette kujutada virtuaalse põrnika liikumisega, mille tihedus on $\rho - \rho_0$. Sellisel juhul liigub virtuaalne põrnikas vedelikku tõrjumata ning ülesanne taandub mugavamale dünaamika ülesandele.
\fi


\ifSolution
Vaatame algul liikuva anumaga seotud taustsüsteemi. Kui põrnikas (massiga $\rho V$) roomab mööda anuma põhja, siis selle peale liigub ka põrnikat ümbritsev vedelik. Põrnika liikumist võib mugavuse mõttes ette kujutada virtuaalse põrnika liikumisega, mille tihedus on $\rho - \rho_0$. Sellisel juhul liigub virtuaalne põrnikas vedelikku tõrjumata ning ülesanne taandub mugavamale dünaamika ülesandele.

Liigume nüüd põrandaga seotud taustsüsteemi. Vastavalt impulsi jäävuse seadusele on põrnika ja anuma koguimpulss kogu aeg konstantne (sest põrnika+anuma süsteemile ei mõju väliseid jõude). Kuna see oli alguses 0, on põrnika ja anuma impulsid vastassuunalised ning absoluutväärtuse poolest võrdsed. Olgu anuma kiirus põranda suhtes $v$. Siis põrnika kiirus põranda suhtes on $u - v$. Impulsi jäävuse seadusest tulenevalt
\[
\rho_0 (V_0 + V ) v = (\rho - \rho_0) V (u - v),
\]
ehk
\[
v=\frac{V\left(\rho_{0}-\rho\right)}{\rho V+\rho_{0} V_{0}}u.
\]
\fi
}