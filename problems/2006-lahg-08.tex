\ylDisplay{Õhk} % Ülesande nimi
{Tundmatu autor} % Autor
{lahtine} % Voor
{2006} % Aasta
{G 8} % Ülesande nr.
{7} % Raskustase
{
% Teema: Gaasid
\ifStatement
Leida niiske (suhteline niiskus $f = \SI{90}{\%}$) ja kuiva õhu tiheduste suhe rõhu $p_0 = \SI{0,1}{MPa}$ ja temperatuuri $t = \SI{27}{\celsius}$ juures. Küllastunud auru tihedus sellel temperatuuril on $\rho_0 = \SI{0,027}{kg/m^3}$. Õhu molaarmass $\mu_1 = \SI{0,029}{kg/mol}$, vee molaarmass $\mu_2 = \SI{0,018}{kg/mol}$.
\fi


\ifHint
Niiske õhu puhul on veeauru ja normaalse õhu rõhkude summa võrdne atmosfäärirõhuga, kusjuures veeauru rõhk on leitav küllastunud veeauru tihedusest.
\fi


\ifSolution
Kuiva õhu tihedus
\[
\rho_1 = \frac{\mu_1p_0}{RT}.
\]
Suhtelise niiskusega $f$ auru tihedus
\[
\rho' = f \rho_0 = \frac{\mu_2p_2}{RT},
\]
kus $p_2$ on auru osarõhk. Siit:
\[
p_2 = \frac{\rho_0fRT}{\mu_2}.
\]
Kuna niiske õhu rõhk on õhu ja auru osarõhkude $p_1$ ja $p_2$ summa, siis
\[
p_1 = p_0 - p_2 = p_0 - \frac{\rho_0fRT}{\mu_2}.
\]
Õhu tihedus (ilma auruta) sellel osarõhul
\[
\rho^{\prime \prime}=\frac{\mu_{1} p_{1}}{R T}=\frac{\mu_{1} p_{0}}{R T}-\frac{\mu_{1} f \rho_{0}}{\mu_{2}}.
\]
Niiske õhu tihedus
\[
\rho_{2}=\rho^{\prime}+\rho^{\prime \prime}=\frac{\mu_{1} p_{0}}{R T}-\left(\frac{\mu_{1}}{\mu_{2}}-1\right) f \rho_{0}.
\]
Niiske ja kuiva õhu tiheduste suhe on seega
\[
\begin{aligned}
\frac{\rho_{2}}{\rho_{1}}&=1-\frac{\left(\mu_{1}-\mu_{2}\right) f \rho_{0} R T}{\mu_{1} \mu_{2} p_{0}} \\ 
&=1-\frac{(\num{0,029}-\num{0,018}) \cdot \num{0,9} \cdot \num{0,027} \cdot \num{8,31} \cdot 300}{\num{0,029} \cdot \num{0,018} \cdot 10^{5}} \approx \num{0,987}.
\end{aligned}
\]
\fi
}