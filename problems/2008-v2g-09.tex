\ylDisplay{Plokid} % Ülesande nimi
{Mihkel Kree} % Autor
{piirkonnavoor} % Voor
{2008} % Aasta
{G 9} % Ülesande nr.
{8} % Raskustase
{
% Teema: Dünaamika
\ifStatement
Polüspast ehk liitplokk koosneb seitsmest plokist (vt. joonist). Koormiste massid $M$ ja $\gamma M$ on näidatud joonisel. Missuguse kiirendusega hakkavad liikuma äärmised koormised? Mis tingimust peab rahuldama suurus $\gamma$, et äärmised koormised hakkaksid langema? Plokkide ja nööri mass jätta arvestamata ning nöör lugeda venimatuks. 

\begin{center}
	\includegraphics[width=0.6\linewidth]{2008-v2g-09-yl}
\end{center}
\fi


\ifHint
Ülesandes on kolm tundmatut: keskmiste plokkide kiirendused, äärmiste plokkide kiirendused ning niidi pinge. Vastavate tundmatute leidmiseks on vaja kolme võrrandit, kaks tulenevad Newtoni II seadusest ning üks tuleb niidi venimatuse tingimusest.
\fi


\ifSolution
Rakendades Newtoni II seadust näeme, et kõik kolm keskmist koormist hakkavad liikuma võrdse kiirendusega $a_0$: 
\[
M a_0 = 2T - Mg,
\]
kus $T$ on niidi pinge. Rakendades Newtoni II seaduste äärmiste koormiste jaoks saame
\[
\gamma M a_1 = T - \gamma Mg,
\]
kus $a_1$ on äärmiste koormiste kiirendus. Elimineerides niidi pinge $T$ saame
\[
2\gamma a_1 - a_0 = g - 2\gamma g.
\] 
Nööri venimatus avaldub kujul $a_1 = -3a_0$, millest tulenevalt 
\[
-2 \gamma a_{1}-\frac{a_{1}}{3}=(2 \gamma-1) g \quad \Rightarrow \quad a_{1}=\frac{1-2 \gamma}{2 \gamma+1 / 3} g.
\]
Äärmised koormised hakkavad langema, kui $a_1$ on negatiivne. Selle jaoks peab kehtima
\[
1-2 \gamma<0 \quad\Rightarrow\quad \gamma>\frac{1}{2}.
\]
\fi
}