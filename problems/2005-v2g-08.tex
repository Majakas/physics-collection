\ylDisplay{Külmutusseade} % Ülesande nimi
{Tundmatu autor} % Autor
{piirkonnavoor} % Voor
{2005} % Aasta
{G 8} % Ülesande nr.
{5} % Raskustase
{
% Teema: Termodünaamika
\ifStatement
Külmutusseadme mootor teeb tööd $A = \SI{2}{kJ}$. Kui suur on maksimaalne külmutusseadmes kehalt äravõetav soojushulk, kui külmutusseadme sisetemperatuur on $t_k = \SI{-15}{\celsius}$ ja seadme jahutusvedeliku temperatuur on $t_j = \SI{10}{\celsius}$?

\emph{Vihje}.
Külmutusseadet võib vaadelda kui pööratud töötsükliga ideaalset soojusmasinat. See tähendab, et kõik tööd ja soojushulgad on vastupidise märgiga.
Niisiis mootor ei tee tööd, vaid tema töös hoidmiseks on vaja teha tööd; madala
temperatuuriga keskkond ei saa soojust, vaid annab soojust ära. 
\fi


\ifHint
Ideaalse soojusmasina kasutegur avaldub kujul $\eta = \frac{T_{j} - T_{k}}{T_{j}}$, kus $T$ on temperatuur kelvinites. Kasutegur on samas tehtud töö ja jahtusvedelikult võetud soojushulga suhe.
\fi


\ifSolution
Külmutusseadme korral on tegu pööratud soojusmasinaga, ideaalse
soojusmasina puhul kehtib seos
\[
\frac{A}{Q_{j}}=\frac{T_{j} - T_{k}}{T_{j}},
\]
kus mootori tehtud töö
on leitav energia jäävusest
\[
A = Q_j - Q_k.
\]
Siinjuures $Q_j$ ja $Q_k$ on vastavalt jahutusvedelikule ära antud ja jahutatavalt kehalt
ära võetud soojushulk. Elimineerides mittevajaliku suuruse $Q_j$ saame
\[
1+\frac{Q_{k}}{A}=\frac{T_{j}}{T_{j}-T_{k}},
\]
millest 
\[
Q_{k}=\frac{A T_{k}}{T_{j}-T_{k}} = \SI{22,6}{kJ}.
\]
\fi
}