\setAuthor{Ott Krikmann}
\setRound{piirkonnavoor}
\setYear{2005}
\setNumber{G 2}
\setDifficulty{3}
\setTopic{Staatika}

\prob{Katus}
Ühtlase lumekihiga kaetud katus on horisondi suhtes kaldu $\alpha = \ang{40}$ nurga all. Katus on ristküliku kujuline ja laius harjast räästani mööda katuse pinda on $L$. Katuse ja lume vaheline hõõrdetegur on $\mu = \num{1}$. Katuse harjast hakkab lumekihi ja katuse vahele voolama vesi, mis muudab märja katuse ja lumekihi vahelise hõõrdeteguri nulliks. Kui vesi jõuab katuseharjast kaugusele $l$, hakkab lumekiht alla libisema. Leidke suhe $l/L$.

\hint
Libisemise piiril tasakaalustab lund katuselt alla lükkavat raskusjõudu hõõrdejõud, kusjuures hõõrdejõud mõjub vaid lume kuivale osale.

\solu
Vaatleme lund katusel kui kahte vastasmõjus olevat keha: üheks kehaks on lumi, mille all on vesi ning millele hõõrdejõud ei mõju, ja teiseks kehaks kuival katusel olev lumi. Nende kahe osa vahel mõjuva jõu $F$ võime lugeda katuse sihiliseks (selle sihi valime $x$-teljeks, $y$-telg on katuse sihiga risti). Arvestame, et lumi on ühtlase paksusega ja seega osade massid on võrdelised nende pikkustega: 
\[
\frac{m_1}{m_2} = \frac{l}{L-l}.
\]
$y$-telje sihiline tasakaaluvõrrand kuiva osa jaoks:
\[
N_2 = m_2g \cos \alpha,
\]
kus $N_2$ on kuivale osale mõjuva katuse rõhumisjõud. $x$-telje sihiline tasakaaluvõrrand vesise ja kuiva osa jaoks:
\[
F = m_1g \sin \alpha,
\]
\[
\mu N_2 = F + m_2g \sin \alpha.
\]
Elimineerides kahest viimasest võrrandist $F$-i leiame
\[
\mu N_2 = (m_1 + m_2) g \sin \alpha.
\]
Asendades siia $N_2$ leiame
\[
m_2\mu g \cos \alpha = (m_1 + m_2) g \sin \alpha.
\]
Jagades läbi $m_2g$-ga ja asendades esimesest võrrandist suhte $m_1/m_2$, saame
\[
\frac{L}{L-l}=\mu \cot \alpha,
\]
millest
\[
\frac{l}{L}=1-\mu^{-1} \tan \alpha \approx \num{0,16}.
\]

\vspace{0.5\baselineskip}

\emph{Alternatiivne lahendus}

Vaatleme lund katusel tervikliku süsteemina. Valime $x$-teljeks katuse sihi, $y$-telg olgu katuse sihiga risti. Lumekihile mõjuvad järgnevad jõud: raskusjõud $mg$, katusepinna toereaktsioonijõud $N$ ning hõõrdejõud
\[
F_h = \frac{L-l}{L} \mu N.
\]
Kordaja $(L-l)/L$ tuleb sellest, et hõõrdejud mõjub vaid $(L-l)$-pikkusel katuseosal.
Tasakaaluvõrrand $y$-telje jaoks on
\[
N = mg \cos \alpha,
\]
$x$-telje jaoks aga
\[
m g \sin \alpha=F_{h}=\frac{L-l}{L} \mu N.
\]
Elimineerides kahest viimasest võrrandist $N$-i leiame
\[
m g \sin \alpha=\mu m g \cos \alpha \frac{L-l}{L}.
\]
Saame, et
\[
\frac{L-l}{L} = \frac{\tan\alpha}{\mu}.
\]
Seega
\[
\frac{l}{L}=1-\frac{\tan \alpha}{\mu} \approx \num{0,16}.
\]
\probend