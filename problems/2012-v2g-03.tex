\ylDisplay{Tünn} % Ülesande nimi
{Koit Timpmann} % Autor
{piirkonnavoor} % Voor
{2012} % Aasta
{G 3} % Ülesande nr.
{2} % Raskustase
{
% Teema: Vedelike-mehaanika
\ifStatement
Vees ujuva tühja tünni ruumalast on $1/10$ vee sees. Pärast tünni täitmist
tundmatu vedelikuga jääb tünn vee peale ujuma, kuid nüüd on vee sees $9/10$ tünni
ruumalast. Kui suur on tünni valatud vedeliku tihedus? Vee tihedus on
\SI{1000}{kg/m^3}.
\fi


\ifHint
Tünnile mõjuv üleslükkejõud peab olema võrdne raskusjõuga.
\fi


\ifSolution
Tühja tünni korral kehtib seos
\[
mg=\frac 1{10}\rho_vVg.
\]
Vedelikku täis tünni korral kehtib seos	
\[
(m+\rho V)g=\frac 9{10}\rho_vVg.
\]
Taandades ruumala $V$ ja $g$, saame
\[
\frac 1{10}\rho_v+\rho=\frac 9{10}\rho_v,
\]
millest 
\[
\rho=\frac 8{10}\rho_v = \SI{800}{kg/m^3}.
\]
\fi


\ifEngStatement
% Problem name: Barrel
An empty barrel floating in water has $1/10$ of its volume inside water. After filling the barrel with an unknown liquid the barrel will stay floating on the water but now $9/10$ of the barrel’s volume is inside the water. What is the density of the liquid that the barrel is filled with? The density of water is $\SI{1000}{kg/m^3}$.
\fi


\ifEngHint
The buoyancy force applied to the barrel has to be equal to the gravity force.
\fi


\ifEngSolution
In the case of an empty barrel the following relation applies: $mg=\frac 1{10}\rho_vVg$. In the case of a barrel filled with liquid the relation applies: $(m+\rho V)g=\frac 9{10}\rho_vVg$. Canceling out the volume $V$ and $g$ we get $\frac 1{10}\rho_v+\rho=\frac 9{10}\rho_v$ from which $\rho=\frac 8{10}\rho_v$. The answer $\rho= \SI{800}{kg/m^3}$.
\fi
}