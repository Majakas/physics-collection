\setAuthor{Tundmatu autor}
\setRound{lahtine}
\setYear{2009}
\setNumber{G 3}
\setDifficulty{4}
\setTopic{Elektriahelad}

\prob{Skeem}
Elemendi $X$ takistus muutub sõltuvalt selle pingest. Kui $U_X \leq \SI{1}{V}$,
siis selle takistus on $R_1 = \SI{1}{\ohm}$, kui aga $U_X > \SI{1}{V}$, siis on takistus $R_2 =
\SI{2}{\ohm}$. Kolm elementi $X$ ühendatakse ideaalse ampermeetriga, nagu näidatud joonisel.
Väljundklemmidele rakendatakse pinge, mille ajaline sõltuvus on toodud graafikul.
Joonestage ampermeetri näidu ajalise sõltuvuse graafik.

\begin{center}
	\includegraphics[width=\linewidth]{2009-lahg-03-yl}
\end{center}

\hint
Süsteem saab töötada kolmes erinevas režiimis. Esiteks, kui pinge on piisavalt madal või kõrge, on kõikide elementide takistus vastavalt \SI{1}{\ohm} või \SI{2}{\ohm}. Vahepealse pinge väärtuse korral on vasakpoolse elemendi takistus \SI{2}{\ohm} ja parempoolsetel \SI{1}{\ohm}. Teisi režiime ei ole, sest vasakpoolse takisti pinge on alati suurem kui parempoolsetel ning seega ei saa vasaku takisti takistus olla väiksem kui parempoolsetel.

\solu
Süsteem saab töötada kolmes režiimis:\\
(I) Kõigi elementide takitus on \SI{1}{\ohm}. Siis süsteemi kogutakistus on $R_I = \SI{1,5}{\ohm}$, vool $I_I = \frac{U}{\num{1,5}}(\si{A})$ ning pinge skeemi vasakpoolsel elemendil $\frac{2}{3}U$ ja parempoolsetel elementidel $\frac 13U$.\\
(II) Vasakpoolse elemendi takistus on \SI{2}{\ohm}, parempoolsemate elementide takistus on \SI{1}{\ohm}. Siis süsteemi kogutakistus on $R_I = \SI{2,5}{\ohm}$, vool $I_I = \frac{U}{\num{2,5}}(\si{A})$ ning pinge vasakpoolsel elemendil $\frac 45U$ ja parempoolsetel elementidel $\frac 15U$.\\
(III) Kõigi elementide takitus on \SI{2}{\ohm}. Siis süsteemi kogutakistus on $R_I = \SI{3}{\ohm}$, vool $I_I = \frac U3(A)$ ning pinge skeemi vasakpoolsel elemendil $\frac 23U$ ja parempoolsetel elementidel $\frac 13U$.

Vaatame süsteemi käitumist, kui klemmipinge kasvab. Alguses töötab süsteem režiimis I kuni hetkeni, mil klemmipinge kasvab väärtuseni $U = \SI{1,5}{V}$. Siis muutub vasakpoolse elemendi takistuse väärtus $R_2 = \SI{2}{\ohm}$-ks ning süsteem jätkab tööd režiimis II. Hetkel, mil klemmipingepinge kasvab väärtuseni $U = \SI{5}{V}$, muutub ka parempoolsete elementide takistus $R_2$-ks ning süsteem jätkab tööd režiimis III.

Vaatame süsteemi käitumist, kui klemmipinge kahaneb. Alguses töötab süsteem režiimis III kuni hetkeni, mil klemmipinge langeb väärtuseni $U = \SI{3}{V}$. Siis muutub parempoolsete elementide takistuse väärtus tagasi $R_1 = \SI{1}{\ohm}$-ks ning süsteem jätkab tööd režiimis II. Hetkel, mil klemmipingepinge langeb väärtuseni $U = \SI{1,25}{V}$, muutub ka vasakpoolse elemendi takistus tagasi $R_1$-ks ning süsteem jätkab tööd režiimis I. Voolutugevuse käitumine on esitatud graafikul.

\begin{center}
	\includegraphics[width=0.9\linewidth]{2009-lahg-03-lah}
\end{center}
\probend