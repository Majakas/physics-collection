\setAuthor{Jaak Kikas}
\setRound{piirkonnavoor}
\setYear{2008}
\setNumber{G 3}
\setDifficulty{2}
\setTopic{Termodünaamika}

\prob{Tulehõõrumine}
Jõuga $F$ otsapidi vastu tasast pinda surutud toru pöörleb sagedusega $f$. Toru läbimõõt on $D$ ja seina paksus $d \ll D$. Toru otspind on risti toru teljega, hõõrdetegur toru ja tasapinna vahel on $\mu$. Kui palju soojusenergiat vabaneb ajavahemiku $\Delta t$ jooksul?

\begin{center}
	\includegraphics[width=0.3\linewidth]{2008-v2g-03-yl}
\end{center}

\hint
Varda pöörlemise käigus muutub hõõrdejõu $F_h$ ületamiseks kulutatud töö soojuseks. Töö on leitav hõõrdejõu ja läbitud tee pikkuse korrutisena.

\solu
Varda pöörlemisel käigus muutub hõõrdejõu ületamiseks tehtud töö soojuseks. Toru otspinna ja aluse vahel mõjub hõõrdejõud $F_h$, mis võrdub pinnaga ristuva rõhumisjõu ja hõõrdeteguri korrutisega. Rõhumisjõuks on jõud $F$, millega surutakse toru vastu alust. Seega $F_h = \mu F$. Kui toru teeb ühe pöörde, siis läbib toru sein teepikkuse $L = \pi D$. Hõõrdejõu ületamiseks tehti ühe pöörde läbimisel töö $A = F_hL$. Kui toru pöörleb sagedusega $f$, siis aja $t$ jooksul teeb toru $N = f t$ pööret. Kokku eraldub toru pöörlemisel soojushulk
\[
Q=A N=\mu F \pi D f \Delta t.
\]
\probend