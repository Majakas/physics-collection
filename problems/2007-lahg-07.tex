\ylDisplay{Kuulikesed} % Ülesande nimi
{Tundmatu autor} % Autor
{lahtine} % Voor
{2007} % Aasta
{G 7} % Ülesande nr.
{4} % Raskustase
{
% Teema: Elektrostaatika
\ifStatement
Kaks ühesugust kuulikest, millest kumbki kannab laengut $q$, asuvad vertikaalsihis kaugusel $H$ üksteisest. Alumine kuulike on jäigalt kinnitatud, ülemine aga hakkab liikuma vertikaalselt alla suunatud algkiirusega $v$. Kui suur on minimaalne kaugus $h$ alumise kuulikeseni, millele suudab läheneda ülemine kuulike? Ülemise kuulikese mass on $m$. Raskuskiirendus on $g$.
\fi


\ifHint
Kehtib energia jäävuse seadus, kus peame arvestama nii gravitatsioonilise kui ka elektrilise potentsiaalse energiaga.
\fi


\ifSolution
Energia jäävuse seaduse kohaselt
\[
mgH + \frac{mv^2}{2} + \frac{kq^2}{H} = mgh + \frac{kq^2}{h},
\]
kus peale gravitatsioonilise potentsiaalse energia arvestasime ka elektrilise potentsiaaliga. Niisiis
\[
h-\frac{v^{2}}{2 g}-H-\frac{k q^{2}}{m g H}+\frac{k q^{2}}{m g h}=0,
\]
\[
h^{2}-\left(\frac{v^{2}}{2 g}+\frac{k q^{2}}{m g H}+H\right) h+\frac{k q^{2}}{m g}=0.
\]
Lahendades saadud ruutvõrrandi, saame
\[
h=\frac{1}{2}\left(\frac{v^{2}}{2 g}+\frac{k q^{2}}{m g H}+H\right)-\sqrt{\frac{1}{4}\left(\frac{v^{2}}{2 g}+\frac{k q^{2}}{m g H}+H\right)^{2}-\frac{k q^{2}}{m g}}.
\]
Lahend \enquote{$+$}-märgiga ruutjuure ees oleks kaugusest $H$ suurem ja vastaks maksimaalsele kõrgusele, mille saavutaks kuulike, kui ta saaks samasuguse kuid ülespoole suunatud algkiiruse.
\fi
}