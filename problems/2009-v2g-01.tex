\ylDisplay{Väike prints} % Ülesande nimi
{Urmo Visk} % Autor
{piirkonnavoor} % Voor
{2009} % Aasta
{G 1} % Ülesande nr.
{2} % Raskustase
{
% Teema: Taevamehaanika
\ifStatement
Väike Prints elab sfäärilisel asteroidil B-612. Jalutades märkas väike prints, et mida kiiremini ta kõnnib, seda kergemaks ta muutub. Kui väike prints jooksis piki asteroidi ekvaatorit kiirusega $v = \SI{6}{m/s}$, siis muutus ta kaalutuks ja hakkas asteroidi pinna kohal hõljuma. Kui suur on asteroidi raadius $R$? Eeldame, et asteroid ei pöörle. Asteroidi tihedus on $\rho = \SI{5200}{kg/m^3}$, gravitatsioonikonstant $G = \SI{6.67e-11}{m^3.kg^{-1}.s^{-2}}$.
\fi


\ifHint
Kui väike prints kõnnib piki asteroidi ekvaatorit, mõjub talle gravitatsioonijõud, normaaljõud ning ringjoonelisest trajektoorist tingitud kesktõmbekiirendus. Hõljuma hakates normaaljõudu ei mõju ning kehtib jõudude tasakaal.
\fi


\ifSolution
Kui väike prints kõnnib piki asteroidi ekvaatorit, mõjub talle gravitatsioonijõud,
mis põhjustab kesktõmbekiirendust. Kaalugu väike prints $m$ kilogrammi. Newtoni
II seaduse põhjal
\[
m \frac{v^{2}}{R}=G \frac{m M}{R^{2}} \quad \Rightarrow \quad v^{2}=\frac{G M}{R}.
\]
Asteroidi mass pole teada, kuid teada on asteroidi tihedus. Kui asteroidi raadius on $R$, siis on asteroidi ruumala $V=\frac{4}{3} \pi R^{3}$ ja mass
\[
M=\rho V=\frac{4}{3} \pi \rho R^{3}.
\]
Asendades massi esialgsesse avaldisse saame
\[
v^{2}=\frac{G M}{R}=\frac{G}{R} \cdot \frac{4}{3} \pi \rho R^{3}=\frac{4}{3} \pi \rho G R^{2}.
\]
Siit avaldame asteroidi raadiuse:
\[
R=v \sqrt{\frac{3}{4 \pi \rho G}}=\frac{v}{2} \sqrt{\frac{3}{\pi \rho G}}=\SI{5}{km}.
\]
\fi
}