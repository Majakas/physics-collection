\ylDisplay{Vesi ja jää} % Ülesande nimi
{Andres Laan} % Autor
{piirkonnavoor} % Voor
{2010} % Aasta
{G 5} % Ülesande nr.
{4} % Raskustase
{
% Teema: Termodünaamika
\ifStatement
Kahte suurt paralleelset metallplaati hoitakse horisontaalselt vastastikku. Üks plaatidest on temperatuuril $T_1 = \SI{-20}{\celsius}$ ja teine temperatuuril $T_2 = \SI{20}{\celsius}$. Metallplaatide vahel on vesi. Ilmselgelt on külma plaadi läheduses vesi tahkes olekus. On teada, et vee tahke ja vedela kihi paksuste suhe on \num{4}. Millisele temperatuurile tuleb soojendada teine metallplaat, et vedela kihi paksus saaks võrdseks tahke kihi paksusega?
\fi


\ifHint
Soojusvoog läbi kihi paksusega $d$, pindalaga $S$, temperatuuride vahega $\Delta T$ ja soojusjuhtivuskoefitsiendinga $D$ on $Q = D\frac{S\Delta T}{d}$. Statsionaarses olekus tasakaalustab jää ja vee eralduspinnal ühe plaadi poolt tulev soojusvoog teiselt poolt tuleva soojusvoo täielikult ära. 
\fi


\ifSolution
Soojusvoog läbi vee kihi on määrtud valemiga
\[
Q = D\frac{S\Delta T}{l},
\]
kus $S$ on kihi pindala, $l$ selle paksus, $\Delta T$ kihi ülemise ja alumise pinna temperatuuride vahe ning $D$ vastava vee faasi soojusjuhtivuskoefitsient. Vaatleme vedela ja tahke faasi piirpinda. Antud pinna temperatuur on \SI{0}{\celsius}. Tahkest poolest tulev soojusvoo võimsus on
$Q_t = D_tST_1/l_t$
ja vedelast faasist tuleva voo võimsus on
$Q_v = D_vST_2/l_v$.
Statsionaarses olukorras, kus piirpinna asukoht ei muutu, tasakaalustavad antud vood üksteist ära. Arvestades, et mõlema faasi pindalad on võrdsed, saame
\[
\frac{D_tT_1}{l_t} = -\frac{D_vT_2}{l_v}.
\]
Kasutame ära asjaolu, et tahke ja vedela kihi paksuste suhe oli alguses \num{4}:
\[
\frac{D_t}{D_v} = -\frac{T_{2} l_{t}}{T_1 l_{v1}}=-\frac{\num{-20}\cdot \num{4}}{\num{20}\cdot \num{1}} = \num{4},
\]
Teisel juhul
\[
T_{2}' = -\frac{D_{t}}{D_{v}}\frac{l_{v}'}{l_{t}'} T_1 = \num{-4}\frac{l_v'}{l_t'} T_2 = \SI{80}{\celsius}.
\]
\fi
}