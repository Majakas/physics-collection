\setAuthor{Tundmatu autor}
\setRound{lõppvoor}
\setYear{2004}
\setNumber{G 3}
\setDifficulty{1}
\setTopic{Teema}

\prob{Kaitsmed}
Rööpselt on lülitatud kaks sulavkaitset voolule $I_{M \max }=\SI{1}{A}$ ja $I_{N \max }=\SI{1,2}{A}$ takistustega vastavalt $R_{M}=\SI{1}{\Omega}$ ja $R_{N}=\SI{2}{\Omega}$. Milline maksimaalne vool vôib taolist süsteemi läbida? Milline oleks maksimaalne vool kui $I_{N \max }=\SI{1,7}{A}$?

\hint

\solu
Kuna ülesande tingimuste kohaselt vool läbi kaitsme $M$ on alati suurem kui vool läbi kaitsme $N$ (kui kumbki kaitsmetest ei ole veel läbi põlenud), siis koguvoolu kasvades põleb esmalt läbi kaitse $M$. Koguvoolu väärtus on siis $\left(1+R_{M} / R_{N}\right) I_{M max }=\SI{1,5}{A}$. Pärast kaitsme $M$ läbipõlemist läbib kogu vool kaitset $N$ ja võib omandada maksimaalse väärtuse $\SI{1,2}{A}$. Kuna see väärtus on väiksem kui $\SI{1,5}{A}$, on maksimaalne vôimalik voolu väärtus $\SI{1,5}{A}$ (või matemaatiliselt täpne olles - sellele väärtusele kuitahes lähedane väiksem väärtus). Juhul, kui $I_{N max }=\num{1,7}$ A saavutab vool oma maksimaalväärtuse \SI{1,7}{A} alles pärast kaitsme läbipõlemist.

\probend