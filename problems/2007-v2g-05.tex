\ylDisplay{Tuulik} % Ülesande nimi
{Valter Kiisk} % Autor
{piirkonnavoor} % Voor
{2007} % Aasta
{G 5} % Ülesande nr.
{3} % Raskustase
{
% Teema: Gaasid
\ifStatement
Teatud tuuleturbiin (tiiviku diameeter $d = \SI{50}{m}$) töötab maksimaalse efektiivsusega tuule kiirusel $v = \SI{9}{m/s}$. Sel juhul õnnestub $\eta = \SI{40}{\%}$ tiiviku poolt haaratava õhuvoolu kineetilisest energiast muundada elektriks (kineetilise energia arvutamisel ei arvestata õhu pidurdumist tiivikul). Leidke nendel tingimustel tuuliku elektriline võimsus. Õhu tihedus on $\rho = \SI{1,3}{kg/m^3}$.
\fi


\ifHint
Kui ajavahemiku $\Delta t$ jooksul kandub tiivikust läbi õhumass kineetilise energiaga $\Delta E$, siis sellele vastav võimsus on $P = \frac{\Delta E}{\Delta t}$.
\fi


\ifSolution
Tiiviku poolt haaratav pindala on
\[
S = \frac{\pi d^2}{4},
\]
Ajavahemikus $\Delta t$ kandub läbi selle pinna õhumass $\Delta m = vS\rho \Delta t$, mille kineetiline energia on
\[
\Delta E=\frac{\Delta m v^{2}}{2}=\frac{S \rho \Delta t v^{3}}{2},
\]
millele vastab võimsus $P_0 = S\rho v^3/2$. Elektriks õnnestub muundada
osa $\eta$ sellest:
\[
P=\eta P_{0}=\frac{\eta S \rho v^{3}}{2}=\frac{\eta \pi d^2 \rho v^{3}}{8} \approx \SI{370}{kW}.
\]
\fi
}