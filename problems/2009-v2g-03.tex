\ylDisplay{Termos} % Ülesande nimi
{Urmo Visk} % Autor
{piirkonnavoor} % Voor
{2009} % Aasta
{G 3} % Ülesande nr.
{2} % Raskustase
{
% Teema: Termodünaamika
\ifStatement
Termoses, mis on ümbritsevatest kehadest soojuslikult isoleeritud, on $m_1 = \SI{300}g$ vett temperatuuriga $t_1 = \SI{20}{\celsius}$. Sellele lisatakse $m_2 = \SI{600}g$ vett temperatuuriga $t_2 = \SI{80}{\celsius}$. Pärast soojusliku
tasakaalu saabumist mõõdeti vee temperatuuriks $T_1$. Järgmisel korral oli
samas anumas alguses $m_2 = \SI{600}g$ vett temperatuuriga $t_2 = \SI{80}{\celsius}$ ja sellele lisati $m_1 = \SI{300}g$ vett temperatuuriga $t_1 = \SI{20}{\celsius}$. Nüüd mõõdeti vee temperatuuriks soojusliku tasakaalu saabumise järel $T_2 = T_1+\SI{2}{\celsius}$. Kui
suur on termose materjali erisoojus? Tühja termose mass on $m = \SI{140}g$ ja vee erisoojus $c = \SI{4200}{J/kg.C}$.

\fi


\ifHint
Ülesandes on kaks tundmatut: vee lõpptemperatuur ja termose erisoojus. Need on leitavad pannes süsteemi jaoks kirja soojusliku tasakaalu võrrandid.
\fi


\ifSolution
Olgu $c_x$ otsitav erisoojus.

Vaatleme esimest juhtu, kus termoses oli algselt külmem vesi. Kuna külmem vesi oli termosega soojuslikus tasakaalus, siis oli ka termose temperatuur $t_1$. Temperatuuride
ühtlustumisel annab soojem vesi energiat ära. Külmem vesi ja termos saavad energiat juurde. Paneme kirja soojusliku tasakaalu võrrandi:
\begin{equation}
m_{1} c\left(T_{1}-t_{1}\right)+m c_{x}\left(T_{1}-t_{1}\right)=m_{2} c\left(t_{2}-T_{1}\right).
\label{2009-v2g-03:eq1}
\end{equation}
Vaatleme teist juhtu, kus termoses oli algselt soojem vesi. Kuna soe vesi oli termosega soojuslikus tasakaalus, siis oli ka termose temperatuur $t_2$. Temperatuuride
ühtlustumisel annavad termos ja soojem vesi energiat ära. Külmem vesi saab energiat juurde. Kirjutame soojusliku tasakaalu võrrandi:
\begin{equation}
m_{1} c\left(T_{2}-t_{1}\right)=m_{2} c\left(t_{2}-T_{2}\right)+m c_{x}\left(t_{2}-T_{2}\right).
\label{2009-v2g-03:eq2}
\end{equation}
Lahutame teineteisest võrrandid (\ref{2009-v2g-03:eq2}) ja (\ref{2009-v2g-03:eq1}).
\[
m_{1} c\left(T_{1}-T_{2}\right)+m c_{x}\left(T_{1}-t_{1}\right)=m_{2} c\left(T_{2}-T_{1}\right)-m c_{x}\left(t_{2}-T_{2}\right).
\]
Tähistame $T_2 - T_1 = \Delta T$. Allpool on toodud $c_x$ tuletuskäik eelnevast valemist.
\[
-m_{1} c \Delta T+m c_{x}\left(t_{2}-t_{1}\right)=m_{2} c \Delta T+m c_{x} \Delta T\implies
\]
\[
-\Delta T c\left(m_{2}+m_{1}\right)=m c_{x}\left(\Delta T+t_{1}-t_{2}\right)\implies
\]
\[
c_{x}=-\frac{\Delta T c\left(m_{2}+m_{1}\right)}{m\left(\Delta T+t_{1}-t_{2}\right)}=\SI{930}{J/kg.C}.
\]
\fi
}