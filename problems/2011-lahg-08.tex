\setAuthor{Kristian Kuppart}
\setRound{lahtine}
\setYear{2011}
\setNumber{G 8}
\setDifficulty{6}
\setTopic{Elektrostaatika}

\prob{Sfäärid}
Kaks juhtivast materjalist sfääri raadiustega $R_1$ ja $R_2$ on ühendatud pika 
juhtmega. Ühele sfääridest antakse mingi laeng. Leidke suhe $\frac{E_1}{E_2}$, kus
$E_1$ ja $E_2$ on elektrivälja tugevused vastavate sfääride pinnal. Eeldage, et
juhtme mahtuvus on tühine ning juhtme pikkus on oluliselt suurem sfääride
raadiustest.

\hint
Kuna sfäärid on traadiga ühendatud, peavad need sama potentsiaaliga olema.

\solu
Et sfäärid on traadiga ühendatud, siis nad omandavad sama potentsiaali. Olgu ühe laeng $Q_1$ ja teise laeng $Q_2$;
sellisel juhul $kQ_1/R_1=kQ_2/R_2$. Jagades selle võrduse vasaku ja parema poole läbi $R_1R_2$-ga ja tähistades $E_1=kQ_1/R_1^2$ ning
$E_2=kQ_2/R_2^2$, saame $E_1/R_2 = E_2/R_1$, millest $E_1/E_2=R_2/R_1$.
\probend