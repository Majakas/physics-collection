\ylDisplay{Veejuga} % Ülesande nimi
{Siim Ainsaar} % Autor
{piirkonnavoor} % Voor
{2006} % Aasta
{G 9} % Ülesande nr.
{5} % Raskustase
{
% Teema: Vedelike-mehaanika
\ifStatement
Vesi voolab kraanist vertikaalselt alla purki. Nagu teada, ei ole kraanist voolav veejuga silindriline. Joa raadius kraani otsa juures on $r_0 = \SI{5}{mm}$, sellest kaugusel $h = \SI{130}{mm}$ allpool aga $r_1 = \SI{3}{mm}$. Leidke aeg $t$, mis kulub purgi täitmiseks, kui raskuskiirendus on $g = \SI{9,8}{m/s}$. Purgi ruumala $V = \SI{1}{liiter}$. Pindpinevusest tingitud efekte pole vaja arvestada. Eeldada, et voolamiskiirus on iga ristlõike piires ühesugune ning keeriseid ei ole. 
\fi


\ifHint
Kuna veejuga kiireneb ühtlaselt, kehtib energia jäävuse seadus. Lisaks kehtib vee massi jäävus ristlõigete ulatuses.
\fi


\ifSolution
Leidmaks purgi täitumise aega, on meil vaja teada veevoolu kiirust mingil kõrgusel. Olgu see kiirus kraanitoru otsa juures $v_0$ ning kaugusel $h$ (seal, kus raadius on $r_1$) $v_1$. Kuna vool kiireneb ühtlaselt, siis
\[
v_1^2 - v_0^2 = 2gh.
\]
Ajaga $\Delta t$ läbib iga veejoa ristlõiget sama kogus vett, sest seda ei kao kuhugi ega tule ka juurde:
\[
\pi r_0^2 v_0\Delta t = \pi r_1^2 v_1\Delta t.
\]
Seega
\[
v_1 = v_0 \left(\frac{r_0}{r_1}\right)^2,
\]
mille asetame avaldisse $v_1^2-v_0^2=2gh$:
\[
\begin{aligned}
\left[v_{0}\left(\frac{r_{0}}{r_{1}}\right)^{2}\right]^{2}-v_{0}^{2}=2 g h \quad&\Rightarrow\quad v_{0}^{2}\left[\left(\frac{r_{0}}{r_{1}}\right)^{4}-1\right]=2 g h \quad\Rightarrow\\
&\Rightarrow \quad v_{0}=\sqrt{\frac{2 g h}{\left(r_{0} / r_{1}\right)^{4}-1}}
\end{aligned}
\]
Ajaga $t$ voolab purki veekogus, mille ruumala on
\[
V = \pi r_0^2 v_0t.
\]
Avaldame viimasest võrrandist aja $t$:
\[
t=\frac{V}{\pi r_{0}^{2} v_{0}}=\frac{V}{\pi r_{0}^{2}} \sqrt{\frac{\left(r_{0} / r_{1}\right)^{4}-1}{2 g h}} = \frac{V}{\pi \sqrt{2 g h}} \sqrt{\frac{1}{r_{1}^{4}}-\frac{1}{r_{0}^{4}}}\approx \SI{21}{s}.
\]
\fi
}