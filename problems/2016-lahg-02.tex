\ylDisplay{Kurv} % Ülesande nimi
{Mihkel Rähn} % Autor
{lahtine} % Voor
{2016} % Aasta
{G 2} % Ülesande nr.
{1} % Raskustase
{
% Teema: Dünaamika
\ifStatement
Kiirusega $v=\SI{90}{km/h}$ sõitev auto läbib kurvi raadiusega $R=\SI{250}{m}$. Kui suur peab olema tee külgkalle (kraadides), et autos istujad ei tunneks kurvist tingitud külgsuunalist jõudu? Raskuskiirendus $g=\SI{9.8}{m/s^{2}}$.
\fi


\ifHint
Autos istujad ei tunne külgsuunalist jõudu siis, kui summaarne jõud on tee pinnaga risti.
\fi


\ifSolution
Olgu auto külgkalle $\alpha$ ning autole mõjuv summaarne jõud $N$. Kuna autos olijad ei tunne külgsuunalist jõudu, on $N$ teega risti ja seega nurga $\alpha$ all vertikaali suhtes. Jõu võrrandid maaga seotud teljestikus on:
\begin{align*}
mg &= N\cos \alpha,\\
N\sin\alpha &= \frac{mv^2}{R}.
\end{align*}
Lahendades võrrandisüsteemi, saame
\[
\alpha = \arctan\left(\frac{v^2}{Rg}\right) = \SI{14}{\degree}.
\]
\fi


\ifEngStatement
% Problem name: Curve
A car with a speed of $v=\SI{90}{km/h}$ is driving through a curve of radius $R=\SI{250}{m}$. How big must the road’s sideways slant be in degrees so that the people in the car wouldn’t feel any side force from the curve? The gravitational acceleration is $g=\SI{9.8}{m/s^{2}}$.
\fi


\ifEngHint
The passengers of the car do not feel a sideways force when the total force is perpendicular to the road’s surface.
\fi


\ifEngSolution
Let the sideways slant of the car be $\alpha$ and the total force applied to the car $N$. Because the passengers in the car do not feel a sideways force then $N$ is perpendicular to the road and therefore at the angle $\alpha$ with respect to the vertical. The force equations in the set of coordinate axes related to the ground are: 
\begin{align*}
mg &= N\cos \alpha,\\
N\sin\alpha &= \frac{mv^2}{R}.
\end{align*}
Solving the system of equations we get $\alpha = \arctan(\frac{v^2}{Rg}) = \SI{14}{\degree}$
\fi
}