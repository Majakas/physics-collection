\ylDisplay{Ratturid} % Ülesande nimi
{Tundmatu autor} % Autor
{lahtine} % Voor
{2009} % Aasta
{G 1} % Ülesande nr.
{1} % Raskustase
{
% Teema: Kinemaatika
\ifStatement
Kolm ratturit sõitsid linnast $A$ linna $B$. Linnast $A$ väljusid nad üheaegselt. Esimese ratturi keskmine kiirus oli $v_1 = \SI{30}{km/h}$, teise ratturi oma $v_2 = \SI{20}{km/h}$. Esimene rattur jõudis sihtpunkti kell 19.00, teine rattur kell 20.00 ning kolmas rattur kell 21.00. Milline oli kolmanda ratturi keskmine kiirus $v_3$?
\fi


\ifHint
Ratturite sõiduaeg on leitav asjaolust, et esimene ja teine rattur läbisid sama pika vahemaa.
\fi


\ifSolution
Olgu esimese ratturi sõiduaeg tundides $t$, siis teise ratturi sõiduaeg on $t + 1$ ning kolmanda ratturi oma $t+2$. Et esimene ja teine rattur sõitsid läbi sama pika vahemaa, saame võrrandi
\[
30 t=20(t+1),
\]
kust saame $t = \num{2}$ tundi. Järelikult linnade $A$ ja $B$ vaheline kaugus on $s = \num{30}·\num{2} = \SI{60}{km}$ kilomeetrit. Kolmanda ratturi keskmine kiirus oli seega
\[
v_{3}=\frac{s}{t+2}=\SI{15}{km/h}.
\]
\fi
}