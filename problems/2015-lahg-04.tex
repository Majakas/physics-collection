\ylDisplay{Veekeedukann} % Ülesande nimi
{Erkki Tempel} % Autor
{lahtine} % Voor
{2015} % Aasta
{G 4} % Ülesande nr.
{4} % Raskustase
{
% Teema: Termodünaamika
\ifStatement
Veekeedukann küttekeha võimsusega $N$ on täidetud veega. Kannu tila ava pindala on $S$. Milline on suurim joonkiirus, millega veeaur kannu avast väljub? Vee aurustumissoojus on $L$, ideaalse gaasi konstant on $R$, õhurõhk on $p$ ning vee molaarmass on $\mu$. Veekeedukannu kasutegur soojuskadusid arvestades on $\gamma$.
\fi


\ifHint
Suurima kiiruse saavutab veeaur siis, kui vesi on kuumutatud keemistemperatuurini. Sellisel juhul kulub kogu küttekeha võimsus vee aurustamiseks.
\fi


\ifSolution
Suurima kiiruse saavutab veeaur siis, kui vesi on kuumutatud keemistemperatuurini $T=\SI{100}{\celsius}=\SI{373}{K}$. Veekeedukannu tehtud töö $A=\gamma Nt$ läheb siis vee aurustamiseks ehk $A=Q=Lm$. Seega aja $t$ jooksul aurustunud vee mass on
\[
m=\frac{\gamma Nt}{L}
\]
ning ruumala
\[
V=\frac{\gamma NtRT}{Lp\mu}.
\]
Kuna veeaur pääseb välja avast pindalaga $S$, siis peab aja $t$ jooksul eraldanud veeaur läbima vahemaa $s=V/S$ ning veeauru väljumise kiiruseks saame
\[
v=\frac{s}{t}=\frac{\gamma NtRT}{Lp\mu S}.
\]
\fi


\ifEngStatement
% Problem name: Kettle
A kettle with a heating power $N$ is filled with water. The area of the kettle’s spout is $S$. What is the biggest speed of the water vapor exiting the kettle? Water’s enthalpy of vaporization is $L$, the universal gas constant is $R$, the air pressure is $p$ and the molar mass of water is $\mu$. The efficiency factor of the kettle without accounting for the heat losses is $\gamma$.
\fi


\ifEngHint
The water vapor obtains the biggest velocity when the water has been heated to boiling temperature. In this case all of the heater’s power goes to vaporizing the water.
\fi


\ifEngSolution
The water vapor achieves the biggest velocity if the water has been heated to the boiling temperature $T=\SI{100}{\celsius}=\SI{373}{K}$. The work $A=\gamma Nt$ done by the kettle then goes to vaporizing of the water, meaning $A=Q=Lm$. Therefore the mass of the vaporized water during the time $t$ is $m=\frac{\gamma Nt}{L}$ and volume $V=\frac{\gamma NtRT}{Lp\mu}$. Because the water vapor gets out from an opening of area $S$ then the water vapor dissipated during the time $t$ has to cover a distance $s=\frac{V}{S}$ and we get the velocity of the exiting water vapor to be $v=\frac{s}{t}=\frac{\gamma NtRT}{Lp\mu S}$.
\fi
}