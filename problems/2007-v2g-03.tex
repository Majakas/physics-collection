\setAuthor{Aleksei Vlassov}
\setRound{piirkonnavoor}
\setYear{2007}
\setNumber{G 3}
\setDifficulty{2}
\setTopic{Termodünaamika}

\prob{Vedelike segamine}
Kahe erineva vedeliku segamisel ruumalade suhtega $1 : 1$ tekib segu temperatuuriga $t_3 = \SI{42}{\degreeCelsius}$. Milline oleks segu temperatuur, kui ruumalade suhe oleks $2 : 1$? Vedelike temperatuurid on vastavalt $t_1 = \SI{27}{\degreeCelsius}$ ning $t_2 = \SI{47}{\degreeCelsius}$.

\hint
Mõlema olukorra jaoks kehtib energia jäävuse seadus, kusjuures vedelike tihedused ning erisoojused on tundmatud.

\solu
Olgu vedelike tihedused vastavalt $\rho_1$ ja $\rho_2$ ning erisoojused $c_1$ ja $c_2$. Olgu otsitav temperatuur $t_4$. Paneme kirja energia jäävuse võrrandid mõlema segu jaoks:
\[
\left\{\begin{array}{l}
	{\rho_{1} V c_{1}\left(t_{3}-t_{1}\right)=\rho_{2} V c_{2}\left(t_{2}-t_{3}\right)} \\ {2 \rho_{1} V c_{1}\left(t_{4}-t_{1}\right)=\rho_{2} V c_{2}\left(t_{2}-t_{4}\right)}\end{array} \Rightarrow\left\{\begin{array}{l}{\rho_{1} c_{1}\left(t_{3}-t_{1}\right)=\rho_{2} c_{2}\left(t_{2}-t_{3}\right)} \\ {2 \rho_{1} c_{1}\left(t_{4}-t_{1}\right)=\rho_{2} c_{2}\left(t_{2}-t_{4}\right).}
\end{array}\right.\right.
\]
Korrutame esimese võrrandi vasaku poole läbi teise võrrandi parema poolega:
\[
\rho_{1} c_{1} \rho_{2} c_{2}\left(t_{3}-t_{1}\right)\left(t_{2}-t_{4}\right) =2 \rho_{1} c_{1} \rho_{2} c_{2}\left(t_{4}-t_{1}\right)\left(t_{2}-t_{3}\right).
\]
ehk
\[
\left(t_{3}-t_{1}\right)\left(t_{2}-t_{4}\right) =2\left(t_{4}-t_{1}\right)\left(t_{2}-t_{3}\right).
\]
Siit avaldame $t_4$:
\[
t_{4}=\frac{t_{1} t_{2}+t_{2} t_{3}-2 t_{1} t_{3}}{2 t_{2}-t_{1}-t_{3}}=\SI{39}{\degreeCelsius}.
\]
\probend