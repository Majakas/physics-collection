\ylDisplay{Elektripliit} % Ülesande nimi
{Tundmatu autor} % Autor
{lahtine} % Voor
{2007} % Aasta
{G 8} % Ülesande nr.
{5} % Raskustase
{
% Teema: Termodünaamika
\ifStatement
Elektripliidi spiraali poolt ajaühikus keskkonnale üle antav soojushulk sõltub lineaarselt spiraali ja toa õhu temperatuuride vahest: $N = \kappa (T - T_0)$. Spiraali takistus sõltub sellest vahest samuti lineaarselt: $R = R_0 [1+\alpha (T -T_0)]$, kus $R_0$ on spiraali takistus toatemperatuuril. Kui suure temperatuurini kuumeneb spiraal, kui seda läbib vool tugevusega $I$?
\fi


\ifHint
Spiraal kuumeneb temperatuurini, mil tekib soojuslik tasakaal spiraali ja ümbritseva keskkonna vahel. Teisisõnu spiraali takistil eralduv võimsus peab olema võrdne soojusvooga spiraalist toa õhku.
\fi


\ifSolution
Spiraal kuumeneb temperatuurini, mil tekib soojuslik tasakaal spiraali ja ümbritseva keskkona vahel. Kuna spiraalil eraldub võimsus $P = I^2R$, peab kehtima $P = N$ (soojuskadusid arvestamata), ehk
\[
\kappa\left(T-T_{0}\right)=I^{2} R_{0}\left[1+\alpha\left(T-T_{0}\right)\right].
\]
Avaldades $T$, saame
\[
T=T_{0}+\frac{I^{2} R_{0}}{\kappa-\alpha I^{2} R_{0}}.
\]
Kui $\kappa \leq \alpha I^2R_0$ suureneb temperatuur lõpmatuseni ning pliit põleb läbi.
\fi
}