\ylDisplay{Fotograaf} % Ülesande nimi
{Jaan Kalda} % Autor
{lõppvoor} % Voor
{2011} % Aasta
{G 6} % Ülesande nr.
{7} % Raskustase
{
% Teema: Kinemaatika
\ifStatement
Fotograaf pildistas kõrgest joast langevat veevoolu; päikesevalguses sätendavad veepiisad venisid piltidel vertikaalseteks triipudeks.
Kui fotoaparaat oli pildistamisel normaalasendis, siis olid kõik triibud pikkusega $l_1 = \num{120}$ pikselit; kui fotoaparaat oli pildistamisel \enquote{jalad ülespidi} (st seda
pöörati ümber optilise telje \num{180} kraadi), siis oli triipude pikkuseks $l_2 = \num{200}$
pikselit. Kui pikad olid triibud siis, kui fotoaparaati hoiti pildistamisel \enquote{portree
asendis} (st seda pöörati ümber optilise telje \num{90} kraadi)? Eeldada, et säriaeg
ja optilise telje suund oli kõigil juhtudel üks ja sama. Kui toodud andmete
põhjal pole vastus üheselt leitav, siis andke kõik võimalikud vastused.

\emph{Vihje}. 
Fotoaparaadi põhikomponendid on objektiiv (lääts) ja katik, millest
esimene tekitab digitaalsensori (või filmi) tasandile pildistatavate esemete kujutise. \enquote{Puhkeasendis} ei lange see kujutis siiski sensorile, sest katik varjab
läbi objektiivi tulnud valguse ära. Päästikule vajutamisel avaneb katik lühikeseks ajavahemikuks (säriajaks): objektide kujutis langeb nüüd tõesti sensorile
ning sensori iga piksel mõõdab ära kogu selle aja vältel langeva valgusenergia.
Harilikult kujutab katik endast kahte \enquote{kardinat}, mis paiknevad vahetult sensori ees ja katavad selle. Alguses varjab sensorit esimene kardin, mille ülemine
serv liigub päästikule vajutamisel konstantse kiirusega $v$ ülevalt alla, avades
sensori. Säriaja lõpetab teine kardin, mille alumine serv liigub samuti ülevalt
alla, samasuguse kiirusega $v$ nagu esimenegi. Kui säriaeg on hästi lühike, siis
ei jõua sensor täielikult avaneda: mõlemad kardinad liiguvad koos ülevalt alla
ning sensor on avatud objektiivist tulevale valgusele vaid kardinate vahelise
kitsa horisontaalse riba ulatuses (kusjuures see valgusele avatud riba liigub
kiirusega $v$ ülevalt alla).
\fi


\ifHint
Antud ülesandes on kolm tundmatud: pilu laius, katiku kiirus ja piisa kujutise kiirus sensori tasandis. Lisaks kirjeldati kahte olukorda, mis seovad antud tundmatuid. Selgub, et nendest piisab, et määrata kolmandas olukorras triibu pikkust.
\fi


\ifSolution
Olgu pilu laius $d$, katiku kiirus $u$ ja piisa kujutise kiirus sensori tasandis $v$. Katiku
taustsüsteemis liigub piisa kujutis kiirusega $u \pm v$; kui fotoaparaat on päripidi, siis tuleb võtta märk \enquote{$+$} ja kui tagurpidi, siis \enquote{$-$}. Seega on piisa jälje tekkimise aeg
$d/|u \pm v|$ ning jälje pikkus $l = vd/|u \pm v|$. Olgu $u \geq v$; siis
\[
l_{1}=\frac{v d}{u+v}, \quad l_{2}=\frac{v d}{u-v}.
\]
Jagades teise võrrandi esimesega saame 
\[
\frac{u+v}{u-v}=\frac{l_{2}}{l_{1}}=\frac{5}{3},
\]
millest
\[
3u+ 3v = 5u-5v,
\]
ehk
\[
u = 4v.
\]
Kui fotoaparaat on portreeasendis, siis viibib piisa kujutis pilus ajavahemiku
$d/u$ jooksul ja jälje pikkus on seega
\[
l_3 = vd/u.
\]
Esimese võrrandiga läbi jagades leiame, et $l_3/l_1 = 1 + \frac{v}{u} = \frac{5}{4}$
ning
\[
l_{3}=\frac{5}{4} l_{1}=150 \text { pikselit. }
\]

Kui $u < v$, siis muutub ainult teine võrrand,
\[
l_2 = \frac{vd}{v - u},
\]
mistõttu $3u + 3v = 5v - 5u$ ja $u = v/4$, mistõttu
\[
l_{3}=5 l_{1}=600 \text { pikselit. }
\]
\emph{Märkus}. Ülesande teksti põhjal on see üks kahest võimalikust vastusest; reaalselt, arvestades tüüpilist katiku liikumiskiirust (\SI{18}{mm} läbimisaeg $\frac{1}{125}\si{s} \implies u = \SI{2,25}{m/s} \implies v = 4u = \SI{9}{m/s}$) on siiski üsna raske saavutada, et $v = 4u$: pildistamine peaks toimuma ohtlikult lähedalt. Kui joa kõrgus oleks nt \SI{100}{m}, siis vaba-langenud piisa
kiirus oleks ca \SI{44}{m/s}, mistõttu pildistamiskauguse ja objektiivi fookuskauguse suhe
(st suurendustegur) tuleks $44/9 \approx 5$ ning isegi teleobjektiivi (nt $f = \SI{300}{mm}$) korral
peaks fotograaf olema joast vaid \SI{1,5}{m} kaugusel.

\emph{Märkus 2}. Eeldusest, et \enquote{pilu laius on $d$} võib jääda mulje, justkui eeldanuks me
vaikimisi, et sensor ei jõua säritamise ajal täielikult avaneda. Ometigi kehtib lahendus
ka siis, kui säriaeg on nii pikk, et sensor jõuab täielikult avaneda: piltlikult võib
ette kujutada, et ikkagi mõlemad kardinad liiguvad samaaegselt, kuid pilu laius on
suurem sensori kõrgusest, st esimene kardin jõuab sensori kohalt eemale minna enne
teise kardina saabumist.
\fi
}