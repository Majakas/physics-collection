\setAuthor{Oleg Košik}
\setRound{piirkonnavoor}
\setYear{2005}
\setNumber{G 7}
\setDifficulty{5}
\setTopic{Vedelike mehaanika}

\prob{Veekahur}
Veekahur tulistab veejuga, mille ristlõikepindala on $S = \SI{8}{cm^2}$ ning võimsus $N = \SI{6000}{W}$. Millise jõuga tabab veejuga märki, kui kahur ja märk asuvad samal kõrgusel? Vee tihedus on $\rho = \SI{1000}{kg/m^3}$, õhutakistust mitte arvestada. Märklaua ja veekahuri vahemaa on väike, st veejoa kõverdumist raskusjõu toimel ei pea arvestama.

\hint
Õhutakistuse puudumisel jõuab veejuga märgini sama kiirusega nagu väljudes (energia jäävusest tulenevalt). Jõu leidmiseks võib vaadelda ajavahemikku $\Delta t$ ning selle jooksul üle antavat vee impulssi.

\solu
Olgu veejoa kiirus kahurist väljudes $v$. Et märk ja kahur asuvad samal kõrgusel ning õhutakistus puudub, siis märgini jõuab veejuga samuti kiirusega $v$. Aja $t$ jooksul väljub kahurist vesi massiga $m = \rho Svt$. Väljuva vee kineetiline energia on $E = mv^2/2$ ning asendades massi leiame
\[
E = \frac{\rho Sv^3t}{2}.
\]
Seega saame avaldada kahuri võimsuse
\[
N=\frac{E}{t}=\frac{\rho S v^{3}}{2} \quad \Rightarrow \quad v=\sqrt[3]{\frac{2 N}{\rho S}}.
\]
Jõud, millega veejuga tabab märki, on määratud vee impulsiga:
\[
F=\frac{m v}{t} \quad \Rightarrow \quad F=\frac{\rho S v^{2} t}{t}=\rho S v^{2}.
\]
Asendades $v$ saame
\[
F=\sqrt[3]{4 N^{2} \rho S} \approx \SI{490}{N}.
\]
\probend