\ylDisplay{Gaasiküte} % Ülesande nimi
{Ardi Loot} % Autor
{piirkonnavoor} % Voor
{2017} % Aasta
{G 10} % Ülesande nr.
{8} % Raskustase
{
% Teema: Termodünaamika
\ifStatement
Poolsfäärikujulist telki raadiusega $R=\SI{4}{m}$ köetakse
gaasipuhuriga. Seinte soojusjuhtivus on $U=\SI{3}{W/\left(m^{2}\cdot K\right)}$.
Ühe massiühiku gaasi põletamisel eraldub $D=\SI{2.25}{}$ massiühikut vett. Gaasi kütteväärtus on $k=\SI{40}{MJ/kg}$. Välisõhu temperatuur on $T_{0}=\SI{-10}{\celsius}$
ja õhuniiskus $\eta_{0}=\SI{50}{\percent}.$ Kui suur peab olema gaasikütte
võimsus $P$ ja telgi ventileerimise õhuruumala $Q$ ajaühikus, et hoida telgis
temperatuuri $T_{1}=\SI{15}{\celsius}$ ja õhuniiskust $\eta_{1}=\SI{80}{\percent}$?
Kui suur osa küttevõimusest kulub ventileeritava õhu soojendamiseks
ja mitu korda tunnis vahetub telgi õhk?

Õhu tihedus $\rho_{\tilde{o}}=\SI{1.2}{kg/m^{3}}$ ja soojusmahtuvus
$c_{\tilde{o}}=\SI{1.0}{kJ/\left(kg\cdot K\right)}$. Temperatuuril
$T_{0}=\SI{-10}{\celsius}$ mahub õhu ruumalaühikusse maksimaalselt
$G_{0}=\SI{2.3}{g/m^{3}}$ veeauru ning temperatuuril $T_{1}=\SI{15}{\celsius}$
vastavalt $G_{1}=\SI{12.8}{g/m^{3}}.$ Eeldada, et soojakaod läbi
telgi põranda puuduvad.
\fi


\ifHint
Telk peab olema soojuslikus ja niiskuslikus tasakaalus. Peale soojakadudele läbi seinte läheb osa soojusest kaotsi ventileeritava õhuga.
\fi


\ifSolution
Telk peab olema soojuslikus ja niiskuslikus tasakaalus. Telk kaotab
sooja läbi telgi seinte soojusjuhtivuse tõttu:
\[
P_{s}=SU\Delta T\approx\SI{7.54}{kW},
\]
kus $S=2\pi R^{2}\approx\SI{100.5}{m^{2}}$
ja $\Delta T=T_{1}-T_{0}=\SI{25}{\celsius},$ ning telgi ventileerimise tõttu:
\[
P_{v}=Q\rho_{\tilde{o}}c_{\tilde{o}}\Delta T\approx Q\cdot\left(\SI{30.0}{kW\cdot s/m^{3}}\right).
\]
Soojusliku tasakaalu korral

\begin{equation}
P_{p}=P_{s}+P_{v}\approx\SI{7.54}{kW}+\left(Q\cdot\SI{30.0}{kW\cdot s/m^{3}}\right).\label{eq:2017-v2g-10-gaas-eq1}
\end{equation}


Niiskusliku tasakaalu jaoks peab ventileerimine telgist välja viima
samapalju niiskust kui gaasi põletamisel tekib. Sooja õhu väljaviskel
viiakse ajaühikus telgist välja niiskust $\Gamma_{v}=QG_{1}\eta_{1}\approx Q\cdot\left(\SI{10.2}{g/m^{3}}\right)$
ning külma õhu sissevooluga siseneb telki ajaühikus $\Gamma_{s}=QG_{0}\eta_{0}\approx Q\cdot\left(\SI{1.15}{g/m^{3}}\right)$
niiskust. Võimsusega $P_{p}$ gaasiküte eraldab ajaühikus
$\Gamma_{p}=D\cdot P_{p}/k\approx P_{p}\cdot\left(10^{-5}\cdot\SI{5.63}{kg/\left(kW\cdot s\right)}\right)$
niiskust. Tasakaalu korral
\vspace{-3pt}
\[
\Gamma_{p}=\Gamma_{v}-\Gamma_{s}
\]
\vspace{-3pt}
\noindent ehk
\vspace{-3pt}
\begin{equation}
P_{p}\cdot\left(10^{-5}\cdot\SI{5.63}{kg/\left(kW\cdot s\right)}\right)\approx Q\cdot\left(\SI{9.09}{g/m^{3}}\right).\label{eq:2017-v2g-10-gaas-eq2}
\end{equation}


Lahendades tasakaaluvõrranditest (\ref{eq:2017-v2g-10-gaas-eq1}) ja (\ref{eq:2017-v2g-10-gaas-eq2})
tekkinud süsteemi saame
\vspace{-3pt}
\begin{eqnarray*}
	Q & = & \frac{SU\Delta T}{\gamma K/D-\rho c_{\tilde{o}}\Delta T}\approx\SI{206}{m^{3}/h}\\
	P_{p} & = & \frac{Q\gamma k}{D}\approx\SI{9.26}{kW}\\
	\gamma & = & G_{1}\eta_{1}-G_{0}\eta_{0}\approx\SI{9.09}{g/m^{3}}.
\end{eqnarray*}


Ventileerimisele kulub $P_{v}/P_{p}\approx\SI{18.6}{\percent}$ küttevõimsusest
ja telgis vahetub õhk $Q/V\approx1,54$ korda tunnis ($V=\frac{2}{3}\pi R^{3}$).
\fi


\ifEngStatement
% Problem name: Gas heating
A half-spherical tent of radius $R=\SI{4}{m}$ is heated with a gas heater. The thermal conductivity of the walls is $U=\SI{3}{W/\left(m^{2}\cdot K\right)}$. $D=\SI{2.25}{}$ units of mass of water segregates when burning one unit of mass of gas. The calorific value of the gas is $k=\SI{40}{MJ/kg}$. The temperature of the outside air is $T_{0}=\SI{-10}{\celsius}$ and the humidity $\eta_{0}=\SI{50}{\percent}$. How big has to be the gas heater’s power $P$ and the tent’s ventilation air volume $Q$ per unit of time so that the temperature in the tent would be at $T_{1}=\SI{15}{\celsius}$ and the humidity $\eta_{1}=\SI{80}{\percent}$? How big part of the heating power goes to the heating of the ventilation air and how many times per hour does the air in the tent change?\\
The air density is $\rho_{a}=\SI{1.2}{kg/m^{3}}$ and the heat capacity $c_{a}=\SI{1.0}{kJ/\left(kg\cdot K\right)}$. At the temperature $T_{0}=\SI{-10}{\celsius}$ maximally $G_{0}=\SI{2.3}{g/m^{3}}$ of water vapor can fit into a unit of volume of air and at the temperature $T_{1}=\SI{15}{\celsius}$ accordingly $G_{1}=\SI{12.8}{g/m^{3}}$. Assume that there are no heat losses through the floor of the tent.
\fi


\ifEngHint
The tent has to have heat and humidity balance. Besides the heat losses through the walls a part of the heat gets lost with the ventilation air.
\fi


\ifEngSolution
The tent has to be in a thermal and humidity equilibrium. The tent loses heat through the tent’s walls due to thermal conductivity:
\[
P_{s}=SU\Delta T\approx\SI{7.54}{kW},
\]
where $S=2\pi R^{2}\approx\SI{100.5}{m^{2}}$ and $\Delta T=T_{1}-T_{0}=\SI{25}{\celsius},$ and due to the tent’s ventilation:
\[
P_{v}=Q\rho_{a}c_{a}\Delta T\approx Q\cdot\left(\SI{30.0}{kW\cdot s/m^{3}}\right).
\]
In the case of thermal equilibrium
\begin{equation}
P_{p}=P_{s}+P_{v}\approx\SI{7.54}{kW}+\left(Q\cdot\SI{30.0}{kW\cdot s/m^{3}}\right).\label{eq:2017-v2g-10-gaas-eq1}
\end{equation}
For humidity equilibrium the ventilation has to take out the same amount of humidity from the tent as is created during the burning of gas. When the warm air flows out the humidity $\Gamma_{v}=QG_{1}\eta_{1}\approx Q\cdot\left(\SI{10.2}{g/m^{3}}\right)$ exits the tent per unit of time and from the cold air inflow the humidity $\Gamma_{s}=QG_{0}\eta_{0}\approx Q\cdot\left(\SI{1.15}{g/m^{3}}\right)$ enters the tent per unit of time. The gas heater of power $P_{p}$ dissipates the humidity $\Gamma_{p}=D\cdot P_{p}/k\approx P_{p}\cdot\left(10^{-5}\cdot\SI{5.63}{kg/\left(kW\cdot s\right)}\right)$ per unit of time. In the case of equilibrium
\[
\Gamma_{p}=\Gamma_{v}-\Gamma_{s}
\]
meaning 
\begin{equation}
P_{p}\cdot\left(10^{-5}\cdot\SI{5.63}{kg/\left(kW\cdot s\right)}\right)\approx Q\cdot\left(\SI{9.09}{g/m^{3}}\right).\label{eq:2017-v2g-10-gaas-eq2}
\end{equation}
Solving the system gotten from the equilibrium equations (1) and (2) we get
\begin{eqnarray*}
	Q & = & \frac{SU\Delta T}{\gamma K/D-\rho c_{\tilde{o}}\Delta T}\approx\SI{206}{m^{3}/h}\\
	P_{p} & = & \frac{Q\gamma k}{D}\approx\SI{9.26}{kW}\\
	\gamma & = & G_{1}\eta_{1}-G_{0}\eta_{0}\approx\SI{9.09}{g/m^{3}}.
\end{eqnarray*}
$P_{v}/P_{p}\approx\SI{18.6}{\percent}$ of heating power goes to ventilation and the air changes in the tent $Q/V\approx1,54$ times per hour ($V=\frac{2}{3}\pi R^{3}$).
\fi
}