\setAuthor{Jaan Kalda}
\setRound{lõppvoor}
\setYear{2015}
\setNumber{G 6}
\setDifficulty{8}
\setTopic{Varia}

\prob{Lööklaine}
Elektrostaatilist lööklainet, mis levib kiirusega $w$ piki $x$-telge, võib kirjeldada elektrilise potentsiaali abil: $U=0$ kui $x<wt$ ning $U=U_0$ kui $x>wt$. Millise kiiruse $v$ omandab lööklaine mõjul algselt paigal seisnud osake massiga $m$ ning laenguga $q$? Vastus andke sõltuvana potentsiaalibarjääri kõrgusest $U_0$. Pöörake tähelepanu asjaolule, et see, kummale poole barjääri osake jääb, sõltub $U_0$ väärtusest. 

\hint
Osakese energia säilib vaid lööklaine taustsüsteemis. Seega tasub vaadelda liikumist vastavas taustsüsteemis.

\solu
Läheme üle lööklainega seotud taustsüsteemi, milles osake läheneb lööklainele kiirusega $w$. Energia jäävusest saame, et juhul kui osake läbib lööklainet, siis 
\[
mw^2/2=qU_0+mu^2/2,
\]
kus $u$ on osakese kiirus pärast lööklainega kohtumist. Sellest saame
\[
u=\sqrt{w^2-2qU_0/m}.
\]
Tagasi laboratoorsesse süsteemi 
minnes saame kiiruseks
\[
v=u-w=\sqrt{w^2-2qU_0/m}-w,
\]
mis kehtib, kui $mw^2>2qU_0$. Vastasel juhul peegeldub osake lööklainelt ning
$u=-w$ ja $v=-2w$.

\probeng{Shock wave}
An electrostatic shock wave that travels with a speed $w$ along the $x$-axis can be described by an electric potential: $U=0$ if $x<wt$ and $U=U_0$ if $x>wt$. What speed $v$ does an initially still particle of mass $m$ and charge $q$ obtain under a shock wave? The answer should depend on the height $U_0$ of the potential barrier. Keep in mind that whether the particle stays on one side of the barrier or the other depends on the value of $U_0$.

\hinteng
The energy of the particle is preserved only in the shock wave’s frame of reference. Thus you should observe the motion in the respective frame of reference.

\solueng
Let us go over to the shock wave's frame of reference where the particle approaches the shock wave with a velocity $w$. From the conservation of energy we get that in the case where the particle goes through the shock wave then $mw^2/2=qU_0+mu^2/2$ where $u$ is the velocity of the particle after meeting the shock wave. From this we get $u=\sqrt{w^2-2qU_0/m}$. Going back to the laboratory frame of reference we get that the velocity is $v=u-w=\sqrt{w^2-2qU_0/m}-w$ which applies when $mw^2>2qU_0$. Otherwise the particle reflects back from the shock wave and $u=-w$ and $v=-2w$.
\probend