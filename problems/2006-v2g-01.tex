\setAuthor{Oleg Košik}
\setRound{piirkonnavoor}
\setYear{2006}
\setNumber{G 1}
\setDifficulty{1}
\setTopic{Kinemaatika}

\prob{Autod}
Tartu ja Tallinna vahemaa on $s = \SI{180}{km}$. Jalgrattur sõidab Tartust Tallinna poole kiirusega $v_1 = \SI{30}{km/h}$. Sõites luges ta kokku, et $t_0 = \SI{5}{min}$ jooksul tuli talle vastu $n_0 = \SI{20}{autot}$. Mitu Tallinnast Tartusse sõitvat autot on korraga maanteel? Eeldada, et autod sõidavad võrdsete vahemaadega kiirusega $v_2 = \SI{90}{km/h}$ kogu maantee ulatuses.

\hint
Ratta liikumist on mugavam vaadelda autodega seotud süsteemis.

\solu
Vaatleme jalgratturi liikumist talle vastu sõitvate autode suhtes. Tema kiirus autode süsteemis on $v = v_1 + v_2 = \SI{120}{km/h}$. Seega katavad $n_0 = 20$ autot vahemaa $vt_0 = \SI{10}{km}$ ning terve maantee ulatuses on autosid
\[
n = n_0 \frac{s}{vt_0} = \num{360}. 
\]
\probend