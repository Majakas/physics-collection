\setAuthor{Jaak Kikas}
\setRound{piirkonnavoor}
\setYear{2006}
\setNumber{G 10}
\setDifficulty{7}
\setTopic{Dünaamika}

\prob{Vai}
Vertikaalset vaia pikkusega $L$ ja massiga $M$ lüüakse pinnasesse nii, et tema otsa pihta lastakse kõrguselt $H\gg L$ vaia otsast kukkuda koormisel massiga $m$. Lööki vaia pihta võib lugeda absoluutselt mitteelastseks, st pärast raskuse ja vaia kokkupuudet liiguvad nad kui üks tervik. Pinnase takistusjõud on $F = F_0 + kl$, kus $l$ on maa sees oleva vaiaosa pikkus. Kui suur on löökide arv $N$, mis on vajalik selleks, et vai täies pikkuses maasse lüüa? Võite eeldada, et ühekordse löögi tagajärjel nihkub vai sügavamale väikese osa võrra oma pikkusest.

\hint
Ühe põrke käigus kehtib impulsi jäävus, aga mitte energia jäävus. Selle põhjal on võimalik määrata hõõrdejõu ületamiseks kulunud töö ühe põrke jooksul ning saadud avaldis summeerida kogu vaia ulatuses.

\solu
Olgu raskuse kiirus enne lööki vaia pihta $v$ ja raskuse ning vaia kiirus vahetult
pärast lööki $v'$. Löögi jooksul säilib impulss (aga mitte energia) 
\[
p = mv = (m + M)v'.
\] 
Kiiruse $v$ või impulsi $p$ saame energia jäävuse seadusest, näiteks kujul
\[
\frac{p^2}{2m} = mgH.
\]
Niisiis on peale lööki süsteem vai+raskus omandanud kiiruse $v'$. Hõõrdejõudude toimel peatuvad need teatava teepikkuse $x$ jooksul, kusjuures tingimuste kohaselt löökide arv $N \gg 1$ ning seega $x \ll L$. Hõõrdejõudude töö on seejuures võrdne energia muuduga.

Eelpooltoodu võib kirja panna kujul
\[
(F_0 + kl)x = (M + m) gx + \frac{p^2}{2(m+M)},
\]
kuid lõppvastuse leidmise seisukohast on meil lihtsam võrrutada summaarne hõõrdejõudude töö (mis on leitav nt graafiku $F(l)$ aluse pindalana),
\[
A_h = F_0L + \frac{kL^2}{2}.
\]
summaarse dissipeeruva energiaga
\[
E=\frac{N p^{2}}{2(m+M)}+(M+m) g L.
\]
Võrdusest $A_h = E$ saame (arvestades eelpooltoodud avaldisi $A_h$, $E$ ja $p^2/2m$ jaoks)
\[
N=\left(F_{0}+\frac{k L}{2}-M g-m g\right) \frac{(m+M) L}{m^{2} g H}.
\]
\probend