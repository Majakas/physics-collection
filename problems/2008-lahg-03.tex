\setAuthor{Jaan Kalda}
\setRound{lahtine}
\setYear{2008}
\setNumber{G 3}
\setDifficulty{2}
\setTopic{Varia}

\prob{Tunnel}
Rong, mis sõidab kiirusega $v = \SI{50}{km/h}$, sisenes hästi pikka tunnelisse. Nii rongi kui tunneli ristlõiget lugeda ruuduks küljepikkusega vastavalt $a = \SI{4}{m}$ ja $b = \SI{6}{m}$. Hinnake, milline on tuule kiirus rongi aknast mõõdetuna.

\hint
Rongi taustsüsteemis on rong paigal ning õhk uhab mööda kiirusega $v$, kusjuures õhu jaoks kehtib massi jäävuse seadus.

\solu
Loeme õhu kokkusurumatuks. Sellisel juhul rongi ja tunneli seina vahele juurde
voolanud õhu ruumala võrdub rongi eest välja surutud õhu ruumalaga:
\[
\left(b^{2}-a^{2}\right) u^{\prime}=a^{2} v,
\]
kus $u'$ on õhu kiirus tunneli suhtes. Seega otsitav kiirus on
\[
u=v+u^{\prime}=v\left(1+\frac{a^{2}}{b^{2}-a^{2}}\right)=\frac{v b^{2}}{b^{2}-a^{2}}=\SI{90}{km/h}.
\]
Tegelikult on kiirus veidi väiksem, sest rongi ees surutakse õhk mingil määral kokku ja rongi ees ning taga tekib teatud õhuvool.
\probend