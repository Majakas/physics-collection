\ylDisplay{Kontraktsioon} % Ülesande nimi
{EFO zürii} % Autor
{piirkonnavoor} % Voor
{2018} % Aasta
{G 1} % Ülesande nr.
{1} % Raskustase
{
% Teema: Varia
\ifStatement
Omavahel segatakse $V_v$ liitrit vett ja $V_p$ liitrit piiritust nii, et tekkinud lahuse ruumala $V=\SI{1}{dm^3}$ ning lahuses on massi järgi $p=\SI{44,1}{\percent}$ piiritust. Leidke omavahel segatud vee ja piirituse ruumalad $V_v$ ja $V_p$. Lahuste kokkuvalamisel esineb $\gamma = \SI{6}{\percent}$-line kontraktsioon -- saadud lahuse ruumala on \SI{6}{\percent} väiksem kui vee ja piirituse ruumalade summa. Vee tihedus $\rho_v=\SI{1000}{kg/m^3}$ ning piirituse tihedus $\rho_p=\SI{790}{kg/m^3}$.
\fi


\ifHint
Esialgu võib leida piirituse ja vee massid ning nende kaudu vastavad ruumalad.
\fi


\ifSolution
Tähistame võetud vee massi $m_v$ ning piirituse massi $m_p$. Teades, piirituse massiprotsenti $p = \SI{44,1}{\percent}$, saame leida vee ja piirituse masside suhte.
\[ \frac{m_p}{m_p+m_v}=\SI{0,441}  \quad\Rightarrow\quad m_p=\SI{0,789}{}m_v.\]
Teades lahuse kontraktsiooni $\gamma = \SI{6}{\percent}$, saame kirjutada seose
\[ (V_v + V_p)\SI{0,94}{} = V.\]
Avaldades vee ja piirituse ruumalad massi ja tiheduse kaudu, saame
\[ \frac{m_v}{\rho_v} + \frac{m_p}{\rho_p} = \SI{1,064}{}V.\]
Masside suhtest saime, et $m_p=\SI{0,789}{}m_v$. Asendades selle eelmisesse võrrandisse, saame leida vee ja piirituse massid.
\[ \frac{m_v}{\SI{1}{kg/dm^3}} + \frac{\SI{0,789}{}{m_v}}{\SI{0,79}{kg/dm^3}} = \SI{1,064}{}\cdot\SI{1}{dm^3} \quad\Rightarrow\quad
m_v = \SI{532}{g},\]
\[ m_p = \SI{0,789}{}m_v =  \SI{420}{g}.\]
Vee ja piirituse ruumalad on seega
\[ V_v = \frac{m_v}{\rho_v} = \SI{532}{cm^3},\]
\[ V_p = \frac{m_p}{\rho_p} =  \SI{532}{cm^3}.\]
\fi


\ifEngStatement
% Problem name: Contraction
$V_w$ liters of water and $V_e$ liters of ethanol are mixed with each other so that the volume of their solution is $V=\SI{1}{dm^3}$ and that by mass there is $p=\SI{44,1}{\percent}$ of ethanol in the solution. Find the volumes $V_w$ and $V_e$ of the water and ethanol mixed with each other. When pouring the solutions together there is a $\gamma = \SI{6}{\percent}$ of contraction – the volume of the acquired solution is 6\% smaller than the total volume of water and ethanol. The density of water is $\rho_w=\SI{1000}{kg/m^3}$ and ethanol $\rho_e=\SI{790}{kg/m^3}$.
\fi


\ifEngHint
Initially you can find the masses of the ethanol and water and through those the corresponding volumes.
\fi


\ifEngSolution
Let us mark the mass of the water to be $m_w$ and the mass of the ethanol $m_e$. Knowing the ethanol’s mass percentage $p = \SI{44,1}{\percent}$ we can find the ratio of the masses of the water and the ethanol. 
\[ \frac{m_e}{m_e+m_w}=\SI{0,441}  \quad\Rightarrow\quad m_e=\SI{0,789}{}m_w.\]
Knowing the contraction $\gamma = \SI{6}{\percent}$ of the solution we can write down the relation
\[ (V_w + V_e)\SI{0,94}{} = V.\]
Expressing the volumes of water and ethanol from the mass and density we get
\[ \frac{m_w}{\rho_w} + \frac{m_e}{\rho_e} = \SI{1,064}{}V.\]
From the ratio of the masses we get $m_e=\SI{0,789}{}m_w$. Replacing it to the previous equation we can find the masses of the water and the ethanol
\[ \frac{m_w}{\SI{1}{kg/dm^3}} + \frac{\SI{0,789}{}{m_w}}{\SI{0,79}{kg/dm^3}} = \SI{1,064}{}\cdot\SI{1}{dm^3} \quad\Rightarrow\quad
m_w = \SI{532}{g},\]
\[ m_e = \SI{0,789}{}m_w =  \SI{420}{g}.\]
The volumes of the water and the ethanol are therefore
\[ V_w = \frac{m_w}{\rho_w} = \SI{532}{cm^3},\]
\[ V_e = \frac{m_e}{\rho_e} =  \SI{532}{cm^3}.\]
\fi
}