\setAuthor{Jaan Kalda}
\setRound{lõppvoor}
\setYear{2009}
\setNumber{G 6}
\setDifficulty{5}
\setTopic{Termodünaamika}

\prob{Õhuaken}
Tuba köetakse elektriradiaatoriga, mille võimsus on $P=\SI{1}{kW}$. Välistemperatuur on $t_0=\SI{0}{\degreeCelsius}$, toas püsib ühtlane temperatuur $t_1=\SI{20}{\degreeCelsius}$.
Nüüd avatakse õhuaken ning õueõhku tuleb tuppa kiirusega $v=\SI{20}{l}$ sekundis. Milliseks kujuneb toatemperatuur? Õhu võib lugeda ideaalseks gaasiks,
mille soojusmahtuvus konstantsel rõhul ühe mooli kohta on $c_P=\frac 72R$. Eeldada, et soojuskaod läbi seinte on võrdelised sise- ja välistemperatuuride vahega.

\hint
Soojusliku tasakaalu tingimuse kohaselt kulub radiaatori võimsus sissetuleva õhu soojendamiseks ja seinte soojuskadude kompenseerimiseks. Seinte soojuskaod on leitavad akna avamise eelsest tasakaalutingimusest.

\solu
Soojusvahetuskiirus läbi seinte jms on $$P_s=\alpha (t-t_0)=P\frac{t-t_0}{t_1-t_0}.$$
Peale selle toimub soojusvahetus sissetuleva õhu abil
$P_1=\dot\nu \frac 72 R (t-t_0)$, kus ajaühikus sisenevate moolide arv on $\dot\nu=v/V$ ja mooli ruumala $V=RT/p_0=\SI{22,4}{l/mol}$.
Seega soojusliku tasakaalu tingimuse saab kirja kujul
$$P= P\frac{t-t_0}{t_1-t_0} + c_p\frac vV(t-t_0),$$
millest $$t=t_0+\frac{P}{\frac{P}{t_1-t_0}+c_p\frac{v}{V}}
\approx \SI{13,2}{\degreeCelsius}.$$
\probend