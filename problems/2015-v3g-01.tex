\ylDisplay{Õhupall} % Ülesande nimi
{Eero Vaher} % Autor
{lõppvoor} % Voor
{2015} % Aasta
{G 1} % Ülesande nr.
{2} % Raskustase
{
% Teema: Gaasid
\ifStatement
Heeliumiga täidetud õhupall suudab tõsta koormist massiga kuni $M=\SI{200}{kg}$. Kui suur on õhupalli ruumala $V$? Koormise ruumala lugeda tühiseks. Õhupalli kesta mass on arvestatud koormise massi sisse. Õhu tihedus on $\rho=\SI{1.2}{kg\per m^3}$, õhu rõhk $p=\SI{100}{kPa}$, õhu temperatuur $T=\SI{20}{\celsius}$. Heeliumi molaarmass on $\mu=\SI{4.0}{g\per mol}$, ideaalse gaasi konstant $R=\SI{8.3}{J\cdot mol^{-1}\cdot K^{-1}}$.
\fi


\ifHint
Suurima võimaliku koormise massi korral on õhupalli keskmine tihedus võrdne õhu tihedusega.
\fi


\ifSolution
Suurima võimaliku koormise massi korral on õhupalli keskmine tihedus võrdne õhu tihedusega ehk $\rho=\frac{m+M}{V}$, kus $m$ on õhupallis oleva gaasi mass. Siit saame avaldada $m=\rho V-M$. Lisaks kehtib ideaalse gaasi seadus $pV=\frac{m}{\mu}RT$, millest saame $m=\frac{pV\mu}{RT}$. Nende kahe võrrandi põhjal saame kirjutada
\[
\frac{pV\mu}{RT}=\rho V-M
\]
ehk
\[
V=\frac{M}{\rho-\frac{p\mu}{RT}}=\SI{193}{m^3}.
\]
\fi


\ifEngStatement
% Problem name: balloon
A balloon filled with helium can lift a weight with a mass up to $M=\SI{200}{kg}$. What is the volume of the balloon $V$? The volume of the weight is negligible. The mass of the balloon’s shell is included in the mass of the weight. The air density is $\rho=\SI{1.2}{kg\per m^3}$, air pressure $p=\SI{100}{kPa}$, air temperature $T=\SI{20}{\celsius}$. The molar mass of helium is $\mu=\SI{4.0}{g\per mol}$, universal gas constant is $R=\SI{8.3}{J\cdot mol^{-1}\cdot K^{-1}}$.
\fi


\ifEngHint
In the case of the biggest possible mass of the weight the average density of the balloon is equal to the density of air.
\fi


\ifEngSolution
The average density of the balloon is equal to the air density if the mass of the weight is as big as possible, meaning $\rho=\frac{m+M}{V}$, where $m$ is the mass of the gas in the balloon. From this we can express $m=\rho V-M$. In addition the ideal gas law applies $pV=\frac{m}{\mu}RT$, where we get $m=\frac{pV\mu}{RT}$. Based on these two equations we can write $\frac{pV\mu}{RT}=\rho V-M$ meaning $V=\frac{M}{\rho-\frac{p\mu}{RT}}$. For the given data $V=\SI{193}{m^3}$.
\fi
}