\ylDisplay{Takistid} % Ülesande nimi
{Hans Daniel Kaimre} % Autor
{lahtine} % Voor
{2015} % Aasta
{G 2} % Ülesande nr.
{3} % Raskustase
{
% Teema: Elektriahelad
\ifStatement
Leidke ühesugustest takistitest koosneva ahela kogutakistus $R_{AB}$. Iga takisti takistus on $R$.

\begin{figure}[h]
\centering
\begin{circuitikz}[scale=0.9] \draw

(0,0) to [resistor] (0,2)
(0,0) to [resistor] (0,-2)
(0,0) to [resistor, *-*] (2,0)
(2,0) to [resistor] (4,0)
(4,0) to [resistor, *-*] (6,0)
(2,0) to [resistor, -*] (2,2)
(2,0) to [resistor, -*] (2,-2)
(4,0) to [resistor, -*] (4,2)
(4,0) to [resistor, -*] (4,-2)
(-1,0) to [short,o-] (0,0)
(6,0) to [short,-o] (7,0)
(0,2) -- (6,2) -- (6,0) -- (6,-2) -- (0,-2)
(-1,0) node[label={above:A}] {}
(7,0) node[label={above:B}] {}
;
\end{circuitikz}
\end{figure}
\fi


\ifHint
Kogutakistuse määramist lihtsustab skeemi kavalam ümber joonistamine.
\fi


\ifSolution
Teades, et ühesuguse potentsiaaliga punkte võib ahelas vabalt lahti ja kokku ühendada, saame ülesandes esitatud elektriskeemi viia allolevale kujule.
\begin{center}
\begin{circuitikz}[scale=0.9] \draw
(0,0) to [resistor, *-*] (2,0)
(2,0) to [resistor] (4,0)
(4,0) to [resistor, *-*] (6,0)
(4,1) to [resistor] (6,1)
(4,-1) to [resistor] (6,-1)
(4,1) -- (4,-1)
(6,1) -- (6,-1)
(2,-1.8) -- (2,1.8)
(6.2,-1.8)-- (6.2,1.8)
(2,1.8) to[resistor] (6.2,1.8)
(2,-1.8) to[resistor] (6.2,-1.8)
(6.4,2.6) -- (6.4,-2.6)
(0,2.6) -- (0,-2.6)
(0,2.6) to[resistor] (6.4,2.6)
(0,-2.6) to[resistor] (6.4,-2.6)
(-1,0) to [short,o-] (0,0)
(6,0) to [short,-o] (7.4,0)
(-1,0) node[label={above:A}] {}
(7.4,0) node[label={above:B}] {}
(6.2,0) to[short, *-*] (6.4,0)
;
\end{circuitikz}
\end{center}
\[
R_{AB}= (2R^{-1}+(R+(2R^{-1}+(R+(3R^{-1})^{-1})^{-1})^{-1})^{-1})^{-1}=\frac{15}{41}R.
\]
\fi


\ifEngStatement
% Problem name: Resistors
Find the total resistance $R_{AB}$ of a circuit diagram which consists of identical resistors. The resistance of each resistor is $R$. 
\begin{figure}[h]
\centering
\begin{circuitikz}[scale=0.9] \draw

(0,0) to [resistor] (0,2)
(0,0) to [resistor] (0,-2)
(0,0) to [resistor, *-*] (2,0)
(2,0) to [resistor] (4,0)
(4,0) to [resistor, *-*] (6,0)
(2,0) to [resistor, -*] (2,2)
(2,0) to [resistor, -*] (2,-2)
(4,0) to [resistor, -*] (4,2)
(4,0) to [resistor, -*] (4,-2)
(-1,0) to [short,o-] (0,0)
(6,0) to [short,-o] (7,0)
(0,2) -- (6,2) -- (6,0) -- (6,-2) -- (0,-2)
(-1,0) node[label={above:A}] {}
(7,0) node[label={above:B}] {}
;
\end{circuitikz}
\end{figure}
\fi


\ifEngHint
It would be easier to find the total resistance if you redraw the sketch in a more clever way.
\fi


\ifEngSolution
Knowing that points with same potential can freely be separated and connected in a diagram it is possible to reduce the diagram given in the problem to the form below. 
\begin{center}
\begin{circuitikz}[scale=0.9] \draw
(0,0) to [resistor, *-*] (2,0)
(2,0) to [resistor] (4,0)
(4,0) to [resistor, *-*] (6,0)
(4,1) to [resistor] (6,1)
(4,-1) to [resistor] (6,-1)
(4,1) -- (4,-1)
(6,1) -- (6,-1)
(2,-1.8) -- (2,1.8)
(6.2,-1.8)-- (6.2,1.8)
(2,1.8) to[resistor] (6.2,1.8)
(2,-1.8) to[resistor] (6.2,-1.8)
(6.4,2.6) -- (6.4,-2.6)
(0,2.6) -- (0,-2.6)
(0,2.6) to[resistor] (6.4,2.6)
(0,-2.6) to[resistor] (6.4,-2.6)
(-1,0) to [short,o-] (0,0)
(6,0) to [short,-o] (7.4,0)
(-1,0) node[label={above:A}] {}
(7.4,0) node[label={above:B}] {}
(6.2,0) to[short, *-*] (6.4,0)
;
\end{circuitikz}
\end{center}
$R_{AB}= (2R^{-1}+(R+(2R^{-1}+(R+(3R^{-1})^{-1})^{-1})^{-1})^{-1})^{-1}=\frac{15}{41}R$
\fi
}