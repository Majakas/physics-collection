\setAuthor{Oleg Košik}
\setRound{piirkonnavoor}
\setYear{2006}
\setNumber{G 3}
\setDifficulty{3}
\setTopic{Kinemaatika}

\prob{Sonar}
Vaatame laeva kiiruse määramiseks järgmist meetodit: rannikult saadetakse eemalduvale laevale ultraheli signaal sagedusega  $f_1$. Laevalt peegeldub signaal tagasi rannikule, kus vastuvõtja fikseerib signaali sagedusega $f_2$. Teades, et heli kiirus õhus on $v_h$, määrake laeva kiirus $v$.

\hint
Lõppsageduse avaldamiseks esialgse sageduse kaudu võib Doppleri seadust kaks korda järjest kasutada.

\solu
Sonarist saadetakse välja heli lainepikkusega $\lambda_1 = v_h/f_1$. Helilaine läheneb laevale kiirusega $v_h - v$, seega jõuab laevani helisignaal sagedusega
\[
f_L = \frac{v_h-v}{\lambda_1} = f_1 \frac{v_h-v}{v_h}.
\]
Peegeldudes laevalt tagasi, liigub helisignaal laeva suhtes kiirusega $v_h + v$. Kuna laevalt peegeldunud kiirt saab vaadelda kui kiirt, mis on alguse saanud laeva pardal olevalt signaaliallikalt sagedusega $f_L$, on signaali lainepikkus $\lambda_2 = (v_h+v)/f_L$. Sonari vastuvõtjasse rannikul jõuab signaal sagedusega
\[
f_{2}=\frac{v_{h}}{\lambda_{2}}=f_{L} \frac{v_{h}}{v_{h}+v}=f_{1} \frac{v_{h}-v}{v_{h}+v}.
\]
Avaldame viimasest võrdusest $v$:
\[
v = v_h \frac{f_1-f_2}{f_1+f_2}.
\]

\vspace{0.5\baselineskip}

\emph{Alternatiivne lahendus}

Kasutame Doppleri valemit
\[
f_2 = f_1 \frac{1+v_2/v_h}{1+v_1/v_h},
\]
kus saatja ja vastuvõtja lähenevad üksteisele kiirustega vastavalt $v_1$ ja $v_2$.

Kui signaal jõuab rannikult laevani, siis antud olukorras on saatja kiirus $v_1 = 0$ ja vastuvõtja kiirus $v_2 = -v$, sest vastuvõtja (laev) kaugeneb saatjast. Doppleri valem saab sel juhul kuju
\[
f_L = f_1 \frac{v_h-v}{v_h}, 
\]
kus $f_l$ on laevani jõudva signaali sagedus.

Olukorras, kus signaal läheb laevalt tagasi rannikule, on saatja kiirus $v_1 = -v$, sest saatja (laev) kaugeneb vastuvõtjast (rannikult). Vastuvõtja kiirus on aga $v_2 = 0$. Vastavalt Doppleri valemile jõuab sonari vastuvõtjasse rannikul signaal sagedusega
\[
f_2 = f_L \frac{v_h}{v_h+v} = f_1 \frac{v_h-v}{v_h+v}. 
\]
Avaldame viimasest võrdusest $v$:
\[
v = v_h \frac{f_1-f_2}{f_1+f_2}. 
\]

\emph{Märkus:} kuivõrd iga realistliku laeva kiiruse puhul $v \ll v_h$, siis on lubatud
kasutada ligikaudset Doppleri valemit
\[
f \approx f_0 \left(1+\frac{v}{v_h}\right).
\]
Sama lähendust saab kasutada ka esimese lahenduse puhul.
\probend