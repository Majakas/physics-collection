\setAuthor{Tundmatu autor}
\setRound{piirkonnavoor}
\setYear{2005}
\setNumber{G 6}
\setDifficulty{4}
\setTopic{Dünaamika}

\prob{Aerud}
Aerude pikkus tullist (punktist, kus aerud kinnituvad paadi kere külge) kuni käepidemeni on $a = \SI{1}{m}$ ning tullist kuni labadeni on $b = \SI{1,5}{m}$. Keskmine jõud, millega aerutaja tõmbab kumbagi aeru, on $F = \SI{60}{N}$. Paadi ja vee vaheline takistusjõud on $F_h = \alpha v^2$, kus $\alpha = \SI{20}{kg/m}$. Kui suure keskmise kiirusega liigub paat? Hinnata aerutaja keskmist võimsust.

\hint
Aerulabadele mõjuv keskmine jõud on leitav jõumomentide tasakaalust tullide suhtes. Keskmise kiirusega liikuva paadi puhul kehtib jõudude tasakaal aerulabadele mõjuva jõu ja takistusjõu vahel.

\solu
Jõumomentide tasakaalu tingimus aeru jaoks tullide suhtes annab aerulabadele mõjuva keskmise jõu: $F_l = F a/b$.

Tasakaalutingimus süsteemi paat+aerutaja+aerud jaoks annab võrrandi: 
\[
\frac{2Fa}{b} = \alpha v^2 ,
\]
millest
\[
v=\sqrt{\frac{2 F a}{\alpha b}}=\SI{2}{m/s}.
\]
Kui aerulabad püsiksid tõmbamise ajal vee suhtes paigal, siis oleks võimsus
\[
P=\frac{2 v F a}{b}=\SI{160}{W}.
\]
Et aga aerulabad nihkuvad ilmselt veidi tagasi, siis on tegelik võimsus mõnevõrra suurem.
\probend