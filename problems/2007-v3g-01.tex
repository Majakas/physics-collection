\ylDisplay{Kütteklaas} % Ülesande nimi
{Tundmatu autor} % Autor
{lõppvoor} % Voor
{2007} % Aasta
{G 1} % Ülesande nr.
{1} % Raskustase
{
% Teema: Termodünaamika
\ifStatement
Ruumide soojendamiseks kasutatava elektriliselt köetava klaasi pind on kaetud õhukese valgust läbilaskva elektrit juhtiva kihiga, mille vastasservadele rakendatakse elektriline pinge (vool kulgeb mööda klaasi pinda). Kuidas suhtuvad sellisest klaasist valmistatud ristkülikukujuliselt aknalt eralduvad soojusvõimsused $P_H$ ja $P_V$ sama pinge rakendamisel vastavalt klaasi horisontaalsete ($P_H$) ja vertikaalsete ($P_V$) servade vahel? Akna horisontaalmõõde $a = \SI{0,5}{m}$ ja vertikaalmõõde $b = \SI{1}{m}$.
\fi


\ifHint
Kahe vastasserva vaheline takistus on $R = \frac{\rho L}{S}$, kus $L$ on servade vaheline kaugus ja $S$ ristlõikepindala.
\fi


\ifSolution
Kehtivad valemid
\[
P=\frac{U^{2}}{R}, \quad R=\frac{\rho L}{S},
\]
kus $\rho$ on kattekihi eritakistus. Seega vastavalt orientatsioonile $R_H = \rho b/da$ ja $R_V = \rho a/db d$, kus $d$ on kattekihi paksus. Niisiis,
\[
P_H/P_V = a ^2/b^2 = \num{0,25}.
\]

\fi
}