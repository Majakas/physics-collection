\ylDisplay{Traat} % Ülesande nimi
{Jaan Kalda} % Autor
{lõppvoor} % Voor
{2007} % Aasta
{G 10} % Ülesande nr.
{9} % Raskustase
{
% Teema: Magnetism
\ifStatement
Horisontaalsel libedal pinnal on fikseeritud kaks klemmi, mille vahekaugus $a$ on väiksem neid ühendava hästi painduva sõlmevaba traadi pikkusest $L$. Süsteem asub vertikaalses homogeenses magnetväljas tugevusega $B$, traati läbib vool tugevusega $I$. Joonistage, millise kuju võtab traat. Kirjutage välja võrrandid, kust saab leida mehaanilise pinge $T$ traadis. Leidke selle väärtus eeldusel, et $L \gg a$.
\fi


\ifHint
Traat võtab kaare kuju (sest Ampere’i jõud mõjub analoogselt täispuhutud palli puhul pallikestale ülerõhu poolt mõjuva jõuga: lühikesele mõttelisele traadijupile mõjuv jõud on risti traadijupiga). Pinge leidmiseks on mugav vaadelda lühikest traadijupi lõiku ning kirja panna jõudude tasakaalu tingimuse.
\fi


\ifSolution
Traat võtab kaare kuju (sest Amper’i jõud mõjub analoogselt täispuhutud palli puhul pallikestale ülerõhu poolt mõjuva jõuga: lühikesele mõttelisele traadijupile mõjuv jõud on risti traadijupiga). Kaare raadiuse $R$ saab leida järgmisest võrrandist:
\[
a = 2R \sin (L/2R).
\]
Väikese kaare-elemendi jaoks (pikkusega $\alpha R$) välja kirjutatud Amper’i jõu ja mehaanilise pinge tasakaalust leiame pinge traadis: $\alpha RIB = T \alpha$. Eeldusel, et $L \gg a$, moodustub kaarest peaaegu täisring, st $R = L/2\pi$; seega
\[
T = LIB/2\pi.
\]
\fi
}