\ylDisplay{Hantel} % Ülesande nimi
{Mihkel Kree} % Autor
{lõppvoor} % Voor
{2008} % Aasta
{G 1} % Ülesande nr.
{2} % Raskustase
{
% Teema: Dünaamika
\ifStatement
Hantel koosneb kahest võrdse massiga kerast (kumbki massiga $m$) ning neid ühendavast massitust jäigast vardast. Alguses hoitakse hantel horisontaalselt õhus paigal. Nüüd antakse ühele kuulidest hetkega vertikaalsuunaline kiirus $v$ ning hantel hakkab vabalt liikuma. Vabalangemise kiirendus on $g$. Missugune on süsteemi kineetiline energia hetkel, mil massikese saavutab maksimaalse kõrguse?
\fi


\ifHint
Massikeskme kulgliikumise energia muundub maksimaalsele kõrgusele jõudes täielikult potentsiaalseks energiaks.
\fi


\ifSolution
Kehale antakse energia $E_k = \frac{mv^2}{2}$. Massikese saab impulsi $p_c = mv$ ning hakkab vertikaalsuunas liikuma kiirusega
\[
v_{c}=\frac{p_{c}}{m+m}=\frac{v}{2}
\]
ning süsteemi kulgliikumise energia on seega
\[
E_{v}=\frac{v_{c}^{2}(m+m)}{2}=\frac{m v^{2}}{4}.
\]
Kui keha saavutab maksimaalse kõrguse, on tema kulgliikumise energia täielikult muutunud potentsiaalseks energiaks ning süsteemi kineetiline energia on nüüd
\[
E = E_k - E_v = \frac{mv^2}{4}.
\]
\fi
}