\ylDisplay{Satelliit} % Ülesande nimi
{Eero Vaher} % Autor
{lõppvoor} % Voor
{2013} % Aasta
{G 5} % Ülesande nr.
{5} % Raskustase
{
% Teema: Taevamehaanika
\ifStatement
Geostatsionaarseks orbiidiks nimetatakse sellist orbiiti, millel asuv satelliit
Maa suhtes ei liigu. Kui suur on maa-ala, mida sellisel orbiidil olevalt
satelliidilt jälgida saab? Vastuseks esitage selle maa-ala läbimõõt mõõdetuna
mööda Maa pinda. Gravitatsioonikonstant $G=\SI{6.7e-11}{N \cdot m^2/kg^2}$, Maa
mass $M=\SI{6,0e24}{kg}$, Maa raadius $r=\SI{6400}{km}$, Maa
pöörlemisperiood $t=\SI{24}{h}$.
\fi


\ifHint
Geostatsionaarse orbiidi raadius on leitav jõudude tasakaalust ning orbiidi perioodist.
\fi


\ifSolution
Satelliidile orbiidil mõjuv kesktõmbejõud on Maa poolt satelliidile avaldatav gravitatsioonijõud. Saame $$\frac{mv^2}{R}=G\frac{Mm}{R^2},$$kus $m$ on satelliidi mass ja $R$ orbiidi raadius. Kuna geostatsionaarne satelliit Maa suhtes ei liigu, peab selle tiirlemisperiood olema samuti 24 h. Saame $v=\frac{2\pi R}{t}$. Neist võrranditest saame $$4\pi^2 R^3=GM \Rightarrow R=\sqrt[3]{\frac{GMt^2}{4\pi^2}}=\SI{42400}{km}.$$ Maa keskpunkt, satelliit ning satelliidilt nähtava maa-ala serval asetsev suvaline punkt moodustavad täisnurkse kolmnurga, mille hüpotenuusiks on satelliidi orbiidi raadius ning üheks kaatetiks Maa raadius. Maa keskmes asuvaks nurgaks saame $\alpha=\arccos{\frac{r}{R}}$, kuid kuna meid huvitab satelliidilt nähtava ala läbimõõt, peame leidma nurga $2\alpha$. Sellele nurgale vastav kaare pikkus Maa pinnal on $d=2r\alpha$ (kui nurk on radiaanides). Lõppvastuseks saame 
$$d=2r\arccos \left(\frac{r}{\sqrt[3]{\frac{GMt^2}{4\pi^2}}}\right) =\SI{18000}{km}.$$
\fi


\ifEngStatement
% Problem name: Satellite
A geostationary orbit is such an orbit that if it has a satellite on it, the satellite does not move with respect to the Earth. How big is the area of ground that can be observed from such a satellite? Give the area’s diameter measured along the surface of the Earth as an answer. The gravitational constant is $G=\SI{6.7e-11}{N \cdot m^2/kg^2}$, the Earth’s mass $M=\SI{6,0e24}{kg}$, the Earth’s radius $r=\SI{6400}{km}$, the Earth’s rotation period $t=\SI{24}{h}$.
\fi


\ifEngHint
The radius of the geostationary orbit can be found from the force balance and the orbital period.
\fi


\ifEngSolution
The centripetal force applied to the satellite on the orbit is the gravity force applied by the Earth to the satellite. We get 
$$\frac{mv^2}{R}=G\frac{Mm}{R^2},$$
where $m$ is the satellite’s mass and $R$ the orbit’s radius. Because a geostationary satellite does not move with respect to the Earth then its orbiting period also has to be 24 h. We get $v=\frac{2\pi R}{t}$. From these equations we get
$$4\pi^2 R^3=GM \Rightarrow R=\sqrt[3]{\frac{GMt^2}{4\pi^2}}=\SI{42400}{km}.$$
The Earth’s center, the satellite and a random point located on the edge of the Earth region seen from the satellite make up a right triangle. Its hypotenuse is the radius of the satellite’s orbit and one of the legs is the Earth’s radius. We get the angle in the Earth’s center to be $\alpha=\arccos{\frac{r}{R}}$ but because we are interested in the diameter of the region seen from the satellite we have to find the angle $2\alpha$. The length of the arc on the Earth’s surface that corresponds to this angle is $d=2r\alpha$ (if the angle is in radians). For the final answer we get
$$d=2r\arccos \left(\frac{r}{\sqrt[3]{\frac{GMt^2}{4\pi^2}}}\right) =\SI{18000}{km}.$$
\fi
}