\setAuthor{Eero Vaher}
\setRound{piirkonnavoor}
\setYear{2016}
\setNumber{G 1}
\setDifficulty{1}
\setTopic{Elektriahelad}

\prob{Voltmeetrid}
Elektriskeemis on pingeallikas pingega $U_0=\SI{30}{V}$ ning neli ühesugust voltmeetrit. Kui suur on iga voltmeetri näit?

\begin{center}
	\tikzset{component/.style={draw,thick,circle,fill=white,minimum size =0.75cm,inner sep=0pt}}
	\begin{resizebox}{0.45\linewidth}{!}{
		\begin{circuitikz}
			\draw
			
			(6,0) to[battery1,l=${U_0=\SI{30}{V}}$] (0,0) to (0,2) to (6,2) to (6,0)
			(2,2) to[short, *-] (2,3) to (6,3) to[short, -*] (6,2)
			;
			\node[component] at (1,2) {V$_1$};
			\node[component] at (3,2) {V$_2$};
			\node[component] at (5,2) {V$_3$};
			\node[component] at (4,3) {V$_4$};
		\end{circuitikz}}
	\end{resizebox}
\end{center}

\hint
Kuna voltmeetrit on ühesugused, võib need asendada takistustega $R$.

\solu
Voltmeetrid mõõdavad alati pingelangu enda klemmidel, kusjuures pingelangud on määratud voltmeetrite takistustega. Olgu iga voltmeetri takistus $R$. Kogu skeemi takistus on 
\[
R_k=\left(\frac{1}{R}+\frac{1}{2R}\right)^{-1}+R=\frac{5}{3}R.
\]
Pingelang esimesel voltmeetril on
\[
U_1=\frac{R}{R_k}U_0=\SI{18}{V}.
\]
Pingelang neljandal voltmeetril on $U_4=U_0-U_1=\SI{12}{V}$ ning pingelangud teisel ja kolmandal voltmeetril on 
\[
U_2=U_3=\frac{U_4}{2}=\SI{6}{V}.
\]

\probeng{Voltmeters}
In a circuit diagram there is a voltage source of voltage $U_0=\SI{30}{V}$ and four identical voltmeters. What is the reading of each voltmeter?
\begin{center}
	\tikzset{component/.style={draw,thick,circle,fill=white,minimum size =0.75cm,inner sep=0pt}}
	\begin{resizebox}{0.45\linewidth}{!}{
		\begin{circuitikz}
			\draw
			
			(6,0) to[battery1,l=${U_0=\SI{30}{V}}$] (0,0) to (0,2) to (6,2) to (6,0)
			(2,2) to[short, *-] (2,3) to (6,3) to[short, -*] (6,2)
			;
			\node[component] at (1,2) {V$_1$};
			\node[component] at (3,2) {V$_2$};
			\node[component] at (5,2) {V$_3$};
			\node[component] at (4,3) {V$_4$};
		\end{circuitikz}}
	\end{resizebox}
\end{center}

\hinteng
Because the voltmeters are identical they can be replaced by the resistance $R$.

\solueng
Voltmeters always measure a voltage drop on their terminals, moreover the voltage drop is determined by the resistances of the voltmeters. Let the resistance of each voltmeter be $R$. The resistance of the whole circuit is $R_k=\left(\frac{1}{R}+\frac{1}{2R}\right)^{-1}+R=\frac{5}{3}R$. The voltage drop on the first voltmeter is $U_1=\frac{R}{R_k}U_0=\SI{18}{V}$. The voltage drop on the fourth voltmeter $U_4=U_0-U_1=\SI{12}{V}$ and the voltage drops on the second and third voltmeter $U_2=U_3=\frac{U_4}{2}=\SI{6}{V}$.
\probend