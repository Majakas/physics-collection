\setAuthor{Mihkel Rähn}
\setRound{lahtine}
\setYear{2014}
\setNumber{G 1}
\setDifficulty{2}
\setTopic{Dünaamika}

\prob{Kaubarong}
Kaubarongi massiga $m=\SI{5000}{t}$ veab vedur võimsusega $N=\SI{2500}{kW}$. Veerehõõrdetegur rataste ja rööpa vahel on $\mu=\SI{0,002}{}$.\\
\osa Leidke rongi kiirus $v_1$ horisontaalsel teel.\\
\osa Leidke rongi kiirus $v_2$ tõusul üks sentimeeter ühe meetri kohta.\\
Õhutakistusega mitte arvestada.

\hint
\osa Horisontaalsel teel peab kaubarong ületama takistava hõõrdejõu.\\
\osa Lisaks hõõrdejõule peab rong nüüd ületama ka raskusjõu.

\solu
Jagades tuntud töö valemi $A=Fs+\Delta E$ ajaga, saame võimsuse jaoks võrrandi 
\[
N=Fv+\frac{\Delta E}{t},
\]
mille kohaselt veduri võimsus on tasakaalustatud takistusjõudude ja kiiruse korrutise ning potentsiaalse energia $E$ muutumise kiiruse summaga. Hõõrdejõud on $F=\mu mg$. Horisontaalsel teel potentsiaalne energia ei muutu ning avaldades saame
\[
v_1=\frac{N}{\mu mg}=\SI{92}{km \per h}.
\]
Tõusul toimub potentsiaalse energia $E=mgh$ suurenemine, kus $h=s\sin a$ ning $\sin a=\frac{1cm}{100cm}$. Seega
\[
\frac{\Delta E}{t}=\frac{smg}{t}\sin a=mgv\sin a.
\]
Seega teisel juhul
\[
N=umgv_2+mgv_2\sin a,
\]
millest
\[
v_2=\frac{N}{mg(u+\sin a)}=\SI{15}{km \per h}.
\]

\probeng{Freight train}
A freight train of mass $m=\SI{5000}{t}$ is carried by a locomotive of power $N=\SI{2500}{kW}$. Coefficient of rolling friction between the wheels and rails is $\mu=\SI{0,002}{}$.\\
a) Find the speed $v_1$ of the train on a horizontal road.\\
b) Find the speed $v_2$ of the train on a rise of one centimeter per meter.\\
Do not account for air resistance.

\hinteng
a) On a horizontal road the freight train has to overcome the hindering friction force.\\
b) In addition to the friction force the train now also has to overcome the gravity force.

\solueng
Dividing the known work equation $A=Fs+\Delta E$ by time we get the equation $N=Fv+\frac{\Delta E}{t}$ for power according to which the power of locomotive is equal to the sum of the product of balanced drag forces and speed and the potential energy $E$’s rate of change. Frictional force is $F=\mu mg$. On a horizontal road the potential energy does not change and by expressing we get $v_1=\frac{N}{\mu mg}=\SI{92}{km \per h}$. During the rise there is the increase of potential energy $E=mgh$ where $h=s\sin a$ and $\sin a= \frac{1cm}{100cm}$. $\frac{\Delta E}{t}=\frac{smg}{t}\sin a=mgv\sin a$. Thus, in the other case $N=umgv_2+mgv_2\sin a$ where $v_2=\frac{N}{mg(u+\sin a)}=\SI{15}{km \per h}$.
\probend