\ylDisplay{Ookean} % Ülesande nimi
{Tundmatu autor} % Autor
{lahtine} % Voor
{2005} % Aasta
{G 8} % Ülesande nr.
{7} % Raskustase
{
% Teema: Vedelike-mehaanika
\ifStatement
Vee kokkusurutavuse tegur $\beta = \SI{5e-5}{atm^{-1}}$.\\
\osa Hinnake ookeani keskmise sügavuse muutumist juhul, kui vesi oleks täielikult kokkusurumatu. Ookeani keskmine sügavus $h \approx \SI{3800}{m}$.\\
\osa Hinnake vee tiheduste vahet $\Delta \rho$ veepinnalähedasel veel ja veel ookeani süvendi põhjas sügavusel $H = \SI{10}{km}$. 

\emph{Märkus}. kokkusurutavuse tegur $\beta$ näitab keha ühikulise ruumala vähenemist rõhu suurenemisel ühe ühiku võrra. Atmosfäär on rõhu mõõtmise ühik, mis võrdub atmosfääri normaalrõhuga merepinna kõrgusel: \SI{1}{atm} = \SI{101325}{Pa}.
\fi


\ifHint
\osa Juhul kui ookeani keskmine tihedus ning lisarõhk on vastavalt $\rho$ ja $p$, kehtib ülesandes mainitud kokkusurutavuse seos $\rho \approx \rho_0 ( 1 + \beta p)$. Lisaks on vee rõhud ookeani põhjas kokkusurumatul ja kokkusurutud juhtudel võrdsed.\\
\osa Tasub vaadelda väikest vee kogust massiga $m$ ning selle ruumala muute sügavusel $H$ võrreldes pinnapealse olukorraga.
\fi


\ifSolution
\osa Tähistame vedeliku keskmise tiheduse pinnast sügavuseni $h$ kui $\rho$. Kuna vedeliku mass on kokkusurumatu ja kokkusurutud olukordades võrdne, siis
\[
\rho_{0} h_{0}=\rho h.
\]
Kui tiheduse muut on väike, avaldub tiheduste muudu kui
\[
\rho = \rho_0 (1 + \beta p\idx{keskmine}),
\]
kus $p\idx{keskmine}$ on keskmine veele mõju rõhk. Vee rõhk sügavusel $h'$ on $p(h') = \rho gh'$, ehk keskmine rõhk avaldub kui $p\idx{keskmine} = p(h' = h/2) = \rho gh/2$ ning
\[
\rho=\rho_{0}\left(1+\frac{\beta \rho g h}{2}\right)=\frac{\rho_{0} h_{0}}{h} \quad \Rightarrow \quad h\left(1+\frac{\beta \rho g h}{2}\right)=h_{0},
\]
Viimasest valemist saame avaldada sügavuste vahe $\Delta h$:
\[
\Delta h=h_{0}-h=\frac{\beta \rho g h^{2}}{2} \approx \SI{36}{m}.
\]
\osa Sügavusel $H = \SI{10000}{m}$ on lisarõhk ligikaudu $p \approx \rho gH$. Sellele vastab ruumala muutus (ühe kuupmeetri kohta): 
\[
\delta=\frac{\Delta V}{V}=\beta \rho g H \approx \num{0,0484}.
\]
Kui veepinnal on ruumalaga $V$ vee mass $\rho V$, siis sama vee massi ruumala sügavusel $H$ on ligikaudu $V (1 - \delta )$ ja tihedus seega:
\[
\rho_{H} V(1-\delta)=\rho V \quad \Rightarrow \quad \rho_{H}=\frac{\rho}{1-\delta}.
\]
Seega sellel sügavusel on vee tiheduse muutus: 
\[
\Delta \rho=\rho_{H}-\rho=\frac{\rho}{1-\delta}-\rho=\frac{\rho \delta}{1-\delta} \approx \SI{51}{kg/m^3}.
\]
\fi
}