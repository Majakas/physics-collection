\setAuthor{Tundmatu autor}
\setRound{lõppvoor}
\setYear{2004}
\setNumber{G 1}
\setDifficulty{1}
\setTopic{Teema}

\prob{Jahipüss}
Jahipüss massiga $M = \SI{3,5}{kg}$ ripub horisontaalselt kahe paralleelse niidi otsas. Tulistamisel kerkis jahipüss $h = \SI{20}{cm}$ võrra tagasilöögi tõttu. Kuuli mass $m = \SI{10}{g}$. Määrata kuuli kiirus $v_0$ torust väljalennu hetkel.

\hint

\solu
Püssi kiiruse $u_{0}$ vahetult peale tulistamist saame leida impulsi jäävuse seadusest:
$M u_{0}=m v_{0}$, kus $v_{0}$ on otsitav kuuli algkiirus. Peale tulistamist alustab püss niidi otsas pendlisarnasat liikumist. Oma algse kineetilise energia $M u_{0}^{2} / 2$ kulutab püss tõusmiseks kõrgusele $h$, kus tema potentsiaalne energia on $M g h$. Energia jäävuse tõttu:
$$
\frac{1}{2} M u_{0}^{2}=M g h \quad \Rightarrow \quad u_{0}=\sqrt{2 g h}.
$$
Siit
$$
v_{0}=\frac{M u_{0}}{m}=\frac{M}{m} \sqrt{2 g h} \approx \SI{693}{m/s}.
$$
\probend