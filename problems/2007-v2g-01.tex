\ylDisplay{Hobune} % Ülesande nimi
{Valter Kiisk} % Autor
{piirkonnavoor} % Voor
{2007} % Aasta
{G 1} % Ülesande nr.
{1} % Raskustase
{
% Teema: Dünaamika
\ifStatement
Puu oksal istub poiss, kes soovib hüpata puu alt mööda galopeeriva hobuse selga. Hobuse kiirus on $v = \SI{10}{m/s}$ ja puuoksa kõrgus sadula suhtes $h = \SI{3}{m}$. Kui suur peab olema horisontaalsihiline distants sadula ja puuoksa vahel sel hetkel kui poiss oksast lahti laseb?
\fi


\ifHint
Langemise aeg on avaldatav valemi $s = \frac{at^2}{2}$ kaudu.
\fi


\ifSolution
Vaba langemise aeg
\[
t=\sqrt{\frac{2 h}{g}}=\sqrt{\frac{2 \cdot 3}{\num{9,81}}} \approx \SI{0,78}{s}
\]
Seega otsitav kaugus on $s = vt = \SI{10}{m/s} \cdot \SI{0,78}{s} = \SI{7,8}{m}$.
\fi
}