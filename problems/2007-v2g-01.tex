\setAuthor{Valter Kiisk}
\setRound{piirkonnavoor}
\setYear{2007}
\setNumber{G 1}
\setDifficulty{1}
\setTopic{Dünaamika}

\prob{Hobune}
Puu oksal istub poiss, kes soovib hüpata puu alt mööda galopeeriva hobuse selga. Hobuse kiirus on $v = \SI{10}{m/s}$ ja puuoksa kõrgus sadula suhtes $h = \SI{3}{m}$. Kui suur peab olema horisontaalsihiline distants sadula ja puuoksa vahel sel hetkel kui poiss oksast lahti laseb?

\hint
Langemise aeg on avaldatav valemi $s = \frac{at^2}{2}$ kaudu.

\solu
Vaba langemise aeg
\[
t=\sqrt{\frac{2 h}{g}}=\sqrt{\frac{2 \cdot 3}{\num{9,81}}} \approx \SI{0,78}{s}
\]
Seega otsitav kaugus on $s = vt = \SI{10}{m/s} \cdot \SI{0,78}{s} = \SI{7,8}{m}$.
\fi


\probeng{Horse}
A boy who is sitting on a tree branch wants to drop onto a horse that happens to be galloping by right under the tree. The horse is moving at a speed of $v = \SI{10}{m/s}$ and the branch is $h = \SI{3}{m}$ higher than the saddle. What does the horizontal distance between the saddle and the branch have to be at the moment when the boy lets go of the branch?
\fi


\hinteng
One can express the free fall time via the equation $s = \frac{at^2}{2}$.


\hintsol
The free fall time is 
\[
t=\sqrt{\frac{2 h}{g}}=\sqrt{\frac{2 \cdot 3}{\num{9,81}}} \approx \SI{0,78}{s}
\]
The desired distance is thus $s = vt = \SI{10}{m/s} \cdot \SI{0,78}{s} = \SI{7,8}{m}$.
\probend