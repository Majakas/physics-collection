\setAuthor{Valter Kiisk}
\setRound{piirkonnavoor}
\setYear{2007}
\setNumber{G 1}
\setDifficulty{1}
\setTopic{Dünaamika}

\prob{Hobune}
Puu oksal istub poiss, kes soovib hüpata puu alt mööda galopeeriva hobuse selga. Hobuse kiirus on $v = \SI{10}{m/s}$ ja puuoksa kõrgus sadula suhtes $h = \SI{3}{m}$. Kui suur peab olema horisontaalsihiline distants sadula ja puuoksa vahel sel hetkel kui poiss oksast lahti laseb?

\hint
Langemise aeg on avaldatav valemi $s = \frac{at^2}{2}$ kaudu.

\solu
Vaba langemise aeg
\[
t=\sqrt{\frac{2 h}{g}}=\sqrt{\frac{2 \cdot 3}{\num{9,81}}} \approx \SI{0,78}{s}
\]
Seega otsitav kaugus on $s = vt = \SI{10}{m/s} \cdot \SI{0,78}{s} = \SI{7,8}{m}$.
\probend