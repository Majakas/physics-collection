\setAuthor{Mihkel Rähn}
\setRound{lõppvoor}
\setYear{2008}
\setNumber{G 3}
\setDifficulty{2}
\setTopic{Gaasid}

\prob{Toaõhk}
Leida seos toaõhu molekulide summaarse kulgliikumise kineetilise energia ja toatemperatuuri vahel. Õhu rõhk on $p$ ja toa ruumala $V$.

\hint
Ühe molekuli keskmine kineetiline energia temperatuuril $T$ avaldub kui $\left\langle E_{m}\right\rangle=\frac{3}{2} k T$.

\solu
Summaarne kineetiline energia on $E = N \cdot \left\langle E_m\right\rangle$, kus $N$ on toas oleva gaasi molekulide arv ja $\left\langle E_m\right\rangle$ ühe molekuli gaasi keskmine kineetiline energia. Kehtib
\[
\left\langle E_{m}\right\rangle=\frac{3}{2} k T.
\]
Ideaalse gaasi võrrandist saab avaldada toas olevate molekulide arvu $N = pV /(kT)$ Pannes need kokku, $E = \frac{3}{2} pV$. Kuna toas on õhurõhk võrdne välisrõhuga, ei sõltu toas olevate õhumolekulide summaarne kineetiline energia temperatuurist.
\probend