\ylDisplay{Toaõhk} % Ülesande nimi
{Mihkel Rähn} % Autor
{lõppvoor} % Voor
{2008} % Aasta
{G 3} % Ülesande nr.
{2} % Raskustase
{
% Teema: Gaasid
\ifStatement
Leida seos toaõhu molekulide summaarse kulgliikumise kineetilise energia ja toatemperatuuri vahel. Õhu rõhk on $p$ ja toa ruumala $V$.
\fi


\ifHint
Ühe molekuli keskmine kineetiline energia temperatuuril $T$ avaldub kui $\left\langle E_{m}\right\rangle=\frac{3}{2} k T$.
\fi


\ifSolution
Summaarne kineetiline energia avaldub kui $E = N \cdot \left\langle E_m\right\rangle$, kus $N$ on toas oleva gaasi molekulide arv ja $\left\langle E_m\right\rangle$ ühe molekuli gaasi keskmine kineetiline energia. Kehtib
\[
\left\langle E_{m}\right\rangle=\frac{3}{2} k T.
\]
Ideaalse kaasi võrrandist saab avaldada toas olevate molekulide arvu $N = pV /(kT)$ Pannes need kokku, $E = \frac{3}{2} pV$. Kuna toas on õhurõhk võrdne välisrõhuga ei sõltu toas olevate õhumolekulide summaarne kineetiline energia temperatuurist.
\fi
}