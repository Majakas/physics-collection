\ylDisplay{Liivahunnik} % Ülesande nimi
{Tundmatu autor} % Autor
{piirkonnavoor} % Voor
{2011} % Aasta
{G 5} % Ülesande nr.
{3} % Raskustase
{
% Teema: Staatika
\ifStatement
Millisele pindalale on võimalik mahutada koonusekujuline liivahunnik, kui liiva ruumala on $V=\SI{50}{m^{3}}$ ja libisevate liivakihtide vaheline efektiivne hõõrdetegur $\mu=\num{0.4}$? Liivahunniku ja aluspinna hõõrdeteguri võib lugeda väga suureks.
\fi


\ifHint
Liivahunniku maksimaalse kõrguse saavutamiseks peavad pindmised liivakihid
olema libisemise äärel. Seega tasub vaadelda jõudude tasakaalu pindmiste liivaterade jaoks.
\fi


\ifSolution
Liivahunniku maksimaalse kõrguse saavutamiseks peavad pindmised liivakihid olema libisemise äärel, ehk kehtib $\tan (\alpha)=\frac{h}{R}=\mu$, kus $\alpha$ on nurk maa ja koonuse moodustaja vahel, $R$ hunniku aluse raadius ja $h$ hunniku kõrgus. Liiva ruumala on
\[
V=\frac{1}{3} \pi R^{2} h=\frac{1}{3} \pi R^{3} \mu,
\]
millest 
\[
R=\sqrt[3]{\frac{3 V}{\pi \mu}}
\]
ning seega hunniku aluse pindala on
\[
S=\pi R^{2}=\sqrt[3]{9 \pi\left(\frac{V}{\mu}\right)^{2}} \approx \SI{52.4}{m^2}.
\]
\fi
}