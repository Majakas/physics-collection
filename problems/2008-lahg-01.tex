\ylDisplay{Auto} % Ülesande nimi
{Tundmatu autor} % Autor
{lahtine} % Voor
{2008} % Aasta
{G 1} % Ülesande nr.
{1} % Raskustase
{
% Teema: Kinemaatika
\ifStatement
Paigalseisust liikuma hakanud autol kulus teatud vahemaa läbimiseks $t = \SI{15}{s}$. Millise ajaga läbis auto viimase viiendiku sellest vahemaast? Auto liikumine lugeda ühtlaselt kiirenevaks.
\fi


\ifHint
Lihtsam on leida, kui palju aega kulus autol esimese $4/5$ läbimiseks ning seejärel võtta $t$ ja leitud aja vahe.
\fi


\ifSolution
Olgu $s$ läbitud vahemaa, $a$ auto kiirendus ning $\tau$ aeg, millega auto läbis esimese $4/5$ teest. Kehtivad võrdused
\[
s=\frac{a t^{2}}{2}, \quad \frac{4}{5} s=\frac{a \tau^{2}}{2}.
\]
Siit $\tau = 2t/\sqrt 5\approx \SI{13,4}{s}$. Seega otsitav ajavahemik on $t-\tau\approx \SI{1,6}{s}$.
\fi
}