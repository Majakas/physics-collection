\ylDisplay{Takisti} % Ülesande nimi
{Tundmatu autor} % Autor
{piirkonnavoor} % Voor
{2007} % Aasta
{G 7} % Ülesande nr.
{5} % Raskustase
{
% Teema: Elektriahelad
\ifStatement
Oletagem, et me tahame teha takisti takistusega $R = 1 \ohm$, mille takistuse temperatuurisõltuvus oleks toatemperatuuri ümbruses võimalikult väike. Olgu meil kasutada raudtraat ristlõikepindalaga $s = \SI{0,030}{mm^2}$ ja grafiitpulk ristlõikepindalaga $S = \SI{3,0}{mm^2}$. Kuidas valmistada soovitud takistit ja kui pikki grafiitpulga ning terastraadi juppe tuleb seejuures kasutada? Grafiidi ja raua eritakistused on vastavalt $\rho_g = \SI{3,0e5}{\ohm.m}$ ning $\rho_r = \SI{9,7e-8}{\ohm.m}$; takistuse temperatuurikoefitsiendid (suhtelised muutused $\Delta R/R$ temperatuuri kasvamisel ühe kraadi võrra) on $\alpha_g = \SI{-5,0e-3}{K^{-1}}$ ning $\alpha_r = \SI{6,41e-3}{K^{-1}}$
\fi


\ifHint
Mõlemat tüüpi traadijuppide pikkused peavad olema sellised, et esiteks, summaarne takistus on \SI{1}{\ohm} ning teiseks, temperatuurist sõltuvus oleks võimalikult väike. Temperatuurist sõltuvuse leidmiseks tuleb kasutada ülesandes antud temperatuurikoefitsente.
\fi


\ifSolution
Traadi ja pulga takistused pikkusühiku kohta on vastavalt $r_r = \rho_r/s = \SI{3,2}{\ohm/m}$ ja $r_g = \rho_g/S = \SI{10}{\ohm/m}$. Olgu traadi ja pulga pikkused vastavalt $l_r$ ja $l_g$. Arvestades temperatuurisõltuvusega, on takistused vastavalt
\[
R_{r}=l_{r} r_{r}\left(1+\alpha_{r} \Delta T\right) \text { ja } R_{g}=l_{g} r_{g}\left(1+\alpha_{g} \Delta T\right).
\]
Järjestikühenduse korral on summaarne takistus
\[
R=\left(l_{r} r_{r}+l_{g} r_{g}\right)+\left(l_{r} r_{r} \alpha_{r}+l_{g} r_{g} \alpha_{g}\right) \Delta T.
\]
Temperatuurisõltuvus on minimaalne (lineaarses lähenduses olematu), kui
\[
l_{r} r_{r} \alpha_{r}+l_{g} r_{g} \alpha_{g}=0.
\]
Sellisel juhul on takistus
\[
R=l_{r} r_{r}+l_{g} r_{g}.
\]
Nendest kahest võrrandist saame avaldada $l_r$ ja $l_g$: esimesest võrrandist leiame $l_rr_r = -l_gr_g\alpha_g/\alpha_r$, mille asendamisel teise saame
\[
R=l_{g} r_{g}\left(1-\frac{\alpha_{g}}{\alpha_{r}}\right) \quad \Rightarrow \quad l_{g}=\frac{R \alpha_{r}}{r_{g}\left(\alpha_{r}-\alpha_{g}\right)} \approx \SI{5,6}{cm}.
\]
Analoogselt
\[
l_{r}=\frac{R \alpha_{g}}{r_{r}\left(\alpha_{g}-\alpha_{r}\right)} \approx \SI{13,6}{cm}.
\]
\fi
}