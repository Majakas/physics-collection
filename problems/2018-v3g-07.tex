\setAuthor{Eero Vaher}
\setRound{lõppvoor}
\setYear{2018}
\setNumber{G 7}
\setDifficulty{6}
\setTopic{Elektriahelad}

\prob{Takistuste tuvastamine}
\begin{wrapfigure}[10]{r}{0.4\textwidth}
\vspace{-20pt}
\begin{resizebox}{\linewidth}{!}{
\begin{circuitikz}
\draw
(1,0) to[battery1,l_=${U_0=\SI{14}{V},\,I_0=\SI{10}{A}}$] (8,0) to (1,0) to (1,2.5) to (2,2.5) to[short, *-] (2,4) to[resistor,l=${R_1}$, -*] (4.5,4) to[resistor,l=${R_4}$] (7,4) to[short, -*] (7,2.5) to (8,2.5) to (8,0)
(2,2.5) to (2,1) to[resistor,l=${R_2}$, -*] (4.5,1) to[resistor,l=${R_5}$] (7,1) to (7,2.5)
(4.5,1) to[resistor,l_=$R_3$] (4.5,4)
;
\draw[->,thick] (4,2) -- (4,3);
\end{circuitikz}}
\end{resizebox}
\end{wrapfigure}

Vooluallikaga on ühendatud viis takistit. Neist kolme takistus on \SI{1}{\ohm}, ülejäänud kaks on tundmatu, kuid ühesuguse takistusega. Vooluallika pinge $U_0=\SI{14}{V}$ ning voolutugevus selles $I_0=\SI{10}{A}$. Pinge ja voolutugevus kolmandal takistil on vastavalt $U_3=\SI{2}{V}$ ning $I_3=\SI{2}{A}$. Joonisel on märgitud elektrivoolu suund takistis $R_3$. Määrake kõigi takistite takistused.

\hint
Takistuste leidmiseks tasub kõik \SI{1}{\ohm} takistuste kombinatsioonid läbi vaadata ning süstemaatiliselt valed konfiguratsioonid elimineerida.

\solu
\begin{wrapfigure}[9]{r}{0.4\linewidth}
 \vspace{-8pt}
	\begin{resizebox}{\linewidth}{!}{
		\begin{circuitikz}
			\draw
				(8,0) to[battery1] (1,0) to (1,2.5) to (2,2.5) to[short, *-] (2,4) to[resistor,l=${R_1=\SI{2}{\ohm}}$, -*] (4.5,4) to[resistor,l=${R_4=\SI{1}{\ohm}}$] (7,4) to[short, -*] (7,2.5) to (8,2.5) to (8,0)
				(2,2.5) to (2,1) to[resistor,l=${R_2=\SI{1}{\ohm}}$, -*] (4.5,1) to[resistor,l=${R_5=\SI{2}{\ohm}}$] (7,1) to (7,2.5)
				(4.5,1) to[resistor,l_=${R_3=\SI{1}{\ohm}}$] (4.5,4)
				;
		\end{circuitikz}}
	\end{resizebox}
\end{wrapfigure}
Elektrivoolul on võimalik ahelat läbida kolmel moel: läbi ülemise haru (takistid $R_1$ ja $R_4$), läbi keskmise haru (takistid $R_2$, $R_3$ ja $R_4$) ning läbi alumise haru (takistid $R_2$ ja $R_5$). Pingelang igal takistil on võrdne selle takistuse ning seda läbiva voolutugevuse korrutisega ning pingelangude summa kõigil kolmel teekonnal peab olema võrdne pingega vooluallikal. Niisiis $I_1R_1+I_4R_4=I_2R_2+I_3R_3+I_4R_4=I_2R_2+I_5R_5=U_0$. Neist võrranditest saab tuletada avaldised $I_1R_1=I_2R_2+I_3R_3$ ning $I_3R_3+I_4R_4=I_5R_5$, mis kirjeldavad pingelange skeemi vasakul ning paremal poolel. Lisaks saame panna kirja võrrandid voolutugevuste jaoks: $I_1=I_0-I_2$, $I_4=I_1+I_3$ ning $I_5=I_2-I_3$.

Paneme tähele, et kui $R_1=R_2$ ja $R_4=R_5$, siis sümmeetriakaalutlustel peaks kehtima $R_3=0$. 

On ilmne, et $R_3=\SI{1}{\ohm}$. Oletame esmalt, et $R_1=R_4=R_3$. Sellisel juhul avaldub pingelang ülemises harus kujul $(2I_1+I_3)R_3=U_0$, millest järeldub $I_1=\SI{6}{A}$ ning $I_2=\SI{4}{A}$. Vaadeldes voolu teekonda läbi keskmise haru, saame kirjutada $I_2R_2+U_3+(I_1+I_3)R_3=U_0$, millest saab järeldada $R_2=R_5=\SI{1}{\ohm}$. Selline skeem rahuldab eespool mainitud sümmeetriat, seega tehtud eeldus pole tõene. 

Oletame nüüd, et $R_2=R_5=R_3$. Skeemi alumise haru jaoks saame kirjutada $(2I_2-I_3)R_3=U_0$ ehk $I_2=\SI{8}{A}$ ning $I_1=\SI{2}{A}$. Pingelangud skeemi vasakus pooles peavad rahuldama võrrandit $I_1R_1=(I_2+I_3)R_3$, millest järeldub $R_1=\SI{5}{\ohm}$. Sellisel juhul on aga pingelang skeemi ülemisel harul $(2I_1+I_3)R_1=\SI{30}{V}\not=U_0$.

Oletame nüüd, et $R_1=R_5=R_3$. Pingelang skeemi keskmisel harul on $I_2R_2+U_3+(I_0-I_2+I_3)R_2=U_3+(I_0+I_3)R_2=U_0$ ehk $R_2=R_4=\SI{1}{\ohm}$. Tegemist on juba vaadeldud juhuga. 

Ainus järelejäänud võimalus on $R_2=R_4=R_3$. Skeemi vasakust poolest saame $(I_0-I_2)R_1=I_2R_3+U_3$, paremast $(I_2-I_3)R_1=U_3+(I_0-I_2+I_3)R_3$. Nende põhjal $(I_0-I_3)R_1=2U_3+(I_0+I_3)R_3$ ehk $R_1=\SI{2}{\ohm}$. On lihtne veenduda, et leitud väärtused rahuldavad kõiki võrrandeid pingelangude jaoks.

\probeng{Finding resistances}
\begin{wrapfigure}[10]{r}{0.4\textwidth}
\vspace{-20pt}
\begin{resizebox}{\linewidth}{!}{
\begin{circuitikz}
\draw
(1,0) to[battery1,l_=${U_0=\SI{14}{V},\,I_0=\SI{10}{A}}$] (8,0) to (1,0) to (1,2.5) to (2,2.5) to[short, *-] (2,4) to[resistor,l=${R_1}$, -*] (4.5,4) to[resistor,l=${R_4}$] (7,4) to[short, -*] (7,2.5) to (8,2.5) to (8,0)
(2,2.5) to (2,1) to[resistor,l=${R_2}$, -*] (4.5,1) to[resistor,l=${R_5}$] (7,1) to (7,2.5)
(4.5,1) to[resistor,l_=$R_3$] (4.5,4)
;
\draw[->,thick] (4,2) -- (4,3);
\end{circuitikz}}
\end{resizebox}
\end{wrapfigure}
Five resistors are connected to a current source. Three of them have a resistance of $\SI{1}{\ohm}$, the other two have a same but an unknown resistance. The voltage of the current source is $U_0=\SI{14}{V}$ and its current is $I_0=\SI{10}{A}$. The voltage and the current strength on the third resistor are respectively $U_3=\SI{2}{V}$ and $I_3=\SI{2}{A}$. The direction of the current in the resistor $R_3$ is shown in the drawing. Determine all the resistances of the resistors.

\hinteng
To find the resistances all the combinations of the $\SI{1}{\ohm}$ resistors should be looked through and the wrong configurations should be systematically removed.

\solueng
\begin{wrapfigure}[9]{r}{0.4\linewidth}
    \vspace{-8pt}
	\begin{resizebox}{\linewidth}{!}{
		\begin{circuitikz}
			\draw
				(8,0) to[battery1] (1,0) to (1,2.5) to (2,2.5) to[short, *-] (2,4) to[resistor,l=${R_1=\SI{2}{\ohm}}$, -*] (4.5,4) to[resistor,l=${R_4=\SI{1}{\ohm}}$] (7,4) to[short, -*] (7,2.5) to (8,2.5) to (8,0)
				(2,2.5) to (2,1) to[resistor,l=${R_2=\SI{1}{\ohm}}$, -*] (4.5,1) to[resistor,l=${R_5=\SI{2}{\ohm}}$] (7,1) to (7,2.5)
				(4.5,1) to[resistor,l_=${R_3=\SI{1}{\ohm}}$] (4.5,4)
				;
		\end{circuitikz}}
	\end{resizebox}
\end{wrapfigure}
The electric current can go through the diagram by three ways: through the upper branch (resistors $R_1$ and $R_4$), through the middle branch (resistors $R_2$, $R_3$ and $R_4$) and through the bottom branch (resistors $R_2$ and $R_5$). The voltage drop on each resistor is equal to the product of its resistance and the current strength going through it and the sum of voltage drops on all the paths has to be equal to the voltage on the current source. Thus, $I_1R_1+I_4R_4=I_2R_2+I_3R_3+I_4R_4=I_2R_2+I_5R_5=U_0$. From these equations we can derive the expressions $I_1R_1=I_2R_2+I_3R_3$ and $I_3R_3+I_4R_4=I_5R_5$, which describe voltage drops on the left and right side of the diagram. In addition we can write down equations for current strengths: $I_1=I_0-I_2$, $I_4=I_1+I_3$ and $I_5=I_2-I_3$.\\
Let us notice that if $R_1=R_2$ and $R_4=R_5$ then by symmetry considerations $R_3=0$ should apply.\\
It is clear that $R_3=\SI{1}{\ohm}$. Let us first suppose that $R_1=R_4=R_3$. In this case the voltage drop in the upper branch is expressed as $(2I_1+I_3)R_3=U_0$, from which follows that $I_1=\SI{6}{A}$ and $I_2=\SI{4}{A}$. Observing the path of the current through the middle branch we can write down that $I_2R_2+U_3+(I_1+I_3)R_3=U_0$ from which follows that $R_2=R_5=\SI{1}{\ohm}$. Such diagram satisfies the symmetry mentioned before, therefore the assumption made is not true. Let us now suppose that $R_2=R_5=R_3$. For the bottom branch of the diagram we can write $(2I_2-I_3)R_3=U_0$ meaning $I_2=\SI{8}{A}$ and $I_1=\SI{2}{A}$. Voltage drops on the left side of the diagram must satisfy the equation $I_1R_1=(I_2+I_3)R_3$ from which follows that $R_1=\SI{5}{\ohm}$. In this case, however, the voltage drop on the upper branch of the scheme is $(2I_1+I_3)R_1=\SI{30}{V}\not=U_0$. Let us now suppose that $R_1=R_5=R_3$. The voltage drop on the middle branch of the scheme is $I_2R_2+U_3+(I_0-I_2+I_3)R_2=U_3+(I_0+I_3)R_2=U_0$ meaning $R_2=R_4=\SI{1}{\ohm}$. Now we are dealing with a situation already looked at.\\
The only possibility left is that $R_2=R_4=R_3$. From the left side of the diagram we get that $(I_0-I_2)R_1=I_2R_3+U_3$ and from the right side $(I_2-I_3)R_1=U_3+(I_0-I_2+I_3)R_3$. Based on these $(I_0-I_3)R_1=2U_3+(I_0+I_3)R_3$ meaning $R_1=\SI{2}{\ohm}$. It is easy to make sure that the values found satisfy all the equations for voltage drops.
\probend