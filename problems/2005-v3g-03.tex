\ylDisplay{Balloon} % Ülesande nimi
{Jaan Susi} % Autor
{lõppvoor} % Voor
{2005} % Aasta
{G 3} % Ülesande nr.
{1} % Raskustase
{
% Teema: Termodünaamika
\ifStatement
Suletud balloon ruumalaga $V = \SI{10}{l}$ oli täidetud veega temperatuuril $t_0 = \SI{0}{\celsius}$. Samal temperatuuril külmutati vesi jääks, mille tulemusena ballooni kest venis välja ja vesi avaldas kogu jäätumise protsessi käigus balloonile rõhku $p = \SI{5e7}{Pa}$.

Leida balloonis olnud vee ($\mathrm{H_2O}$) siseenergia muut koos märgiga. Jää tihedus $\rho_j = \SI{900}{kg/m^3}$ ja sulamissoojuseks antud rõhul $\lambda = \SI{317}{kJ/kg}$. Jää ja vee kokkusurutavust mitte arvestada. 
\fi


\ifHint
Termodünaamika I seaduse kohaselt $\Delta U = Q - A$, kus $Q$ on süsteemi antud soojushulk ning $A$ on välisjõudude vastu tehtud töö. Antud üesande kontekstis on $Q$ negatiivne ja $A$ positiivne.
\fi


\ifSolution
Termodünaamika I seaduse kohaselt $\Delta U = Q - A$, kus $Q$ on süsteemi antud soojushulk ning $A$ on välisjõudude vastu tehtud töö. Antud ülesande kontekstis on $Q$ negatiivne ja $A$ positiivne ning $A = p\Delta V$ ja $Q = -\lambda \rho_vV$. Niisiis,
\[
\Delta U=-\lambda \rho_{v} V-p\left(\frac{V \rho_{v}}{\rho_{j}}-V\right)=-\rho_{v} V\left[\lambda+p\left(\rho_{j}^{-1}-\rho_{v}^{-1}\right)\right].
\]
Paneme tähele, et avaldis nurksulgudes peaks kujutama endast sulamissoojust normaaltingimustel (sest vee võib viia samasse lõppolekusse ka teisel viisil --- muutes ta jääks normaaltingimustel ning seejärel viies rõhu etteantud väärtuseni; et jää loeme kokkusurumatuks, siis rõhu tõstmisel tööd ei tehta). Paistab, et tegemist pole siiski päris hariliku veega, sest
\[
\lambda+p\left(\rho_{j}^{-1}-\rho_{v}^{-1}\right)= \SI{323}{kJ/kg} \neq \lambda_{0}=\SI{334}{kJ/kg}.
\]
Arvandmete asendamisel leiame $\Delta U = \SI{-3,23}{MJ}$.
\fi
}