\ylDisplay{Kelk} % Ülesande nimi
{Andreas Valdmann} % Autor
{piirkonnavoor} % Voor
{2018} % Aasta
{G 9} % Ülesande nr.
{6} % Raskustase
{
% Teema: Staatika
\ifStatement
Juku läks sõpradega kelgutama. Teel tagasi istusid Juku kaks sõpra kelgule ja Juku üritas kelku horisontaalsel lumisel teel enda järel vedada. Kui suur on minimaalne kelgunööri nurk maapinnaga, mille korral on Jukul võimalik kelk liikuma tõmmata? Juku mass $m_1 = \SI{60}{kg}$ ja hõõrdetegur Juku saabaste ning lume vahel $\mu_1 = \SI{0.30}{}$. Kelgu mass koos Juku sõpradega $m_2 = \SI{110}{kg}$ ja hõõrdetegur kelgu ning lume vahel $\mu_2 = \SI{0.20}{}$.
\fi


\ifHint
Mõistlik on nööri pinge jaotada horisontaalseks ja vertikaalseks komponendiks $T_x$ ja $T_y$. Sellisel juhul kasvab Juku saabastele mõjuv toereaktsioon $T_y$ võrra ning kelgule mõjuv toereaktsioon kahaneb $T_y$ võrra. Minimaalse kelgunööri nurga puhul on Jukust tulenev hõõrdejõud võrdne kelgu hõõrdejõuga.
\fi


\ifSolution
Esiteks näeme, et kui Juku tõmbaks kelgunööri horisontaalselt, siis ei hakkaks kelk liikuma ükskõik kui suure tõmbejõu korral, sest Jukule mõjuv hõõrdejõud on väiksem kui kelgu liikumapanemiseks vajalik jõud:
\[
\mu_1 m_1 g < \mu_2 m_2 g.
\]
Esimesena hakkavad libisema hoopis Juku tallad.

Kui Juku tõmbab nööri teatud nurga all ülespoole, siis tekib nööris vertikaalne jõu komponent $F_{\mathrm{v}}$, mis tõstab kelku ülespoole ja surub Jukut allapoole. Seega mõjub Jukule tema libisemise piiril hõõrdejõud $F_{\mathrm{h}1} = \mu_1 \left(m_1 g + F_{\mathrm{v}}\right)$ ja kelgule tema libisemise piiril hõõrdejõud $F_{\mathrm{h}2} = \mu_2 \left(m_2 g - F_{\mathrm{v}}\right)$. Kuna küsiti minimaalset nurka, siis peab kelk olema libisemise piiri napilt ületanud ja Juku sellele napilt alla jääma ehk piirjuhul $F_{\mathrm{h}1} = F_{\mathrm{h}2} = F_{\mathrm{h}}$, kus $F_{\mathrm{h}}$ tähistab nööris tekkiva jõu horisontaalset komponenti.
Jõudude tasakaalu võrrandist
\begin{equation*}
\mu_1 \left(m_1 g + F_{\mathrm{v}}\right) = \mu_2 \left(m_2 g - F_{\mathrm{v}}\right)
\end{equation*}
saame avaldada nööris tekkiva jõu vertikaalse komponendi:
\begin{equation*}
\mu_1 m_1 g + \mu_1 F_{\mathrm{v}} = \mu_2 m_2 g - \mu_2 F_{\mathrm{v}},
\end{equation*}
\begin{equation*}
F_{\mathrm{v}} \left(\mu_1 + \mu_2\right) = g\left(\mu_2 m_2 - \mu_1 m_1\right),
\end{equation*}
\begin{equation*}
F_{\mathrm{v}} = g\frac{\mu_2 m_2 - \mu_1 m_1}{\mu_1 + \mu_2}.
\end{equation*}

Jõu horisontaalkomponendi leidmiseks asendame saadud tulemuse näiteks Jukule mõjuva hõõrdejõu võrrandisse
\begin{equation*}
F_{\mathrm{h}} = \mu_1 \left(m_1 g + g\frac{\mu_2 m_2 - \mu_1 m_1}{\mu_1 + \mu_2}\right)
\end{equation*}
ja avaldame:
\begin{align*}
F_{\mathrm{h}} &= g \mu_1 \frac{m_1 \left(\mu_1 + \mu_2\right) + \mu_2 m_2 - \mu_1 m_1}{\mu_1 + \mu_2} = \\
&= g \mu_1 \frac{\mu_2 m_1 + \mu_2 m_2}{\mu_1 + \mu_2} = g\mu_1\mu_2\frac{m_1+m_2}{\mu_1+\mu_2}.
\end{align*}

Nurk kelgunööri ja maapinna vahel on
\begin{equation*}
\alpha = \arctan\left(\frac{F_{\mathrm{v}}}{F_{\mathrm{h}}}\right) = \arctan\left(\frac{\mu_2 m_2 - \mu_1 m_1}{\mu_1 \mu_2 \left(m_1 + m_2\right)}\right) = \SI{21}{\degree}.
\end{equation*}
\fi


\ifEngStatement
% Problem name: Sledge
Juku went sledging with his friends. When going back, two of Juku’s friends sat on his sledge and Juku tried to pull the sledge after him on a horizontal and snowy road. What is the minimal angle between the rope of the sledge and the ground so that it is possible for Juku to pull the sledge into motion? Juku’s mass is $m_1 = \SI{60}{kg}$ and the coefficient of friction between Juku’s boots and the snow is $\mu_1 = \SI{0.30}{}$. The mass of the sledge together with Juku’s friends is $m_2 = \SI{110}{kg}$ and the coefficient of friction between the sledge and the snow is $\mu_2 = \SI{0.20}{}$.
\fi


\ifEngHint
It is convenient to divide the rope’s tension into a horizontal and a vertical component $T_x$ and $T_y$. In this case the normal force applied to Juku’s boots increases by $T_y$ and the normal force applied to the sledge decreases by $T_y$. For the minimal angle of the sledge’s rope the friction coming from Juku has to be equal to the sledge’s friction.
\fi


\ifEngSolution
Firstly we see that if Juku pulled the rope of the sledge horizontally then no matter how big the pulling force is the sledge would not start moving because the friction applied to Juku is smaller from the force that is needed for the sledge to start moving:
\[
\mu_1 m_1 g < \mu_2 m_2 g.
\]
Juku’s soles are actually the first to start moving.\\
If Juku pulls the rope at a certain angle upwards then a vertical force component $F_{\mathrm{v}}$ is created in the rope, this force pulls the sledge upwards and presses Juku downwards. Therefore on the verge of sliding Juku is affected by a friction $F_{\mathrm{h}1} = \mu_1 \left(m_1 g + F_{\mathrm{v}}\right)$ and the sledge is affected by a friction $F_{\mathrm{h}2} = \mu_2 \left(m_2 g - F_{\mathrm{v}}\right)$ when it is on the verge of sliding. Because the minimal angle was asked then the sledge has to have barely exceeded the sliding limit and Juku has to barely stay below it, meaning in the limit case $F_{\mathrm{h}1} = F_{\mathrm{h}2} = F_{\mathrm{h}}$ where $F_{\mathrm{h}}$ marks the horizontal component of the force occurring in the rope. From the force balance equation
\begin{equation*}
\mu_1 \left(m_1 g + F_{\mathrm{v}}\right) = \mu_2 \left(m_2 g - F_{\mathrm{v}}\right)
\end{equation*}
we can express the vertical component of the force occurring in the rope:
\begin{equation*}
\mu_1 m_1 g + \mu_1 F_{\mathrm{v}} = \mu_2 m_2 g - \mu_2 F_{\mathrm{v}},
\end{equation*}
\begin{equation*}
F_{\mathrm{v}} \left(\mu_1 + \mu_2\right) = g\left(\mu_2 m_2 - \mu_1 m_1\right),
\end{equation*}
\begin{equation*}
F_{\mathrm{v}} = g\frac{\mu_2 m_2 - \mu_1 m_1}{\mu_1 + \mu_2}.
\end{equation*}
To find the force’s horizontal component we can replace the gotten result in the equation of the friction applied to Juku
\begin{equation*}
F_{\mathrm{h}} = \mu_1 \left(m_1 g + g\frac{\mu_2 m_2 - \mu_1 m_1}{\mu_1 + \mu_2}\right)
\end{equation*}
and express
\begin{align*}
F_{\mathrm{h}} &= g \mu_1 \frac{m_1 \left(\mu_1 + \mu_2\right) + \mu_2 m_2 - \mu_1 m_1}{\mu_1 + \mu_2} = \\
&= g \mu_1 \frac{\mu_2 m_1 + \mu_2 m_2}{\mu_1 + \mu_2} = g\mu_1\mu_2\frac{m_1+m_2}{\mu_1+\mu_2}.
\end{align*}
The angle between the sledge’s rope and the ground is
\begin{equation*}
\alpha = \arctan\left(\frac{F_{\mathrm{v}}}{F_{\mathrm{h}}}\right) = \arctan\left(\frac{\mu_2 m_2 - \mu_1 m_1}{\mu_1 \mu_2 \left(m_1 + m_2\right)}\right) = \SI{21}{\degree}.
\end{equation*}
\fi
}