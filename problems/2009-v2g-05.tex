\setAuthor{Mihkel Heidelberg}
\setRound{piirkonnavoor}
\setYear{2009}
\setNumber{G 5}
\setDifficulty{3}
\setTopic{Dünaamika}

\prob{Auto}
Auto kiirendab nii, et rattad libisevad. Hetkel on auto kiirus stabiilselt $v$, vedavate rataste nurkkiirus $\omega$ ja raadius $r$. Kui oletada, et mootori võimsus läheb ainult auto liikumisse ja vedavate rataste libisemisse, siis kui suur on kasutegur?

\hint
Nii auto liikumisse kui ka mootori tööse minevad võimsused on avaldatavad rataste ja maa vahelise hõõrdejõu kaudu.

\solu
Vedavate rataste ja maa vahel mõjub mingi horisontaalsihiline hõõrdejõud $F$. Jõust ja kiirusest saame auto liikumisse mineva võimsuse $N_l = Fv$. Ratastele mõjub jõumoment $Fr$, mille tõttu mootori võimsus $N_{m}=\omega F r$.

Kasutegur on niisiis
\[
\nu=\frac{N_{l}}{N_{m}}=\frac{F v}{F \omega r}=\frac{v}{\omega r}.
\]
\probend