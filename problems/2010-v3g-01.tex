\ylDisplay{Sild} % Ülesande nimi
{Valter Kiisk} % Autor
{lõppvoor} % Voor
{2010} % Aasta
{G 1} % Ülesande nr.
{2} % Raskustase
{
% Teema: Dünaamika
\ifStatement
Risti üle $l=\SI{100}{m}$ laiuse jõe kulgeb kumer sild, mille keskel on
autotee $h=\SI{5}{m}$ võrra kõrgemal kaldapealsest tasemest. Silla profiiliks on
ringjoone kaar. Auto massiga $m=\SI{1000}{kg}$ ületab silla muutumatu kiirusega $v=\SI{60}{km/h}$.
Kui suure jõuga rõhub auto silla keskkohta? Kui suure kiiruse juures hakkab kaduma
kontakt rataste ja tee vahel?
\fi


\ifHint
Silla kõverusraadius on leitav Pythagorase teoreemist. Autole mõjub silla peal kaks jõudu: raskusjõud ja rõhumisjõud. Antud jõudude resultant annab kesktõmbekiirenduse.
\fi


\ifSolution
Olgu
silla kõverusraadius $r$. Pythagorase teoreemist
\[
r^2=(l/2)^2+(r-h)^2\implies 0=l^2/4-2rh+h^2.
\]
Kuna $h\ll l$ ja seega $h\ll r$, siis $h^2$ võib ära jätta ja $r=l^2/8h=\SI{250}{m}$.
Auto raskusjõu $mg$ ja toereaktsiooni $N$ resultant annab kesktõmbekiirenduse $v^2/r$.
Seega $N=mg-mv^2/r\approx\SI{8700}{N}$.

Kontakt rataste ja maapinna vahel hakkab kaduma, kui $N=0$. Seega $v=\sqrt{gr}\approx
\SI{180}{km/h}$.
\fi
}