\setAuthor{Tundmatu autor}
\setRound{lõppvoor}
\setYear{2004}
\setNumber{G 10}
\setDifficulty{1}
\setTopic{Teema}

\prob{Elektriväli}
Kaks metallkuuli raadiusega $r$ ja massiga $m$ on joodetud kerge metallvarda (pikkus $l$) otste külge. Süsteem asub kaalutuse tingimustes homogeenses elektriväljas tugevusega $E$. Kui suur on varrast pingestav jōud $F$ ja süsteemi väikeste võnkumiste periood $T$? Võite lugeda, et $l \gg r $.

\hint

\solu
Kuulidele indutseeritakse sellised laengud $\pm q$, et kuulide vaheline pinge oleks null,
$$
U=E l-2 k q\left(r^{-1}-l^{-1}\right)=0 \quad \Rightarrow \quad q \approx \frac{E l r}{2 k}
$$
varrast pingestav jõud on
$$
F=q E=\frac{E^{2} l r}{2 k}.
$$
Väikeste võnkumiste puhul on tegemist matemaatilise pendliga, kus raskusjõu asemel on elektrivälja jõud $F$ ning pendli pikkuseks $l / 2$ (võnkumine toimub massikeskme ümber). Seega
$$
T=2 \pi \sqrt{\frac{m l}{2 F}}=\frac{2 \pi}{E} \sqrt{\frac{m k}{r}}.
$$
\probend