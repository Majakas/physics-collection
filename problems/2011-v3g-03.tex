\ylDisplay{Tormituul} % Ülesande nimi
{Mihkel Kree} % Autor
{lõppvoor} % Voor
{2011} % Aasta
{G 3} % Ülesande nr.
{4} % Raskustase
{
% Teema: Staatika
\ifStatement
Vaatleme tugeva külgtuule kätte jäänud veoautot lihtsustatult homogeense risttahukana. Auto laius on $a = \SI{2}{m}$, kõrgus $b = \SI{3}{m}$, pikkus
$c = \SI{5}{m}$. Missugune peaks olema hõõrdetegur rataste ja maapinna vahel, et piisavalt tugev külgtuul saaks auto tuulepoolsed rattad maast lahti kergitada?
\fi


\ifHint
Kriitilise hõõrdeteguri väärtuse korral kehtib jõumomentide tasakaal tuulepoolsete rataste telje suhtes.
\fi


\ifSolution
Tuule poolt avaldatav horistonaalsuunaline jõud $F$ peab olema niisugune, et selle poolt tekitatud jõumoment $Fb/2$ ületab raskusjõu poolt tekitatud jõumomendi $Mga/2$. Jõumomentide võrdsuse korral $F=Mga/b$. Et niisugune jõud autot libisema ei paneks, peab hõõrdejõud $F_h=\mu Mg$ selle tasakaalustama, millest saame nõutud tingimuseks: $\mu > a/b = 2/3$.
\fi
}