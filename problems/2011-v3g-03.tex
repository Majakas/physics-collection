\setAuthor{Mihkel Kree}
\setRound{lõppvoor}
\setYear{2011}
\setNumber{G 3}
\setDifficulty{4}
\setTopic{Staatika}

\prob{Tormituul}
Vaatleme tugeva külgtuule kätte jäänud veoautot lihtsustatult homogeense risttahukana. Auto laius on $a = \SI{2}{m}$, kõrgus $b = \SI{3}{m}$, pikkus
$c = \SI{5}{m}$. Missugune peaks olema hõõrdetegur rataste ja maapinna vahel, et piisavalt tugev külgtuul suudaks auto tuulepoolsed rattad maast lahti kergitada?

\hint
Kriitilise hõõrdeteguri väärtuse korral kehtib jõumomentide tasakaal tuulepoolsete rataste telje suhtes.

\solu
Tuule poolt avaldatav horistonaalsuunaline jõud $F$ peab olema niisugune, et selle poolt tekitatud jõumoment $Fb/2$ ületab raskusjõu poolt tekitatud jõumomendi $Mga/2$. Jõumomentide võrdsuse korral $F=Mga/b$. Et niisugune jõud autot libisema ei paneks, peab hõõrdejõud $F_h=\mu Mg$ selle tasakaalustama, millest saame nõutud tingimuseks: $\mu > a/b = 2/3$.
\probend