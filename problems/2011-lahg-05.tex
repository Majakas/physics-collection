\setAuthor{Tundmatu autor}
\setRound{lahtine}
\setYear{2011}
\setNumber{G 5}
\setDifficulty{5}
\setTopic{Gaasid}

\prob{Süstal}
Kord sooritas noor füüsik eksperimendi, et leida süstlakolvile mõjuvat
hõõrdejõudu. Ta tõmbas $V_{0}=\SI{10}{ml}$ mahuga süstlasse \SI{5,0}{ml} õhku ja sulges siis
süstla otsa sõrmega. Seejärel tõmbas ta süstla kolvi näiduni
$V_{1}=\SI{9,2}{ml}$ ja
lasi sellel seejärel aeglaselt tagasi liikuda. Kolb liikus, kuni näiduks jäi $V_{2}=\SI{5,8}{ml}$.
Mõõtmisel selgus, et süstlakolvi sisediameeter oli $d=\SI{9}{mm}$ ja kraadiklaas
näitas, et ruumis oli $t=\SI{27}{\degreeCelsius}$, õhu suhteline niiskus $R=30\%$ ja õhurõhk
$p_{0}=\SI{103,6}{kPa}$. Milline oli süstlakolvile mõjuv hõõrdejõud?\\
\textit{Märkus.} Kuna tegu on praktilise probleemiga, siis ei pruugi kõik
algandmed vajalikud olla.

\hint
Kolvi lõppasendis tasakaalustab kolvile mõjuv hõõrdejõud rõhkude vahest tekitatud rõhumisjõu.

\solu
Kirjutame ideaalse gaasi olekuvõrrandi süstla jaoks vahetult peale sõrmega sulgemist:
\[p_{0}V=nRT.\]
Peale kolvi välja tõmbamist ja vabastamist:
\[p_{2}V_{2}=nRT \Rightarrow p_{2}=p_{0}\frac{V}{V_{2}}. \]
Kui kolb (ristlõikepindalaga $S=\frac{\pi d^{2}}{4}$)peale vabastamist seiskub, siis on kolvi hõõrdejõud $F_{h}$ tasakaalustanud rõhkude vahest tekitatud jõu:
\[F_{h}=S(p_{0}-p_{2})=Sp_{0} \left(1-\frac{V}{V_{2}}\right)=\frac{\pi
	d^{2}}{4}p_{0} \left(1-\frac{V}{V_{2}}\right) \approx \SI{3,6}{N} \]
\probend