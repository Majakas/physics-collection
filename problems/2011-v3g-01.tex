\setAuthor{Taavi Pungas}
\setRound{lõppvoor}
\setYear{2011}
\setNumber{G 1}
\setDifficulty{2}
\setTopic{Termodünaamika}

\prob{Vesi}
Avatud termoses on vesi temperatuuril $t_0 = \SI{100}{\degreeCelsius}$. Sellest \SI{1}{\%}
aurustub. Hinnata, kui palju muutub termosesse jäänud vee temperatuur $t$.
Vee erisoojus $c_v = \SI{4,2}{kJ.kg^{-1}.K^{-1}}$, veeauru erisoojus $c_a = \SI{1,9}{kJ.kg^{-1}.K^{-1}}$ ning
vee aurustumissoojus temperatuuril \SI{100}{\degreeCelsius} on $L = \SI{2,26}{MJ/kg}$. Eeldada, et termose seinte kaudu soojuskadusid ei ole.

\hint
Vee aurustumise käigus kehtib energia jäävuse seadus. Seega tuleb väikese veekoguse aurustumiseks vajalik soojushulk järelejäänud vee temperatuuri langemise arvelt.

\solu
Energia jäävusest teame, et väikese koguse vee aurustumiseks kuluv soojushulk tuleb järelejäänud vee temperatuuri langemise arvelt.

Kuigi aurustumise alghetkel tekib veeaur temperatuuriga \SI{100}{\degreeCelsius}, on hiljem nii vee
kui tekkiva veeauru temperatuur veidi madalam. Uuel temperatuuril aga ei ole enam
väikese koguse vee aurustumiseks kuluv soojushulk otseselt arvutatav vee aurustumissoojusest temperatuuril \SI{100}{\degreeCelsius} (ülesandes antud $L$).

Seega teeme lihtsustuse, et vee aurustumissoojus on selles temperatuurivahemikus kogu aeg $L$. Olgu esialgselt termoses oleva vee mass $m$. Saame $\num{0,01}mL = \num{0,99}mc_v\Delta t$, mis annab vastuseks
\[
\Delta t=\frac{1}{99} \frac{L}{c_{v}}=\SI{5,4}{\degreeCelsius}.
\]
\probend