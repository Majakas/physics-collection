\ylDisplay{Soojusvaheti} % Ülesande nimi
{Kaur Aare Saar} % Autor
{lõppvoor} % Voor
{2016} % Aasta
{G 1} % Ülesande nr.
{2} % Raskustase
{
% Teema: Termodünaamika
\ifStatement
Tagasivoolu soojusvahetis jahutatakse sissetulevat naftat temperatuuriga $T_{n}=\SI{90}{\celsius}$ temperatuurini $T_{0}=\SI{20}{\celsius}$. 
Jahutusvesi liigub soojusvahetis vastupidises suunas naftaga ja siseneb soojusvahetisse temperatuuriga $T_{v}=\SI{10}{\celsius}$.
Vesi liigub kiirusega $v_{v}=\SI{6}{\m^3\per\minute}$ 
ja nafta liigub kiirusega $v_{n}=\SI{15}{\m^3\per\minute}$. 
Leidke, millise temperatuuriga väljub soojusvahetist vesi? Vee erisoojus $c_{v}=\SI{4200}{\joule\per\kilogram\per\celsius}$ 
ja nafta erisoojus $c_{n}=\SI{1800}{\joule\per\kilogram\per\celsius}$. Vee tihedus $\rho_{v}=\SI{1000}{\kilogram\per\metre\cubed}$ ja nafta tihedus $\rho_{n}=\SI{850}{\kilogram\per\metre\cubed}$.
\fi


\ifHint
Nafta jahtumisel eraldunud soojus kulub vee soojendamiseks.
\fi


\ifSolution
Nafta jahtumisel eraldunud soojus kulub vee soojendamiseks: $Q\idx{nafta} = Q\idx{vesi},$
\[ m_nc_n\Delta t_n = m_vc_v\Delta t_v \quad\Rightarrow \]
\[ \rho_nv_nc_n\Delta t_n = \rho_v v_vc_v\Delta t_v \quad\Rightarrow\quad \Delta t_v = \frac{\rho_nv_nc_n\Delta t_n}{\rho_vv_vc_v} \approx \SI{64}{\celsius}. \]
Seega väljub vesi soojusvahetist temperatuuril $T = \SI{64}{\celsius} + \SI{10}{\celsius} = \SI{74}{\celsius}$.
\fi


\ifEngStatement
% Problem name: Heat exchanger
In a backflow heat exchanger incoming oil of temperature $T_{o}=\SI{90}{\celsius}$ is cooled to a temperature $T_{0}=\SI{20}{\celsius}$. The cooling water in the heat exchanger moves to the opposite direction with respect to the oil and enters the heat exchanger with a temperature $T_{v}=\SI{10}{\celsius}$. The water moves with a speed $w_{v}=\SI{6}{\m^3\per\minute}$ and the oil with a speed $v_{o}=\SI{15}{\m^3\per\minute}$. At what temperature does the water exit the heat exchanger? The specific heat of water is $c_{w}=\SI{4200}{\joule\per\kilogram\per\celsius}$, the specific heat of oil $c_{o}=\SI{1800}{\joule\per\kilogram\per\celsius}$. The density of water $\rho_{w}=\SI{1000}{\kilogram\per\metre\cubed}$ and the density of oil $\rho_{o}=\SI{850}{\kilogram\per\metre\cubed}$.
\fi


\ifEngHint
The heat released due to the oil cooling down goes to the heating of water.
\fi


\ifEngSolution
The heat dissipated during the oil's cooling goes to the heating of water: $Q\idx{oil} = Q\idx{water}$,
\[ m_oc_o\Delta t_o = m_wc_w\Delta t_w \quad\Rightarrow \]
\[ \rho_ov_oc_o\Delta t_o = \rho_w v_wc_w\Delta t_w \quad\Rightarrow\quad \Delta t_w = \frac{\rho_ov_oc_o\Delta t_o}{\rho_wv_wc_w} \approx \SI{64}{\celsius}.  \]
Therefore the water leaves from the heat exchanger at a temperature $T = \SI{64}{\celsius} + \SI{10}{\celsius} = \SI{74}{\celsius}$.
\fi
}