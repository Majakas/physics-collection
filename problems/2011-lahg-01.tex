\ylDisplay{Segadus optikalaboris} % Ülesande nimi
{Roland Matt} % Autor
{lahtine} % Voor
{2011} % Aasta
{G 1} % Ülesande nr.
{2} % Raskustase
{
% Teema: Geomeetriline-optika
\ifStatement
Optik koristas vana laborit ja leidis sealt ühe markeerimata nõgusläätse ja ühe
kumerläätse. Nende optiliste tugevuste määramamiseks paigutas ta läätsed
üksteise taha ja lasi neist läbi kaks paralleelset laserkiirt, mille vahekaugus
oli $x_{1}=\SI{1,0}{cm}$. Ta muutis läätsede vahekaugust seni, kui süsteemist väljunud
kiired olid jällegi paralleelsed (seda kontrollis ta, määrates nende vahekaugust
paberilehekesega erinevatel kaugustel), nüüd oli kiirte vahekauguseks
$x_{2}=\SI{26}{mm}$. Sellises olukorras oli läätsedevaheliseks kauguseks 
$d=\SI{32}{cm}$.
Millise optilise tugevusega olid läätsed?
\fi


\ifHint
Selleks, et kiirtekimp laieneks ja jääks paralleelseks, pidi optik paigutama
nõgusläätse kumerläätse ette niimoodi, et läätsede fookused ühtiksid nõgusläätse
ees. Fookuste ühtimises saab veenduda, kasutades läätse valemit ning arvestades, et paralleelse kiirtekimbu kujutis asub lõpmatuses.
\fi


\ifSolution
Selleks, et kiirtekimp laieneks ja jääks paralleelseks, pidi optik paigutama
nõgusläätse kumerläätse ette niimoodi, et läätsede fookused ühtiksid nõgusläätse
ees. Tähistades nõgusläätse fookuskauguse $f_{n}$ ja vastavalt kumerläätsel
$f_{k}$, saame kirja panna $d=f_{k}-f_{n}$. Vaatame üht kiirt, mis langeb
nõgusläätsele paralleelselt läätsede optiliste peatelgedega, läätse keskpunktist
kaugusel $x_{1}$. Teine kiir liikugu lihtsuse mõttes mööda süsteemi optilist
peatelge. See kiir läbib kumerläätse tema keskpunktist kaugusel $x_{2}$.
Tekkivatest sarnastest kolmnurkadest saame kirja panna
\[ \frac{x_{1}}{f_{n}}=\frac{x_{2}}{f_{k}}=\frac{x_{2}}{f_{k}-d}, \] millest
\[ f_{k}=\frac{x_{2}d}{x_{2}-x{1}}=\SI{52}{cm},\quad f_{n}=f_{k}-d=\SI{20}{cm}.\]
Optilised tugevused saame, võttes fookuskauguste pöördväärtused:
$D_{k}\approx\SI{1,9}{dptr}$ ja $D_{n}=\SI{5}{dptr}$.
\fi


\ifEngStatement
% Problem name: Disorder in an optics laboratory
An optician was cleaning an old laboratory and found an unmarked concave lens and a convex lens. To determine their optical powers he positioned the lenses in a straight line and directed two parallel laser rays through them, the distance between the rays was $x_{1}=\SI{1,0}{cm}$. He changed the distance between the lenses until the rays leaving the system were also parallel (he controlled that by determining the distance between them with a sheet of paper at different lengths), now the distance between the rays was $x_{2}=\SI{26}{mm}$. In that situation the distance between the lenses was $d=\SI{32}{cm}$. What was the optical power of the lenses?
\fi


\ifEngHint
For the beam of light to widen and stay parallel, the optic had to have placed the concave lens in front of the convex lens so that the focal points of the lenses would coincide in front of the concave lens. One can make sure of the coinciding by using the thin lens formula and taking into account that the image of the parallel beam of light is located at infinity.
\fi


\ifEngSolution
For the rays to widen and stay parallel the optician had to place the concave lens in front of the convex lens so that the focal points of the lenses would coincide in front of the concave lens. Marking the focal point of the concave lens as $f_{n}$ and for the convex lens as $f_{k}$ we can write down that $d=f_{k}-f_{n}$. Let us observe one ray that falls on the concave lens parallel to the optical axis of the lenses, at a distance $x_{1}$ from the center of the lens. Let the other ray move along the optical axis of the system. This ray goes through the convex lens at a distance $x_{2}$ from its center. From the similar triangles that appear we can write down
\[ \frac{x_{1}}{f_{n}}=\frac{x_{2}}{f_{k}}=\frac{x_{2}}{f_{k}-d}, \] 
where
\[ f_{k}=\frac{x_{2}d}{x_{2}-x{1}}=\SI{52}{cm},\quad  f_{n}=f_{k}-d=\SI{20}{cm}.\] 
Finding the reverse values of the focal lengths we get the optical powers to be: $D_{k}\approx\SI{1,9}{dptr}$ and $D_{n}=\SI{5}{dptr}$.
\fi
}