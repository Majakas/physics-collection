\ylDisplay{Mullitaja} % Ülesande nimi
{Jaak Kikas} % Autor
{lõppvoor} % Voor
{2005} % Aasta
{G 7} % Ülesande nr.
{5} % Raskustase
{
% Teema: Dünaamika
\ifStatement
Veekogu põhjas asub mullitaja --- õhuballoon väikese avausega, millest võrdsete ajavahemike $\Delta t = \SI{1}{s}$ järel väljuvad õhumullid raadiusega $R = \SI{0,3}{mm}$. Taolise mullikese liikumisel vees mõjub sellele takistusjõud $F = 6\pi \eta Rv$, kus $\eta$ on vedeliku voolamistakistust iseloomustav tegur (vedeliku viskoossus, vee korral on selle suuruse väärtuseks \SI{1e-3}{N.s/m^2} ) ja $v$ on mullikese kiirus. Võite lugeda, et mullikese liikumine toimub kogu aeg kiirusega, mis on määratud tingimusega, et kõigi talle mõjuvate jõudude resultant on null. Vee tihedus $\rho = \SI{1000}{kg/m^3}$, raskuskiirendus $g = \SI{9,8}{m/s^2}$, õhurõhk $p_0 = \SI{100}{kPa}$. Mitu korda muutub vahemaa naabermullikeste vahel tõusul põhjast pinnale, kui veekogu sügavus on $H = \SI{27}{m}$?
\fi


\ifHint
Arvestades mullide arvu jäävust on ruumiline vahemaa nende vahel võrdeline mullikeste kiirusega. Viimane on leitav võrdsustades takistus- ja üleslükkejõu ning arvestades mullikese ruumala muutust rõhu tõttu.
\fi


\ifSolution
Arvestades mullide arvu jäävust on ruumiline vahemaa nende vahel võrdeline mullikeste kiirusega. Viimase leiame võrdsustades takistus- ja üleslükkejõu ning arvestades mullikese ruumala muutust rõhu muutumisel sügavuse vähenemisel.

Mullikesele mõjuvad vee takistusjõud $F$ ja üleslükkejõud $F_A$. Nende võrdusest
\[
6\pi \eta Rv = g (\rho - \rho\idx{õhk}) V,
\]
kus $V$ on mullikese ruumala. Kuna ülesande tingimuste kohaselt $\rho \ll \rho\idx{õhk}$ ja tähistades indeksitega \enquote{H} ja \enquote{0} vastavalt situatsioone veekogu põhjas ja pinna lähedal, saame
\begin{align}
6\pi \eta R_Hv_H &= g\rho V_H, \label{2005-v3g-07:eq1}\\
6\pi \eta R_0v_0 &= g\rho V_0. \label{2005-v3g-07:eq2}
\end{align}
Võrrandeid \ref{2005-v3g-07:eq1} ja \ref{2005-v3g-07:eq2} omavahel jagades ning kasutades seost $V_i = (4/3)\pi R_i^3$, saame mullikeste vahekauguse suhte
\[
\frac{L_{0}}{L_{H}}=\frac{v_{0}}{v_{H}}=\frac{V_{0} R_{H}}{V_{H} R_{0}}=\left(\frac{V_{0}}{V_{H}}\right)^{2 / 3}.
\]
Kuna $V_0p_0 = V_Hp_H$ ja $p_H = p_0 + g\rho H$, siis
\[
\frac{L_{0}}{L_{H}}=\left(1+\frac{\rho g H}{p_{0}}\right)^{2 / 3}=\num{2,4},
\]
st mullikeste vahemaa suureneb \num{2,4} korda.
\fi
}