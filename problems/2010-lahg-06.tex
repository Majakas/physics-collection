\ylDisplay{Ioonmootor} % Ülesande nimi
{Mihkel Pajusalu} % Autor
{lahtine} % Voor
{2010} % Aasta
{G 6} % Ülesande nr.
{4} % Raskustase
{
% Teema: Elektrostaatika
\ifStatement
Kosmosesond on varustatud lihtsa ioonmootoriga, mis koosneb ksenooni
ioonide $\mathrm{Xe}^+$ allikast ja kahest elektroodist, mille vahele rakendatakse pinge
$U$ ja mille vahelist maad läbides ioone kiirendatakse. Kui suurt kogust
(mass) ksenooni on vaja, et selle mootoriga sondi kiirust tõsta
$\Delta v=\SI{1}{km/s}$ võrra?
%
Ksenooni aatommass $\mu=\SI{131,29}{g/mol}$, kosmosesondi mass $M=\SI{1000}{kg}$, kiirendav pinge
$U=\SI{100}{kV}$, elementaarlaeng $e=\SI{1,60 e-19}{C}$, Avogadro arv
$N_A= \SI{6,02 e23}{mol^{-1}}$.
\fi


\ifHint
Ioonide kiirus on leitav energia jäävusest. Sondi ja ioonide kiiruse sidumiseks on kõige mugavam rakendada impulsi jäävust.
\fi


\ifSolution
Ioonide kiiruse leiame energia jäävuse seadusest:
$$mu^2/2=Ue \Rightarrow u=\sqrt{2Ue/m},$$
kus $m=\mu/N_a$.
Impulsi jäävuse seadusest süsteemi ``laev+kiirendatud kütus'' jaoks saame (eeldusel, et $M\gg m_k$)
$$m_ku=Mv \Rightarrow m_k=Mv/u=Mv\sqrt{\mu/2N_AUe}=\SI{2,61}{kg}.$$
Näeme, et tehtud eeldus $m_k\ll M$ tõepoolest kehtib.
\fi
}