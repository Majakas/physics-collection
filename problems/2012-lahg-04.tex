\setAuthor{Taavi Pungas}
\setRound{lahtine}
\setYear{2012}
\setNumber{G 4}
\setDifficulty{3}
\setTopic{Geomeetriline optika}

\prob{Kärbes}
Kumerläätse optilisel peateljel, kaugusel $a$ läätsest, lendab kärbes.
Kärbse kiirus on $v$ ning tema suund on risti optilise peateljega.
Leidke kärbse kujutise kiirus (nii suund kui väärtus). Läätse
fookuskaugus on $f < a$.

\hint
Kärbse kiiruse leidmiseks on kasulik vaadelda, kuidas kärbse poolt aja $t$ jooksul läbitud lõik pikkusega $vt$ läbi läätse välja venib.

\solu
Kärbse trajektoor lühikese aja $t$ jooksul on sirgjoon pikkusega $h=vt$. Konstrueerime joonisel kärbse kujutise trajektoori. Sarnastest kolmnurkadest saame leida kujutise trajektoori pikkuse $h'$:
$$\frac{h}{a-f}=\frac{h'}{f} \Rightarrow h'=\frac{f}{a-f} h.$$
Kujutise kiirus on seega
$$v' = \frac{h'}{t} = \frac{f}{a-f} \frac{h}{t} = \frac{f}{a-f} v. $$
ning see on vastassuunaline kärbse kiirusega.

\probeng{Fly}
A fly is flying on the optical axis of a convex lens at a distance $a$ from the lens. The speed of the fly is $v$ and its direction is perpendicular to the optical axis. Find the velocity of the fly’s image (both the direction and value). The focal length of the lens is $f < a$.

\hinteng
To find the speed of the fly it is useful to observe how the section of length $vt$ traveled by the fly during the time $t$ stretches out through the lens.

\solueng
The trajectory of the fly during a short time $t$ is a line of length $h=vt$. In the figure we construct the trajectory of the fly’s image. From similar triangles we can find the length of the image’s trajectory $h'$:
$$\frac{h}{a-f}=\frac{h'}{f} \Rightarrow h'=\frac{f}{a-f} h.$$ 
The velocity of the image is therefore
$$v' = \frac{h'}{t} = \frac{f}{a-f} \frac{h}{t} = \frac{f}{a-f} v. $$ 
and it has the opposite direction to the fly’s velocity.
\probend