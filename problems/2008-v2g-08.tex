\setAuthor{Valter Kiisk}
\setRound{piirkonnavoor}
\setYear{2008}
\setNumber{G 8}
\setDifficulty{7}
\setTopic{Elektriahelad}

\prob{Aku laadimine}
Teatava akumulaatori elektromotoorjõud kasvab laadimise käigus nõnda, nagu kujutatud joonisel. Samas on toodud ka elektriskeem, mida Juku kavatseb kasutada sellise akumulaatori laadimiseks. Pingeallika klemmidel on pinge \SI{6}{V}. Nii pingeallika kui ka aku sisetakistust võib lugeda tühiseks. Kuidas peaks Juku valima takistite $R_1$ ja $R_2$ väärtused, kui ta taotleb, et maksimaalne laadimisvool ei ületaks \SI{100}{mA} ja laadimisvool muutuks nulliks, kui akumulaator on täielikult laetud? 

\begin{center}
	\includegraphics[width=\linewidth]{2008-v2g-08-yl}
\end{center}

\hint
Mõlema ülesandes antud tingimuse jaoks on võimalik kirja panna vastav võrrand (kasutades näiteks Kirchhoffi seadusi) ning saadud võrrandisüsteemi lahendid peaksidki olema $R_1$ ja $R_2$.

\solu
Olgu aku klemmide pinge $U$ ning voolutugevus $I$. Voolutugevus takistis $R_2$ on seega $U/R_2$ ja voolutugevus takistis $R_1$ avaldub kui $U/R_2 + I$. Teise Kirchhoffi seaduse kohaselt 
\[
U+\left(\frac{U}{R_{2}}+I\right) R_{1}=U_{0} \quad \Rightarrow \quad U R_{1}-\left(U_{0}-U\right) R_{2}+I R_{1} R_{2}=0,
\]
kus $U_0 = \SI{6}{V}$. Laadimisgraafikult leiame, et maksimaalne vool $I = \SI{0,1}{A}$ vastab pingele $U = \SI{1,2}{V}$, kui aga $U = \SI{1,5}{V}$ siis peab olema $I = 0$. Seega $R_1$ ja $R_2$ määramiseks saame võrrandisüsteemi 
\[
\num{1,2}R_1 - \num{4,8}R_2 + \num{0,1}R_1R_2 = 0, \quad \num{1,5}R_1 - \num{4,5}R_2 = 0. 
\]
Selle lahend on $R_1 = \SI{12}{\ohm}, R_2 = \SI{4}{\ohm}$.
\probend