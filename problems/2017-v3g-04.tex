\ylDisplay{Korvpall} % Ülesande nimi
{Aigar Vaigu} % Autor
{lõppvoor} % Voor
{2017} % Aasta
{G 4} % Ülesanne nr.
{3} % Raskustase
{
% Teema: Dünaamika
\ifStatement
Kui täita korvpall õhu asemel heeliumiga, siis lennu kaugus muutub. Korvpalli läbimõõt $D=\SI{23}{cm}$, kesta mass $m=\SI{0,63}{kg}$, õhu tihedus $\rho_0=\SI{1,2}{kg/m^3}$, heeliumi tihedus $\rho\idx{He}=\SI{0,17}{kg/m^3}$. 
Mitu korda kaugemale on võimalik visata heeliumiga täidetud korvpalli võrreldes õhuga täidetud korvpalliga, kui nende algkiirused on võrdsed? Arvutustes õhutakistusega mitte arvestada.
\fi


\ifHint
Kõige kaugema viskekauguse jaoks on optimaalne viskenurk \SI{45}{\degree}. Viskekaugus sõltub pallile mõjuvast kiirendusest, mida saab omakorda leida, kasutades Archimedese jõudu.
\fi


\ifSolution
Kui palli algkiirus on $v$, siis kiiruse vertikaalne ja horisontaalne komponent on vastavalt $v\sin\theta$ ja $v\cos\theta$, kus $\theta$ on viskenurk. Pall liigub mööda parabooli ja poole lennuaja möödudes on vertikaalne kiiruse komponent null. Kuna kiirendus $a$ on konstantne, siis $\frac{at}{2} = v\sin\theta$, kust $t = \frac{2v\sin\theta}{a}$. Horisontaalne kiiruskomponent on konstantne, seega on horisontaalselt läbitud vahemaa
$$L=v\cos\theta t =\frac{v^2}{a}\cdot 2\sin\theta\cos\theta = \frac{v^2}{a}\sin (2\theta ).$$
Kõige kaugemale saab palli mõlemal juhul visata siis, kui viskenurk on \num{45} kraadi. Seega nurk kahe palli puhul ei muutu ja oluline on see, et kaugus on pöördvõrdeline kiirendusega
$$ L \propto 1/a.$$
Peame arvestama, et lisaks palli enda raskusjõule mõjub pallile ka üleslükkejõud ja ka pallis oleva gaasi raskusjõudu tuleb arvestada, seega on kahele pallile mõjuvad kiirendused erinevad. Kuna korvpalli kest on õhukene, siis on korvpalli sees oleva gaasi ruumala $V=\frac{4}{3}\pi \left( \frac{D}{2} \right)^3$. Pallidele mõjuvateks summaarseteks jõududeks tuleb kahel juhul
$$ F\idx{õhk}=\rho_0Vg-(m+\rho_0V)g=-mg,$$
$$F_{He}= \rho_0Vg-(m+\rho\idx{He}V)g=-mg + (\rho_0-\rho\idx{He})Vg.$$

Kuna jõud on võrdelised kiirendustega, siis jõudude suhe on ka kiirenduste suhe.
$$\frac{a\idx{õhk}}{a\idx{He}}=\frac{F\idx{õhk}}{F\idx{He}}=\frac{mg}{mg - (\rho_0-\rho\idx{He})Vg}.$$
Kuna viske kaugus on võrdeline kiirenduse pöördväärtusega ($L\propto 1/a$), saame kauguste suhteks
$$\frac{L\idx{He}}{L\idx{õhk}} = \frac{a\idx{õhk}}{a\idx{He}} = \frac{mg}{mg - (\rho_0-\rho\idx{He})Vg} = 1 + \frac{(\rho_0-\rho\idx{He})V}{mg - (\rho_0-\rho\idx{He})Vg}.$$
Seega saab heeliumiga täidetud korvpalli visata
$\frac{m}{m - (\rho_0-\rho\idx{He})V}$
korda kaugemale.
\fi


\ifEngStatement
% Problem name: Basketball
If you fill a basketball with helium instead of air, then the length of its flight will change. The basketball’s diameter is $D=\SI{23}{cm}$, shell’s mass is $m=\SI{0,63}{kg}$, air density $\rho_0=\SI{1,2}{kg/m^3}$, helium’s density $\rho\idx{He}=\SI{0,17}{kg/m^3}$. How many times further is it possible to throw a basketball filled with helium compared to a basketball filled with air if their initial speeds are equal? Do not account for air resistance.
\fi


\ifEngHint
The optimal throwing angle for the furthest throw would be $\SI{45}{\degree}$. The throw length depends on the acceleration applied on the ball, which in turn can be found by using Archimedes’ force.
\fi


\ifEngSolution
IIf the initial velocity of the ball is $v$ then the vertical and horizontal component of the velocity are accordingly $v\sin\theta$ and $v\cos\theta$ where $\theta$ is the throwing angle. The ball moves along a parabola and after half of the flight’s time has passed the vertical component of the velocity is zero. Because the acceleration $a$ is constant then $\frac{at}{2} = v\sin\theta$ where $t = \frac{2v\sin\theta}{a}$. The horizontal component of the velocity is constant, therefore the distance covered horizontally is
$$L=v\cos\theta t =\frac{v^2}{a}\cdot 2\sin\theta\cos\theta = \frac{v^2}{a}\sin (2\theta ).$$ 
The ball can be thrown furthest in both cases when the throwing angle is 45 degrees. This angle does not change for the two balls and what is important is that the distance is reversely proportional to acceleration
$$ L \propto 1/a.$$
We have to consider that in addition to the ball’s gravity force there is also buoyancy force applied and inside the ball there is the gravity force of the gas, thus the accelerations applied to the two balls are different. Since the shell of the ball is thin then the volume of the gas inside the ball is $V=\frac{4}{3}\pi \left( \frac{D}{2} \right)^3$. The total forces applied to the ball for two cases are
$$ F\idx{õhk}=\rho_0Vg-(m+\rho_0V)g=-mg,$$
$$F_{He}= \rho_0Vg-(m+\rho\idx{He}V)g=-mg + (\rho_0-\rho\idx{He})Vg.$$
Because the forces are proportional to accelerations then the ratio of the forces is also the ratio of the accelerations.
$$\frac{a\idx{õhk}}{a\idx{He}}=\frac{F\idx{õhk}}{F\idx{He}}=\frac{mg}{mg - (\rho_0-\rho\idx{He})Vg}.$$ 
Because the length of the throw is proportional to the inverse of acceleration ($L\propto 1/a$) we get the ratio of lengths
$$\frac{L\idx{He}}{L\idx{õhk}} = \frac{a\idx{õhk}}{a\idx{He}} = \frac{mg}{mg - (\rho_0-\rho\idx{He})Vg} = 1 + \frac{(\rho_0-\rho\idx{He})V}{mg - (\rho_0-\rho\idx{He})Vg}.$$ 
Thus, the basketball filled with helium can be thrown $\frac{m}{m - (\rho_0-\rho\idx{He})V}$ times further.
\fi
}