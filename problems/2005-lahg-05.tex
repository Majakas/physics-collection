\ylDisplay{Juhe} % Ülesande nimi
{Tundmatu autor} % Autor
{lahtine} % Voor
{2005} % Aasta
{G 5} % Ülesande nr.
{3} % Raskustase
{
% Teema: Elektrostaatika
\ifStatement
Sirgjooneline juhe asub sügaval maa all ühtlases pinnases. Lekkevool ühikulise pikkusega juhtmest on $i$. Leidke lekkevoolu tihedus (\si{A/m^2}) kaugusel $r$ juhtmest. Juhtme pikkus on palju suurem kui $r$. Lekkevool on konstantne piki juhet.

\emph{Märkus}. lekkevooluks nimetatakse voolu, mis levib isolaatorites. 
\fi


\ifHint
Kehtib laengute jäävus --- valides suvalise suletud pinna, peab pinda läbiv vool olema võrdne pinna sisse jääva summaarse lekkevooluga.
\fi


\ifSolution
Valime mõttelise ühikulise pikkusega silindrilise kontuuri, mille raadius on $r$. Selle
pindala on $S = 2\pi r$, sellest voolab läbi vool $i$. Voolutihedus tuleb seega
\[
j = \frac{i}{2\pi r}.
\]
\fi
}