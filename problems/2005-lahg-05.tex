\setAuthor{Tundmatu autor}
\setRound{lahtine}
\setYear{2005}
\setNumber{G 5}
\setDifficulty{3}
\setTopic{Elektrostaatika}

\prob{Juhe}
Sirgjooneline juhe asub sügaval maa all ühtlases pinnases. Lekkevool ühikulise pikkusega juhtmest on $i$. Leidke lekkevoolu tihedus (\si{A/m^2}) kaugusel $r$ juhtmest. Juhtme pikkus on palju suurem kui $r$. Lekkevool on konstantne piki juhet.

\emph{Märkus}. Lekkevooluks nimetatakse voolu, mis levib isolaatorites. 

\hint
Kehtib laengute jäävus --- valides suvalise suletud pinna, peab pinda läbiv vool olema võrdne pinna sisse jääva summaarse lekkevooluga.

\solu
Valime mõttelise ühikulise pikkusega silindrilise kontuuri, mille raadius on $r$. Selle
pindala on $S = 2\pi r$, sellest voolab läbi vool $i$. Voolutihedus tuleb seega
\[
j = \frac{i}{2\pi r}.
\]
\probend