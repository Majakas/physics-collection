\ylDisplay{Traatrõngad} % Ülesande nimi
{Jaan Kalda} % Autor
{lõppvoor} % Voor
{2014} % Aasta
{G 9} % Ülesande nr.
{8} % Raskustase
{
% Teema: Kinemaatika
\ifStatement
Kaks ühesugust traatrõngast raadiusega $R$ on üksteise vahetus läheduses, rõngaste tasandid on paralleelsed ning rõngad puudutavad üksteist punktides $A$ ja $B$. Kaarele $AB$ vastav kesknurk on vaadeldaval ajahetkel $\alpha$. Alumine rõngas on paigal, ülemine pöörleb nurkkiirusega $\omega$ ümber punkti $A$ läbiva ning rõngaste tasanditega risti oleva telje. Leidke rõngaste puutepunkti $B$ kiirus antud ajahetkel.
\fi


\ifHint
Liikudes süsteemi, mis pöörleb nurkkiirusega $\omega/2$, on võimalik olukorra sümmeetriat ära kasutada.
\fi


\ifSolution
Läheme süsteemi, mis pöörleb nurkkiirusega $\omega/2$; seal on näha, et lõikepunkt ei pöörle, vaid liigub radiaalselt.
Seega, laboratoorses süsteemis on selle nurkkiirus $\omega/2$; sellise nurkkiirusega pöörleb kõõl $AB$; et kesknurk on kahekordne piirdenurk, siis 
raadius $OB$ (kus $O$ on seisva rõnga keskpunkt) pöörleb nurkkiirusega $\omega$ ning järelikult on lõikepunkti kiirus samaselt võrdne $\omega R$-ga.
\fi


\ifEngStatement
% Problem name: Wire rings
Consider two identical rings made of wire with a radius $R$. The planes of the rings are parallel and the rings are touching each other at points $A$ and $B$. The central angle for the arc $AB$ is observable at a moment of time $\alpha$. The lower ring is still, the upper is rotating with an angular velocity $\omega$ around the axis that intersects with the point $A$ and that is perpendicular to the planes of the rings. Find the speed of the point $B$ at the given moment of time.
\fi


\ifEngHint
Observing the system that rotates with the angular velocity $\omega/2$ it is possible to use the symmetry of the situation.
\fi


\ifEngSolution
Let us go to a frame of reference that rotates with an angular velocity $\omega/2$; there it can be seen that the intersection point does not rotate but moves radially. Therefore in a laboratory frame of reference its angular velocity is $\omega/2$; the chord $AB$ rotates with this angular velocity; since a central angle is double an inscribed angle then the radius $OB$ (where $O$ is the center of the still ring) rotates with the angular velocity $\omega$ and therefore the velocity of the intersection point is similarly equal to $\omega R$.
\fi
}