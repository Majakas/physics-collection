\ylDisplay{Tanker}
{Koit Timpmann}
{Lõppvoor}
{2018}
{PK 1} % Autor
{1} % Raskustase
{
\ifStatement

    Kaks matkaselli pidid jõudma võimalikult kiiresti ja üheaegselt $s = \SI{40}{km}$ kaugusel olevasse laagrisse. Kuna neil oli kahe peale ainult üks äbarik jalgratas, otsustasid nad, et sõidavad jalgrattaga vaheldumisi. Kui kaua „igavles'' ratas tee ääres, kui matkasellid jooksid kiirusega $v_1 = \SI{8}{km/h}$ ja sõitsid rattaga kiirusega $v_2 = \SI{15}{km/h}$?


\fi


\ifHint

Ratta seismisaeg on mõlema matkaselli summaarse teel viibitud aja ja liikumisele kulunud aja vahe.

\fi


\ifSolution

Et jõuda laagrisse üheaegselt, peab kumbki matkasell pool teed läbima joostes ja pool jalgrattal. Seega on aeg, mis kulub tee läbimiseks
\[ t = \frac{s}{2v_1}+\frac{s}{2v_2} = \SI{230}{min}. \]
Ratas seisis tee ääres
\[ t_{ratas} = t-t_2,\quad \text{kus}\quad t_2=\frac{s}{v_2} = \SI{160}{min} \]
Ratas „igavles'' tee ääres $t_{ratas} = \SI{230}{min} - \SI{160}{min} = \SI{70}{min}.$

\fi
}