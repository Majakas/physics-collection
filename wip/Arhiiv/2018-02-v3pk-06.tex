\ylDisplay{Jää sulamine} % Ülesande nimi
{Koit Timpmann} % Autor
{Lõppvoor} % Voor
{2018} % Aasta
{PK 6} % Ülesanne nr.
{2} % Raskustase
{
\ifStatement
Jäätükk massiga $m = \SI{100}{g}$ ja temperatuuriga $t_0 = \SI{0}{\celsius}$ ümbritseti soojusisolatsiooni kihiga ja paigutati hüdraulilise pressi alla, kus sellele jäätükile avaldati rõhku $p = \SI{550}{atm}$ (\SI{1}{atm} on rõhk, mis on võrdne õhurõhuga normaaltingimustel). Leidke selles protsessis tekkiva vee mass, kui on teada, et jää sulamistemperatuuri alanemine on võrdeline jääle avaldatud rõhuga ning rõhu suurenemisel $\Delta p = \SI{138}{atm}$ alaneb jää sulamistemperatuur $\Delta t = \SI{1}{\celsius}$ võrra. Jää erisoojus $c = \SI{2100}{J/(kg\cdot\celsius})$ ja sulamissoojus on $\lambda = \SI{330}{kJ/kg}$.
\fi


\ifHint
Kuna sulamistemperatuuri alanemine on võrdeline jääle avaldatud rõhuga, on rõhu ja sulamistemperatuuri alanemise suhe konstantne. Sulanud jää massi saab leida kasutades energia jäävuse seadust.
\fi


\ifSolution
Leiame jää sulamistemperatuuri rõhu $p$ all
\[ t_1 = \frac{p}{\frac{\Delta p}{\Delta t}} = \SI{-4}{\celsius} \]
Jää saab sulada vabanenud soojushulga arvelt, seega
\[ \lambda m_{sulanud} = cm_j(t_0 - t_1) \hence m_{sulanud} = \SI{2,5}{g}. \]
\fi
}