\documentclass{article}

\usepackage[T1]{fontenc} % Output font encoding
\usepackage[utf8]{inputenc} % Input encoding
\usepackage[estonian]{babel}
\usepackage{fontspec}
\usepackage{ifthen}
\usepackage[per = fraction, expproduct=cdot, decimalsymbol=comma]{siunitx}
\usepackage[xetex]{graphicx} % Built with XeLaTeX
\usepackage{wrapfig}
\usepackage{tikz}
\usepackage{pgfplots} % For tikz graphs
\usetikzlibrary{intersections} % Lõikumispunktide ja joonte nimede jaoks
\usetikzlibrary{decorations.pathreplacing, positioning}
\usetikzlibrary{arrows,calc,decorations.markings,math,arrows.meta}
\usepackage[european]{circuitikz}
\usepackage{adjustbox}
\usepackage{pgffor} % For loop
\usepackage[nointegrals]{wasysym} % Fullmoon, \iint conflict with amsmath
\usepackage{amsmath} % cases
\usepackage{amssymb} % \measuredangle
\usepackage[shortlabels]{enumitem} % enumerate, [\alph*)], conflict with package enumerate, 2013lahg5

\newcommand{\D}{\textrm{d}}

\newcommand{\add}[2]{\((#1+#2)\)}

\hbadness=10000 % Getting rid of the warnings
\hfuzz=\maxdimen
\vbadness=10000
\vfuzz=\maxdimen


%\newenvironment{boxed}[1]
%	{\begin{center}
%	#1\\
%	\begin{tabular}{|p{0.9\textwidth}|}
%	\hline\\
%	}
%	{
%	\\\hline
%	\end{tabular}
%	\end{center}
%	}

\newcounter{ylCounter}%[section]



\newif\ifStatement
\newif\ifHint
\newif\ifSolution

% The arguments are:
% 1) File name
% 2) Author name
% 3) Round name
% 4) Year
% 5) Problem number
% 6) Difficulty
% 7) Problem statement, hints and solutions seperated by if
%    conditionals
\newcommand{\ylDisplayy}[7]
{
	\stepcounter{ylCounter}
	\textbf{\arabic{ylCounter}.}
	\textbf{#1}
	\foreach \f in {1, ..., #6} {
		\newmoon
	}\\
	\small{\emph{Autor: #2, #3, #4, #5}}
	
	#7
}

% Displays the problem statement, hint and solution all at once
\newcommand{\ylDisplay}[7]
{
	\setcounter{equation}{0} % Make the equation labels start at 1 for every problem
	\stepcounter{ylCounter}
	\textbf{\arabic{ylCounter}.}
	\textbf{#1}
	\foreach \f in {1, ..., #6} {
		\newmoon
	}\\
	{\small\emph{Autor: #2, #3, #4, #5}}
	
	\Hintfalse
	\Solutionfalse
	
	\Statementtrue
	{\small\textbf{Statement.}}
	#7
	\Statementfalse
	
	\Hinttrue
	{\small\textbf{Hint.}}
	#7
	\Hintfalse
	
	\Solutiontrue
	{\small\textbf{Solution.}}
	#7
	\Solutionfalse
	
}

\newcommand{\hence}{\quad \Rightarrow \quad} % Mõned lahendused kasutavad seda.
\newcommand{\idx}[1]{_{\mathrm{#1}}} % Püstise kirjaga alaindeks
\newcommand{\TwoDigits}[1]{\ifnum#1<10 0#1\else #1\fi} % Paneb ühekohalisele arvule 0 ette


\begin{document}


\Statementtrue
\Hinttrue
\Solutiontrue


\foreach \teema in {Dünaamika, Elektriahelad, Elektriväljad, Gaasid, Geomeetriline-Optika, Kinemaatika, Laineoptika, Magnetväljad, Staatika, Termodyn, Varia, Vedelike-mehaanika, Taevamehaanika} {%
	\section{\teema}
    \graphicspath{{\teema/}} % Define the directory for the graphics
    
	\foreach \dif in {1, ..., 10} {%
    	\foreach \year in {2011, ..., 2018} {%
    		\foreach \round in {lahg, v2g, v3g} {%
    			\foreach \problem in {1, ..., 10} {%
    				\edef\FileName{\teema/\year-\round-\TwoDigits{\problem}-\TwoDigits{\dif}}
    				%\FileName\\
    				\IfFileExists{\FileName} {%
            			\input{\FileName}%
       				} { }
       			}%
        	}%
    	}%
	}%
}%

%\foreach \year in {2010, ..., 2018} {%
%    	\foreach \round in {lahg, v2g, v3g} {%
%    		\foreach \problem in {1, ..., 10} {%
%    			\foreach \dif in {1, ..., 10} {%
%    				\edef\FileName{\year-\round-\TwoDigits{\problem}-\TwoDigits{\dif}}
%    				
%    				\IfFileExists{\FileName} {%
%        				%\FileName\\
%            			\input{\FileName}%
%       				} { }
%       			}%
%        	}%
%    	}%
%	}%


%\input{2018-03-v3pk-10}

\end{document}
