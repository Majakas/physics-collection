\ylDisplay{Vooluring} % Ülesande nimi
{Koit Timpmann} % Autor
{Lõppvoor} % Voor
{2018} % Aasta
{PK 7} % Ülesanne nr.
{3} % Raskustase
{
\ifStatement
\begin{wrapfigure}{r}{0.28\textwidth}%
\vspace{-30 pt}%
\begin{adjustbox}{scale=0.7}
\begin{circuitikz}[scale=1.0]
  \ctikzset{ label/align = straight }
  %\ctikzset{resistor/length=0.6}
  \draw
  
  (4,-0.25) to[battery1, l=$U$] (0,-0.25) -* (0,2)
  to[resistor, l=$R_1$] (2,3)
  to[resistor, l=$R_2$] (4,2) -- (4,-0.25)
  (4,2) to[resistor, l=$R_4$] (2,1)
  to[resistor, l=$R_3$] (0,2)
  (2,3) -- (2,1);

  \node[circle,fill=black,inner sep=0pt,minimum size=3pt,label=left:{A}] () at (0,2) {};
  \node[circle,fill=black,inner sep=0pt,minimum size=3pt,label=right:{B}] () at (4,2) {};
  \node[circle,fill=black,inner sep=0pt,minimum size=3pt,label=above:{C}] () at (2,3) {};
  \node[circle,fill=black,inner sep=0pt,minimum size=3pt,label=below:{D}] () at (2,1) {};

\end{circuitikz}
\end{adjustbox}
\vspace{-40 pt}%
\end{wrapfigure}

Vooluringis on neli takistit väärtustega $R_1=\SI{2}{\ohm}$, $R_2=\SI{3}{\ohm}$, $R_3=\SI{6}{\ohm}$ ja $R_4=\SI{7}{\ohm}$. 
Punktide $C$ ja $D$ vahel on juhe, mille takistus on $\SI{0}{\ohm}$. Punktide $A$ ja $B$ vahele on rakendatud pinge suurusega $U=\SI{18}{V}$. Kui suur on voolutugevus juhtmes $CD$?

\fi


\ifHint
Vooluring koosneb kahest rööbiti ühendusest. Vaadeldes punkti $C$ on näha, et elektrivool juhtmes $CD$ on voolutugevuste vahe läbi takistite $R_1$ ja $R_2$.
\fi


\ifSolution
Kuna juhtmel $CD$ takistus puudub, võime vooluringi punktid $C$ ja $D$ lugeda samaks punktiks $C$. Sel juhul on vooluringis rööbiti ühendatud takistused $R_1$ ja $R_3$ ning takistused $R_2$ ja $R_4$ omavahel jadamisi ühendatud.\\
Esimese rööpühenduse takistus 
\[ R_{AC} = \frac{R_1R_3}{R_1+R_3} = \SI{1,5}{\ohm} \]
Teise rööpühenduse takistus
\[ R_{CB} = \frac{R_2R_4}{R_2+R_4} = \SI{2,1}{\ohm} \]
Vooluringi kogutakistus $R = \SI{3,6}{\ohm}$\\
Arvutame voolutugevuse vooluringis
\[ I = \frac{U}{R} = \SI{5}{A} \]
Vooluringi otstele rakendatud pinge jaguneb esimesele ja teisele rööpühendusele
\[ U_{AC} = IR_{AC} = \SI{7,5}{V} \] 
\[ U_{CB} = IR_{CB} = \SI{10,5}{V} \] 
Kuna takistid $R_1$ ja $R_3$ on rööbiti ühendatud, on pinge mõlema takisti otstel \SI{7,5}{V}. Takistite $R_2$ ja $R_4$ otstel on pinge \SI{10,5}{V}.\\
Ohmi seadusest saame, et voolutugevus takistis $R_1$ on 
\[ I_1 = \frac{U_1}{R_1} = \SI{3,75}{A} \]
Voolutugevus sellega jadamisi olevas takistis $R_2$ on aga 
\[ I_2 = \frac{U_2}{R_2} = \SI{3,5}{A} \]
Seega punktist $C$ peab osa elektrivoolust liikuma mööda juhet $CD$ punkti $D$. Voolutugevus juhtmes $CD$ on $I_{CD} = I_1 - I_2 = \SI{0,25}{A}$.
\fi
}